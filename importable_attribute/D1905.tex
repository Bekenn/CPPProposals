\documentclass{wg21}


\usepackage{xcolor}
\usepackage{soul}
\usepackage{ulem}
\usepackage{fullpage}
\usepackage{parskip}
\usepackage{csquotes}
\usepackage{listings}
\usepackage{enumitem}
\usepackage{makecell}
\usepackage{longtable}


%\DeclareUnicodeCharacter{201F}{‟}


\newcommand{\cc}[1]{\mintinline{c++}{#1}}
%\newminted[cpp]{c++}{}


\title{In-Source Mechanism to Identify Importable Headers}
\docnumber{P1905R0}
\audience{SG-15, EWG}
\author{Corentin Jabot}{corentin.jabot@gmail.com}

\begin{document}
\maketitle

\section{Target}

C++20

\section{Abstract}

The standard specifies that \emph{an importable header is a member of an implementation-defined set of headers}.
This paper proposes a mechanism to specify that a header is importable from within the source code of that header.

\section{Motivation}

Defining the set of importable header is accepted to require the collaboration between

\begin{itemize}
\item Developers, as determining whether a header is importable (not affected by the state of the preprocessor), cannot be easily computed automatically.
\item Build systems which are the primary tool by which developers can specify how a program can be compiled. 
\item The compiler which needs to know the list of importable header.
\end{itemize}

And of course, compiling a program often involves multiple build systems that do not share the same build file formats.

SG-15 is looking at recommending a set of formats and/or protocols that could be used to share whether a given header can be considered
importable across these entities, however that presents several limitations:

\begin{itemize}
\item All tools involved need to be able to specify, share and consume that information which, given the state of the ecosystem, will take many years.
\item It puts more knowledge about a program in the build system, and less in the source which makes maintenance more complicated and notably makes it harder to replace the build system.
\item It makes it harder to develop and use zero-configuration build systems as well as header-only libraries - which would now need
to be configured in build scripts so they can benefit from being importable.
\item It put the decision of whether a library is importable on the build system maintainer rather than on the person who wrote the header - But only the people maintaining the code know whether a header is and will remain importable.
\end{itemize}

\section{Proposal}

We propose a \tcode{\#pragma importable} which, when it appears at the beginning of a header indicates that the included header is importable.

This can be used: 

\begin{itemize}
\item By compilers to treat the included header as an imported header unit (see 15.2.7)
\item By build systems providing pre-scanning to pre-compile the importable header if the implementation supports that use case.
\item By IDE and other tools to treat the header as importable which can have drastic performance benefits.
\end{itemize}

This attribute would not replace other implementation-defined mechanisms to specify that a header is importable,
but rather add a new one.

\begin{note}
Because the proposed mechanism is designed to be ignorable, it does not replace nor alleviate the need for include guards or \tcode{pragma once}. 
\end{note}

\subsection{Why a pragma?}

The syntax chosen to identify importable headers mustn't make the program ill-formed 
if compiled with a C++17 compiler or one that does not understand that syntax.
In general, whether an importable header is imported or included should not affect the behavior of a program.

This leave us with 2 choices:
\begin{itemize}
\item A pragma
\item An attribute (eg. \tcode{[[importable]]})
\end{itemize}

However, a pragma has a few benefits, most notably it can be handled in phase 4 and is already specified
to appear at the beginning of a line, which makes it easier for tooling to deal with.
It also matches the behavior of whether an include is treated as in import - decision handled during prepossessing.

Having a standard \tcode{\#pragma} directive is novel, however, all existing implementation tested (gcc, clang, msvc, icc)
correctly ignore a \tcode{\#pragma importable} (none of the implementation give a meaning to \tcode{\#pragma importable}), 
so the standard can safely claim that syntax.

It would ultimately be possible to use an attribute if there was a preference for that.


\section{Proposed Wording}

\begin{quote}
\rSec1[cpp.pragma]{Pragma directive}%
\indextext{preprocessing directive!pragma}%
\indextext{\idxcode{\#pragma}|see{preprocessing directives, pragma}}

\begin{addedblock}
\pnum
A preprocessing directive of the form\\
\begin{ncsimplebnf}
    \terminal{\# pragma importable} 
\end{ncsimplebnf}
Indicates that the header or source file being processed and identified by its \emph{header-name} is an importable header ([module.import]).

In the header or source file being processed, this directive shall not appear after a \tcode{\#define} or \tcode{\#include} preprocessing directive or any \emph{preprocessing-token}.

\begin{note}
It is implementation defined whether the header or source file being processed will be treated as if it were imported by an \tcode{import} directive ([cpp.import]) of the form:

\begin{ncsimplebnf}
    import header-name ; new-line
\end{ncsimplebnf}
\end{note}

\end{addedblock}

\pnum
A preprocessing directive of the form\\
\begin{ncsimplebnf}
\terminal{\# pragma} \opt{pp-tokens} new-line
\end{ncsimplebnf}
causes the implementation to behave
in an \impldef{\tcode{\#pragma}} manner.
The behavior might cause translation to fail or cause the translator or
the resulting program to behave in a non-conforming manner.
Any pragma that is not recognized by the implementation is ignored.

\end{quote}

 
\begin{thebibliography}{9}
    
    \bibitem[N4830]{N4830}
    Richard Smith
    \emph{Working Draft, Standard for Programming Language C++}\newline
    \url{https://wg21.link/n4830}
    
\end{thebibliography}

\end{document}