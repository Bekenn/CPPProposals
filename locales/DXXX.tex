\documentclass{wg21}


\usepackage{xcolor}
\usepackage{soul}
\usepackage{ulem}
\usepackage{fullpage}
\usepackage{parskip}
\usepackage{csquotes}
\usepackage{listings}
\usepackage{enumitem}
\usepackage{makecell}
\usepackage{longtable}
\usepackage[utf8x]{inputenc}
\usepackage{graphicx}
\usepackage{amssymb}


%\DeclareUnicodeCharacter{201F}{‟}


\lstdefinestyle{base}{
  language=c++,
  breaklines=false,
  basicstyle=\ttfamily\color{black},
  moredelim=**[is][\color{green!50!black}]{@}{@},
  escapeinside={(*@}{@*)}
}

\newcommand{\cc}[1]{\mintinline{c++}{#1}}
%\newminted[cpp]{c++}{}


\title{Locales and encodings in a globalized world}
\docnumber{DXXXXXR0}
\audience{SG-16, EWG}
\author{Corentin Jabot}{corentin.jabot@gmail.com}

\begin{document}

\maketitle

\begin{flushright}
	\hfill \break
	\hfill \break
	\textit{It's just semantic! - Kevlin Henney}
\end{flushright}


\section{Abstract}

For historical reasons, all text encodings mentioned in the standard are derived from
a locale object, which does not necessarily match the reality of how programs and system interact.

This models works poorly with modern understanding of text, ie the Unicode model separates encoding
from locales which are purely rules for formatting and text transformations but do not affect
which characters are represented by a sequence of code units.

Moreover, the standard does not provide a way to query which encoding are expected or used by the system,
leading to guess work and unavoidable UB.

\section{The many text encodings of a C++ system}

Text in a technical sense is a sequence of bytes to which is virtually attached an encoding.
Without encoding, a blob of data simply cannot be interpreted as text.

In many case the encoding used to encode a string is not communicated along with that string and its
encoding is therefore presumed with more or less success.

Generally, it is useful to know the encoding of a string when

\begin{itemize}
	\item Transferring data as text between systems or processes (io)
	\item Textual transforming of data
	\item Interpretation of a piece of data
\end{itemize}

In the purview of the standard, text i/o text originates from
\begin{itemize}
	\item The source code (literals)
	\item The iostream library as well as system functions
\end{itemize}

Locales provide transformations and conversions facilities and as such, in the current model have an encoding attached to them.

There are therefore 2 set of encodings of primary interest:

\begin{itemize}
	\item The encoding of narrow and wide characters and string literals
	\item The narrow and wide encodings used by a program when sending of receiving string from it's environment
\end{itemize}

\note Because they have different code units sizes, narrow and wide strings have different encodings.
char8_t, char16_t, char32_t literals are assumed to be respectively UTF-8, UTF-16 and UTF-32 encoded.
\endnote

C++ does not differentiate at runtime



\section{Impact on the standard and implementations}

\section{Proposed wording}

 
\section{References}
%\begin{thebibliography}
%
%
%
%\bibitem[N4830]{N4830}
%Richard Smith
%\emph{Working Draft, Standard for Programming Language C++}\newline
%\url{https://wg21.link/n4830}
%
%\end{thebibliography}

\end{document}
