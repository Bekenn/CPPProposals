% !TeX program = luatex
% !TEX encoding = UTF-8


\documentclass{wg21}

\title{Referencing The Unicode Standard}
<<<<<<< HEAD
\docnumber{P2736R2}
=======
\docnumber{P2736R1}
>>>>>>> afb1661 (referencing the unicode standard r1)
\audience{SG-16}
\author{Corentin Jabot}{corentin.jabot@gmail.com}

\newcommand{\replaceucs}{\changed{UCS}{Unicode}}

\usepackage{color, colortbl}
\begin{document}
\maketitle

\section{Abstract}

We propose to reference The Unicode Standard instead of ISO 10646.
This proposal has no impact on implementations.
This resolves NB comments FR-010-133 and FR-021-013


\section{Revisions}

\subsection{R2}

- Wording improvements following SG16 feedback.\\
- Provide editorial guidance to resolve the merge conflict with \paper{P2713R1}.
- Applied Feedback from Core.

\subsection{R1}

- Wording improvements following SG16 feedback.

\subsection{R0}

Initial revision

\subsection{Motivation}

Unicode is developed as The Unicode Standard by The Unicode Consortium.
From that is derived the ISO 10646 standard.
However, Unicode has a much larger scope as it describes algorithms for normalization,
clusterization, comparison, case transformation, bi-directional text handling, and more.
C++ makes use of these features.

As such, in the current state, we reference ISO 10646 and Unicode, and The Unicode annexes.
Care must be taken that the versions referenced are coherent as Unicode algorithms
are designed to work with a matching specification and accompanying Unicode data.
This is challenging as Unicode and ISO 10646 have different release cycles.

Additionally, there are subtle differences in terminology, the main impact of which
is to consume time in study groups.
It is also an issue for implementers who usually reference
Unicode directly as Unicode delivers the necessary Unicode data in an easily extractible
and toolable format (whereas 1SO 10646 is just a PDF).

By referring to a single Unicode Standard, we can reduce the number of references
that we have to maintain (and thereby the number of dependencies the standard carries),
as well as making the standard clearer.

\subsection{Difference between Unicode and ISO 10646}

The Unicode standard explains:

\begin{quoteblock}
The Unicode Consortium maintains a strong working relationship with ISO/IEC
JTC1/SC2/WG2, the working group developing International Standard 10646. Today both
organizations are firmly committed to maintaining the synchronization between The Unicode Standard and ISO/IEC 10646. Each standard nevertheless uses its own form of reference and, to some degree, separate terminology.
\end{quoteblock}

\begin{quoteblock}
ISO/IEC 10646:2011 has significantly revised its discussion of encoding forms, compared
to earlier editions of that standard. The terminology for encoding forms (and encoding
schemes) in 10646 now matches exactly the terminology used in The Unicode Standard.
Furthermore, 10646 is now described in terms of a codespace U+0000..U+10FFFF, instead
of a 31-bit codespace, as in earlier editions. This convergence in codespace description has
eliminated any discrepancies in possible interpretation of the numeric values greater than
0x10FFFF.
\end{quoteblock}

As such, the definition of UTF-8, UTF-16, and UTF-32, as well as the set of supported abstract characters and their code point value is the same.
Most of the terminology changes operated consist of replacing "UCS" with "Unicode".

There are, however, a few points to mention.

ISO 10646 defines "character" as
\begin{quoteblock}
member of a set of elements used for the organization, control, or representation of textual data
Note 1 to entry – A graphic symbol can be represented by a sequence of one or several coded characters.
\end{quoteblock}

Unicode defines instead "abstract character"
\begin{quoteblock}
A unit of information used for the organization, control, or representation of textual data.
\end{quoteblock}

This is a positive change as the term character is both defined by the C++ standard and constantly misused.
in the long term, having a term to talk about an abstract character when we mean it is a positive change.

\subsection{Conformance with Unicode}

Unicode specifies a list of conformance requirements (Chapter 3.2)

\begin{itemize}
    \item C1: \emph{A process shall not interpret a high-surrogate code point or a low-surrogate code point as an abstract character.}\\
    C++ satisfies this requirement.

    \item C2: \emph{A process shall not interpret a noncharacter code point as an abstract character}
    \item C3: \emph{A process shall not interpret an unassigned code point as an abstract character.}\\

    We create characters out of thin air for specifying the translation set.
    However, we never exchange or materialize, or use these things except for specification purposes.

    \item C4: \emph{A process shall interpret a coded character sequence according to the character semantics established by this standard if that process does interpret that coded character sequence.}
    \item C4: \emph{A process shall not assume that it is required to interpret any particular coded character sequence.}
    \item C6: \emph{A process shall not assume that the interpretations of two canonical-equivalent character sequences are distinct}
    \item C7: \emph{When a process purports not to modify the interpretation of a valid coded character
        sequence, it shall make no change to that coded character sequence other than the possible replacement of character sequences by their canonical-equivalent sequences.}

    C++ satisfies these requirements.

    \item C8: \emph{When a process interprets a code unit sequence which purports to be in a Unicode
        character encoding form, it shall interpret that code unit sequence according to the
        corresponding code point sequence.}
    \item C9: \emph{When a process generates a code unit sequence which purports to be in a Unicode
        character encoding form, it shall not emit ill-formed code unit sequences.}

    C++ satisfies these requirements.

    \item C10: \emph{When a process interprets a code unit sequence which purports to be in a Unicode
        character encoding form, it shall treat ill-formed code unit sequences as an error condition and shall not interpret such sequences as characters}

    C++ satisfies this requirement, arguably. The C functions do behave weirdly in the presence of invalid code units, but do they actually purports to support Unicode if they don't support multi bytes encodings?

    \item C11: \emph{When a process interprets a byte sequence which purports to be in a Unicode character
        encoding scheme, it shall interpret that byte sequence according to the byte order and
        specifications for the use of the byte order mark established by this standard for that
        character encoding scheme}

    I think we do.

    \item C14: \emph{A process that tests Unicode text to determine whether it is in a Normalization Form
        shall do so in accordance with the specifications in Section 3.11, Normalization Forms}


    \item C16: \emph{Normative references to The Unicode Standard itself, to property aliases, to property
        value aliases, or to Unicode algorithms shall follow the formats specified in Section 3.1,
        Versions of The Unicode Standard.}

    This requirement should be satisfied with the adoption of this paper.
    Care must be taken when referencing properties.

    \item C17: \emph{Higher-level protocols shall not make normative references to provisional properties.}
    • Higher-level protocols may make normative references to informative properties.

    \item C17: \emph{If a process purports to implement a Unicode algorithm, it shall conform to the specification of that algorithm in the standard, including any tailoring by a higher-level protocol as permitted by the specification.}

    Yes.
\end{itemize}
Omitted requirements are not applicable.

\subsection{Terminology}
\begin{itemize}
\item D4 Character name: A unique string used to identify each abstract character encoded in
the standard.
\item D5 Character name alias: An additional unique string identifier, other than the character name, associated with an encoded character in the standard.
\item D7 Abstract character: A unit of information used for the organization, control, or representation of textual data.
\item D10 Code point: Any value in the Unicode codespace.
\item D11 Encoded character: An association (or mapping) between an abstract character and
a code point.
\item D76 Unicode scalar value: Any Unicode code point except high-surrogate and low-surrogate code points.
\item D77 Code unit: The minimal bit combination that can represent a unit of encoded text for processing or interchange.
\item D78 Code unit sequence: An ordered sequence of one or more code units.
\item D79 A Unicode encoding form assigns each Unicode scalar value to a unique code unit sequence.
\item D94 Unicode encoding scheme: A specified byte serialization for a Unicode encoding
form, including the specification of the handling of a byte order mark (BOM), if
allowed.
\item D16 Higher-level protocol: Any agreement on the interpretation of Unicode characters
that extends beyond the scope of this standard
\end{itemize}

\subsection{Character names}

Consistently refering to Unicode resolves the observation made by NB comment FR-021-01
that the set of character that can be spelled by named escape sequences \tcode{\textbackslash{N}\{\}} was, in the standard,
a subset of the set of code point with the XID_Continue or XID_Start properties, as the names were derived from ISO
and the list of elligible code points from Unicode.

\subsection{\tcode{__STDC_ISO_10646__}}

This macro, defined in [cpp] was the most difficult change.
Indeed its \tcode{yyyymm} is supposed to refer to the publication date of an ISO1646 revision.
Here is the thing though.
Past discussions in LLVM and the WG14 reflector have established that the wide execution encoding
may depend on the C library, and that macro is therefore defined by the C library.
On some platforms, the \tcode{wchar_t} is actually a runtime property, such that the use of a macro
is not suitable.

In addition, its value is not always updated and doesn't indicate much.
Indeed, all ISO1646 publications require a 4 bytes \tcode{wchar_t} for the storing of code points.

Therefore, we change the wording of that macro to state that its value is implementation-defined,
such that we do not have to reference ISO10646 and reword it slightly.

Further work is encouraged to maybe deprecate this macro.
Inspecting a large amount of open source code, the only use of the value we found was \tcode{(__STDC_ISO_10646__ <= 198700L)}.
Of course, there are numerous uses of the macro for its presence.

Note that we don't expect implementers to change the value of that macro, but it lets us not mention ISO 10646 there

\subsection{Removing the reference to ISO1646:2003}

The deprecated \tcode{codecvt_utf8} and \tcode{codecvt_utf16} refers to UCS-2.
Because this has long been obsoleted and removed from ISO 10646, we carry a dependency on ISO 10646:2003.

But, UCS-2 is a strict subset of UTF-16. So we reword the appropriate paragraph to say
"code points in the range U+0000-U+FFFF encoded as UTF-16" instead.

Any code point outside that range doesn't satisfy that requirement, and therefore cannot be encoded and causes the conversion to fail with \tcode{result::error}, preserving the existing behavior.

\subsection{Non-Normative reference to UAX \#29, Unicode Text Segmentation}

Revision 35 (Unicode 12) -> Revision 41 (Unicode 15)

This is used in [format.string.std] for the purpose of width estimation.
The only non-editorial applicable change in that document between that version and the last were
\begin{itemize}
\item Moved surrogate code points from Control to XX
\item Excluded prepended concatenation marks from Control.
\end{itemize}

Both are bug fixes.
Referencing version 15 ensures that the width of recent code points can be correctly estimated.
Some implementations refer to Unicode 15 already.

Once we can use the same version, we can remove the reference entierly, as unicode mentions
\begin{quoteblock}
This core specification, together with The Unicode code charts, The Unicode Character Database, and The Unicode Standard Annexes, defines The Unicode Standard.
The Unicode Standard Annexes form an integral part of The Unicode Standard.
\end{quoteblock}__

\subsection{Non-Normative reference to UAX \#31}

Left untouched, someone needs to go through the UAX31 Annex in the C++ standard to ensure consistency.
Given this UAX is changing a lot to support source code spoofing mitigation over the next year, we may consider
delaying that work?
If we do decided to delay that work, we should make sure UAX31 constently refer to the referenced version in the annex.


\section{Wording}

\rSec1[intro.refs]{Normative references}%
\indextext{normative references|see{references, normative}}%

\pnum
\indextext{references!normative|(}%
The following documents are referred to in the text
in such a way that some or all of their content
constitutes requirements
of this document. For dated references, only the edition cited applies.
For undated references, the latest edition of the referenced document
(including any amendments) applies.
\begin{itemize}
    % ISO documents in numerical order.
    \item ISO/IEC 2382, \doccite{Information technology --- Vocabulary}
    \item ISO 8601:2004, \doccite{Data elements and interchange formats ---
        Information interchange --- Representation of dates and times}
    \item ISO/IEC 9899:2018, \doccite{Programming languages --- C}
    \item ISO/IEC/IEEE 9945:2009, \doccite{Information Technology --- Portable
        Operating System Interface (POSIX)}
    \item ISO/IEC/IEEE 9945:2009/Cor 1:2013, \doccite{Information Technology --- Portable
        Operating System Interface (POSIX), Technical Corrigendum 1}
    \item ISO/IEC/IEEE 9945:2009/Cor 2:2017, \doccite{Information Technology --- Portable
        Operating System Interface (POSIX), Technical Corrigendum 2}
    \begin{removedblock}
    \item ISO/IEC 10646, \doccite{Information technology ---
        Universal Coded Character Set (UCS)}
    \item ISO/IEC 10646:2003,
    \doccite{Information technology ---
        Universal Multiple-Octet Coded Character Set (UCS)}
    \end{removedblock}
    \item ISO/IEC/IEEE 60559:2020, \doccite{Information technology ---
        Microprocessor Systems --- Floating-Point arithmetic}
    \item ISO 80000-2:2009, \doccite{Quantities and units ---
        Part 2: Mathematical signs and symbols
        to be used in the natural sciences and technology}
    % Other international standards.
    \item Ecma International, \doccite{ECMAScript Language Specification},
    Standard Ecma-262, third edition, 1999.
    \begin{addedblock}
    \item
    The Unicode Consortium.
    The Unicode Standard.
    Available from: \url{https://www.unicode.org/versions/latest/}
    \end{addedblock}

    \ednote{Using recommandation from https://unicode.org/versions/}

    \begin{removedblock}
    \item
    The Unicode Consortium.
    Unicode Standard Annex, UAX \#44, \doccite{Unicode Character Database}.
    Edited by Ken Whistler and Lauren\c{t}iu Iancu.
    Available from: \tcode{http://www.unicode.org/reports/tr44/}
    \item
    The Unicode Consortium.
    The Unicode Standard, \doccite{Derived Core Properties}.
    Available from: \tcode{https://www.unicode.org/Public/UCD/latest/ucd/DerivedCoreProperties.txt}
    \end{removedblock}

    \ednote{The reference to The Unicode standard encompasses its annexes and associated}
\end{itemize}

\pnum
The library described in ISO/IEC 9899:2018, Clause 7,
is hereinafter called the
\defnx{C standard library}{C!standard library}.

\pnum
The operating system interface described in ISO/IEC 9945:2009 is
hereinafter called \defn{POSIX}.

\pnum
The ECMAScript Language Specification described in Standard Ecma-262 is
hereinafter called \defn{ECMA-262}.
\indextext{references!normative|)}


\pnum
\begin{removedblock}
\begin{note}
    References to ISO/IEC 10646:2003 are used only
    to support deprecated features\iref{depr.locale.stdcvt}.
\end{note}
\end{removedblock}

\textcolor{noteclr}{[...]}

\rSec1[lex.phases]{Phases of translation}%

\pnum
\indextext{translation!phases|(}%
The precedence among the syntax rules of translation is specified by the
following phases.
\begin{wfootnote}
    Implementations behave as if these separate phases
    occur, although in practice different phases can be folded together.
\end{wfootnote}

\begin{enumerate}
    \item
    \indextext{character!source file}%
    \indextext{character set!basic source}%
    An implementation shall support input files
    that are a sequence of UTF-8 code units (UTF-8 files).
    It may also support
    an \impldef{supported input files} set of other kinds of input files, and,
    if so, the kind of an input file is determined in
    an \impldef{determination of kind of input file} manner
    that includes a means of designating input files as UTF-8 files,
    independent of their content.
    \begin{note}
        In other words,
        recognizing the U+feff byte order mark is not sufficient.
    \end{note}
    If an input file is determined to be a UTF-8 file,
    then it shall be a well-formed UTF-8 code unit sequence and
    it is decoded to produce a sequence of \replaceucs{} scalar values\changed{
    that constitutes the sequence of elements of the translation character set}{. A sequence of translation character set elements is then formed by mapping each Unicode scalar value to the corresponding translation character set element.}.

    For any other kind of input file supported by the implementation,
    characters are mapped, in an
    \impldef{mapping physical source file characters to translation character set} manner,
    to a sequence of translation character set elements\iref{lex.charset}
    (introducing new-line characters for end-of-line indicators).
\end{enumerate}

\rSec1[lex.charset]{Character sets}

\pnum
\indextext{character set|(}%
The \defnadj{translation}{character set} consists of the following elements:
\begin{itemize}
    \item
    each \added{abstract} character \changed{named by ISO/IEC 10646,
    as identified by its unique UCS scalar value}{assigned a code point in the Unicode codespace}, and
    \item
    a distinct character for each \replaceucs{} scalar value
    \changed{where no named character is assigned}{not assigned to an abstract character}.
\end{itemize}
\begin{note}
    \changed{ISO/IEC 10646}{Unicode} code points are integers
    in the range $[0, \mathrm{10FFFF}]$ (hexadecimal).
    A surrogate code point is a value
    in the range $[\mathrm{D800}, \mathrm{DFFF}]$ (hexadecimal).
    A \replaceucs{} scalar value is any code point that is not a surrogate code point.
\end{note}

\pnum
The \defnadj{basic}{character set} is a subset of the translation character set,
consisting of 96 characters as specified in \tref{lex.charset.basic}.
\begin{note}
    Unicode short names are given only as a means to identifying the character;
    the numerical value has no other meaning in this context.
\end{note}

%%

\begin{floattable}{Basic character set}{lex.charset.basic}{lll}
    \topline
    \lhdrx{2}{character} & \rhdr{glyph} \\ \capsep
    \ucode{0009} & \uname{character tabulation} & \\
    \ucode{000b} & \uname{line tabulation} & \\
    \ucode{000c} & \uname{form feed} & \\
    \ucode{0020} & \uname{space} & \\
    \ucode{000a} & \uname{line feed} & new-line \\
    \ucode{0021} & \uname{exclamation mark} & \tcode{!} \\
    \ucode{0022} & \uname{quotation mark} & \tcode{"} \\
    \ucode{0023} & \uname{number sign} & \tcode{\#} \\
    \ucode{0025} & \uname{percent sign} & \tcode{\%} \\
    \ucode{0026} & \uname{ampersand}  & \tcode{\&} \\
    \ucode{0027} & \uname{apostrophe} & \tcode{'} \\
    \ucode{0028} & \uname{left parenthesis} & \tcode{(} \\
    \ucode{0029} & \uname{right parenthesis} & \tcode{)} \\
    \ucode{002a} & \uname{asterisk} & \tcode{*} \\
    \ucode{002b} & \uname{plus sign} & \tcode{+} \\
    \ucode{002c} & \uname{comma} & \tcode{,} \\
    \ucode{002d} & \uname{hyphen-minus} & \tcode{-} \\
    \ucode{002e} & \uname{full stop} & \tcode{.} \\
    \ucode{002f} & \uname{solidus} & \tcode{/} \\
    \ucode{0030} .. \ucode{0039} & \uname{digit zero .. nine} & \tcode{0 1 2 3 4 5 6 7 8 9} \\
    \ucode{003a} & \uname{colon} & \tcode{:} \\
    \ucode{003b} & \uname{semicolon} & \tcode{;} \\
    \ucode{003c} & \uname{less-than sign} & \tcode{<} \\
    \ucode{003d} & \uname{equals sign} & \tcode{=} \\
    \ucode{003e} & \uname{greater-than sign} & \tcode{>} \\
    \ucode{003f} & \uname{question mark} & \tcode{?} \\
    \ucode{0041} .. \ucode{005a} & \uname{latin capital letter a .. z} & \tcode{A B C D E F G H I J K L M} \\
    & & \tcode{N O P Q R S T U V W X Y Z} \\
    \ucode{005b} & \uname{left square bracket} & \tcode{[} \\
    \ucode{005c} & \uname{reverse solidus} & \tcode{\textbackslash} \\
    \ucode{005d} & \uname{right square bracket} & \tcode{]} \\
    \ucode{005e} & \uname{circumflex accent} & \tcode{\caret} \\
    \ucode{005f} & \uname{low line} & \tcode{_} \\
    \ucode{0061} .. \ucode{007a} & \uname{latin small letter a .. z} & \tcode{a b c d e f g h i j k l m} \\
    & & \tcode{n o p q r s t u v w x y z} \\
    \ucode{007b} & \uname{left curly bracket} & \tcode{\{} \\
    \ucode{007c} & \uname{vertical line} & \tcode{|} \\
    \ucode{007d} & \uname{right curly bracket} & \tcode{\}} \\
    \ucode{007e} & \uname{tilde} & \tcode{\textasciitilde} \\
\end{floattable}

\pnum
The \grammarterm{universal-character-name} construct provides a way to name
other characters.

\begin{bnf}
    \nontermdef{n-char} \textnormal{one of}\br
    \terminal{A B C D E F G H I J K L M N O P Q R S T U V W X Y Z}\br
    \terminal{0 1 2 3 4 5 6 7 8 9}\br
    \textnormal{\unicode{002d}{hyphen-minus}}\br
    \textnormal{\unicode{0020}{space}}
\end{bnf}

\begin{bnf}
    \nontermdef{n-char-sequence}\br
    n-char\br
    n-char-sequence n-char
\end{bnf}

\begin{bnf}
    \nontermdef{named-universal-character}\br
    \terminal{\textbackslash N\{} n-char-sequence \terminal{\}}
\end{bnf}

\begin{bnf}
    \nontermdef{hex-quad}\br
    hexadecimal-digit hexadecimal-digit hexadecimal-digit hexadecimal-digit
\end{bnf}

\begin{bnf}
    \nontermdef{simple-hexadecimal-digit-sequence}\br
    hexadecimal-digit\br
    simple-hexadecimal-digit-sequence hexadecimal-digit
\end{bnf}

\begin{bnf}
    \nontermdef{universal-character-name}\br
    \terminal{\textbackslash u} hex-quad\br
    \terminal{\textbackslash U} hex-quad hex-quad\br
    \terminal{\textbackslash u\{} simple-hexadecimal-digit-sequence \terminal{\}}\br
    named-universal-character
\end{bnf}

\pnum
A \grammarterm{universal-character-name}
of the form \tcode{\textbackslash u} \grammarterm{hex-quad},
\tcode{\textbackslash U} \grammarterm{hex-quad} \grammarterm{hex-quad}, or
\tcode{\textbackslash u\{\grammarterm{simple-hexadecimal-digit-sequence}\}}
designates the character in the translation character set
whose \replaceucs{} scalar value is the hexadecimal number represented by
the sequence of \grammarterm{hexadecimal-digit}s
in the \grammarterm{universal-character-name}.
The program is ill-formed if that number is not a \replaceucs{} scalar value.

\pnum
\begin{removedblock}
A \grammarterm{universal-character-name}
that is a \grammarterm{named-universal-character}
designates the character named by its \grammarterm{n-char-sequence}.
A character is so named if the \grammarterm{n-char-sequence} is equal to
\begin{itemize}
    \item
    the associated character name or associated character name alias
    or
    \item
    the control code alias given in [lex.charset.ucn].
    \begin{note}
        The aliases in [lex.charset.ucn] are provided for control characters
        which otherwise have no associated character name or character name alias.
        These names are derived from
        the Unicode Character Database's \tcode{NameAliases.txt}.
        For historical reasons, control characters are formally unnamed.
    \end{note}
\end{itemize}
\begin{note}
    None of the associated character names,
    associated character name aliases,
    control code aliases have leading or trailing spaces.
\end{note}
\end{removedblock}

\begin{addedblock}
A \grammarterm{universal-character-name} that is a \grammarterm{named-universal-character} designates the corresponding character in the Unicode Standard (chapter 4.8 Name) if the \grammarterm{n-char-sequence} is equal to its character name or to one of its character name aliases of type “control”, “correction”, or “alternate”; otherwise, the program is ill-formed.

\begin{note}
	These aliases are listed in the Unicode Character Database’s \tcode{NameAliases.txt}. None of these names or aliases have leading or trailing spaces
\end{note}
\end{addedblock}

\ednote{Remove the table "Table 2: Control code aliases [tab:lex.charset.ucn]"}

\pnum
If a \grammarterm{universal-character-name} outside
the \grammarterm{c-char-sequence}, \grammarterm{s-char-sequence}, or
\grammarterm{r-char-sequence} of
a \grammarterm{character-literal} or \grammarterm{string-literal}
(in either case, including within a \grammarterm{user-defined-literal})
corresponds to a control character or
to a character in the basic character set, the program is ill-formed.
\begin{note}
    A sequence of characters resembling a \grammarterm{universal-character-name} in an
    \grammarterm{r-char-sequence}\iref{lex.string} does not form a
    \grammarterm{universal-character-name}.
\end{note}

\pnum
The \defnadj{basic literal}{character set} consists of
all characters of the basic character set,
plus the control characters specified in \tref{lex.charset.literal}.
\begin{removedblock}
\begin{note}
    The alias \uname{bell} for \ucode{0007} shown in ISO 10646
    is ambiguous with \unicode{1f514}{bell}.
\end{note}
\end{removedblock}

\begin{floattable}{Additional control characters in the basic literal character set}{lex.charset.literal}{ll}
    \topline
    \ohdrx{2}{character} \\ \capsep
    \ucode{0000} & \uname{null} \\
    \ucode{0007} & \uname{alert} \\
    \ucode{0008} & \uname{backspace} \\
    \ucode{000d} & \uname{carriage return} \\
\end{floattable}

\pnum
A \defn{code unit} is an integer value
of character type\iref{basic.fundamental}.
Characters in a \grammarterm{character-literal}
other than a multicharacter or non-encodable character literal or
in a \grammarterm{string-literal} are encoded as
a sequence of one or more code units, as determined
by the \grammarterm{encoding-prefix}\iref{lex.ccon,lex.string};
this is termed the respective \defnadj{literal}{encoding}.
The \defnadj{ordinary literal}{encoding} is
the encoding applied to an ordinary character or string literal.
The \defnadj{wide literal}{encoding} is the encoding applied
to a wide character or string literal.

\pnum
A literal encoding or a locale-specific encoding of one of
the execution character sets\iref{character.seq}
encodes each element of the basic literal character set as
a single code unit with non-negative value,
distinct from the code unit for any other such element.
\begin{note}
    A character not in the basic literal character set
    can be encoded with more than one code unit;
    the value of such a code unit can be the same as
    that of a code unit for an element of the basic literal character set.
\end{note}
\indextext{character!null}%
\indextext{wide-character!null}%
The \unicode{0000}{null} character is encoded as the value \tcode{0}.
No other element of the translation character set
is encoded with a code unit of value \tcode{0}.
The code unit value of each decimal digit character after the digit \tcode{0} (\ucode{0030})
shall be one greater than the value of the previous.
The ordinary and wide literal encodings are otherwise
\impldef{ordinary and wide literal encodings}.
\indextext{UTF-8}%
\indextext{UTF-16}%
\indextext{UTF-32}%
For a UTF-8, UTF-16, or UTF-32 literal,
the \replaceucs{} scalar value
corresponding to each character of the translation character set
is encoded as specified in \changed{ISO/IEC 10646}{the Unicode Standard} for the respective \replaceucs{} encoding form.
\indextext{character set|)}

\textcolor{noteclr}{[...]}

\rSec1[lex.name]{Identifiers}

\indextext{identifier|(}%
\begin{bnf}
    \nontermdef{identifier}\br
    identifier-start\br
    identifier identifier-continue
\end{bnf}

\begin{bnf}
    \nontermdef{identifier-start}\br
    nondigit\br
    \textnormal{an element of the translation character set \changed{of class XID_Start}{with the Unicode property XID_Start}}
\end{bnf}

\begin{bnf}
    \nontermdef{identifier-continue}\br
    digit\br
    nondigit\br
    \textnormal{an element of the translation character set \changed{of class XID_Continue}{with the Unicode property XID_Continue}}
\end{bnf}

\begin{bnf}
    \nontermdef{nondigit} \textnormal{one of}\br
    \terminal{a b c d e f g h i j k l m}\br
    \terminal{n o p q r s t u v w x y z}\br
    \terminal{A B C D E F G H I J K L M}\br
    \terminal{N O P Q R S T U V W X Y Z _}
\end{bnf}

\begin{bnf}
    \nontermdef{digit} \textnormal{one of}\br
    \terminal{0 1 2 3 4 5 6 7 8 9}
\end{bnf}

\pnum
\indextext{name!length of}%
\indextext{name}%
\added{[Note:}The character classes XID_Start and XID_Continue
are Derived Core Properties as described by UAX \#44 \added{of the Unicode Standard}.\added{--end note]}
The program is ill-formed
if an \grammarterm{identifier} does not conform to
Normalization Form C as specified in \changed{ISO/IEC 10646}{the Unicode Standard}.
\begin{note}
    Identifiers are case-sensitive.
\end{note}
\begin{note}
    In translation phase 4,
    \grammarterm{identifier} also includes
    those \grammarterm{preprocessing-token}s\iref{lex.pptoken}
    differentiated as keywords\iref{lex.key}
    in the later translation phase 7\iref{lex.token}.
\end{note}

\textcolor{noteclr}{[...]}

\rSec2[lex.string]{String literals}

\textcolor{noteclr}{[...]}


\pnum
Evaluating a \grammarterm{string-literal} results in a string literal object
with static storage duration\iref{basic.stc}.
\indextext{string!distinct}%
Whether all \grammarterm{string-literal}s are distinct (that is, are stored in
nonoverlapping objects) and whether successive evaluations of a
\grammarterm{string-literal} yield the same or a different object is
unspecified.
\begin{note}
    \indextext{literal!string!undefined change to}%
    The effect of attempting to modify a string literal object is undefined.
\end{note}

\pnum
\indextext{\idxcode{0}|seealso{zero, null}}%
\indextext{\idxcode{0}!string terminator}%
\indextext{\idxcode{0}!null character|see {character, null}}%
String literal objects are initialized with
the sequence of code unit values
corresponding to the \grammarterm{string-literal}'s sequence of
\grammarterm{s-char}s (originally from non-raw string literals) and
\grammarterm{r-char}s (originally from raw string literals),
plus a terminating \unicode{0000}{null} character,
in order as follows:
\begin{itemize}
    \item
    The sequence of characters denoted by each contiguous sequence of
    \grammarterm{basic-s-char}s,
    \grammarterm{r-char}s,
    \grammarterm{simple-escape-sequence}s\iref{lex.ccon}, and
    \grammarterm{universal-character-name}s\iref{lex.charset}
    is encoded to a code unit sequence
    using the \grammarterm{string-literal}'s associated character encoding.
    If a character lacks representation in the associated character encoding,
    then the \grammarterm{string-literal} is conditionally-supported and
    an
    \impldef{code unit sequence for non-representable \grammarterm{string-literal}}
    code unit sequence is encoded.
    \begin{note}
        No character lacks representation in any \removed{of the} \replaceucs{} encoding form\removed{s}.
    \end{note}
    When encoding a stateful character encoding,
    implementations should encode the first such sequence
    beginning with the initial encoding state and
    encode subsequent sequences
    beginning with the final encoding state of the prior sequence.
    \begin{note}
        The encoded code unit sequence can differ from
        the sequence of code units that would be obtained by
        encoding each character independently.
    \end{note}
    \item
\end{itemize}


\rSec1[cpp.predefined]{Predefined macro names}
\begin{itemize}
\item
\indextext{__stdc_version__@\mname{STDC_VERSION}}%
\mname{STDC_VERSION}\\
Whether \mname{STDC_VERSION} is predefined and if so, what its value is,
are \impldef{definition and meaning of \mname{STDC_VERSION}}.

\item
\indextext{__stdc_iso_10646__@\mname{STDC_ISO_10646}}%
\mname{STDC_ISO_10646}\\
An integer literal of the form \tcode{yyyymmL} (for example,
\tcode{199712L}).
\removed{If this symbol is defined, then every character in the Unicode required set, when
stored in an object of type \keyword{wchar_t}, has the same value as the code point
of that character. The \defn{Unicode required set} consists of all the characters
that are defined by ISO/IEC 10646, along with all amendments and technical corrigenda
as of the specified year and month}\added{Whether \mname{STDC_ISO_10646}  is predefined and if so, what its value is, are implementation-defined}.
\end{itemize}

\ednote{
-----
Library wording below
-----
}


\rSec3[format.string.escaped]{Formatting escaped characters and strings}

\pnum
\indextext{string!formatted as escaped}%
\indextext{character!formatted as escaped}%
A character or string can be formatted as \defn{escaped}
to make it more suitable for debugging or for logging.

\pnum
The escaped string \placeholder{E} representation of a string \placeholder{S}
is constructed by encoding a sequence of characters as follows.
The associated character encoding \placeholder{CE}
for \tcode{charT}~(\tref{lex.string.literal})
is used to both interpret \placeholder{S} and construct \placeholder{E}.

\begin{itemize}
    \item
    \unicode{0022}{quotation mark} (\tcode{"}) is appended to \placeholder{E}.

    \item
    For each code unit sequence \placeholder{X} in \placeholder{S} that either
    encodes a single character,
    is a shift sequence, or
    is a sequence of ill-formed code units,
    processing is in order as follows:

    \begin{itemize}
        \item
        If \placeholder{X} encodes a single character \placeholder{C}, then:

        \begin{itemize}
            \item
            If \placeholder{C} is one of the characters in \tref{format.escape.sequences},
            then the two characters shown as the corresponding escape sequence
            are appended to \placeholder{E}.

            \item
            Otherwise, if \placeholder{C} is not \unicode{0020}{space} and

            \begin{itemize}
                \item
                \placeholder{CE} is \changed{a Unicode encoding}{UTF-8, UTF-16, or UTF-32} and
                \placeholder{C} corresponds to either
                a \changed{UCS scalar value whose Unicode property}{Unicode scalar value whose Unicode property} \tcode{General_Category}
                has a value in the groups \tcode{Separator} (\tcode{Z}) or \tcode{Other} (\tcode{C}) or to
                a \changed{UCS scalar value whose Unicode property}{Unicode scalar value with the Unicode property} \tcode{Grapheme_Extend=Yes},
                as described by \removed{table 12 of} UAX \#44 \added{of the Unicode Standard}.


 \ednote{
 Merging with \paper{P2713R1} should produce the following wording:
 \item CE is UTF-8, UTF-16, or UTF-32 and C corresponds to a Unicode scalar value whose Unicode property \tcode{General_Category} has a value in the groups \tcode{Separator} (\tcode{Z}) or \tcode{Other} (\tcode{C}), as described by UAX \#44 of the Unicode Standard, or
\item CE is UTF-8, UTF-16, or UTF-32 and C corresponds to a Unicode scalar value with the Unicode property \tcode{Grapheme_Extend=Yes} as described by UAX \#44 of the Unicode Standard and C is not immediately preceded in S by a character P appended to E without translation to an escape sequence, or \\
}


                \item
                \placeholder{CE} is \changed{not a Unicode encoding}{neither UTF-8, UTF-16, nor UTF-32} and
                \placeholder{C} is one of an implementation-defined set
                of separator or non-printable characters
            \end{itemize}


            then the sequence \tcode{\textbackslash u\{\placeholder{hex-digit-sequence}\}}
            is appended to \placeholder{E},
            where \tcode{\placeholder{hex-digit-sequence}}
            is the shortest hexadecimal representation
            of \placeholder{C} using lower-case hexadecimal digits.

            \item
            Otherwise, \placeholder{C} is appended to \placeholder{E}.
        \end{itemize}

        \item
        Otherwise, if \placeholder{X} is a shift sequence,
        the effect on \placeholder{E} and further decoding of \placeholder{S}
        is unspecified.

        \recommended
        A shift sequence should be represented in \placeholder{E}
        such that the original code unit sequence of \placeholder{S}
        can be reconstructed.

        \item
        Otherwise (\placeholder{X} is a sequence of ill-formed code units),
        each code unit \placeholder{U} is appended to \placeholder{E} in order
        as the sequence \tcode{\textbackslash x\{\placeholder{hex-digit-sequence}\}},
        where \tcode{\placeholder{hex-digit-sequence}}
        is the shortest hexadecimal representation of \placeholder{U}
        using lower-case hexadecimal digits.
    \end{itemize}

    \item
    Finally, \unicode{0022}{quotation mark} (\tcode{"})
    is appended to \placeholder{E}.
\end{itemize}

\textcolor{noteclr}{[...]}

\rSec3[format.string.std]{Standard format specifiers}

\textcolor{noteclr}{[...]}

\pnum
For the purposes of width computation,
a string is assumed to be in
a locale-independent,
\impldef{encoding assumption for \tcode{format} width computation} encoding.
Implementations should use \changed{a Unicode encoding}{either UTF-8, UTF-16, or UTF-32},
on platforms capable of displaying Unicode text in a terminal.

\pnum
For a string in \changed{a Unicode encoding}{UTF-8, UTF-16, or UTF-32},
implementations should estimate the width of a string
as the sum of estimated widths of
the first code points in its extended grapheme clusters.
The extended grapheme clusters of a string are defined by UAX 29.
The estimated width of the following code points is 2:
\begin{itemize}
	\item \ucode{1100} -- \ucode{115f}
	\item \ucode{2329} -- \ucode{232a}
	\item \ucode{2e80} -- \ucode{303e}
	\item \ucode{3040} -- \ucode{a4cf}
	\item \ucode{ac00} -- \ucode{d7a3}
	\item \ucode{f900} -- \ucode{faff}
	\item \ucode{fe10} -- \ucode{fe19}
	\item \ucode{fe30} -- \ucode{fe6f}
	\item \ucode{ff00} -- \ucode{ff60}
	\item \ucode{ffe0} -- \ucode{ffe6}
	\item \ucode{1f300} -- \ucode{1f64f}
	\item \ucode{1f900} -- \ucode{1f9ff}
	\item \ucode{20000} -- \ucode{2fffd}
	\item \ucode{30000} -- \ucode{3fffd}
\end{itemize}
The estimated width of other code points is 1.

\pnum
For a string in \changed{a non-Unicode encoding}{neither UTF-8, UTF-16, nor UTF-32}, the width of a string is unspecified.

\pnum
% FIXME: What if it's an arg-id?
The \grammarterm{nonnegative-integer} in
\grammarterm{precision} is a decimal integer defining
the precision or maximum field size.
It can only be used with floating-point and string types.
For floating-point types this field specifies the formatting precision.
For string types, this field provides an upper bound
for the estimated width of the prefix of
the input string that is copied into the output.
For a string in \changed{a Unicode encoding}{UTF-8, UTF-16, or UTF-32},
the formatter copies to the output
the longest prefix of whole extended grapheme clusters
whose estimated width is no greater than the precision.

\textcolor{noteclr}{[...]}

\rSec3[ostream.formatted.print]{Print}{}

\textcolor{noteclr}{[...]}


\indexlibraryglobal{vprint_unicode}%
\indexlibraryglobal{vprint_nonunicode}%
\begin{itemdecl}
    void vprint_unicode(ostream& os, string_view fmt, format_args args);
    void vprint_nonunicode(ostream& os, string_view fmt, format_args args);
\end{itemdecl}

\begin{itemdescr}
    \pnum
    \effects
    Behaves as a formatted output function\iref{ostream.formatted.reqmts}
    of \tcode{os}, except that:
    \begin{itemize}
        \item
        failure to generate output is reported as specified below, and
        \item
        any exception thrown by the call to \tcode{vformat} is propagated
        without regard to the value of \tcode{os.exceptions()} and
        without turning on \tcode{ios_base::badbit} in the error state of \tcode{os}.
    \end{itemize}
    After constructing a \tcode{sentry} object,
    the function initializes an automatic variable via
    \begin{codeblock}
        string out = vformat(os.getloc(), fmt, args);
    \end{codeblock}
    If the function is \tcode{vprint_unicode} and
    \tcode{os} is a stream that refers to a terminal capable of displaying Unicode
    which is determined in an implementation-defined manner,
    writes \tcode{out} to the terminal using the native Unicode API;
    if \tcode{out} contains invalid code units,
    \indextext{undefined}%
    the behavior is undefined and
    implementations are encouraged to diagnose it.
    Otherwise (if \tcode{os} is not such a stream or
    the function is \tcode{vprint_nonunicode}),
    inserts the character sequence
    \range{out.begin()}{out.end()} into \tcode{os}.
    If writing to the terminal or inserting into \tcode{os} fails,
    calls \tcode{os.setstate(ios_base::badbit)}
    (which may throw \tcode{ios_base::failure}).

    \pnum
    \recommended
    For \tcode{vprint_unicode},
    if invoking the native Unicode API requires transcoding,
    implementations should substitute invalid code units
    with \unicode{fffd}{replacement character} per
    the Unicode Standard \removed{Version 14.0 - Core Specification}, Chapter 3.9 \added{U+FFFD Substitution in Conversion}.
\end{itemdescr}

\textcolor{noteclr}{[...]}

\indexlibraryglobal{vprint_unicode}%
\begin{itemdecl}
    void vprint_unicode(FILE* stream, string_view fmt, format_args args);
\end{itemdecl}

\begin{itemdescr}
    \pnum
    \expects
    \tcode{stream} is a valid pointer to an output C stream.

    \pnum
    \effects
    The function initializes an automatic variable via
    \begin{codeblock}
        string out = vformat(fmt, args);
    \end{codeblock}
    If \tcode{stream} refers to a terminal capable of displaying Unicode,
    writes \tcode{out} to the terminal using the native Unicode API;
    if \tcode{out} contains invalid code units,
    \indextext{undefined}%
    the behavior is undefined and
    implementations are encouraged to diagnose it.
    Otherwise writes \tcode{out} to \tcode{stream} unchanged.
    \begin{note}
        On POSIX and Windows, \tcode{stream} referring to a terminal means that,
        respectively,
        \tcode{isatty(fileno(\linebreak{}stream))} and
        \tcode{GetConsoleMode(_get_osfhandle(_fileno(stream)), ...)}
        return nonzero.
    \end{note}
    \begin{note}
        On Windows, the native Unicode API is \tcode{WriteConsoleW}.
    \end{note}

    \pnum
    \throws
    Any exception thrown by the call to \tcode{vformat}\iref{format.err.report}.
    \tcode{system_error} if writing to the terminal or \tcode{stream} fails.
    May throw \tcode{bad_alloc}.

    \pnum
    \recommended
    If invoking the native Unicode API requires transcoding,
    implementations should substitute invalid code units
    with \unicode{fffd}{replacement character} per
    the Unicode Standard \removed{Version 14.0 - Core Specification}, Chapter 3.9 \added{U+FFFD Substitution in Conversion}.
\end{itemdescr}

\rSec1[depr]{Annex D}
\rSec2[depr.locale.stdcvt.req]{Requirements}

\pnum
For each of the three code conversion facets \tcode{codecvt_utf8},
\tcode{codecvt_utf16}, and \tcode{codecvt_utf8_utf16}:
\begin{itemize}
    \item
    \tcode{Elem} is the wide-character type, such as
    \keyword{wchar_t}, \keyword{char16_t}, or \keyword{char32_t}.
    \item
    \tcode{Maxcode} is the largest wide-character code that the facet
    will read or write without reporting a conversion error.
    \item
    If \tcode{(Mode \& consume_header)}, the facet shall consume an
    initial header sequence, if present, when reading a multibyte sequence
    to determine the endianness of the subsequent multibyte sequence to be read.
    \item
    If \tcode{(Mode \& generate_header)}, the facet shall generate an
    initial header sequence when writing a multibyte sequence to advertise
    the endianness of the subsequent multibyte sequence to be written.
    \item
    If \tcode{(Mode \& little_endian)}, the facet shall generate a
    multibyte sequence in little-endian order,
    as opposed to the default big-endian order.
    \begin{addedblock}
    \item UCS-2 is the same encoding as UTF-16, except that it encodes scalar values in the range U+0000-U+FFFF (Basic Multilingual Plane) only.
    \end{addedblock}

    \ednote{Surrogates cannot appear in a valid utf-8 sequence (nor in a valid utf-16 sequence)}
\end{itemize}

\pnum
\indextext{UTF-8}%
\indextext{UCS-2}%
\indextext{UTF-32}%
For the facet \tcode{codecvt_utf8}\indexlibraryglobal{codecvt_utf8}:
\begin{itemize}
    \item
    The facet shall convert between UTF-8 multibyte sequences
    and UCS-2 or UTF-32 (depending on the size of \tcode{Elem})
    \removed{within the program}.
    \item
    Endianness shall not affect how multibyte sequences are read or written.
    \item
    The multibyte sequences may be written as either a text or a binary file.
\end{itemize}

\pnum
\indextext{UTF-16}%
\indextext{UCS-2}%
\indextext{UTF-32}%
For the facet \tcode{codecvt_utf16}\indexlibraryglobal{codecvt_utf16}:
\begin{itemize}
    \item
    The facet shall convert between UTF-16 multibyte sequences
    and UCS-2 or UTF-32 (depending on the size of \tcode{Elem})
    \removed{within the program}.
    \item
    Multibyte sequences shall be read or written
    according to the \tcode{Mode} flag, as set out above.
    \item
    The multibyte sequences may be written only as a binary file.
    Attempting to write to a text file produces undefined behavior.
\end{itemize}

\pnum
\indextext{UTF-8}%
\indextext{UTF-16}%
For the facet \tcode{codecvt_utf8_utf16}\indexlibraryglobal{codecvt_utf8_utf16}:
\begin{itemize}
    \item
    The facet shall convert between UTF-8 multibyte sequences
    and UTF-16 (one or two 16-bit codes) within the program.
    \item
    Endianness shall not affect how multibyte sequences are read or written.
    \item
    The multibyte sequences may be written as either a text or a binary file.
\end{itemize}

\pnum
\removed{The encoding forms UTF-8, UTF-16, and UTF-32 are specified in ISO/IEC 10646.
The encoding form UCS-2 is specified in ISO/IEC 10646:2003.}


\subsection{Bibliography}

\begin{itemize}
    \renewcommand{\labelitemi}{---}
    % ISO documents in numerical order.
    \item
    ISO 4217:2015,
    \doccite{Codes for the representation of currencies}
    \item
    ISO/IEC 10967-1:2012,
    \doccite{Information technology --- Language independent arithmetic ---
        Part 1: Integer and floating point arithmetic}
    \item
    ISO/IEC TS 18661-3:2015,
    \doccite{Information Technology ---
        Programming languages, their environments, and system software interfaces ---
        Floating-point extensions for C --- Part 3: Interchange and extended types}
    % Other international standards.
    \item
    %%% Format for the following entry is based on that specified at
    %%% http://www.iec.ch/standardsdev/resources/draftingpublications/directives/principles/referencing.htm
    \begin{removedblock}
    The Unicode Consortium. Unicode Standard Annex, UAX \#29,\\
    \doccite{Unicode Text Segmentation} [online].\\
    Edited by Mark Davis. Revision 35; issued for Unicode 12.0.0. 2019-02-15 [viewed 2020-02-23].\\
    Available from: \tcode{http://www.unicode.org/reports/tr29/tr29-35.html}
    \end{removedblock}\\ \begin{addedblock}
    %Edited by Christopher Chapman, an integral part of The Unicode Standard.\\
    %Version Unicode 15.0.0. 2022-08-26 [viewed 2022-12-02].\\
    %Available from: \tcode{https://www.unicode.org/reports/tr29/tr29-41.html}
    \end{addedblock}
    \ednote{Referencing Unicode encompasses the annexes and the motivation explains that we can and should reference the version of the annex that correspond to The Unicode Standard}\\
    \item
    \removed{The Unicode Consortium. Unicode Standard Annex, UAX \#31,\\
    \doccite{Unicode Identifier and Pattern Syntax} [online].}
    %\begin{removedblock}
    %Edited by Mark Davis. Revision 33; issued for Unicode 13.0.0.\\
    %2020-02-13 [viewed 2021-06-08].\\
    %Available from: \tcode{https://www.unicode.org/reports/tr31/tr31-33.html}
    %\end{removedblock}\\ \begin{addedblock}
    %Edited by Mark Davis and Robin Leroy, an integral part of The Unicode Standard.\\
    %Version Unicode 15.0.0. 2022-08-31 [viewed 2022-12-02].\\
    %Available from: \tcode{https://www.unicode.org/reports/tr31/tr31-37.html}\\
    %\end{addedblock}
    \begin{removedblock}
    \item
    The Unicode Standard Version 14.0,
    \doccite{Core Specification}.
    Unicode Consortium, ISBN 978-1-936213-29-0, copyright \copyright 2021 Unicode, Inc.\\
    Available from: \tcode{https://www.unicode.org/versions/Unicode14.0.0/UnicodeStandard-14.0.pdf}
    \end{removedblock}
    \item
    IANA Time Zone Database.
    Available from: \url{https://www.iana.org/time-zones}
    % Literature references.
    \item
    Bjarne Stroustrup,
    \doccite{The \Cpp{} Programming Language, second edition}, Chapter R.
    Addison-Wesley Publishing Company, ISBN 0-201-53992-6, copyright \copyright 1991 AT\&T
    \item
    Brian W. Kernighan and Dennis M. Ritchie,
    \doccite{The C Programming Language}, Appendix A.
    Prentice-Hall, 1978, ISBN 0-13-110163-3, copyright \copyright 1978 AT\&T
    \item
    P.J. Plauger,
    \doccite{The Draft Standard \Cpp{} Library}.
    Prentice-Hall, ISBN 0-13-117003-1, copyright \copyright 1995 P.J. Plauger
\end{itemize}

The arithmetic specification described in ISO/IEC 10967-1:2012 is
called \defn{LIA-1} in this document.



\section{References}

\renewcommand{\section}[2]{}%
\bibliographystyle{plain}
\bibliography{wg21, extra}

\begin{thebibliography}{9}

\bibitem[N4892]{N4892}
Thomas Köppe
\emph{Working Draft, Standard for Programming Language C++}\newline
\url{https://wg21.link/N4892}


\end{thebibliography}

\end{document}
