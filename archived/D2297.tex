% !TeX program = luatex
% !TEX encoding = UTF-8


\RequirePackage{luatex85}

\documentclass{wg21}

% \usepackage[twitter]{emoji}

\title{Wording improvements for encodings and character sets}
\docnumber{P2297R1}
\audience{SG-16}
\author{Corentin Jabot}{corentin.jabot@gmail.com}

\begin{document}
\maketitle

\section{Revisions}

\subsection{R1}
\begin{itemize}
\item Remove an incorrect note in the section about UDL
\end{itemize}


\section{Abstract}


\section{Summary of behavior changes}


\subsection{Alert \& backspace} % \emoji{1F514}

The wording mandated that the executions encoding be able to encode "alert, backspace, and carriage return". This requirement is not used
in the core wording (Tweaks of [5.13.3.3.1] may be needed), nor in the library wording, and therefore does not seem useful, so it was not added in the new wording.
This will not have any impact on existing implementations.

\subsection{Unicode in raw string delimiters}

Falls out of the wording change. should we?

\section{New terminology}

\subsection{Basic character set}

Formerly \defn{basic source character set}. Represent the set of abstract (non-coded) characters in the graphic subset of the ASCII character set.
The term "source" has been dropped because the source code encoding is not observable nor relevant past phase 1.

The \defn{basic character set} is used:
\begin{itemize}
    \item As a subset of other encodings
    \item To restric accepted characters in grammar elements
    \item To restrict values in library
\end{itemize}

\subsection{\defn{literal character set}, \defn{literal character encoding}, \defn{wide literal character set}, \defn{wide literal character encoding}}

Encodings and associated character sets of narrow and wide character and string literals. implementation-defined, and locale agnostic.



\subsection{\defn{execution character set}, \defn{execution character encoding}, \defn{wide execution character set}, \defn{wide execution character encoding}}

Encodings and associated character sets of the encoding used by the library. isomorphic or supersets of their literal counterparts.
Separating literal encodings from libraries encoding allows:

\begin{itemize}
    \item To make a distinction that exists in practice and which was not previously admitted by the standard previous.
    \item To keep the core wording locale agnostic.
\end{itemize}

The definition for these encodings has been moved to [library.intro]

\section{Questions and bikesheding}

\begin{itemize}
    \item Do the terms of art code unit, code point, abstract character need to be defined?
    \item Are we happy with \tcode{execution} for library encodings?
    \item Do we prefer \defn{literal character encoding} or \defn{literal ordinary character encoding} ?
\end{itemize}

\subsection{Codepoints vs scalar values}

Only code points that are scalar can appear in a valid program.
This is implied by the sentence "mapped to the Unicode character set" (as opposition to codespace) in phase 1.
As such it doesn't matter which term we use, but we should be consistent.

\section{Differences with P2314}

This paper and P2314 are two separate effort which accidentally occurred at the same time.
They share many similarities as they are born of the same SG-16 efforts to improve lexing.

There are some differences between the two approaches that we feel are important enough that
we think it is important for SG-16 to consider and compare bother papers!

Note that unlike \paper{P2194R0} this paper does not try to fix phase 5 \& 6!

\subsection{The character set of C++ source code is Unicode}

We explored this point in great detail in \paper{P2194R0}.
In particular:
\begin{itemize}
    \item If the source character set is Unicode, codepoints can be mapped 1 to 1 (see also P2290R0).
    Whether a codepoint is assigned or not is irrelevant for the compiler as long as the properties of the codepoint are not observed.
    In particular identifier validation which looks at XID properties implies codepoints are assigned.

    \item If the source character set is not Unicode then the existence of a mapping to a codepoint implies the mapped-to codepoint is assigned (bare an evil implementation).
\end{itemize}

Therefore, unlike P2314, this paper does not introduce a different character set to describe the Unicode character set, referring instead
to the Unicode character set directly, which we think is important as it makes things simpler.
The fact that we need to look at Unicode properties of codepoints to determine the validity of identifiers further implies that we understand the Unicode character set as being a Coded Character set.


\subsection{Encodings assumed at runtime should be described!}

In the current wording, it is implied that there exists an execution encoding that is both the encoding of literals and
the encoding assumed by C functions affected by locale.
This is a side effect of the current wording not distinguishing between the compile and runtime environment.
It is presumably also why the encodings of literals are "locale-specific".

It seems evident that \emph{literal encoding} is a better term to describe the encodings of literals.
But that renaming is less innocuous than it might first appear:
\textbf{It admits the encoding of literals may not be the same as the encodings assumed by local specific libraries functions!}

This for sure is a good thing as it matches reality but left us with a question: What is then the encoding assumed by the local specific functions and do they have a relation with the encodings of literals?

Ultimately, I think the question boils down to whether it is sensible for the following assertion to ever fail:

\begin{colorblock}
assert(std::isalpha('a'));
\end{colorblock}

A wider question is maybe "Is a program which exposes mojibake well-defined?"

The functions in \tcode{<cctype>}, \tcode{<cstdlib>} expect ""characters"" in the local specific encoding.
Digging into the C++ standard, it seems to be the intent that \tcode{fputc}
and all functions that use it (including \tcode{printf}) treat in characters.

The effect of \tcode{setlocale} on the execution encoding is never quite described by neither standard.

Assuming \tcode{putc('a')} should print the character \tcode{'a'}, and assuming \tcode{'a'} is a lower case alphabetic character,
or assuming simply that \tcode{'a'} is a character than there must exist a relation between the encoding used to encode \tcode{'a'}
and the encoding used to later interpret it.

Requiring the same encoding, however, would be over constraining.
For example, if the literal encoding is ASCII the runtime encoding can be UTF-8 as UTF8 is a superset of ASCII.
The precise requirement to avoid mojibake is that all code units sequences present in literals are valid code units sequences in
the execution encoding associated with the corresponding character type.

A less precise, but simpler and therefore more useful requirement is that the execution character encoding is a "superset" of the literal character encoding, to the extent encodings can be supersets of one another.

Without this precondition, no function can be expected to behave reasonably in the extreme case where the literal encoding is EBCDIC-derived and the execution encoding is ASCII derived (or vice-versa)
Starting with \tcode{setlocale}. What does a call to \tcode{setlocale(LC_ALL, "C")} possibly mean if \tcode{"C"} is not the "C" locale?
Confused yet?
In fact, any function taking a character or string as a parameter would behave non-sensically if initialized with a literal.
Without a relationship between the literal and execution encodings, literals can never be used in any function as the literal values would be in a different domain!

This observation is reinforced and aggravated by the fact that it is generally not possible to distinguish literals from other objects at runtime!

\section{Issues Fixed}

\begin{itemize}
    \item Phase 1 modification resolves \paper{CWG1403}, \paper{CWG578}, \paper{CWG1335}
\end{itemize}

\section{Future works}

\begin{itemize}
\item Align \tcode{wchar_t} with existing practices.
\item Review more thoroughly usages of the terms \emph{character}.
\end{itemize}

\section{Wording}



\rSec1[intro.defs]{Terms and definitions}

[...]

\textbf{multibyte character}\\
\changed{
sequence of one or more bytes representing a member of the extended
character set of either the source or the execution environment}
{Sequence of one or more code units representing a member of one of the literal or execution character sets.}

\begin{removedblock}
\begin{note}
    The extended character set is a superset of the basic character
    set\iref{lex.charset}.
\end{note}
\end{removedblock}

\begin{quoteblock}
Rationale: The notion of extended characters is removed, as, while the notion of basic character is useful,
there are only a few places where basic characters should be handled differently from other characters (character meaning code point here).

TODO: Should that definition apply to the UTF-8 (char8_t) encoding?
\end{quoteblock}



[...]

\rSec1[basic.memobj]{Memory and objects}

\rSec2[intro.memory]{Memory model}

\pnum
\indextext{memory model|(}%
The fundamental storage unit in the \Cpp{} memory model is the
\defn{byte}.
A byte is at least large enough to \changed{contain any member of the basic
\indextext{character set!basic execution}%
execution character set\iref{lex.charset}}{represent any code unit of the literal and execution character encodings}
and the eight-bit code units of the Unicode
UTF-8 encoding form
and is composed of a contiguous sequence of
bits, the number of which is \impldef{bits in a byte}.

\rSec2[basic.fundamental]{Fundamental types}

[...]


\begin{quoteblock}
Rationale: The wording was not clear that it meant the basic source (rather than execution) character set.
"implementation's basic character set" is also a fuzzy term. Is the basic source character set a coded character set?
\end{quoteblock}


\indextext{type!\idxcode{char}}%
\indextext{type!character}%
\indextext{type!ordinary character}%
\indextext{type!narrow character}%
\indextext{\idxcode{char}!implementation-defined sign of}%
\indextext{type!\idxcode{signed char}}%
\indextext{character!\idxcode{signed}}%
\indextext{type!\idxcode{unsigned char}}%
Type \tcode{char} is a distinct type
that has an \impldef{underlying type of \tcode{char}} choice of
``\tcode{signed char}'' or ``\tcode{unsigned char}'' as its underlying type.
The values of type \tcode{char} can represent \changed{distinct codes
for all members of the implementation's basic character set}{all code units of the literal and execution character encodings}.
The three types \tcode{char}, \tcode{signed char}, and \tcode{unsigned char}
are collectively called
\defnx{ordinary character types}{type!ordinary character}.
The ordinary character types and \tcode{char8_t}
are collectively called \defnx{narrow character types}{narrow character type}.
For narrow character types,
each possible bit pattern of the object representation represents
a distinct value.
\begin{note}
 This requirement does not hold for other types.
\end{note}
\begin{note}
 A bit-field of narrow character type whose width is larger than the width of that type has padding bits; see \iref{basic.types}.
\end{note}

\begin{quoteblock}
Rationale: The wording was implying that UTF-16 could not be used with wchar_t (as it is a multibyte encoding and therefore can not represent all values in a single wchar_t)
\end{quoteblock}


\indextext{\idxcode{wchar_t}|see{type, \tcode{wchar_t}}}%
\indextext{type!\idxcode{wchar_t}}%
\indextext{type!underlying!\idxcode{wchar_t}}%
Type \tcode{wchar_t} is a distinct type that has
an \impldef{underlying type of \tcode{wchar_t}}
signed or unsigned integer type as its underlying type.
The values of type \tcode{wchar_t} can represent
\changed{distinct codes for all members of the largest extended character set
specified among the supported locales\iref{locale}}{all code units of the wide literal and wide execution character encodings}.


\rSec1[lex.phases]{Phases of translation}%

\begin{enumerate}
\item
\indextext{character!source file}%
\indextext{character set!basic source}
\changed{Physical source file characters}{Each abstract character in the source file is} mapped, in an implementation-defined mannner,
to \changed{the basic source character set}{a Unicode scalar value} (introducing new-line characters for end-of-line
indicators) \removed{if necessary}.
The set of \removed{physical} source file character\changed{s}{ sets} accepted is \impldef{physical source file characters}.

\removed{Any source file character not in the basic source character
set \iref{lex.charset} is replaced by the
\indextext{universal character name}\\ \grammarterm{universal-character-name} that
designates that character. An implementation may use any internal
encoding, so long as an actual extended character encountered in the
source file, and the same extended character expressed in the source
file as a\\ \grammarterm{universal-character-name} (e.g., using the \tcode{\textbackslash
    uXXXX} notation), are handled equivalently
except where this replacement is reverted\iref{lex.pptoken} in a raw string literal.}

\begin{quoteblock}
Do we need to mention anything about the encoding of the Unicode character set being implementation-defined given it is not observable?
\end{quoteblock}

\item
\indextext{line splicing}%
Each instance of a backslash character (\textbackslash)
immediately followed by a new-line character is deleted, splicing
\removed{physical} source input lines to form logical source lines. Only the last
backslash on any \removed{physical} source input line shall be eligible for being part
of such a splice.
Except for splices reverted in a raw string literal, if a splice results in
a character sequence that matches the
syntax of a \grammarterm{universal-character-name}, the behavior is
undefined. A source file that is not empty and that does not end in a new-line
character, or that ends in a new-line character immediately preceded by a
backslash character before any such splicing takes place,
shall be processed as if an additional new-line character were appended
to the file.

\item The source file is decomposed into preprocessing
tokens\iref{lex.pptoken} and sequences of whitespace characters
(including comments). A source file shall not end in a partial
preprocessing token or in a partial comment.
Each comment is replaced by one space character. New-line characters are
retained. Whether each nonempty sequence of whitespace characters other
than new-line is retained or replaced by one space character is
unspecified.
\begin{addedblock}
Each universal-character-name outside of a character or string literal is replaced by the Unicode codepoint it represents.
\end{addedblock}

The process of dividing a source file's
characters into preprocessing tokens is context-dependent.
\begin{example}
    See the handling of \tcode{<} within a \tcode{\#include} preprocessing
    directive.
\end{example}

\end{enumerate}

\rSec1[lex.pptoken]{Preprocessing tokens}

[...]

\pnum
If the input stream has been parsed into preprocessing tokens up to a
given character:
\begin{itemize}
    \item
    \indextext{literal!string!raw}%
    If the next character begins a sequence of characters that could be the prefix
    and initial double quote of a raw string literal, such as \tcode{R"}, the next preprocessing
    token shall be a raw string literal. Between the initial and final
    double quote characters of the raw string, \changed{any transformations performed in phases
    1 and 2 (\grammarterm{universal-character-name}{s} and line splicing) are}{line splicing performed in phase 2 is} reverted; this reversion
    shall apply before any \grammarterm{d-char}, \grammarterm{r-char}, or delimiting
    parenthesis is identified. The raw string literal is defined as the shortest sequence
    of characters that matches the raw-string pattern
\begin{ncbnf}\opt{encoding-prefix} \terminal{R} raw-string\end{ncbnf}
    
    \item Otherwise, if the next three characters are \tcode{<::} and the subsequent character
    is neither \tcode{:} nor \tcode{>}, the \tcode{<} is treated as a preprocessing token by
    itself and not as the first character of the alternative token \tcode{<:}.
\end{itemize}
\rSec1[lex.charset]{Character sets}



\pnum
\indextext{character set|(}%
The \defnx{basic \removed{source} character set}{character set!basic source} consists of 96 characters: the space character,
the control characters representing horizontal tab, vertical tab, form feed, and
new-line, plus the following 91 graphical characters:

\begin{quoteblock}
Edit the footnote associated with the above paragraph as follows:

Rationale:
\begin{itemize}
\item Abstract character is a more precise terminology to talk about the same characters in different character sets or not in any character set.
\item The second sentence seems incorrect. While the mapping in phase 1 must be documented, neither the source files nor the internal representation should be observable by the program and as such do not need to be documented. The paragraph further seems to imply that the formerly-source basic character set applies to source files
\end{itemize}

\begin{quoteblock}
The glyphs for
the members of the basic \removed{source} character set are intended to
identify \added{abstract} characters from the subset of ISO/IEC 10646 which corresponds to the ASCII
character set. \removed{However, the mapping from source file characters to the source
    character set (described in translation phase 1) is specified as
    \impldef{mapping from physical source file characters to basic source character set},
    and therefore implementations must document how the basic source characters are
    represented in source files.}
\end{quoteblock}
\end{quoteblock}

\begin{codeblock}
    a b c d e f g h i j k l m n o p q r s t u v w x y z
    A B C D E F G H I J K L M N O P Q R S T U V W X Y Z
    0 1 2 3 4 5 6 7 8 9
    _ { } [ ] # ( ) < > % : ; . ? * + - / ^ & | ~ ! = , @\textbackslash@ " '
\end{codeblock}

\begin{quoteblock}
The above paragraph, footnote and table should be replaced by a table of codepoints identified by their Unicode name and values!
\end{quoteblock}

\pnum
The \grammarterm{universal-character-name} construct provides a way to name
other characters.

\begin{bnf}
    \nontermdef{hex-quad}\br
    hexadecimal-digit hexadecimal-digit hexadecimal-digit hexadecimal-digit
\end{bnf}

\begin{bnf}
    \nontermdef{universal-character-name}\br
    \terminal{\textbackslash u} hex-quad\br
    \terminal{\textbackslash U} hex-quad hex-quad
\end{bnf}

A \grammarterm{universal-character-name}
designates the character in ISO/IEC 10646 (if any)
whose code point is the hexadecimal number represented by
the sequence of \grammarterm{hexadecimal-digit}s
in the \grammarterm{universal-character-name}.
The program is ill-formed if that number is not a code point
or if it is a surrogate code point.
\removed{Noncharacter code points and reserved code points
are considered to designate separate characters distinct from
any ISO/IEC 10646 character.}

\begin{quoteblock}
Noncharacters, reserved characters
are still valid codepoints and valid scalar values.

Consequences of that:
\begin{itemize}
\item When they appear in a literal whose associated character set is Unicode, then the codepoint can be preserved in the evaluated string
\item When they appear in a literal whose associated character set is \textbf{not} Unicode, then the codepoint designates a non-encodable character literal
\item In identifier they do not designate a code point with the XID_Start or XID_Continue which makes the program ill-formed.
\end{itemize}

\end{quoteblock}

If a \grammarterm{universal-character-name} outside
the \grammarterm{c-char-sequence}, \grammarterm{s-char-sequence}, or
\grammarterm{r-char-sequence} of
a \grammarterm{character-literal} or \grammarterm{string-literal}
(in either case, including within a \grammarterm{user-defined-literal})
corresponds to a control character or
to a character in the basic
\removed{source} character set, the program is ill-formed.
%\begin{footnote}
    A sequence of characters resembling a \grammarterm{universal-character-name} in an
    \grammarterm{r-char-sequence}\iref{lex.string} does not form a
    \grammarterm{universal-character-name}.
%\end{footnote}
\begin{note}
    ISO/IEC 10646 code points are integers in the range $[0, \mathrm{10FFFF}]$ (hexadecimal).
    A surrogate code point is a value in the range $[\mathrm{D800}, \mathrm{DFFF}]$ (hexadecimal).
    A control character is a character whose code point is
    in either of the ranges $[0, \mathrm{1F}]$ or $[\mathrm{7F}, \mathrm{9F}]$ (hexadecimal).
\end{note}

\pnum
\begin{removedblock}
The \defnx{basic execution character set}{character set!basic execution} and the
\defnx{basic execution wide-character set}{wide-character set!basic execution}
shall each contain all the members of the
basic source character set, plus control characters representing alert,
backspace, and carriage return, plus a \defnx{null character}{character!null}
(respectively, \defnx{null wide character}{wide-character!null}), whose value is 0.
For each basic execution character set, the values of the
members shall be non-negative and distinct from one another. In both the
source and execution basic character sets, the value of each character
after \tcode{0} in the above list of decimal digits shall be one greater
than the value of the previous. The \defnx{execution character set}{character set!execution}
and the \defnx{execution wide-character set}{wide-character set!execution} are
\impldef{execution character set and execution wide-character set}
supersets of the
basic execution character set and the basic execution wide-character
set, respectively. The values of the members of the execution character sets
and the sets of additional members
are locale-specific.%
\indextext{character set|)}
\end{removedblock}

\begin{addedblock}

The \defn{literal character set} and \defn{wide literal character set} are implementation-defined characters set which shall contain all members
of the \defn{basic character set} plus an implementation-defined set of additional members.

The \defn{literal character encoding} and \defn{wide literal character encoding} are the implementation-defined character encodings of the \defn{literal character set} and \defn{wide literal character set} respectively such that:

\begin{itemize}
\item Each code unit is represented by a single \tcode{char} or \tcode{wchar_t} respectively
\item Each member of the \defn{basic character set} is uniquely represented by a single code unit whose value is positive
\item The NULL character (U+0000) is represented as a single code unit whose value, as read via a glvalue of type \tcode{char}, is 0

\begin{quoteblock}
Do we still need the gvalue bit above? My understanding is that we are trying to say \tcode{char(L'\textbackslash 0') == 0}.
\end{quoteblock}

\item The code units representing each digit in the basic character set (U+0030 to U+0039) have consecutive values
\end{itemize}

Each member of the \defn{wide literal character set} shall be represented by a single code units in the \defn{wide literal character encoding}.
\begin{quoteblock}
This is... an interesting restriction that does not match existing practices but maintains the status-quo brokenness of \tcode{wchar_t}!
\end{quoteblock}
Members of the \defn{wide literal character set} are represented in one or more code units in the \defn{wide literal character encoding}.

\end{addedblock}

[...]



\rSec1[lex.header]{Header names}

\indextext{header!name|(}%
\begin{bnf}
    \nontermdef{header-name}\br
    \terminal{<} h-char-sequence \terminal{>}\br
    \terminal{"} q-char-sequence \terminal{"}
\end{bnf}

\begin{bnf}
    \nontermdef{h-char-sequence}\br
    h-char\br
    h-char-sequence h-char
\end{bnf}

\begin{bnf}
    \nontermdef{h-char}\br
    \textnormal{any \changed{member of the source basic character set}{Unicode codepoint} except new-line and \terminal{>}}
\end{bnf}

\begin{bnf}
    \nontermdef{q-char-sequence}\br
    q-char\br
    q-char-sequence q-char
\end{bnf}

\begin{bnf}
    \nontermdef{q-char}\br
    \textnormal{any \changed{member of the source basic character set}{Unicode codepoint} except new-line and \terminal{"}}
\end{bnf}


\rSec2[lex.ccon]{Character literals}

\begin{quoteblock}
The grammar below will be further impacted by work to not replace non-basic characters in phase 1
\end{quoteblock}

\begin{bnf}
\nontermdef{basic-c-char}\br
\textnormal{any \changed{member of the source basic character set}{Unicode codepoint} except the single-quote \terminal{'}, backslash \terminal{\textbackslash}, or new-line character}
\end{bnf}


\begin{bnf}

\nontermdef{conditional-escape-sequence-char}\br
\textnormal{any member of the \removed{source} basic character set} that is not an octal-digit\textnormal{, a} simple-escape-sequence-char\textnormal{, or the characters \terminal{u}, \terminal{U}, or \terminal{x}}
\end{bnf}

\begin{quoteblock}
    I think we want to limit to basic characters here
\end{quoteblock}

[...]

\pnum
The kind of a \grammarterm{character-literal},
its type, and its associated character encoding
are determined by
its \grammarterm{encoding-prefix} and its \grammarterm{c-char-sequence}
as defined by \tref{lex.ccon.literal}.
The special cases for
non-encodable character literals and multicharacter literals
take precedence over their respective base kinds.
\begin{note}
    The \changed{associated character encoding for ordinary and wide character literals}{ordinary and wide literal character encodings}
    determines encodability,
    but does not determine the value of
    non-encodable ordinary or wide character literals or
    ordinary or wide multicharacter literals.
    The examples in [lex.ccon.literal]
    for non-encodable ordinary and wide character literals assume that
    the specified character lacks representation in
    the \changed{execution}{literal} character set or \changed{execution}{literal} wide-character set, respectively, or
    that encoding it would require more than one code unit.
\end{note}

\begin{tabular}
    %{Character literals}{lex.ccon.literal}
{lllll}
%\topline
Encoding & Kind & Type & Associated char- & Example \\
prefix & & & acter encoding & \\
%\capsep
none &
\defnx{ordinary character literal}{literal!character!ordinary} &
\keyword{char} &
\changed{encoding of}{literal} &
\tcode{'v'} \\
&
non-encodable ordinary character literal &
\keyword{int} &
\changed{the execution}{encoding} &
\tcode{'\textbackslash U0001F525'} \\
&
ordinary multicharacter literal &
\keyword{int} &
\removed{character set} &
\tcode{'abcd'} \\ \hline
\tcode{L} &
\defnx{wide character literal}{literal!character!wide} &
\keyword{wchar_t} &
\changed{encoding of}{wide literal}&
\tcode{L'w'} \\
&
non-encodable wide character literal &
\keyword{wchar_t} &
\changed{the execution}{encoding} &
\tcode{L'\textbackslash U0001F32A'} \\
&
wide multicharacter literal &
\keyword{wchar_t} &
\removed{wide-character set}&
\tcode{L'abcd'} \\ \hline
\tcode{u8} &
\defnx{UTF-8 character literal}{literal!character!UTF-8} &
\keyword{char8_t} &
UTF-8 &
\tcode{u8'x'} \\ \hline
\tcode{u} &
\defnx{UTF-16 character literal}{literal!character!UTF-16} &
\keyword{char16_t} &
UTF-16 &
\tcode{u'y'} \\ \hline
\tcode{U} &
\defnx{UTF-32 character literal}{literal!character!UTF-32} &
\keyword{char32_t} &
UTF-32 &
\tcode{U'z'} \\
\end{tabular}


\rSec2[lex.string]{String literals}


\begin{quoteblock}
The grammars below will be further impacted by work to not replace non-basic characters in phase 1
\end{quoteblock}

\begin{bnf}
\nontermdef{basic-s-char}\br
\textnormal{any \changed{member of the source basic character set}{Unicode codepoint} except the double-quote \terminal{"}, backslash \terminal{\textbackslash}, or new-line character}
\end{bnf}

\begin{bnf}
    \nontermdef{raw-string}\br
    \terminal{"} \opt{d-char-sequence} \terminal{(} \opt{r-char-sequence} \terminal{)} \opt{d-char-sequence} \terminal{"}
\end{bnf}

\begin{bnf}
    \nontermdef{r-char-sequence}\br
    r-char\br
    r-char-sequence r-char
\end{bnf}

\begin{bnf}
    \nontermdef{r-char}\br
    \textnormal{any \changed{member of the source basic character set}{Unicode codepoint}, except a right parenthesis \terminal{)} followed by}\br
    \bnfindent\textnormal{the initial \grammarterm{d-char-sequence} (which may be empty) followed by a double quote \terminal{"}.}
\end{bnf}

\begin{bnf}
    \nontermdef{d-char-sequence}\br
    d-char\br
    d-char-sequence d-char
\end{bnf}


\begin{bnf}
    \nontermdef{d-char}\br
    \textnormal{any \changed{member of the source basic character set}{Unicode codepoint} :}\br
    \bnfindent\textnormal{space, the left parenthesis \terminal{(}, the right parenthesis \terminal{)}, the backslash \terminal{\textbackslash}, and the control characters}\br
    \bnfindent\textnormal{representing horizontal tab, vertical tab, form feed, and newline.}
\end{bnf}

\begin{quoteblock}
\textbf{Because \tcode{\textbackslash} is not allowed, and threfore universal-character-name, allowing unicode here is a change of behavior.}
\end{quoteblock}


[...]

\begin{tabular}
    % {String literals}{lex.string.literal}
{llp{2.6cm}p{2.3cm}p{4.7cm}}
%\topline
Encoding & Kind & Type & Associated & Examples \\
prefix & & & character encoding & \\
% \capsep
none &
\defnx{ordinary string literal}{literal!string!ordinary} &
array of $n$\newline \tcode{\keyword{const} \keyword{char}} &
\changed{encoding of the execution character set}{literal encoding} &
\tcode{"ordinary string"}\newline
\tcode{R"(ordinary raw string)"} \\
\tcode{L} &
\defnx{wide string literal}{literal!string!wide} &
array of $n$\newline \tcode{\keyword{const} \keyword{wchar_t}} &
\changed{encoding of the execution wide-character set}{wide literal encoding} &
\tcode{L"wide string"}\newline
\tcode{LR"w(wide raw string)w"} \\
\tcode{u8} &
\defnx{UTF-8 string literal}{literal!string!UTF-8} &
array of $n$\newline \tcode{\keyword{const} \keyword{char8_t}} &
UTF-8 &
\tcode{u8"UTF-8 string"}\newline
\tcode{u8R"x(UTF-8 raw string)x"} \\
\tcode{u} &
\defnx{UTF-16 string literal}{literal!string!UTF-16} &
array of $n$\newline \tcode{\keyword{const} \keyword{char16_t}} &
UTF-16 &
\tcode{u"UTF-16 string"}\newline
\tcode{uR"y(UTF-16 raw string)y"} \\
\tcode{U} &
\defnx{UTF-32 string literal}{literal!string!UTF-32} &
array of $n$\newline \tcode{\keyword{const} \keyword{char32_t}} &
UTF-32 &
\tcode{U"UTF-32 string"}\newline
\tcode{UR"z(UTF-32 raw string)z"} \\
\end{tabular}

\pnum
\indextext{literal!string!raw}%
A \grammarterm{string-literal} that has an \tcode{R}
\indextext{prefix!\idxcode{R}}%
in the prefix is a \defn{raw string literal}. The
\grammarterm{d-char-sequence} serves as a delimiter. The terminating
\grammarterm{d-char-sequence} of a \grammarterm{raw-string} is the same sequence of
characters as the initial \grammarterm{d-char-sequence}. A \grammarterm{d-char-sequence}
shall consist of at most 16 characters.

\pnum
\begin{note}
    The characters \tcode{'('} and \tcode{')'} are permitted in a
    \grammarterm{raw-string}. Thus, \tcode{R"delimiter((a|b))delimiter"} is equivalent to
    \tcode{"(a|b)"}.
\end{note}

\pnum
\begin{note}
    A source-file new-line in a raw string literal results in a new-line in the
    \changed{resulting execution}{evaluated} string literal. Assuming no
    whitespace at the beginning of lines in the following example, the assert will succeed:
    \begin{codeblock}
        const char* p = R"(a\
        b
        c)";
        assert(std::strcmp(p, "a\\\nb\nc") == 0);
    \end{codeblock}
\end{note}

[...]

\rSec2[lex.ext]{User-defined literals}

[...]

\pnum
If \placeholder{L} is a \grammarterm{user-defined-integer-literal}, let \placeholder{n} be the literal
without its \grammarterm{ud-suffix}. If \placeholder{S} contains a literal operator with
parameter type \tcode{unsigned long long}, the literal \placeholder{L} is treated as a call of
the form
\begin{codeblock}
    operator "" @\placeholder{X}@(@\placeholder{n}@ULL)
\end{codeblock}
Otherwise, \placeholder{S} shall contain a raw literal operator
or a numeric literal operator template\iref{over.literal} but not both.
If \placeholder{S} contains a raw literal operator,
the literal \placeholder{L} is treated as a call of the form
\begin{codeblock}
    operator "" @\placeholder{X}@("@\placeholder{n}@")
\end{codeblock}
Otherwise (\placeholder{S} contains a numeric literal operator template),
\placeholder{L} is treated as a call of the form
\begin{codeblock}
    operator "" @\placeholder{X}@<'@$c_1$@', '@$c_2$@', ... '@$c_k$@'>()
\end{codeblock}
where \placeholder{n} is the \changed{source character}{codepoint} sequence $c_1c_2...c_k$.
\begin{note}
    The sequence
    $c_1c_2...c_k$ can only contain characters from the basic \removed{source} character set.
\end{note}

\pnum
If \placeholder{L} is a \grammarterm{user-defined-floating-point-literal}, let \placeholder{f} be the
literal without its \grammarterm{ud-suffix}. If \placeholder{S} contains a literal operator
with parameter type \tcode{long double}, the literal \placeholder{L} is treated as a call of
the form
\begin{codeblock}
    operator "" @\placeholder{X}@(@\placeholder{f}@L)
\end{codeblock}
Otherwise, \placeholder{S} shall contain a raw literal operator
or a numeric literal operator template\iref{over.literal} but not both.
If \placeholder{S} contains a raw literal operator,
the \grammarterm{literal} \placeholder{L} is treated as a call of the form
\begin{codeblock}
    operator "" @\placeholder{X}@("@\placeholder{f}@")
\end{codeblock}
Otherwise (\placeholder{S} contains a numeric literal operator template),
\placeholder{L} is treated as a call of the form
\begin{codeblock}
    operator "" @\placeholder{X}@<'@$c_1$@', '@$c_2$@', ... '@$c_k$@'>()
\end{codeblock}
where \placeholder{f} is the  \changed{source character}{codepoint} sequence $c_1c_2...c_k$.
\begin{note}
    The sequence
    $c_1c_2...c_k$ can only contain characters from the basic \removed{source} character set.
\end{note}


\rSec1[library]{Library introduction}

\rSec1[library.c]{ Method of description}

\rSec2[conventions]{Other conventions}

\rSec3[type.descriptions]{Type descriptions}

\rSec3[character.seq]{Character sequences}


\begin{addedblock}
\rSec3[execution encodings]{Execution encodings}

The \defn{execution encoding} is the character encoding of the \defn{execution character set}, such that
all members of the \defn{literal character set} are represented, with the same value in the \defn{execution character set} and any character
in the \defn{literal character encoding} represent the same character when interpreted as being in the \defn{execution encoding}.

The \defn{wide execution encoding} is the character encoding of the \defn{wide execution character set}, such that
all members of the \defn{wide literal character set} are represented, with the same value in the \defn{wide execution character set} and any character
in the \defn{wide literal character encoding} represent the same character when interpreted as being in the \defn{wide execution encoding}.

The \defn{execution encoding} and \defn{wide execution encoding} are implementation-defined and may be be affected by a call to \tcode{setlocale(int, const char*)}, or by a change to a \tcode{locale} object, as described in locales and input.output.

\end{addedblock}

\begin{quoteblock}
The paragraph below only becomes relevant if we have constexpr text transformation, encodings or classification functions.
I don't think that's the case yet.
\end{quoteblock}

\begin{addedblock}
During constant evaluation, the \defn{execution encoding} and \defn{wide execution encoding} are the \defn{literal encoding} and \defn{wide literal encoding} respectively and are not affected by locale.

\end{addedblock}

\rSec3[character.seq.general]{General}
\pnum
The C standard library makes widespread use
\indextext{library!C standard}%
of characters and character sequences that follow a few uniform conventions:

\begin{itemize}
    \item
    A \defn{letter} is any of the 26 lowercase or 26
    \indextext{lowercase}%
    \indextext{uppercase}%
    uppercase letters in the \changed{basic execution}{basic} character set.
    \item
    The
    \defnx{decimal-point character}{character!decimal-point}
    is the
    (single-byte) character used by functions that convert between a (single-byte)
    character sequence and a value of one of the floating-point types.
    It is used
    in the character sequence to denote the beginning of a fractional part.
    It is
    represented in [??] through  [??]
    and \iref{depr} by a period,
    \indextext{period}%
    \tcode{'.'},
    which is
    also its value in the \tcode{"C"}
    locale, but may change during program
    execution by a call to
    \tcode{setlocale(int, const char*)},
    %\begin{footnote}
    %    declared in
    %    \libheaderref{clocale}.
    %    \indexlibraryglobal{setlocale}%
    % \end{footnote}
    or by a change to a
    \tcode{locale}
    object, as described in \iref{locales} and \iref{input.output}.
    \item
    A
    \defn{character sequence}
    is an array object\iref{dcl.array} \tcode{\placeholdernc{A}} that
    can be declared as
    \tcode{\placeholdernc{T\;A}[\placeholder{N}]},
    where \tcode{\placeholder{T}} is any of the types
    \tcode{char},
    \tcode{unsigned char},
    or
    \tcode{signed char}\iref{basic.fundamental}, optionally qualified by any combination of
    \tcode{const}
    or
    \tcode{volatile}.
    The initial elements of the
    array have defined contents up to and including an element determined by some
    predicate.
    A character sequence can be designated by a pointer value
    \tcode{\placeholder{S}} that points to its first element.
\end{itemize}

\rSec3[byte.strings]{Byte strings}

\indextext{string!null-terminated byte|see{\ntbs{}}}%
\pnum
A \defnx{null-terminated byte string}{NTBS@\ntbs{}},
or \ntbs{},
is a character sequence whose highest-addressed element
with defined content has the value zero
(the \defnx{terminating null character}{character!terminating null});
no other element in the sequence has the value zero.

\pnum
The \defnx{length of an \ntbs{}}{NTBS@\ntbs{}!length}
is the number of elements that
precede the terminating null character.
An \defnx{empty \ntbs{}}{NTBS@\ntbs{}!empty}
has a length of zero.

\pnum
The \defnx{value of an \ntbs{}}{NTBS@\ntbs{}!value}
is the sequence of values of the
elements up to and including the terminating null character.

\pnum
A \defnx{static \ntbs{}}{NTBS@\ntbs{}!static}
is an \ntbs{} with
static storage duration.

\rSec3[multibyte.strings]{Multibyte strings}

\indextext{string!null-terminated multibyte|see{\ntmbs{}}}%
\pnum
A \defnx{null-terminated multibyte string}{NTMBS@\ntmbs{}},
or \ntmbs{},
is an \ntbs{} that constitutes a
sequence of valid multibyte characters, beginning and ending in the initial
shift state.

\begin{quoteblock}
Edit the footnote attached to the above sentence as follow:
\begin{quoteblock}
An \ntbs{} that \changed{contains characters only from the
basic execution character set is also an \ntmbs{}.
Each multibyte character then consists of a single byte}{only contains characters represented as a single byte is also an \ntmbs}.
\end{quoteblock}
\end{quoteblock}

\pnum
A \defnx{static \ntmbs{}}{NTMBS@\ntmbs{}!static}
is an \ntmbs{} with static storage duration.


\rSec1[locales]{Locales}

\rSec2[locale]{Class \tcode{locale}}

\rSec3[locale.ctype.members]{\tcode{ctype} members}

\indexlibrarymember{ctype}{do_widen}%
\begin{itemdecl}
    charT        do_widen(char c) const;
    const char*  do_widen(const char* low, const char* high, charT* dest) const;
\end{itemdecl}

\begin{itemdescr}
\pnum
\effects
Applies the simplest reasonable transformation
from a \tcode{char} value or sequence of \tcode{char} values
to the corresponding \tcode{charT} value or values.
The only characters for which unique transformations are required
are those in the basic \removed{source} character set\iref{lex.charset}.

For any named \tcode{ctype} category with
a \tcode{ctype<charT>} facet \tcode{ctc} and
valid \tcode{ctype_base::mask} value \tcode{M},
\tcode{(ctc.\brk{}is(M, c) || !is(M, do_widen(c)) )} is \tcode{true}.

The second form transforms
each character \tcode{*p} in the range \range{low}{high},
placing the result in \tcode{dest[p - low]}.

\pnum
\returns
The first form returns the transformed value.
The second form returns \tcode{high}.
\end{itemdescr}

\indexlibrarymember{ctype}{do_narrow}%
\begin{itemdecl}
char         do_narrow(charT c, char dfault) const;
const charT* do_narrow(const charT* low, const charT* high, char dfault, char* dest) const;
\end{itemdecl}

\begin{itemdescr}
\pnum
\effects
Applies the simplest reasonable transformation
from a \tcode{charT} value or sequence of \tcode{charT} values
to the corresponding \tcode{char} value or values.

For any character \tcode{c} in the basic \removed{source} character set\iref{lex.charset}
the transformation is such that
\begin{codeblock}
    do_widen(do_narrow(c, 0)) == c
\end{codeblock}
\end{itemdescr}

\rSec1[time]{Time library}

\begin{LongTable}{Meaning of \tcode{parse} flags}{time.parse.spec}{lx{.8\hsize}}
    \\ \topline
    \lhdr{Flag} & \rhdr{Parsed value} \\ \capsep
    \endfirsthead
    \continuedcaption\\
    \hline
    \lhdr{Flag} & \rhdr{Parsed value} \\ \capsep
    \endhead
    \tcode{\%a} &
    The locale's full or abbreviated case-insensitive weekday name.
    \\ \rowsep
    \tcode{\%Z} &
    The time zone abbreviation or name.
    A single word is parsed.
    This word can only contain characters
    from the basic \removed{source} character set\iref{lex.charset}
    that are alphanumeric, or one of
    \tcode{'_'}, \tcode{'/'}, \tcode{'-'}, or \tcode{'+'}.
    \\ \rowsep
    \tcode{\%\%} &
    A \tcode{\%} character is extracted.
    \\
\end{LongTable}

\rSec1[diff.cpp14]{\Cpp{} and ISO \CppXIV{}}


\rSec2[diff.cpp14.lex]{lexical conventions}


\indextext{trigraph sequence}%
\change
Removal of trigraph support as a required feature.
\rationale
Prevents accidental uses of trigraphs in non-raw string literals and comments.
\effect
Valid \CppXIV{} code that uses trigraphs may not be valid or may have different
semantics in this revision of \Cpp{}. Implementations may choose to
translate trigraphs as specified in \CppXIV{} if they appear outside of a raw
string literal, as part of the \impldef{mapping from physical source file characters
    to basic character set} mapping from physical source file characters to
the basic \removed{source} character set.

\section{References}
\renewcommand{\section}[2]{}%
\bibliographystyle{plain}
\bibliography{wg21}

\begin{thebibliography}{9}

\bibitem[N4878]{N4878}
Richard Smith
\emph{Working Draft, Standard for Programming Language C++}\newline
\url{https://wg21.link/N4878}

\end{thebibliography}
\end{document}
