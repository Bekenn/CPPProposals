% !TeX program = luatex
% !TEX encoding = UTF-8


\documentclass{wg21}

\title{UB? In My Lexer?}
\docnumber{P2621R3}
\audience{EWG, SG-22}
\author{Corentin Jabot}{corentin.jabot@gmail.com}

\begin{document}
\maketitle

\section{Abstract}

The mere act of Lexing c++ can result in undefined behavior.
This paper removes that undefined behavior.
Further work will be needed to remove all undefined behavior in [cpp].


\section{Revisions}

\subsection*{Revision 3}
\begin{itemize}
 \item Rebase and fix the context around the modified wording - the modified wording is left untouched
\end{itemize}

\subsection*{Revision 2}
\begin{itemize}
\item Apply CWG feedback from Issaquah
\end{itemize}

\subsection*{Revision 1}
\begin{itemize}
\item Fix typos
\end{itemize}


\section {Motivation}

According to the standard, the following examples expose undefined behavior:

\begin{colorblock}
int \\ // UB : universal character name accross spliced lines
u\
0\
3\
9\
1 = 0;


#define CONCAT(x, y) x ## y
int CONCAT(\, u0393) = 0; // UB: universal character name formed by macro expansion

// UB: unterminated string
const char * foo = "
\end{colorblock}

It does seem unfortunate that lexing C++ would incur UB.
As such, we propose to change the specification to remove the UB by either well-defining the behavior or making it ill-formed,
closely matching implementations.
The status-quo, as well as the proposed changes, are summarized below.
The red cell highlights the only impact this paper would have on existing implementations.

\renewcommand*\arraystretch{1.4}
\begin{tabular}{|c|c|c|c|c|c|}
     \cline{2-6}
    \multicolumn{1}{c|}{} & GCC & CLANG & EDG & MSVC & Proposed \\
    \hline
    Spliced UCN & Supported & Supported & Supported & \textcolor{red}{Error} & Well-formed \\
    \hline
    UCN Produced BY \#\# & Supported & Supported & Supported & Supported & Well-formed \\
    \hline
    Unterminated " or ' & ill-formed & ill-formed & ill-formed & ill-formed & ill-formed \\
    \hline
\end{tabular}\\

We propose that spliced UCNs be supported because, in addition to 3/4 of surveyed compilers supporting it,
it falls off naturally of the specification: splicing happens before any other form of tokenization and supporting it avoid
special-casing this oddity.

\section{Wording}

\rSec1[lex.phases]{Phases of translation}

\pnum{2.}
\indextext{line splicing}%
If the first translation character is U+FEFF BYTE ORDER MARK, it is deleted.
Each sequence of a backslash character (\textbackslash)
immediately followed by
zero or more whitespace characters other than new-line followed by
a new-line character is deleted, splicing
physical source lines to form logical source lines. Only the last
backslash on any physical source line shall be eligible for being part
of such a splice.
\removed{Except for splices reverted in a raw string literal, if a splice results in
a character sequence that matches the
syntax of a \grammarterm{universal-character-name}, the behavior is
undefined}.
\begin{addedblock}
\begin{note}
Line splicing can form a \grammarterm{universal-character-name} [lex.charset].
\end{note}
\end{addedblock}
A source file that is not empty and that does not end in a new-line
character, or that ends in a splice,
shall be processed as if an additional new-line character were appended
to the file.

\rSec1[lex.pptoken]{Preprocessing tokens}

\pnum{2.}
A preprocessing token is the minimal lexical element of the language in translation
phases 3 through 6.
In this document,
glyphs are used to identify
elements of the basic character set\iref{lex.charset}.
The categories of preprocessing token are: header names,
placeholder tokens produced by preprocessing \tcode{import} and \tcode{module} directives
(\grammarterm{import-keyword}, \grammarterm{module-keyword}, and \grammarterm{export-keyword}),
identifiers, preprocessing numbers, character literals (including user-defined character
literals), string literals (including user-defined string literals), preprocessing
operators and punctuators, and single non-whitespace characters that do not lexically
match the other preprocessing token categories.
If a U+0027 APOSTROPHE or a U+0022 QUOTATION MARK
matches the last category, the \changed{behavior is undefined}{program is ill-formed}.
If any character not in the basic character set matches the last category,
the program is ill-formed.

\rSec2[cpp.concat]{The \tcode{\#\#} operator}%
\indextext{\#\#1 operator@\tcode{\#\#} operator}%
\indextext{concatenation!macro argument|see{\tcode{\#\#} operator}}

\pnum{1.}
A
\tcode{\#\#}
preprocessing token shall not occur at the beginning or
at the end of a replacement list for either form
of macro definition.

\pnum{2.}
If, in the replacement list of a function-like macro, a parameter is
immediately preceded or followed by a
\tcode{\#\#}
preprocessing token, the parameter is replaced by the
corresponding argument's preprocessing token sequence; however, if an argument consists of no preprocessing tokens, the parameter is
replaced by a placemarker preprocessing token instead.

\pnum{3.}
For both object-like and function-like macro invocations, before the
replacement list is reexamined for more macro names to replace,
each instance of a
\tcode{\#\#}
preprocessing token in the replacement list
(not from an argument) is deleted and the
preceding preprocessing token is concatenated
with the following preprocessing token.
Placemarker preprocessing tokens are handled specially: concatenation
of two placemarkers results in a single placemarker preprocessing token, and
concatenation of a placemarker with a non-placemarker preprocessing token results
in the non-placemarker preprocessing token.
\begin{removedblock}
 If the result begins with a sequence matching the syntax of \grammarterm{universal-character-name},
the behavior is undefined.

\begin{note}
    This determination does not consider the replacement of
    \grammarterm{universal-character-name}s in translation phase 3\iref{lex.phases}.
\end{note}
\end{removedblock}
\begin{addedblock}
\begin{note}
Concatenation can form a \grammarterm{universal-character-name}.
\end{note}
\end{addedblock}

If the result is not a valid preprocessing token,
the behavior is undefined.
The resulting token is available for further macro replacement.
The order of evaluation of
\tcode{\#\#}

\section{Future work}

The reader will have noticed that [cpp] has a few other undefined behaviors.
This should equally be fixed, however, this work is best left to someone with greater preprocessor expertise.

\end{document}
