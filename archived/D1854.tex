% !TeX program = luatex
% !TEX encoding = UTF-8


\documentclass{wg21}
%\usepackage{luatexja-fontspec}
%\setmainjfont{Noto Sans CJK KR}

\title{Making non-encodable string literals ill-formed}
\docnumber{P1854R4}
\audience{EWG}
\author{Corentin Jabot}{corentin.jabot@gmail.com}

\usepackage{tikz}
\begin{document}

    \maketitle

\paperquote{It's just semantic! - Kevlin Henney}

\section{Target}

C++26

\section{Abstract}

This paper makes non-encodable characters (characters that cannot be represented in the literal encoding) in character and string literals ill-formed.
We also restrict the valid characters in a multicharacter literal to avoid visual ambiguity caused by graphemes constituted of multiple code points.

Both proposed changes are unrelated but proposed in the same paper as their wording is co-located.
The overarching motivation is to make lexing less suprising in the presence of non-latin1 characters in the source.

\section{Revisions}

\subsection{R4}
\begin{itemize}
\item Apply wording changes suggested by CWG in Issaquah
\end{itemize}


\subsection{R3}
\begin{itemize}
\item Fix spelling of code point
\item Make it clearer than both proposals are independent, but the wording is jumbled together.
\end{itemize}

\subsection{R2}
\begin{itemize}
\item Modify the wording to mandate that each element of a multicharacter literal is representable as a single code unit,
rather than restricting them to members of the basic character set.
\item Expand motivation to clarify that multicharacter literals are changed only to avoid visual ambiguities.
\end{itemize}

\subsection{R1}
\begin{itemize}
\item Rebase the wording
\item Propose to make non-basic characters in multi-characters ill-formed
\item Add more motivation.
\end{itemize}


\section{Non-encodable character-literals}

Implementation defined behaviors related to conversion to literal encoding reduce the portability of C++ programs and lead to silently incorrect
programs as implementations are allowed to substitute the characters they cannot represent.
Strings are text which carries intent and meaning. We believe an implementation should not be able to alter that meaning.

A program should either conserve the same sequence of abstract character as in the source or be ill-formed.

This constitutes a breaking change in the wording, as well as some implementations(MSVC), and matches other existing implementations' behavior.


This is a follow-up to \paper{P2362R3} and a realization of the plan outlined in \paper{P2178R1}.

\subsection{Impact on the standard and implementations}

\begin{itemize}
\item Clang always use UTF-8 to encode narrow literals
\item GCC emits a diagnostics

\begin{quoteblock}
    converting to execution character set: Invalid or incomplete multibyte or wide character
\end{quoteblock}
\item MSVC uses \tcode{?} as a replacement character.
For example, the string \tcode{"こんにちは"} becomes \tcode{' ?? ?? ', 00H}.
MSVC does emits warning for this scenario (enabled by default).

\begin{quoteblock}
    \begin{codeblock}
        <source>(4): warning C4566: character represented by universal-character-name
        '\u3053' cannot be represented in the current code page (20127)
        <source>(4): warning C4566: character represented by universal-character-name
        '\u3093' cannot be represented in the current code page (20127)
        <source>(4): warning C4566: character represented by universal-character-name
        '\u306B' cannot be represented in the current code page (20127)
        <source>(4): warning C4566: character represented by universal-character-name
        '\u3061' cannot be represented in the current code page (20127)
        <source>(4): warning C4566: character represented by universal-character-name
        '\u306F' cannot be represented in the current code page (20127)
    \end{codeblock}
\end{quoteblock}

\end{itemize}

A demonstration of existing behavior is available on \href{https://compiler-explorer.com/z/4vheqx1oj}{Compiler Explorer}.

We argue that the code which breaks never matches the developer's intent.

\subsection{Are we removing a capability?}

\begin{itemize}
\item The exact nature of the literal encoding can be observed by a dedicated API \paper{P1885R7}, and in general, the relying on non-encodable characters to detect the literal encoding
is non-portable as it can only work on windows. It is also very difficult to use such clever tricks in a way that has no false positives or false negatives.
\item \tcode{?} can be inserted in string and character literals.
\item \tcode{u8} strings can be used portably.
\item If the code's author does not care about the content of a string being preserved, then presumably that character can be removed.
\end{itemize}

\subsection{Impact on C}

This makes a behavior that is implementation-defined in C ill-formed in C++.
GCC exposes the same behavior (the one proposed by this paper) in all language modes.

\section{Multi character literals}

Narrow multicharacter literals such as \tcode{'intl'} are widely used. We are not proposing to remove them.
However, grapheme clusters - for example, 'é' (e, ACUTE ACCENT) - read as single characters.
A multicharacter literal can be visually indistinguishable from a character literal, leading to the accidental creation of multicharacter literals.
This is the same issue described for wide literals in \paper{P2362R3}.

Unlike what we proposed for wide-literals, we can't make all the multi-characters literals ill-formed.
Instead, we propose that multi-characters literals can only contain characters representable as a single code unit.

This has two benefits
\begin{itemize}
\item It excludes all combining characters or characters that do not constitute a full grapheme. That takes care of the visual ambiguity.
\item It makes multi-characters literals slightly less confusing as it is difficult to imagine how multiple code points over 0x80 could be stuffed into an \tcode{int} in any sensible way.
\end{itemize}
And indeed, the documentation of GCC shows that code points that do not fit in a single byte are not preserved but instead truncated.
\begin{quoteblock}
The compiler evaluates a multicharacter character constant a character at a time, shifting the previous value left by the number of bits per target character and then or-ing in the bit-pattern of the new character truncated to the width of a target character. The final bit-pattern is given type int, and is therefore signed, regardless of whether single characters are signed or not. If there are more characters in the constant than would fit in the target int the compiler issues a warning, and the excess leading characters are ignored.

For example, 'ab' for a target with an 8-bit char would be interpreted as \tcode{(int) ((unsigned char) 'a' * 256 + (unsigned char) 'b')}, and '\textbackslash 234a' as \tcode{(int) ((unsigned char) '\textbackslash 234' \* 256 + (unsigned char) 'a')}.
\end{quoteblock}

Because this proposal only cares about visual ambiguities between character literal and multicharacter literals, we do not propose to
make any escape sequences in multicharacter literal ill-formed (the number of escape sequences in a multicharacter or character literal is visually identifiable).

It is likely that escape sequences in multicharacter literals are in the general case non-sensible but given the implementation-defined nature of
multicharacter literals, we do not think there is value in adding further restriction; our only goal is for users to accidentally introduce them.

\subsection{Alternatives considered}

We considered
\begin{itemize}
\item Restricting elements of multi-characters literals to elements of the basic translation set. However, that would exclude \$ and @.
\item Restricting elements of multi-characters literals to U+0000-U+007F.
\end{itemize}

These solutions are virtually isomorphic (given that characters in these ranges are visually distinguishable). However, restricting encodability by a single unit is arguably the more direct expression of intent.


\subsection{Impact on the standard and implementations}

GCC, Clang ICC(EDG) emit a warning for any multi-characters literals in general. They also emit a warning when the computed value exceeds the size of \tcode{int}.

No compiler emits a warning for non-encodable characters in multicharacter literals.
Because this feature cannot produce a sensible result, we do not think its removal would affect users.

\section{Feature macro}

No feature macro is proposed because the transformation to characters literals and string literals is not observable by the program.


\section{Proposed wording}


\rSec2[lex.ccon]{Character literals}

\pnum
\indextext{literal!character}%
\indextext{literal!\idxcode{char8_t}}%
\indextext{literal!\idxcode{char16_t}}%
\indextext{literal!\idxcode{char32_t}}%
\indextext{literal!type of character}%
\indextext{type!\idxcode{char8_t}}%
\indextext{type!\idxcode{char16_t}}%
\indextext{type!\idxcode{char32_t}}%
\indextext{wide-character}%
\indextext{type!\idxcode{wchar_t}}%
\begin{removedblock}
A \defnx{non-encodable character literal}{literal!character!non-encodable}
is a \grammarterm{character-literal}
whose \grammarterm{c-char-sequence} consists of a single \grammarterm{c-char}
that is not a \grammarterm{numeric-escape-sequence} and
that specifies a character
that either lacks representation in the literal's associated character encoding
or that cannot be encoded as a single code unit.
\end{removedblock}
A \defnadj{multicharacter}{literal} is a \grammarterm{character-literal}
whose \grammarterm{c-char-sequence} consists of
more than one \grammarterm{c-char}.
\begin{addedblock}
A multicharacter literal shall not have an \grammarterm{encoding-prefix}.
If a multicharacter literal contains a \grammarterm{c-char} that is not encodable as a single code unit in the ordinary literal encoding, the program is ill-formed.
\end{addedblock}

\begin{removedblock}
The \grammarterm{encoding-prefix} of
a non-encodable character literal or
a multicharacter literal
shall be absent \removedprev{or \tcode{L}}.
\end{removedblock}

\changed{Such \grammarterm{character-literal}{s}}{Multicharacter literals} are conditionally-supported.

\pnum
The kind of a \grammarterm{character-literal},
its type, and its associated character encoding
are determined by
its \grammarterm{encoding-prefix} and its \grammarterm{c-char-sequence}
as defined by [lex.ccon.literal].
\removed{The special cases for
non-encodable character literals and multicharacter literals
take precedence over the\removedprev{ir respective} base kind\removedprev{s}.}

\begin{removedblock}
\begin{note}
    The associated character encoding for ordinary \removedprev{and wide} character literals
    determines encodability,
    but does not determine the value of
    non-encodable ordinary \removedprev{or wide} character literals or
    ordinary or wide multicharacter literals.
    The examples in [lex.ccon.literal]
    for non-encodable ordinary \removedprev{and wide} character literals assume that
    the specified character lacks representation in
    the execution character set \removedprev{or execution wide-character set, respectively,} or
    that encoding it would require more than one code unit.
\end{note}
\end{removedblock}

\begin{floattable}{Character literals}{lex.ccon.literal}
{lllll}
\topline
Encoding & Kind & Type & Associated char- & Example \\
prefix & & & acter encoding & \\
\capsep
none &
\defnx{ordinary character literal}{literal!character!ordinary} &
\keyword{char} &
encoding of &
\tcode{'v'} \\
&
\removed{non-encodable ordinary character literal} &
\removed{\keyword{int}} &
the execution &
\removed{\tcode{'\textbackslash U0001F525'}} \\
&
\removed{ordinary} multicharacter literal &
\keyword{int} &
character set &
\tcode{'abcd'} \\ \hline
\tcode{L} &
\defnx{wide character literal}{literal!character!wide} &
\keyword{wchar_t} &
encoding of &
\tcode{L'w'} \\
&
\removedprev{non-encodable wide character literal} &
\removedprev{\keyword{wchar_t}} &
the execution &
\removedprev{\tcode{L'\textbackslash U0001F32A'}} \\
&
\removedprev{wide multicharacter literal} &
\removedprev{\keyword{wchar_t}} &
wide-character set &
\removedprev{\tcode{L'abcd'}} \\ \hline
\tcode{u8} &
\defnx{UTF-8 character literal}{literal!character!UTF-8} &
\keyword{char8_t} &
UTF-8 &
\tcode{u8'x'} \\ \hline
\tcode{u} &
\defnx{UTF-16 character literal}{literal!character!UTF-16} &
\keyword{char16_t} &
UTF-16 &
\tcode{u'y'} \\ \hline
\tcode{U} &
\defnx{UTF-32 character literal}{literal!character!UTF-32} &
\keyword{char32_t} &
UTF-32 &
\tcode{U'z'} \\
\end{floattable}

\pnum
In translation phase 4,
the value of a \grammarterm{character-literal} is determined
using the range of representable values
of the \grammarterm{character-literal}'s type in translation phase 7.
A \removed{non-encodable character literal or a} multicharacter literal
has an implementation-defined value.
The value of any other kind of \grammarterm{character-literal}
is determined as follows:
\begin{itemize}
\item
A \grammarterm{character-literal} with
a \grammarterm{c-char-sequence} consisting of a single
\grammarterm{basic-c-char},
\grammarterm{simple-escape-sequence}, or
\grammarterm{universal-character-name}
is the code unit value of the specified character
as encoded in the literal's associated character encoding.
\removed{[Note:}
If the specified character lacks
representation in the literal's associated character encoding or
if it cannot be encoded as a single code unit,
then \changed{the literal is a non-encodable character literal}{the program is ill-formed}.
\removed{- end note]}
\item
A \grammarterm{character-literal} with
a \grammarterm{c-char-sequence} consisting of
a single \grammarterm{numeric-escape-sequence}
that specifies an integer value $v$ has a value as follows:
\begin{itemize}
    \item
    If $v$ does not exceed
    the range of representable values of the \grammarterm{character-literal}'s type,
    then the value is $v$.
    \item
    Otherwise,
    if the \grammarterm{character-literal}'s \grammarterm{encoding-prefix}
    is absent or \tcode{L}, and
    $v$ does not exceed the range of representable values of the corresponding unsigned type for the underlying type of the \grammarterm{character-literal}'s type,
    then the value is the unique value of the \grammarterm{character-literal}'s type \tcode{T} that is congruent to $v$ modulo $2^N$, where $N$ is the width of \tcode{T}.
    \item
    Otherwise, \changed{the \grammarterm{character-literal}}{the program} is ill-formed.
\end{itemize}
\item
A \grammarterm{character-literal} with
a \grammarterm{c-char-sequence} consisting of
a single \grammarterm{conditional-escape-sequence}
is conditionally-supported and
has an \impldef{value of \grammarterm{conditional-escape-sequence}} value.
\end{itemize}


%Modify 5.2.5 [lex.phases] as follow
%
%\begin{quote}
%
%Each basic source character set member in a character literal or a string literal, as well as each escape sequence and universal-character-name in a character literal or a non-raw string literal, is converted to the corresponding member of the execution character set ([lex.ccon], [lex.string]); if there is no corresponding member, \changed{it is converted to an implementation-defined member other than the null (wide) character}{the program is ill-formed}.
%
%\end{quote}
%
%
%Modify 5.13.3.2 [lex.icon] as follow
%
%\begin{quote}
%A character literal that does not begin with
%\tcode{u8}, \tcode{u}, \tcode{U}, or \tcode{L}
%is an \grammarterm{ordinary character literal}.
%An ordinary character literal that contains a
%single \grammarterm{c-char} representable in the execution character
%set has type \tcode{char}, with value equal to the
%numerical value of the encoding of the \grammarterm{c-char} in the
%execution character set.
%An ordinary character literal that contains more than one \grammarterm{c-char} is a
%\grammarterm{multicharacter literal}.
%A multicharacter literal\removed{, or an ordinary character literal containing
%a single \grammarterm{c-char} not representable in the execution
%character set}, is conditionally-supported, has type \tcode{int},
%and has an implementation-defined value.
%\added{An ordinary character literal containing a single \grammarterm{c-char} not representable in the execution character set is ill-formed.}
%
%\end{quote}
%
%Modify 5.13.3.6 [lex.icon] as follow
%
%\begin{quote}
%\pnum
%A character literal that
%begins with the letter \tcode{L}, such as \tcode{L'z'}, is a wide-character literal. A wide-character literal has type
%\tcode{wchar_t}.
%The value of a wide-character literal containing a single
%\grammarterm{c-char} has value equal to the numerical value of the encoding
%of the \grammarterm{c-char} in the execution wide-character set, unless the
%\grammarterm{c-char} has no representation in the execution wide-character set, in which
%case the \changed{value is implementation-defined}{program is ill-formed}. \begin{note} The type \tcode{wchar_t} is able to
%	represent all members of the execution wide-character set (see~\ref{basic.fundamental}).
%\end{note}
%The value of a wide-character literal containing multiple \grammarterm{c-char}{s} is implementation-defined.
%
%\end{quote}
%
%Modify 5.13.3.8 [lex.icon] as follow
%
%\begin{quote}
%The escape
%\tcode{\textbackslash\numconst{ooo}} consists of the backslash followed by one,
%two, or three octal digits that are taken to specify the value of the
%desired character. The escape
%\tcode{\textbackslash x\numconst{hhh}}
%consists of the backslash followed by \tcode{x} followed by one or more
%hexadecimal digits that are taken to specify the value of the desired
%character. There is no limit to the number of digits in a hexadecimal
%sequence. A sequence of octal or hexadecimal digits is terminated by the
%first character that is not an octal digit or a hexadecimal digit,
%respectively.
%
%\begin{removedblock}
%The value of a character literal is implementation-defined if it falls outside
%of the implementation-defined
%range defined for \tcode{char} (for character literals with no prefix) or
%\tcode{wchar_t} (for character literals prefixed by \tcode{L}).
%
%\begin{note}
%	If the value of a character literal prefixed by
%	\tcode{u}, \tcode{u8}, or \tcode{U}
%	is outside the range defined for its type,
%	the program is ill-formed.
%\end{note}
%\end{removedblock}

\rSec2[lex.string]{String literals}

\begin{quote}

String literal objects are initialized with
the sequence of code unit values
corresponding to the \grammarterm{string-literal}'s sequence of
\grammarterm{s-char}s (for a non-raw string literal) and
\grammarterm{r-char}s (for a raw string literal)
in order as follows:
\begin{itemize}
    \item
    The sequence of characters denoted by each contiguous sequence of
    \grammarterm{basic-s-char}s,
    \grammarterm{r-char}s,
    \grammarterm{simple-escape-sequence}s\iref{lex.ccon}, and
    \grammarterm{universal-character-name}s\iref{lex.charset}
    is encoded to a code unit sequence
    using the \grammarterm{string-literal}'s associated character encoding.
    If a character lacks representation in the associated character encoding, then \changed{:}{the program is ill-formed.}
    \begin{removedblock}
        \begin{itemize}
            \item
            If the \grammarterm{string-literal}'s \grammarterm{encoding-prefix}
            is absent or \tcode{L},
            then the \grammarterm{string-literal} is conditionally-supported and
            an
            \impldef{code unit sequence for non-representable \grammarterm{string-literal}}
            code unit sequence is encoded.
            \item
            Otherwise, the \grammarterm{string-literal} is ill-formed.
        \end{itemize}
    \end{removedblock}
    When encoding a stateful character encoding,
    implementations should encode the first such sequence
    beginning with the initial encoding state and
    encode subsequent sequences
    beginning with the final encoding state of the prior sequence.
    \begin{note}
        The encoded code unit sequence can differ from
        the sequence of code units that would be obtained by
        encoding each character independently.
    \end{note}
    \item
    Each \grammarterm{numeric-escape-sequence}\iref{lex.ccon}
    that specifies an integer value $v$
    contributes a single code unit with a value as follows:
    \begin{itemize}
        \item
        If $v$ does not exceed the range of representable values of
        the \grammarterm{string-literal}'s array element type,
        then the value is $v$.
        \item
        Otherwise,
        if the \grammarterm{string-literal}'s \grammarterm{encoding-prefix}
        is absent or \tcode{L}, and
        $v$ does not exceed the range of representable values of
        the corresponding unsigned type for the underlying type of
        the \grammarterm{string-literal}'s array element type,
        then the value is the unique value of
        the \grammarterm{string-literal}'s array element type \tcode{T}
        that is congruent to $v$ modulo $2^N$, where $N$ is the width of \tcode{T}.
        \item
        Otherwise, the \changed{\grammarterm{string-literal}}{program} is ill-formed.
    \end{itemize}
    When encoding a stateful character encoding,
    these sequences should have no effect on encoding state.
    \item
    Each \grammarterm{conditional-escape-sequence}\iref{lex.ccon}
    contributes an
    \impldef{code unit sequence for \grammarterm{conditional-escape-sequence}}
    code unit sequence.
    When encoding a stateful character encoding,
    it is
    \impldef{effect of \grammarterm{conditional-escape-sequence} on encoding state}
    what effect these sequences have on encoding state.
\end{itemize}



%\added{If the value of a character literal is outside the range defined for its type, the program is ill-formed.}

\end{quote}

\section{Acknowledgments}

Many thanks to JeanHeyd Meneide, Peter Bindels, Zach Laine, Tom Honermann, and Steve Downey, who reviewed this paper and offered valuable feedback.


\bibliographystyle{plain}
\bibliography{wg21}


\renewcommand{\section}[2]{}%
\begin{thebibliography}{9}
\bibitem[N4885]{N4885}
Thomas Köppe
\emph{Working Draft, Standard for Programming Language C++}\newline
\url{https://wg21.link/N4885}
\end{thebibliography}

\end{document}
