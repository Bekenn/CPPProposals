\documentclass{wg21}

\usepackage{xcolor}
\usepackage{soul}
\usepackage{ulem}
\usepackage{fullpage}
\usepackage{parskip}
\usepackage{csquotes}
\usepackage{listings}
\usepackage{minted}
\usepackage{enumitem}
\usepackage{minted}


\lstdefinestyle{base}{
  language=c++,
  breaklines=false,
  basicstyle=\ttfamily\color{black},
  moredelim=**[is][\color{green!50!black}]{@}{@},
  escapeinside={(*@}{@*)}
}

\newcommand{\cc}[1]{\mintinline{c++}{#1}}
\newminted[cpp]{c++}{}


\title{Iterators pair constructors for stack and queue}
\docnumber{DXXXX}
\audience{LEWG}
\author{Corentin Jabot}{corentin.jabot@gmail.com}


\begin{document}
\maketitle

\section{Abstract}

This paper proposes to add iterators-pair constructors to \tcode{std::stack} and \tcode{std::queue}

\section{Tony tables}
\begin{center}
\begin{tabular}{l|l}
Before & After\\ \hline

\begin{minipage}[t]{0.5\textwidth}
\begin{minted}[fontsize=\footnotesize]{cpp}
std::vector<int> v(42);
std::queue<int> q({v.begin(), v.end()});
std::stack<int> s({v.begin(), v.end()});
\end{minted}
\end{minipage}
&
\begin{minipage}[t]{0.3\textwidth}
\begin{minted}[fontsize=\footnotesize]{cpp}
std::vector<int> v(42);
std::queue q(v.begin(), v.end());
std::queue s(v.begin(), v.end());
\end{minted}
\end{minipage}
\\\\ \hline

\end{tabular}
\end{center}


\section{Motivation}

\tcode{std::stack} and \tcode{std::queue} do not provide iterators based constructors which is inconsistent.
This paper is an offshoot of \cite{P1206}, for which i conducted a review of existing containers and containers adapters constructors.

The lack of these constructors forces the implementation of \tcode{ranges::to} to special case container-adapters or to not support them.
Their absence make it also impossible to deduce their type using CTAD.

While this is a a small change, we believe its impact on the standard is low and consistent designs are less surprising and therefore easier
to use : With this change, all container-like types, whether they are \emph{Containers} or container adapters can be constructed from 
an iterators pair, making them more compatible with \tcode{ranges}.

\pagebreak

\section{Proposed Wording}

\begin{quote}

\rSec2[queue.defn]{Definition}
	
\begin{codeblock}
	
	
namespace std {
	template<class T, class Container = deque<T>>
	class queue {
		public:
		using value_type      = typename Container::value_type;
		using reference       = typename Container::reference;
		using const_reference = typename Container::const_reference;
		using size_type       = typename Container::size_type;
		using container_type  =          Container;
		
		protected:
		Container c;
		
		public:
		queue() : queue(Container()) {}
		explicit queue(const Container&);
		explicit queue(Container&&);
		
		@\added{template<class InputIterator>}@
		@\added{queue(InputIterator first, InputIterator last);}@
				
		template<class Alloc> explicit queue(const Alloc&);
		template<class Alloc> queue(const Container&, const Alloc&);
		template<class Alloc> queue(Container&&, const Alloc&);
		template<class Alloc> queue(const queue&, const Alloc&);
		template<class Alloc> queue(queue&&, const Alloc&);
		
		//...
	};
	
	template<class Container>
	queue(Container) -> queue<typename Container::value_type, Container>;
	
	@\added{template<class InputIterator>}@
	@\added{queue(InputIterator, InputIterator)}@
	@\added{-> queue<typename iterator_traits<InputIterator>::value_type>;}@
	
	template<class Container, class Allocator>
	queue(Container, Allocator) -> queue<typename Container::value_type, Container>;
	
	template<class T, class Container>
	void swap(queue<T, Container>& x, queue<T, Container>& y) noexcept(noexcept(x.swap(y)));
	
	template<class T, class Container, class Alloc>
	struct uses_allocator<queue<T, Container>, Alloc>
	: uses_allocator<Container, Alloc>::type { };
}
\end{codeblock}

\rSec2[queue.cons]{Constructors}

\begin{itemdecl}
	explicit queue(const Container& cont);
\end{itemdecl}

\begin{itemdescr}
	\pnum
	\effects\ Initializes \tcode{c} with \tcode{cont}.
\end{itemdescr}

\begin{itemdecl}
	explicit queue(Container&& cont);
\end{itemdecl}

\begin{itemdescr}
	\pnum
	\effects\ Initializes \tcode{c} with \tcode{std::move(cont)}.
\end{itemdescr}

\begin{addedblock}

\begin{itemdecl}
template<class InputIterator>
queue(InputIterator first, InputIterator last);
\end{itemdecl}

\begin{itemdescr}
	\pnum
	\effects
	Initializes \tcode{c} from the range \range{first}{last}
\end{itemdescr}

\end{addedblock}

\rSec3[stack.defn]{Definition}

\begin{codeblock}
namespace std {
	template<class T, class Container = deque<T>>
	class stack {
		public:
		using value_type      = typename Container::value_type;
		using reference       = typename Container::reference;
		using const_reference = typename Container::const_reference;
		using size_type       = typename Container::size_type;
		using container_type  = Container;
		
		protected:
		Container c;
		
		public:
		stack() : stack(Container()) {}
		explicit stack(const Container&);
		explicit stack(Container&&);
		
		@\added{template<class InputIterator>}@
		@\added{stack(InputIterator first, InputIterator last);}@
		
		template<class Alloc> explicit stack(const Alloc&);
		template<class Alloc> stack(const Container&, const Alloc&);
		template<class Alloc> stack(Container&&, const Alloc&);
		template<class Alloc> stack(const stack&, const Alloc&);
		template<class Alloc> stack(stack&&, const Alloc&);
		
		//...
	};
	
	template<class Container>
	stack(Container) -> stack<typename Container::value_type, Container>;
	
	@\added{template<class InputIterator>}@
	@\added{stack(InputIterator, InputIterator)}@
	@\added{-> stack<typename iterator_traits<InputIterator>::value_type>;}@
	
	template<class Container, class Allocator>
	stack(Container, Allocator) -> stack<typename Container::value_type, Container>;
	
	template<class T, class Container, class Alloc>
	struct uses_allocator<stack<T, Container>, Alloc>
	: uses_allocator<Container, Alloc>::type { };
}
\end{codeblock}

\rSec3[stack.cons]{Constructors}

\begin{itemdecl}
	explicit stack(const Container& cont);
\end{itemdecl}

\begin{itemdescr}
	\pnum
	\effects Initializes \tcode{c} with \tcode{cont}.
\end{itemdescr}

\begin{itemdecl}
	explicit stack(Container&& cont);
\end{itemdecl}

\begin{itemdescr}
	\pnum
	\effects Initializes \tcode{c} with \tcode{std::move(cont)}.
\end{itemdescr}


\begin{addedblock}
	
\begin{itemdecl}
	template<class InputIterator>
	stack(InputIterator first, InputIterator last);
\end{itemdecl}

\begin{itemdescr}
	\pnum
	\effects
	Initializes \tcode{c} from the range \range{first}{last}
\end{itemdescr}
	
\end{addedblock}


\end{quote}

\section{References}
\renewcommand{\section}[2]{}%
\begin{thebibliography}{9}
	
	\bibitem[P1206]{P1206}
	Corentin Jabot
	\emph{A function to convert any range to a container}\newline
	\url{https://wg21.link/P1206}

\end{thebibliography}
\end{document}