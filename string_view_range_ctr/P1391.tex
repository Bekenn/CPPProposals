\documentclass{wg21}

\usepackage{xcolor}
\usepackage{soul}
\usepackage{ulem}
\usepackage{fullpage}
\usepackage{parskip}
\usepackage{csquotes}
\usepackage{listings}
\usepackage{minted}
\usepackage{enumitem}
\usepackage{minted}


\lstdefinestyle{base}{
language=c++,
breaklines=false,
basicstyle=\ttfamily\color{black},
moredelim=**[is][\color{green!50!black}]{@}{@},
escapeinside={(*@}{@*)}
}

\newcommand{\cc}[1]{\mintinline{c++}{#1}}
\newminted[cpp]{c++}{}


\title{Range constructor for std::string\_view}
\docnumber{P1391R4}
\audience{LEWG, LWG}
\author{Corentin Jabot}{corentin.jabot@gmail.com}


\begin{document}
\maketitle

\section{Abstract}

This paper proposes that  \tcode{string_view} be constructible from any contiguous range of characters.
The idea was extracted from P1206.

\section{Tony tables}
\begin{center}
\begin{tabular}{l|l}
Before & After\\ \hline

\begin{minipage}[t]{0.5\textwidth}
\begin{minted}[fontsize=\footnotesize]{cpp}
void foo(string_view);
vector<char8_t> vec = get_some_unicode();
foo(string_view{vec.data(), vec.size()});

\end{minted}
\end{minipage}
&
\begin{minipage}[t]{0.5\textwidth}
\begin{minted}[fontsize=\footnotesize]{cpp}
void foo(string_view);
vector<char8_t> vec = get_some_unicode();
foo(vec);
\end{minted}
\end{minipage}
\\\\ \hline

\end{tabular}
\end{center}

\section{Motivation}

While P1206 gives a general motivation for range constructors, it's especially important for string_view because there exist in a lot of codebases
string types that would benefit from being convertible to string_view \tcode{string_view}. For example, \tcode{llvm::StringRef}, \tcode{QByteArray}, \tcode{fbstring}, \tcode{boost::container::string} ...

Manipulating the content of a vector as a string is also useful.

Finally, this makes contiguous views operating on characters easier to use with \tcode{string_view}.


\section{Design considerations}

\begin{itemize}
    \item instantiations of \tcode{basic_string} are specifically excluded because \tcode{std::basic_string} already provides a conversion operator and more importantly,
    strings with different char_traits should not be implicitly convertible
    \item Because \tcode{basic_string_view} doesn't mutate the underlying data, there is no reason to accept a range by something other than const lvalue reference.
    \item The construction is implicit because it is cheap and a contiguous range of character is the same platonic thing as a string_view.

\end{itemize}


\section{Arrays and null terminated strings}

During review by LWG, it was noticed that the proposed change introduces this arguably surprising behavior:

\begin{codeblock}

char const t[] = "text";
std::string_view s(t); // s.size() == 4;


std::span<char> tv(t);
std::string_view s(tv); // s.size() == 5;

\end{codeblock}

This is not an ambiguity of the overload set but rather a consequences of both null-terminated terminated strings and array of characters being both
sequence of characters with array of characters implicitly convertible to pointers.

To be consistent with C++17 and not introduce a behavior change, we make sure arrays of characters decay to const charT*.
We think this proposed design is consistent with existing practices of having to be explicit about the size in the presence of embedded nulls as well as the general behavior of C functions, and does not introduce a new problem - how unfortunate that problem might be.
It is also worth noting that while embedded nulls have a lot of known usages they are not the common case.

Finding a better solution to that problem is not possible at the level of this proposal and would require major breaking language changes.





\section{Proposed wording}

Change in \textbf{[string.view] 20.4.2}:
\begin{quote}
\begin{codeblock}

template<class charT, class traits = char_traits<charT>>
class basic_string_view {
public:
    [...]

    // construction and assignment
    constexpr basic_string_view() noexcept;
    constexpr basic_string_view(const basic_string_view&) noexcept = default;
    constexpr basic_string_view& operator=(const basic_string_view&) noexcept = default;
    constexpr basic_string_view(const charT* str);
    constexpr basic_string_view(const charT* str, size_type len);

    

    @\added{template <class It, class End>}@
    @\added{constexpr basic_string_view(It begin, End end);}@

    [...]
};
@\added{template<class It, class End>}@
@\added{basic_string_view(It, End) -> basic_string_view<iter_value_t<It>>;}@


\end{codeblock}
\end{quote}

%@\added{template <class R>}@
%@\added{constexpr basic_string_view(const R\& r);}@
%@\added{template<class R>}@
%@\added{basic_string_view(const R\&)}@
%@\added{    -> basic_string_view<ranges::range_value_t<const R>>;}@

Change in \textbf{[string.view.cons] 20.4.2.1}:

Add after 7

\begin{quote}
\begin{addedblock}
%\begin{itemdecl}
%template <class R>
%constexpr basic_string_view(const R& r);
%
%\end{itemdecl}
%
%\begin{itemdescr}
%    \constraints
%    \begin{itemize}
%        \item \tcode{const R} satisfies \tcode{ranges::contiguous_range},
%        \item \tcode{const R} satisfies \tcode{ranges::sized_range},
%        \item \tcode{is_same_v<ranges::range_value_t<const R>, charT>} is \tcode{true},
%        \item \tcode{is_convertible_v<const R\&, const charT*>} is \tcode{false}, and
%        \item If the qualified-id \tcode{R::traits_type} is valid and denotes a type, \tcode{is_same_v<R::traits\_type,traits>} is \tcode{true}.
%    \end{itemize}
% 
%	\expects
%	 \begin{itemize}
%		\item \tcode{const R} models \tcode{ranges::contiguous_range}, and
%		\item \tcode{const R} models \tcode{ranges::sized_range}.
%	\end{itemize}
%
%    \effects
%    Initializes \tcode{data\_} with \tcode{ranges::data(r)} and \tcode{size\_} with \tcode{ranges::size(r)}.
%
%
%    \throws
%    What and when \tcode{ranges::data(r)} and \tcode{ranges::size(r)} throw.
%


%\end{itemdescr}

\begin{itemdecl}
template <class It, class End>
constexpr basic_string_view(It first, End last);

\end{itemdecl}

\begin{itemdescr}
    \constraints
    \begin{itemize}
        \item \tcode{It} satisfies \tcode{contiguous_iterator},
        \item \tcode{End} satisfies \tcode{sized_sentinel_for<It>},
        \item \tcode{is_same_v<iter_value_t<It>, charT>} is \tcode{true}, and
        \item \tcode{is_convertible_v<End, size_type>} is \tcode{false}.
    \end{itemize}

    \expects
    \begin{itemize}
	    \item \range{first}{last} is a valid range,
	    \item \tcode{It} models \tcode{contiguous_iterator}, and
    	\item \tcode{End} models \tcode{sized_sentinel_for<It>}.
	\end{itemize}

    \effects
    Initializes \tcode{data_} with \tcode{to_address(first)}, and \tcode{size_} with \tcode{last - first}.

\end{itemdescr}
\end{addedblock}
\end{quote}


Add a new section [string.view.deduction] to describe the following deduction guides:



\begin{quote}
\begin{addedblock}
\begin{itemdecl}
template <class It, class End>
basic_string_view(It, End) -> basic_string_view<iter_value_t<It>>;
\end{itemdecl}
\begin{itemdescr}
    \constraints
    \begin{itemize}
        \item \tcode{It} satisfies \tcode{contiguous_iterator},
        \item \tcode{End} satisfies \tcode{sized_sentinel_for<It>}.
    \end{itemize}
\end{itemdescr}

%\begin{itemdecl}
%template<class R>
%basic_string_view(const R&)
%-> basic_string_view<ranges::range_value_t<const R>>;
%\end{itemdecl}
%\begin{itemdescr}
%    \constraints \tcode{const R} satisfies \tcode{ranges::contiguous_range}.
%\end{itemdescr}

\end{addedblock}
\end{quote}


\end{document}