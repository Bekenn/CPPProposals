\documentclass{wg21}

\usepackage{xcolor}
\usepackage{soul}
\usepackage{ulem}
\usepackage{fullpage}
\usepackage{parskip}
\usepackage{csquotes}
\usepackage{listings}
\usepackage{minted}
\usepackage{enumitem}
\usepackage{minted}


\lstdefinestyle{base}{
  language=c++,
  breaklines=false,
  basicstyle=\ttfamily\color{black},
  moredelim=**[is][\color{green!50!black}]{@}{@},
  escapeinside={(*@}{@*)}
}

\newcommand{\cc}[1]{\mintinline{c++}{#1}}
\newminted[cpp]{c++}{}


\title{Range constructor for std::string\_view}
\docnumber{D1391R1}
\audience{LEWG}
\author{Corentin Jabot}{corentin.jabot@gmail.com}


\begin{document}
\maketitle

\section{Abstract}

This paper proposes that  \tcode{string_view} be constructible from any contiguous range of characters.
The idea was extracted from P1206.

\section{Tony tables}
\begin{center}
\begin{tabular}{l|l}
Before & After\\ \hline

\begin{minipage}[t]{0.5\textwidth}
\begin{minted}[fontsize=\footnotesize]{cpp}
void foo(string_view);
vector<char8_t> vec = get_some_unicode();
foo(string_view{vec.data(), vec.size()});

\end{minted}
\end{minipage}
&
\begin{minipage}[t]{0.5\textwidth}
\begin{minted}[fontsize=\footnotesize]{cpp}
void foo(string_view);
vector<char8_t> vec = get_some_unicode();
foo(vec);
\end{minted}
\end{minipage}
\\\\ \hline

\end{tabular}
\end{center}

\section{Motivation}

While P1206 gives a general motivation for range constructors, it's especially important for string_view because there exist in a lot of codebases
string types that would benefit from being convertible to string_view \tcode{string_view}. For example, \tcode{llvm::StringRef}, \tcode{QByteArray}, \tcode{fbstring}, \tcode{boost::container::string} ...

Manipulating the content of a vector as a string is also useful.

Finally, this makes contiguous views operating on characters easier to use with \tcode{string_view}.


\section{Design considerations}

\begin{itemize}
    \item instantiations of \tcode{basic_string} are specifically excluded because \tcode{std::basic_string} already provides a conversion operator and more importantly,
    strings with different char_traits should not be implicitly convertible
    \item Because \tcode{basic_string_view} doesn't mutate the underlying data, there is no reason to accept a range by something other than const lvalue reference.
    \item The construction is implicit because it is cheap and a contiguous range of character is the same platonic thing as a string_view.

\end{itemize}

\section{Proposed wording}

Change in \textbf{[string.view] 20.4.2}:
\begin{quote}
\begin{codeblock}

template<class charT, class traits = char_traits<charT>>
class basic_string_view {
public:
    [...]

    // construction and assignment
    constexpr basic_string_view() noexcept;
    constexpr basic_string_view(const basic_string_view&) noexcept = default;
    constexpr basic_string_view& operator=(const basic_string_view&) noexcept = default;
    constexpr basic_string_view(const charT* str);
    constexpr basic_string_view(const charT* str, size_type len);

    @\added{template <ContiguousRange R>}@
    @\added{requires ranges::SizedRange<const R> \&\& Same<iter_value_t<iterator_t<const R>>, charT>}@
    @\added{constexpr basic_string_view(const R\& r);}@

    @\added{template <ContiguousIterator It, Sentinel<It> End>}@
    @\added{requires Same<iter_value_t<It>, charT>}@
    @\added{constexpr basic_string_view(It begin, End end );}@

    [...]
};

\end{codeblock}
\end{quote}

Change in \textbf{[string.view.cons] 20.4.2.1}:

Add after 7

\begin{quote}
\begin{addedblock}
\begin{itemdecl}
template <ranges::ContiguousRange R>
requires ranges::SizedRange<const R> && Same<iter_value_t<iterator_t<const R>>, charT>
constexpr basic_string_view(const R& r);

\end{itemdecl}

\begin{itemdescr}
    \effects
    Constructs a \tcode{basic_string_view} over the \tcode{ContiguousRange} r.


    \throws If  \tcode{data(r)} or \tcode{size(r)} throw

     \remarks This constructor shall not participate in overload resolution unless
    \begin{itemize}
        \item \tcode{is_array<R>} is \tcode{false}.
        \item \tcode{R} does not derive from a specialization of \tcode{std::basic_string}
        \item \tcode{R} does not derive from a specialization of \tcode{std::basic_string_view}
        \end{itemize}
    %
\end{itemdescr}

\begin{itemdecl}
	template <ContiguousIterator It, Sentinel<It> End>
	requires Same<iter_value_t<It>>, charT>
	constexpr basic_string_view(It begin, End end );

\end{itemdecl}

\begin{itemdescr}
	\effects
	Constructs a \tcode{basic_string_view} over the range [begin, end) .

	\remarks This constructor shall not participate in overload resolution unless
	\begin{itemize}
		\item \tcode{It} does not derive from an instantiation of  \tcode{std::basic_string}::iterator or  \tcode{std::basic_string}::const_iterator
		\item \tcode{It} does not derive from an instantiation of  \tcode{std::basic_string_view}::iterator  \tcode{std::basic_string_view}::const_iterator
		\item \tcode{It} and \tcode{End} are not of the same type or \tcode{End} is not convertible to a pointer of \tcode{charT}
	\end{itemize}
	%
\end{itemdescr}
\end{addedblock}
\end{quote}


\end{document}