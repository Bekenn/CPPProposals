% !TeX program = luatex
% !TEX encoding = UTF-8


\RequirePackage{luatex85}


\documentclass{wg21}



\newcommand{\UnicodeLetter}[1]{\textbf{\textcolor{BrickRed}{\Large\tcode{#1}}}}

\title{Reinterpreting pointers of character types}
\docnumber{P2292R0}
\audience{LEWG, EWG, SG-16}
\author{Corentin Jabot}{corentin.jabot@gmail.com}

\begin{document}
\maketitle

\paperquote{The best way to get the right answer is to write bad wording}

\section{Abstract}

We propose a function \tcode{reinterpret_string_cast} to cast between \tcode{char*} and c\tcode{char8_t*} without UB

\section{Tony tables}
\begin{center}
\begin{tabular}{l|l}
Before & After\\ \hline

\begin{minipage}[t]{0.45\textwidth}
\begin{colorblock}
const char* str = "UB";
auto res = reinterpret_cast<char8_t*>(str);
\end{colorblock}
\end{minipage}
&
\begin{minipage}[t]{0.5\textwidth}
\begin{colorblock}
const char* str = "Well defined";
auto res = std::reinterpret_string_cast<const char8_t*>(str);
\end{colorblock}
\end{minipage}
\\\\ \hline

\end{tabular}
\end{center}


\section{Motivation}

\tcode{char8_t} introduced in C++20 by \paper{P0482R6} is a character type designed to handle UTF-8 code units.
An important motivation for \tcode{char8_t} is that it doesn't alias with \tcode{char} and \tcode{char8_t*} are not implicitly convertible to \tcode{char_t}.
This allows users to make sure they never lose the utf8-ness of their strings and that they can handle both utf-8 and non-utf8 text and data within the same overload set.

There however exists a large number of libraries (iconv, ICU, gtk), and C functions that may use \tcode{char} to represent UTF-8 code units.
As such, many people have expressed an interest in a way to interpret a range of \tcode{char8_t} to \tcode{char} or inversely for 2 way compatibilities with existing libraries, and system functions.

\section{Design and constraints}

\begin{itemize}
    \item We don't want to change the design of \tcode{char8_t} (no aliasing or implicit conversion), as this would defeat the purpose of  \tcode{char8_t}
    \item Because such a cast would only be safe if the preconditions (valid UTF code units sequence) are met, we think the syntax should be explicit.
\end{itemize}

\subsection{Aliasing}

As to not break the memory model and compilers, the proposed function assumes that after a call to \tcode{reinterpret_string_cast} the range passed as parameter will only be accessed
through the return pointer, thereby sidesteping aliasing issues.

\section{Open Questions}

\begin{itemize}
\item Can it be constexpr?
\item Does it needs to be?
\item How much do we need to modify the object model and core wording?
\end{itemize}


\section{Alternative design}

It might be possible to add more rules to \tcode{reinterpret_cast}, to achieve similar results, however

\begin{itemize}
    \item We want this conversion to be as explicit as possible
    \item \tcode{reinterpret_cast} is not \tcode{constexpr},
    \item We think the magic function should work for span and string_view
\end{itemize}

\section{Wording}

The following wording is provided in an attempt to nerd-snipe a willing CWG expert into providing better wording.

\begin{quote}
\begin{addedblock}
\begin{codeblock}

// header <new> ?
namespace std {
    template <typename To, typename From>
    constexpr To reinterpret_string_cast(From from, std::size_t n) noexcept;
}

// header <span>
namespace std {
    template <typename To, typename From>
    constexpr std::span<To> reinterpret_string_cast(std::span<From> from);
}

// header string_view
namespace std {
    template <typename To, typename From>
    constexpr reinterpret_string_cast(std::basic_string_view<From> from) noexcept;
}

\end{codeblock}


\begin{itemdecl}
template <typename To, typename From>
constexpr To string_cast(From from, std::size_t n) noexcept;
\end{itemdecl}

\begin{itemdescr}
\constraints
\begin{itemize}
\item \tcode{From} is a pointer,
\item \tcode{To} is a pointer,
\item \tcode{std::remove_const_t<std::remove_pointer_t<From>>} is a character type,
\item \tcode{std::remove_const_t<std::remove_pointer_t<To>>} is a character type,
\item \tcode{sizeof(To) == sizeof(From)} is \tcode{true},
\item \tcode{From} and \tcode{To} have the same alignement.
\end{itemize}
\expects
\begin{itemize}
\item \range{from}{from + n} denotes a valid range
\item \range{from}{from + n} denotes a valid sequence of code units \emph{seq} such that
\begin{itemize}
    \item if \tcode{same_as<From, char8_t>} || \tcode{same_as<To, char8_t>} is \tcode{true}, \emph{seq} is a valid sequence of UTF-8 code units,
    \item otherwise if \tcode{same_as<From, char16_t>} || \tcode{same_as<To, char16_t>} is \tcode{true}, \emph{seq} is a valid sequence of UTF-16 code units,
    \item otherwise if \tcode{same_as<From, char32_t>} || \tcode{same_as<To, char32_t>} is \tcode{true}, \emph{seq} is a valid sequence of UTF-32 code units.
    \item otherwise the behavior is undefined.
\end{itemize}
\end{itemize}

\effects
This function behaves as if:

\begin{enumerate}
\item The range \range{from}{from + n} is copied to a temporary buffer \tcode{b} of \tcode{To} elements as if using \tcode{memcpy}
\item The destructor \tcode{\tilde From()} is called on each element of the range \range{from}{from + n}, ending their lifetime. Any pointer or reference to any address in the range \range{from}{from + n} is invalidated.
\item A new array \tcode{a} of type \tcode{To} and size \tcode{n} is constructed at address \tcode{p} as if \tcode{new (p) To[n]}
\item Each element of \tcode{b} is copied in \tcode{a} as if per \tcode{memcpy}

\returns
A pointer to the first element in \tcode{a}.
\end{enumerate}

\complexity Constant time

\begin{note}
After a call to this function accessing any object in the range \range{from}{from + n} by any pointer or reference other than the one returned by the function is undefined.
\end{note}

\end{itemdescr}


\end{addedblock}
\end{quote}

\section{Acknowledgments}

\section{References}
\renewcommand{\section}[2]{}%
\bibliographystyle{plain}
\bibliography{wg21}

\begin{thebibliography}{9}

    \bibitem[N4861]{N4861}
    Richard Smith
    \emph{Working Draft, Standard for Programming Language C++}\newline
    \url{https://wg21.link/N4861}

\end{thebibliography}
\end{document}
