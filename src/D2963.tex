% !TeX program = luatex
% !TEX encoding = UTF-8


\documentclass{wg21}

\title{Ordering of constraints involving fold expressions}
\docnumber{P2963R2}
\audience{EWG, CWG}
\author{Corentin Jabot}{corentin.jabot@gmail.com}
\usepackage[dvipsnames]{xcolor}


\begin{document}
\maketitle

\section{Abstract}

Fold expressions, which syntactically look deceptively like conjunctions/subjections for the purpose of constraint ordering are in fact atomic constraints
We propose rules for the normalization and ordering of fold expressions over \tcode{\&\&} and \tcode{||}.

\section{Revisions}

\subsection{R2}

\begin{itemize}
\item Wording improvements following CWG review in Tokyio. Notably, we added a description of how satisfaction is established.
\item Clarify that subsumption checking short-circuits. Add a design discussion.
\item A fold expression constraint can now only subsume another if they both have the same fold operator. This avoid inconsistent subsumption and checking results in the presence of empty packs.
\end{itemize}


\subsection{R1}

\begin{itemize}
\item Wording improvements: The previous version of this paper incorrectly looked at the size of the packs involved in the fold expressions (as it forced partial ordering to look at template arguments).
The current design does not look at the template argument/parameter mapping to establish subsumption of fold expressions.
\item A complete implementation of this proposal is available on Compiler Explorer. The implementation section was expanded.
\item Add an additional example.
\end{itemize}

\section{Motivation}

This paper is an offshoot of \paper{P2841R0} which described the issue with lack of subsumption for fold expressions.
This was first observed in a \href{http://cplusplus.github.io/concepts-ts/ts-active.html#28}{Concept TS issue}.

This question comes up ever so often on online boards and various chats.

\begin{itemize}
\item \href{https://stackoverflow.com/questions/34843745}{[StackOverflow] How are fold expressions used in the partial ordering of constraints?}
\item \href{https://stackoverflow.com/questions/58724459/}{[StackOverflow] How to implement the generalized form of std::same_as?}
\end{itemize}

In Urbana, core observed "We can't constrain variadic templates without fold-expressions" and almost folded (!) fold expressions into the concept TS.
The expectation that these features should interoperate well then appear long-standing.

\subsection{Subsumption and fold expressions over \tcode{\&\&} and \tcode{||}}

Consider:

\begin{colorblock}
template <class T> concept A = std::is_move_constructible_v<T>;
template <class T> concept B = std::is_copy_constructible_v<T>;
template <class T> concept C = A<T> && B<T>;

template <class... T>
requires (A<T> && ...)
void g(T...);

template <class... T>
requires (C<T> && ...)
void g(T...);
\end{colorblock}

We want to apply the subsumption rule to the normalized form of the requires clause (and its arguments). As of C++23, the above \tcode{g} is ambiguous.


This is useful when dealing with algebraic-type classes. Consider a concept constraining a (simplified) environment implementation via a type-indexed \tcode{std::tuple}. (In real code, the environment is a type-tag indexed map.)

\begin{colorblock}
template <typename X, typename... T>
concept environment_of = (... && requires (X& x) { { get<T>(x) } -> std::same_as<T&>; } );

auto f(sender auto&& s, environment_of<std::stop_token> auto env); // uses std::allocator
auto f(sender auto&& s, environment_of<std::stop_token, std::pmr::allocator> auto env); // uses given allocator
\end{colorblock}
Without the subsumption fixes to fold expressions, the above two overloads conflict, even though they should be partially ordered.\\

A similar example courtesy of Barry Revzin:\\


\begin{colorblock}
template <std::ranges::bidirectional_range R> void f(R&&); // #1
template <std::ranges::random_access_range R> void f(R&&); // #2

template <std::ranges::bidirectional_range... R> void g(R&&...); // #3
template <std::ranges::random_access_range... R> void g(R&&...); // #4
\end{colorblock}

\begin{center}
\begin{tabular}{l|l}
C++23 & This Paper\\ \hline

\begin{minipage}[t]{0.65\textwidth}
\begin{colorblock}
f(std::vector{1, 2, 3}); // Ok
@\textcolor{red}{g(std::vector\{1, 2, 3\})}@; // Error: call to 'g' is ambiguous
\end{colorblock}
\end{minipage}
&
\begin{minipage}[t]{0.4\textwidth}
\begin{colorblock}
f(std::vector{1, 2, 3}); // Ok, calls #2
g(std::vector{1, 2, 3}); // Ok, calls #4
\end{colorblock}
\end{minipage}
\\\\ \hline
\end{tabular}
\end{center}

\href{https://compiler-explorer.com/z/xojh8eo4x}{[Compiler Explorer Demo]}

\subsection{Impact on the standard}

This change makes ambiguous overload valid and should not break existing valid code.

\subsection{Implementabiliy}

This was implemented in Clang.
Importantly, what we propose does not affect compilers' ability to partially order functions by constraints without instantiating them, nor does it affect
the caching of subsumption, which is important to minimize the cost of concepts on compile time: The template arguments of the constraint expressions do not need to be observed
to establish subsumption.

An implementation does need to track whether an atomic constraint that contains an unexpanded pack was originally part of a and/or fold expression to properly
implement the subsumption rules (\tcode{\&\&} subsumes \tcode{||} \& \tcode{\&\&} and \tcode{||} subsumes \tcode{||}).

\subsection{Subsection with mixed fold operators}

Consider this example provided by Hubert Tong

\begin{colorblock}
template <typename ...T> struct Tuple { };
template <typename T> concept P = true;

template <typename T, typename U, typename V, typename X> struct A;


template <typename ...T, typename ...U, typename V, typename X>
requires P<X> || ((P<V> || P<U>) || ...)         // #A
void foo(A<Tuple<T ...>, Tuple<U ...>, V, X> *); // #1

template <typename ...T, typename ...U, typename V, typename X>
requires P<X> || ((P<V> && P<T>) && ...)         // #B
void foo(A<Tuple<T ...>, Tuple<U ...>, V, X> *); // #2


void bar(A<Tuple<int>, Tuple<>, int, int> *p) { foo(p); }
\end{colorblock}

In this example, under the rule proposed in R1, of this paper, \#A subsumed \#B, and so \#1 would have been be a better choice.
However here, \tcode{U} is empty. So A's constraints are equivalent to just \tcode{P<X>} which make B more constrained.

To avoid inconsistencies between constraint checking and subsumption, a fold expression can only subsume another if they both have the same fold operator
(they are both folding over \tcode{\&\&} or they are both folding over \tcode{||}).

\subsection{Short circuiting}

To be consistent with conjunction constraint and disjection constraints, we propose that fold expanded constrait short circuit (both their evaluation and substitution).



\subsection{What this paper is not}

When the pattern of the \tcode{fold-expressions} is a `concept` template parameter, this paper does not apply. In that case, we need different rules which are covered in \paper{P2841R0} along with the rest of the "concept template parameter" feature (specifically, for concept patterns we need to decompose each concept into its constituent atomic constraints and produce a fully decomposed sequence of conjunction/disjunction)


\subsection{Design and wording strategy}

To simplify the wording, we first normalize fold expressions to extract the non-pack expression of binary folds into its own normalized form,
and transform \tcode{(... \&\& A)} into \tcode{(A \&\& ...)} as they are semantically identical for the purpose of subsumption.
We then are left with either \tcode{(A \&\& ...)} or \tcode{(A || ...)}, and for packs of the same size, the rules of subsumptions are the same as for that of atomic constraints.

\section{Wording}

\rSec2[temp.constr.constr]{Constraints}

\rSec3[temp.constr.constr.general]{General}

\indextext{satisfy|see{constraint, satisfaction}}%

\pnum
A \defn{constraint} is a sequence of logical operations and
operands that specifies requirements on template arguments.
The operands of a logical operation are constraints.
There are \changed{three}{four} different kinds of constraints:
\begin{itemize}
    \item conjunctions \iref{temp.constr.op},
    \item disjunctions \iref{temp.constr.op}, \removed{and}
    \item atomic constraints \iref{temp.constr.atomic}\added{, and}
\begin{addedblock}
    \item fold expanded constraints [temp.constr.fold].
\end{addedblock}
\end{itemize}

\pnum
In order for a constrained template to be instantiated \iref{temp.spec},
its associated constraints \iref{temp.constr.decl}
shall be satisfied as described in the following subclauses.
\begin{note}
    Forming the name of a specialization of
    a class template,
    a variable template, or
    an alias template \iref{temp.names}
    requires the satisfaction of its constraints.
    Overload resolution \iref{over.match.viable}
    requires the satisfaction of constraints
    on functions and function templates.
\end{note}

\rSec3[temp.constr.op]{Logical operations}

\pnum
There are two binary logical operations on constraints: conjunction
and disjunction.
\begin{note}
    These logical operations have no corresponding \Cpp{} syntax.
    For the purpose of exposition, conjunction is spelled
    using the symbol $\land$ and disjunction is spelled using the
    symbol $\lor$.
    The operands of these operations are called the left
    and right operands. In the constraint $A \land B$,
    $A$ is the left operand, and $B$ is the right operand.
\end{note}

\pnum
A \defn{conjunction} is a constraint taking two
operands.
To determine if a conjunction is
\defnx{satisfied}{constraint!satisfaction!conjunction},
the satisfaction of
the first operand is checked.
If that is not satisfied, the conjunction is not satisfied.
Otherwise, the conjunction is satisfied if and only if the second
operand is satisfied.

\pnum
A \defn{disjunction} is a constraint taking two
operands.
%
To determine if a disjunction is
\defnx{satisfied}{constraint!satisfaction!disjunction},
the satisfaction of
the first operand is checked.
If that is satisfied, the disjunction is satisfied.
Otherwise, the disjunction is satisfied if and only if the second
operand is satisfied.

\pnum
\begin{example}
    \begin{codeblock}
        template<typename T>
        constexpr bool get_value() { return T::value; }

        template<typename T>
        requires (sizeof(T) > 1) && (get_value<T>())
        void f(T);      // has associated constraint \tcode{sizeof(T) > 1 $\land$ get_value<T>()}

        void f(int);

        f('a'); // OK, calls \tcode{f(int)}
    \end{codeblock}
    In the satisfaction of the associated constraints\iref{temp.constr.decl}
    of \tcode{f}, the constraint \tcode{sizeof(char) > 1} is not satisfied;
    the second operand is not checked for satisfaction.
\end{example}

\pnum
\begin{note}
A logical negation expression \iref{expr.unary.op} is an atomic constraint;
the negation operator is not treated as a logical operation on constraints.
As a result, distinct negation \grammarterm{constraint-expression}{s}
that are equivalent under~\ref{temp.over.link}
do not subsume one another under~\ref{temp.constr.order}.
Furthermore, if substitution to determine
whether an atomic constraint is satisfied \iref{temp.constr.atomic}
encounters a substitution failure, the constraint is not satisfied,
regardless of the presence of a negation operator.
\begin{example}
\begin{codeblock}
template <class T> concept sad = false;

template <class T> int f1(T) requires (!sad<T>);
template <class T> int f1(T) requires (!sad<T>) && true;
int i1 = f1(42);        // ambiguous, \tcode{!sad<T>} atomic constraint expressions \iref{temp.constr.atomic}
// are not formed from the same \grammarterm{expression}

template <class T> concept not_sad = !sad<T>;
template <class T> int f2(T) requires not_sad<T>;
template <class T> int f2(T) requires not_sad<T> && true;
int i2 = f2(42);        // OK, \tcode{!sad<T>} atomic constraint expressions both come from \tcode{not_sad}

template <class T> int f3(T) requires (!sad<typename T::type>);
int i3 = f3(42);        // error: associated constraints not satisfied due to substitution failure

template <class T> concept sad_nested_type = sad<typename T::type>;
template <class T> int f4(T) requires (!sad_nested_type<T>);
int i4 = f4(42);        // OK, substitution failure contained within \tcode{sad_nested_type}
\end{codeblock}
    Here,
    \tcode{requires (!sad<typename T::type>)} requires
    that there is a nested \tcode{type} that is not \tcode{sad},
    whereas
    \tcode{requires (!sad_nested_type<T>)} requires
    that there is no \tcode{sad} nested \tcode{type}.
\end{example}
\end{note}

\rSec3[temp.constr.atomic]{Atomic constraints}

\pnum
An \defnadj{atomic}{constraint} is formed from
an expression \tcode{E}
and a mapping from the template parameters
that appear within \tcode{E} to
template arguments that are formed via substitution during constraint normalization
in the declaration of a constrained entity (and, therefore, can involve the
unsubstituted template parameters of the constrained entity),
called the \defn{parameter mapping}\iref{temp.constr.decl}.
\begin{note}
    Atomic constraints are formed by constraint normalization\iref{temp.constr.normal}.
    \tcode{E} is never a logical and expression\ref{expr.log.and}
    nor a logical or expression\ref{expr.log.or}.
\end{note}

\pnum
Two atomic constraints, $e_1$ and $e_2$, are
\indextext{identical!atomic constraints|see{atomic constraint, identical}}%
\defnx{identical}{atomic constraint!identical}
if they are formed from the same appearance of the same
\grammarterm{expression}
and if, given a hypothetical template $A$
whose \grammarterm{template-parameter-list} consists of
\grammarterm{template-parameter}s corresponding and equivalent\iref{temp.over.link} to
those mapped by the parameter mappings of the expression,
a \grammarterm{template-id} naming $A$
whose \grammarterm{template-argument}s are
the targets of the parameter mapping of $e_1$
is the same\iref{temp.type} as
a \grammarterm{template-id} naming $A$
whose \grammarterm{template-argument}s are
the targets of the parameter mapping of $e_2$.
\begin{note}
    The comparison of parameter mappings of atomic constraints
    operates in a manner similar to that of declaration matching
    with alias template substitution\iref{temp.alias}.
    \begin{example}
        \begin{codeblock}
            template <unsigned N> constexpr bool Atomic = true;
            template <unsigned N> concept C = Atomic<N>;
            template <unsigned N> concept Add1 = C<N + 1>;
            template <unsigned N> concept AddOne = C<N + 1>;
            template <unsigned M> void f()
            requires Add1<2 * M>;
            template <unsigned M> int f()
            requires AddOne<2 * M> && true;

            int x = f<0>();     // OK, the atomic constraints from concept \tcode{C} in both \tcode{f}s are \tcode{Atomic<N>}
            // with mapping similar to $\tcode{N} \mapsto \tcode{2 * M + 1}$

            template <unsigned N> struct WrapN;
            template <unsigned N> using Add1Ty = WrapN<N + 1>;
            template <unsigned N> using AddOneTy = WrapN<N + 1>;
            template <unsigned M> void g(Add1Ty<2 * M> *);
            template <unsigned M> void g(AddOneTy<2 * M> *);

            void h() {
                g<0>(nullptr);    // OK, there is only one \tcode{g}
            }
        \end{codeblock}
    \end{example}
    As specified in \ref{temp.over.link},
    if the validity or meaning of the program depends on
    whether two constructs are equivalent, and
    they are functionally equivalent but not equivalent,
    the program is ill-formed, no diagnostic required.
    \begin{example}
        \begin{codeblock}
            template <unsigned N> void f2()
            requires Add1<2 * N>;
            template <unsigned N> int f2()
            requires Add1<N * 2> && true;
            void h2() {
                f2<0>();          // ill-formed, no diagnostic required:
                // requires determination of subsumption between atomic constraints that are
                // functionally equivalent but not equivalent
            }
        \end{codeblock}
    \end{example}
\end{note}

\pnum
To determine if an atomic constraint is
\defnx{satisfied}{constraint!satisfaction!atomic},
the parameter mapping and template arguments are
first substituted into its expression.
If substitution results in an invalid type or expression
in the immediate context of the atomic constraint\iref{temp.deduct.general},
the constraint is not satisfied.
Otherwise, the lvalue-to-rvalue conversion\iref{conv.lval}
is performed if necessary,
and \tcode{E} shall be a constant expression of type \tcode{bool}.
The constraint is satisfied if and only if evaluation of \tcode{E}
results in \tcode{true}.
If, at different points in the program, the satisfaction result is different
for identical atomic constraints and template arguments,
the program is ill-formed, no diagnostic required.
\begin{example}
    \begin{codeblock}
        template<typename T> concept C =
        sizeof(T) == 4 && !true;      // requires atomic constraints \tcode{sizeof(T) == 4} and \tcode{!true}

        template<typename T> struct S {
            constexpr operator bool() const { return true; }
        };

        template<typename T> requires (S<T>{})
        void f(T);                      // \#1
        void f(int);                    // \#2

        void g() {
            f(0);                         // error: expression \tcode{S<int>\{\}} does not have type \tcode{bool}
        }                               // while checking satisfaction of deduced arguments of \#1;
        // call is ill-formed even though \#2 is a better match
    \end{codeblock}
\end{example}

\begin{addedblock}
\rSec3[temp.constr.fold]{Fold expanded constraint}

A \defn{fold expanded constraint} is formed from a constraint and a \defn{fold-operator} which can either be \tcode{\&\&} or \tcode{||}.

A fold expanded constraint whose \defn{fold-operator} is \tcode{\&\&} is a \defn{fold expanded conjunction constraint}.

A fold expanded constraint whose \defn{fold-operator} is \tcode{||} is a \defn{fold expanded disjunction constraint}.

A fold expanded constraint is a pack expansion. Let $N$ be the number of elements in the pack expansion that would result from its instantiation [temp.variadic].

A fold expanded conjunction constraint is satisfied if $N$ equals $0$ or if for each $0 \le i < N$ in increasing order, its constraint is satisfied when replacing each pack expansion parameter with the corresponding $i$th element. No substitution takes place for any $i$ greater than the smallest $i$ for which the constraint is not satisfied.


A fold expanded disjunction constraint is satisfied, if $N$ is not $0$ and for any $0 \le i < N$ in increasing order its constraint is satisfied when replacing each pack expansion parameter with the corresponding $i$th element. No substitution takes place for any $i$ greater than the smallest $i$ for which the constraint is satisfied.


\end{addedblock}




\ednote{[...]}


\rSec2[temp.constr.normal]{Constraint normalization}
\indextext{constraint!normalization|(}%

\pnum
The \defnx{normal form}{normal form!constraint} of an \grammarterm{expression} \tcode{E} is
a constraint\iref{temp.constr.constr} that is defined as follows:
%
\begin{itemize}
\item
The normal form of an expression \tcode{( E )} is
the normal form of \tcode{E}.

\item
The normal form of an expression \tcode{E1 || E2} is
the disjunction\iref{temp.constr.op} of
the normal forms of \tcode{E1} and \tcode{E2}.

\item
The normal form of an expression \tcode{E1 \&\& E2}
is the conjunction of
the normal forms of \tcode{E1} and \tcode{E2}.

\item
The normal form of a concept-id \tcode{C<A$_1$, A$_2$, ..., A$_n$>}
is the normal form of the \grammarterm{constraint-expression} of \tcode{C},
after substituting \tcode{A$_1$, A$_2$, ..., A$_n$} for
\tcode{C}{'s} respective template parameters in the
parameter mappings in each atomic constraint.
If any such substitution results in an invalid type or expression,
the program is ill-formed; no diagnostic is required.
\begin{example}
\begin{codeblock}
    template<typename T> concept A = T::value || true;
    template<typename U> concept B = A<U*>;
    template<typename V> concept C = B<V&>;
\end{codeblock}
Normalization of \tcode{B}{'s} \grammarterm{constraint-expression}
is valid and results in
\tcode{T::value} (with the mapping $\tcode{T} \mapsto \tcode{U*}$)
$\lor$
\tcode{true} (with an empty mapping),
despite the expression \tcode{T::value} being ill-formed
for a pointer type \tcode{T}.
Normalization of \tcode{C}{'s} \grammarterm{constraint-expression}
results in the program being ill-formed,
because it would form the invalid type \tcode{V\&*}
in the parameter mapping.
\end{example}

\begin{addedblock}
\item For a \grammarterm{fold-operator}  [expr.prim.fold] that is either \tcode{\&\&} or \tcode{||}, the normal form of an expression \tcode{( ... \grammarterm{fold-operator} E )} is the normal form of \tcode{( E \grammarterm{fold-operator}...)}.
\item The normal form of an expression \tcode{( E1 \grammarterm{fold-operator}  ... \grammarterm{fold-operator}  E2 )} is the the normal form of
\begin{itemize}
    \item \tcode{(E1 \grammarterm{fold-operator}...) \grammarterm{fold-operator}  E2} if \tcode{E1} contains an unexpanded pack, or
    \item \tcode{E1 \grammarterm{fold-operator} (E2 \grammarterm{fold-operator}...)} otherwise.
\end{itemize}
\item The normal form of \tcode{(E \&\& ...)} is a fold expanded conjunction constraint \mbox{[temp.constr.fold]} whose constraint is the normal form of \tcode{E}.
\item The normal form of \tcode{(E || ...)} is a fold expanded disjunction constraint whose constraint is the normal form of \tcode{E}.
\end{addedblock}

\item
The normal form of any other expression \tcode{E} is
the atomic constraint
whose expression is \tcode{E} and
whose parameter mapping is the identity mapping.
\end{itemize}

\rSec2[temp.constr.order]{Partial ordering by constraints}
\indextext{subsume|see{constraint, subsumption}}


\pnum
A constraint $P$ \defnx{subsumes}{constraint!subsumption} a constraint $Q$
if and only if,
for every disjunctive clause $P_i$
in the disjunctive normal form
of $P$, $P_i$ subsumes every conjunctive clause $Q_j$
in the conjunctive normal form of $Q$, where
\begin{itemize}
    \item
    a disjunctive clause $P_i$ subsumes a conjunctive clause $Q_j$ if and only
    if there exists an atomic constraint $P_{ia}$ in $P_i$ for which there exists
    an atomic constraint $Q_{jb}$ in $Q_j$ such that $P_{ia}$ subsumes $Q_{jb}$, and

    \item an atomic constraint $A$ subsumes another atomic constraint
    $B$ if and only if $A$ and $B$ are identical using the
    rules described in [temp.constr.atomic].

   \begin{addedblock}
    \item a fold expanded constraint $A$ subsumes another fold expanded constraint
    $B$ if both $A$ and $B$ have the same \grammarterm{fold-operator} and the constraint of $A$ subsumes that of $B$.
    \end{addedblock}
\end{itemize}
%
\begin{example}
    Let $A$ and $B$ be atomic constraints [temp.constr.atomic].
    %
    The constraint $A \land B$ subsumes $A$, but $A$ does not subsume $A \land B$.
    %
    The constraint $A$ subsumes $A \lor B$, but $A \lor B$ does not subsume $A$.
    %
    Also note that every constraint subsumes itself.
\end{example}


\section{Acknowledgments}

Thanks to Robert Haberlach for creating the original  \href{http://cplusplus.github.io/concepts-ts/ts-active.html#28}{Concept TS issue}.

Thanks to Jens Mauer and Barry Revzin for their input on the wording.

\bibliographystyle{plain}
\bibliography{wg21}


\renewcommand{\section}[2]{}%

\begin{thebibliography}{9}

\bibitem[N4958]{N4958}
Thomas Köppe
\emph{Working Draft, Standard for Programming Language C++}\newline
\url{https://wg21.link/N4958}


\end{thebibliography}
\end{document}
