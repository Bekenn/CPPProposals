% !TeX document-id = {9322a846-f757-4574-9231-a2e85c743b21}
% !TeX program = luatex
% !TEX encoding = UTF-8


\RequirePackage{luatex85}%
\documentclass{wg21}



\title{Thread attributes}
\docnumber{P2019R4}
\audience{LEWG}
\author{Corentin Jabot}{corentin.jabot@gmail.com}

\begin{document}

\maketitle

\setlength{\arrayrulewidth}{0.2mm}
\setlength{\tabcolsep}{8pt}
\renewcommand{\arraystretch}{1.}

\def\changemargin#1#2{\list{}{\rightmargin#2\leftmargin#1}\item[]}
\let\endchangemargin=\endlist
\setlength\extrarowheight{5pt}


\section{Abstract}

We propose a way to set a thread stack size and name before the start of its execution, both of which are,
as we demonstrate, current practices in many domains.

The absence of these features make \tcode{std::thread} and \tcode{std::jthread} unfit or unsatisfactory for many use cases.

\section{Revisions}

\subsection{R4}

During the review of \paper{P2019R3}, LEWG did not like the proposed \tcode{make_with_attributes} and said they would prefer a constructor
that allows both attributes and arguments, such that attributes could be prepended to existing constructor calls of \tcode{std::thread}, without further modifications.
The \tcode{thread_attribute} base class was removed as LEWG felt that we should not prescribe a way for implementation to detect that a type is a thread attribute.

\subsection{R3}

We propose a \tcode{make_with_attributes} static factory function
instead of an additional constructor.
This is because LEWG expressed a slight preference for that design durting the 24th January 23 telecon.
The new design, and the names should avoid confusion with the existing constructor, and make the intent clearer.

The design section add a note on the viability of NTTP and getter and setters as these questions came up during the meeting.

Fix some typos.

\subsection{R2}

Wording improvements.

\subsection{R1}

Rework the design to take a list of attributes as parameters, instead of a single type.
This is done to alleviate ABI and extensibility concerns raised by SG1.

SG1 suggested exploring a property-based design, but this ship doesn't seem to still be sailing,
So instead there is still a constructor interface, but with a separate type for each attribute.

Both \tcode{thread_name} and \tcode{thread_stack_size} are preserved in this revision despite the
mixed feelings SG1 expressed for the \tcode{thread_stack_size} attribute (see polls section).

\section{Example}

The following code illustrates the totality of the proposed additions:

\begin{colorblock}
void f(int);
int main() {
    std::jthread thread(std::thread_name("Worker"), std::thread_stack_size(512*1024), f, 42);
    return 0;
}
\end{colorblock}


This code suggests a thread name as well as a stack size
the implementation should use when creating a new thread of execution.

Achieving the same result in C++20 requires duplicating the entire \tcode{std::thread}
class, which would be difficult to fit in a Tony table.

Here is how to set the name and stack size of a thread on most POSIX implementation

\begin{colorblock}
int libcpp_thread_create(libcpp_thread_t *t, void *(*func)(void *),
                        void *arg,
                        size_t stack_size,
                        const libcpp_threadname_char_t* name)
{
    int res = 0;
    if(stack_size != 0) {
        pthread_attr_t attr;
        res = pthread_attr_init(&attr);
        if (res != 0) {
            return res;
        }
         // Ignore errors
        pthread_attr_setstacksize(&attr, stack_size);
        res = pthread_create(t, &attr, func, arg);
        // Ignore errors
        pthread_attr_destroy(&attr);
    }
    else {
        res = pthread_create(t, 0, func, arg);
    }
    if (res == 0) {
        // Ignore errors
        pthread_setname_np(*t, name);
    }
    return res;
}
\end{colorblock}

\section{Previous Polls}

\begin{quoteblock}
SG1, July 2020
We want the ability to provide attributes to thread (full threads: thread, jthread) constructors, even if we can't specify their semantics fully.

\begin{tabular}{|c|c|}
\hline
Yay & Nay\\
\hline
12 & 1 \\
\hline
\end{tabular}
\end{quoteblock}
\begin{quoteblock}
The name attribute.

\begin{tabular}{|c|c|}
\hline
Yay & Nay\\
\hline
14 & 1 \\
\hline
\end{tabular}
\end{quoteblock}
\begin{quoteblock}
The stack size attribute.

\begin{tabular}{|c|c|}
\hline
Yay & Nay\\
\hline
7 & 7 \\
\hline
\end{tabular}
\end{quoteblock}


\section{Motivation}

\subsection{Threads have a name}

Most operating systems, including real-time operating systems for embedded platforms, provide a way to name threads.

Names of threads are usually stored in the control structure the kernel uses to manage threads or tasks.

The name can be used by:
\begin{itemize}
    \item Debuggers such as GDB, LLDB, WinDBG, and IDEs using these tools
    \item Platforms and third-party crash dump and trace reporting tools
    \item System task and process monitors
    \item Other profiling tracing and diagnostic tools
    \item Windows Performance Analyzer and ETW tracing
\end{itemize}

The \href{https://docs.microsoft.com/en-us/visualstudio/debugger/how-to-set-a-thread-name-in-native-code?view=vs-2019}{Visual Studio documentation for SetThreadDescription} explains:

\begin{quoteblock}
Thread naming is possible in any edition of Visual Studio. Thread naming is useful for identifying threads of interest in the Threads window when debugging a running process. Having recognizably-named threads can also be helpful when performing post-mortem debugging via crash dump inspection and when analyzing performance captures using various tools.
\end{quoteblock}

This non-exhaustive table shows that most platforms do in fact provide a way to set and often query a thread name.

\begin{center}
\begin{scriptsize}
% \makesavenoteenv{tabular}
\begin{tabular}{ |c|c|c|c| }

    %\hline
    %\multicolumn{5}{|c|}{System APIS for Naming Threads} \\
    \hline
    Platform &  At Creation & After  & Query \\
    \hline
    Linux & \multicolumn{2}{c|}{\tcode{pthread_setname_np}\footnote{GLIBC 2.12+, MUSL}} & \tcode{pthread_getname_np} \\
    QNX & \multicolumn{2}{c|}{\tcode{pthread_setname_np}} & \tcode{pthread_getname_np} \\
    NetBSD & \multicolumn{2}{c|}{\tcode{pthread_setname_np}} & \tcode{pthread_getname_np} \\
    \hline
    Win32 & & \tcode{SetThreadDescription}\footnote{Since Windows 10 1607 - In older versions, a name can be set only when a debugger is attached, by throwing an exception from the calling thread. See  \href{https://stackoverflow.com/a/59490438/877556}{Windows Documentation} and  \href{https://randomascii.wordpress.com/2015/10/26/thread-naming-in-windows-time-for-something-better/}{this article by Bruce Dawson}} & \tcode{GetThreadDescription} \\
    \hline
    Darwin & & \tcode{pthread_setname_np}\footnote{Can only be called from the new thread} & \tcode{pthread_getname_np} \\
    \hline
    Fuchsia &  \tcode{zx_thread_create} & & \\
    \hline
    Android & \tcode{JavaVMAttachArgs}\footnote{See \url{https://stackoverflow.com/a/59490438/877556}} & &  \\
    \hline
    FreeBSD & \multicolumn{2}{c|}{\tcode{pthread_setname_np}} &  \\
    OpenBSD & \multicolumn{2}{c|}{\tcode{pthread_setname_np}} &  \\
    \hline
    RTEMS \footnote{since 2017} & \tcode{pthread_setname_np}  & \tcode{pthread_setname_np}  & \tcode{pthread_getname_np}   \\
    \hline
    FreeRTOS & \tcode{xTaskCreate} & & \tcode{pcTaskGetName} \\
    \hline
    VxWorks &  \tcode{taskSpawn} & & \\
    \hline
    eCos & \tcode{cyg_thread_create} & & \\
    \hline
    Plan 9 &\tcode{threadsetname}\footnote{Can only be called from the new thread}  & \tcode{threadsetname}\footnote{Can only be called from the new thread}&\\
    \hline
    Haiku & \tcode{spawn_thread} & \tcode{rename_thread} &\tcode{get_thread_info} \\
    \hline
    Keil RTX & \tcode{osThreadNew} & & \tcode{osThreadGetName}  \\
    \hline
    WebAssembly & & &  \\
    \hline
\end{tabular}
\end{scriptsize}
\end{center}

Web assembly was the only platform for which we didn't find a way to set a thread name.

A cursory review of programming language reveals that at least the following languages/environments provide a way to set thread names:

\href{https://doc.rust-lang.org/std/thread/struct.Builder.html}{Rust}, \href{https://docs.python.org/3/library/threading.html#thread-objects}{Python}, \href{https://dlang.org/phobos/core_thread_osthread.html#.Thread.name}{D}
, \href{https://docs.microsoft.com/en-us/dotnet/api/system.threading.thread.name?view=netframework-4.8}{C\#}
, \href{https://docs.oracle.com/javase/7/docs/api/java/lang/Thread.html}{Java}
, \href{https://docs.raku.org/type/Thread#method_name}{Raku}
, Swift
, \href{https://doc.qt.io/qt-5/qthread.html#managing-threads}{Qt}
, \href{https://github.com/facebook/folly/blob/master/folly/system/ThreadName.h}{Folly}

We also found \href{https://stackoverflow.com/questions/10121560/stdthread-naming-your-thread}{multiple} questions \href{https://stackoverflow.com/questions/57477053/how-to-set-custom-name-of-this-thread}{related} to setting \href{https://stackoverflow.com/questions/16486937/name-a-thread-created-by-beginthread-in-c}{name thread}
on StackOverflow.

Thread names are also the object of a C proposal \cite{N2419}

All of that illustrates that giving a name to os threads is standard practice.


\subsection{Threads have a stack size}

In the following, non-exhaustive table, we observe that almost all APIs across a wide range
of environments expose a stack size that can either be queried or set.
The necessity for such a parameter results from the unfortunate non-existence of infinite tape.

A stack size refers to the number of bytes an application can use to store variables of static storage duration and other implementation-defined information
necessary to store the sequence of stack entries making the stack.

Because of that, all implementations which let a stack size be set, do so during the creation of the thread
of execution.

We observe fewer variations of APIs across platforms (compared to names) as the parameter is a simple integer
that can be no greater than the total system memory.

\tcode{pthread_attr_setstacksize} is part of the POSIX specification since Issue 5 (1997).
However, platforms vary in the minimum and maximum stack sizes supported.


\begin{changemargin}{-1.7cm}{0.5cm}
\begin{center}
\begin{footnotesize}
\begin{tabular}{ |c|c|c| }

    %\hline
    %\multicolumn{5}{|c|}{System APIS for Naming Threads} \\
    \hline
    Platform &  At Creation & Query \\
    \hline
    Linux & \tcode{pthread_attr_setstacksize} & \tcode{pthread_attr_getstacksize} \\
    QNX & \tcode{pthread_attr_setstacksize} & \tcode{pthread_attr_getst\textbf{}acksize} \\
    \hline
    Win32 & \tcode{CreateThread} &    \\
    \hline
    Darwin & \tcode{pthread_attr_setstacksize} & \tcode{pthread_attr_getstacksize}  \\
    \hline
    Fuchsia & &  \\
    \hline
    Android & \tcode{pthread_attr_setstacksize} & \tcode{pthread_attr_getstacksize} \\
    \hline
    FreeBSD & \tcode{pthread_attr_setstacksize} & \tcode{pthread_attr_getstacksize}  \\
    OpenBSD & \tcode{pthread_attr_setstacksize} & \tcode{pthread_attr_getstacksize}  \\
    NetBSD & \tcode{pthread_attr_setstacksize} & \tcode{pthread_attr_getstacksize}  \\
    \hline
    RTEMS & \tcode{pthread_attr_setstacksize} & \tcode{pthread_attr_getstacksize}  \\
    \hline
    FreeRTOS & \tcode{xTaskCreate} &  \\
    \hline
    VxWorks &  \tcode{taskSpawn} &  \\
    \hline
    eCos & \tcode{cyg_thread_create} &  \\
    \hline
    Plan 9 & \tcode{threadcreate}  &  \\
    \hline
    Haiku &  &\tcode{get_thread_info} \\
    \hline
    Keil RTX & & \tcode{osThreadGetStackSize}  \\
    \hline
    WebAssembly & & \tcode{pthread_attr_getstacksize}  \\
    \hline
\end{tabular}
\end{footnotesize}
\end{center}
\end{changemargin}

We observe that \href{https://docs.oracle.com/javase/7/docs/api/java/lang/Thread.html#Thread()}{Java}, \href{https://doc.rust-lang.org/std/thread/struct.Builder.html}{Rust},
\href{https://docs.python.org/3/library/threading.html}{Python}, \href{https://docs.microsoft.com/en-us/dotnet/api/system.threading.thread.-ctor?view=netframework-4.8#System_Threading_Thread__ctor_System_Threading_ParameterizedThreadStart_System_Int32_}{C\#}, \href{https://downloads.haskell.org/~ghc/latest/docs/html/users_guide/runtime_control.html}{Haskell} (through a compile-time parameter), \href{https://dlang.org/phobos/core_thread_osthread.htm}{D}, \href{https://perldoc.perl.org/threads.html#THREAD-STACK-SIZE}{Perl}, Swift,
\href{https://www.boost.org/doc/libs/1_72_0/doc/html/thread/thread_management.html#thread.thread_management.tutorial.attributes}{Boost},  \href{https://doc.qt.io/qt-5/qthread.html#setStackSize}{Qt}
support constructing threads with a stack size.

There are many reasons why a program may need to set a stack size:
\begin{itemize}
\item Ensuring a consistent stack size across platforms for portability and reliability as some applications are designed to be run with a specific amount of stack size.

More generally, such inconsistencies are a source of bugs and expensive testing.

\item Ability to use less than the platform default (usually 1MB on windows, 2MB on many Unixes), which, when not used is a waste (on systems without virtual memory), especially if a large number of threads is started.
\item Some applications will set a larger stack trace for the main thread, which is then inherited by spawn threads, which might be undesirable.
\item Some applications, notably big games, and other large applications will require a stack larger than the default.
\end{itemize}

\section{Motivation for standardization}

Libc++ \tcode{std::thread} implementation is (very approximately) 1000 lines of code.
Because stack size needs to be set before thread creation, an application wishing to use a non-default stack size has to duplicate that effort.

We found threads classes supporting names and stack size in many open source projects, including POCO, Chromium, Firefox, LLVM, Bloomberg Basic Development Environment, Folly, Intel TBB, Tensorflow...
In many cases, these classes are very similar to \tcode{std::thread}, except they support a stack size.

Like thread names, adding this support to \tcode{std::thread} would be standardizing existing practices.

People working on AAA games told us that the lack of stack size support prevented them to use \tcode{std::thread}, which therefore fails to be a vocabulary type.
As such this proposal is more about rounding an existing feature rather than proposing a new one.

\section{FAQ}

In which we try to answer all the questions we heard about this proposal

\subsection{What about queries?}

We observe that
\begin{itemize}
\item It is rarely useful to query the stack size (except to assert that it is in a range acceptable to the application).\\
Querying the stack size could be done by storing a \tcode{std::size_t} within the \tcode{std::thread} instance, which is rather cheap,
but we still don't think it is worth it.
\item It is rarely useful to query the thread name, nor is there a portable way to do so (some platforms have API to do that). Use cases for querying a thread name include printing stack traces \cite{P0881R5}\\
\item It is also more difficult to design a query API for the name that would not pull in \tcode{<string>}
\item While less convenient facility, it is at least possible to query available properties after creation from \tcode{native_handle}.
\end{itemize}


\subsection{Threads should have names??? What next, \tcode{mutex} should have a name? \tcode{vector}?}

Naming threads is standard practice across many operating systems and environments.
This proposal merely proposes to expose this widely available and used system feature.
We observe that it is common for threads to have names as processes do.

Windows indeed has the concept of named mutexes which are used to share mutexes across processes.
However, \tcode{std::mutex} is not intended to be shared across processes and as such does not need
a name nor should it have one.
A quick review of platforms reveals that it is not standard practice to use a mutex across processes (many UNIX systems rely on lock files).

\tcode{std::vector} and other C++ objects are not visible outside the program, except by debuggers which can identify them
by their identifiers. Giving them a name would make little sense.


\subsection{I don't need that and don't want to pay for it}

None of the proposed attributes is stored in the thread object nor in any object associated with the thread or its associated \tcode{thread} object.
The proposed \tcode{thread::attribute} object can be destroyed after the thread creation.
The behavior of preexisting constructors remains unchanged.

On many implementations, including Linux, the space for the thread name is allocated regardless of whether it is used or not.

\subsection{It's an ABI break ???}

No.
Because none of the attributes is stored in the thread or its associated \tcode{std::thread} object, the ABI is not changed.
We proposed adding a single template constructor.


\subsection{We cannot speak about stack size in the standard?}
There exist a POSIX function that makes the wording more palatable.
Setting a stack size insufficient for the correct execution of a well-formed program isn't different
than if the default stack size is insufficient ([intro.compliance])

\subsection{This is not something that the committee has the bandwidth to deal with?}

We spent resources standardizing 2 (!) thread classes, which are not used in many cases.
This proposal will help more people use \tcode{std::thread}.

The author of this proposal is aware of the limited resources of the committee, and that informed the design.
The cost of re-implementing classes similar to \tcode{std::thread} is great for the industry.

\subsection{I cannot implement that on my platform?}
Here is a conforming minimal implementation

\begin{colorblock}
namespace std {

struct thread_name  {
   constexpr thread_name(std::string_view str) noexcept {};
   constexpr thread_name(std::u8string_view str) noexcept {};
};

struct thread_stack_size {
    constexpr thread_stack_size(std::size_t) noexcept {}
};

\end{colorblock}

\subsection{This belongs in a library?}

Because the proposed attributes may need to be set during the thread creation, a library would have no choice but
to reimplement all of \tcode{std::thread}.
Besides the cost of doing that implementation, it poses composability challenges (cannot put a \tcode{custom_thread} in a \tcode{std::thread_poool} for example)

\subsection{What about GPU threads?}

While \tcode{std::thread} has no mechanism to specify an execution context, an implementation that wishes
to use \tcode{std::thread} on a GPU or other hardware could ignore all attributes or the ones not relevant on their platform.

\subsection{What about other properties}

Depending on platforms, threads may have

\begin{itemize}
\item A CPU affinity such that they are only executed on a given CPU or set of CPU
\item A CPU preference such that they preferentially executed on a given CPU
\item A priority compared to the thread in the process
\item A priority compared to threads in the system
\end{itemize}

The meaning of each value and parameter has more variation across implementations,
as it is tied to the scheduler or the system.

It is also less generally useful and mostly used in HPC and embedded platforms, where there is the greatest variety of implementation.

As such, thread priorities and other properties are not proposed in this paper.
However, the API is designed to allow adding support for more properties in the future.

Note that priorities can often be changed after the thread creation making it easier for third
parties libraries to support thread priorities.


\section{Proposed design}

\subsection{Constraints}

\begin{itemize}
\item Some environments do not support naming threads.
\item Thread names can be either narrow encoded or, in the case of win32, Unicode (UTF-16) encoded
\item There is a platform-specific limit on thread name length (15(+1) on Linux, 32K on windows)
\item All platforms expect names to be null-terminated.
\item Some platforms set the name during the thread creation, while on Darwin (and plan 9) it can only be set in the thread in which the name is set.
\item The stack size is always set prior to the thread creation.
\item Platforms have minimum and maximum stack sizes that are not always possible to expose
\item Implementation may allocate more stack size than requested (it is usually aligned on a memory page)
\item Implementation may ignore stack size requests
\item On some platforms, the thread stack size is not configurable.
\item On some platforms, the thread stack size is not query-able.
\item Defining these features in terms of wording may be challenging.
\item \textbf{Users who do not care about these features should not have to pay for it}
\end{itemize}

\subsection{Design}

We propose to modify the constructor of \tcode{std::thread} and \tcode{std::jthread}
so that thread attributes can be passed before the invocable.
This allows adding attribute without modifying the rest of the code.

Having separate types for individual attributes alleviates ABI concerns, and makes extension easier.
Individual attributes - which only serve to hold a value, have constexpr constructors so they can be cheaply initialized.


Supporting both attributes and arguments in the same function call will require implementation some meta programming
contortion to extract the different parameters (attribute, invocable, arguments).
This is made slightly easier by pack indexing (\paper{P2662R2}) \href{https://compiler-explorer.com/z/n8vYb7Tar}{[Compiler Explorer]}.


\tcode{thread_name} makes a copy of this argument and supports non-null terminated strings, and \tcode{u8} strings, as to
portably support environments with utf-8 execution encodings (Linux), environments where Unicode encodings can be preserved (Windows),
and other environments.

This interface is made possible by the realization that lambdas alleviate the needs for \tcode{(j)thread} to support parameter forwarding,
and so we don't need attributes + function + arguments but just function + attributes.


\subsubsection{What about using NTTP?}

During this review of this paper, it was suggested we could pass thread attributes as NTTP.
However, in many cases these attributes may depend on runtime values (thread names depending on a counter would be a good example of that).

\subsubsection{Can we add setters/getters later?}

As explained, this would not be portable, however it would be possible to add the interface to support these later:

\begin{colorblock}
class thread {
          template <typename Attr>
          auto get_attribute() -> optional<typename Attr>;
          template <typename Attr>
          auto set_attribute(Attr);
};
\end{colorblock}


This might require to extend attributes with (static, constexpr) members specifying
\begin{itemize}
	\item Whether they can be retrieved/read after construction
	\item Whether they can be set
	\item The type of the attribute (int, string, etc), as returning the attribute itself may not be the best way to interact with its value
\end{itemize}

This would allow to place better compile-time constraints:

\begin{colorblock}
	class thread {
		template <typename Attr>
		requires @\placeholder{is-readable-attribute}@
		auto get_attribute() -> optional<typename Attr::value_type>;
		template <typename Attr>
		requires @\placeholder{is-settable-attribute}@
		auto set_attribute(Attr);
	};
\end{colorblock}

In that design, the attributes types are then only used by the \tcode{getter} as a tag to indicate to the implementation which attribute to return.

This could be added later (and isn't ABI breaking).

\section{Implementation}

A \href{https://github.com/cor3ntin/llvm-project/tree/corentin/thread_name_p2019}{prototype implementation} for libc++ (supporting only POSIX) threads has been created to validate the design.
This is available on \href{https://compiler-explorer.com/z/MzGsndc7W}{Compiler explorer}, but of limited usefulness in the absence of debugging tools.

\section{Alternatives considered}

\subsection{\paper{P0320R1}}

P0320R1 proposes \tcode{thread::attributes} holding a set of attributes that would all be implementation-defined (the standard would specify no attributes).
This puts the burden on the user to check which attributes are present, presumably using \tcode{\#ifdef}.
We feel very strongly that such an approach fails to improve portability and only improves the status quo marginally.
There is little value in standardizing a class without standardizing its members.

A class is also less extensible and portable than individually declared attributes and poses more ABI concerns.

Moreover, it proposes a \tcode{get_attributes()} function which would returned an implementation-defined object
with all the supported attributes of that platform.
The problem is that not all attributes that can be set can be queried (and reciprocally), and that interface would force
and implementation to return all the attributes it supports, which is wasteful (would have to allocate for the name if a user wants to check the stack size).

\subsection{P0484R1}

P0484 proposes several solutions in the same design space:

\begin{itemize}
\item  A constructor taking a native handle as a parameter
\begin{codeblock}
std::thread thread::thread(native_handle_type h);
\end{codeblock}

This is probably a good idea, regardless of the attributes presented here,
to interface with C libraries or third-party code.

This solves the problem of having to rewrite an entire thread class just to set a stack size.
However, it would still be painful to do so portably, as described in P0484.
A standard library that targets a limited number of platforms can set the attributes more easily than a library that may desire to work in an environment where C++ is deployed.


\item A factory function for creating a thread with attributes
\begin{codeblock}
template <class F, class ... Args>
unicorn<std::thread, ??> make_with_attributes(thread::attributes, F && f, Args && ... args);
\end{codeblock}

We think this is trying to solve two problems:

\begin{itemize}
\item  Threads cannot be used without exceptions support
\item  Some users want the stack size to be guaranteed
\end{itemize}

We are sympathetic to the first concern, however, it seems orthogonal to thread attributes.
If a unicorn type (\tcode{expected}?) or a cheaper exception mechanism is ever standardized,
such a factory function will be welcome, but it doesn't prevent a thread constructor to support attributes.
As for guaranteed stack size:

\begin{itemize}
\item  Some platforms do not support stack size at all - doesn't mean they won't use the desired amount
\item  Some platforms may ignore stack size requests silently
\item  Some platforms may allocate more than request to align with memory pages
\item  Trying to check after the thread has started is not possible (aka it would throw an exception even though the new thread has started)
\end{itemize}

As such, we allow but do not require an implementation to throw when a stack size request cannot be fulfilled.
\end{itemize}


\section{Wording}

\rSec1[thread.threads]{Threads}

\rSec2[thread.threads.general]{General}

\pnum
\ref{thread.threads} describes components that can be used to create and manage threads.
\begin{note}
    These threads are intended to map one-to-one with operating system threads.
\end{note}

\rSec2[thread.syn]{Header \tcode{<thread>} synopsis}

\begin{codeblock}
#include <compare>              // see \ref{compare.syn}

namespace std {
\end{codeblock}
\begin{addedblock}
\begin{codeblock}
    class thread_name;
    class thread_stack_size;
\end{codeblock}
\end{addedblock}
\begin{codeblock}
    // \ref{thread.thread.class}, class \tcode{thread}
    class thread;

    void swap(thread& x, thread& y) noexcept;

    // \ref{thread.jthread.class}, class \tcode{jthread}
    class jthread;

    // \ref{thread.thread.this}, namespace \tcode{this_thread}
    namespace this_thread {
        thread::id get_id() noexcept;

        void yield() noexcept;
        template<class Clock, class Duration>
        void sleep_until(const chrono::time_point<Clock, Duration>& abs_time);
        template<class Rep, class Period>
        void sleep_for(const chrono::duration<Rep, Period>& rel_time);
    }
}
\end{codeblock}
\begin{addedblock}

\rSec2[thread.attributes]{Thread Attributes}

Thread attributes can be used to define additional implementation-defined behaviors on a \tcode{thread} or \tcode{jthread} object.

%\recommended
%Implementations may provide additional platform-specific thread attributes derived from \tcode{thread_attribute} using reserved names.

\rSec3[thread.thread.name.class]{Class \tcode{thread_name}}

\begin{codeblock}
class thread_name {
public:
    constexpr thread_name(std::string_view name) noexcept;
    constexpr thread_name(std::u8string_view name) noexcept;
private:
    @\impdefnc@ __name[@\impdefnc@]; // \expos
};
\end{codeblock}

\recommended

The \tcode{thread_name} thread attribute, can be used
to set the name of a thread such that the name could be used for debugging or platform-specific display mechanisms.
The name should not be stored in the \tcode{std::thread} or \tcode{std::jthread} object.

\begin{itemdecl}
constexpr thread_name(std::string_view name) noexcept;
constexpr thread_name(std::u8string_view name) noexcept;
\end{itemdecl}

\begin{itemdescr}
\effects Initializes \tcode{__name} with \tcode{name} in an implementation-defined manner.

\recommended If \tcode{__name} is not large enough to store the value of \tcode{name},
             the thread name might be truncated. If the implementation performs a text conversion
             during the initialization of \tcode{__name} and if \tcode{name} is a valid code unit sequence
             in the encoding associated with its type, \tcode{__name} should be a valid code unit sequence in its
             associated encoding.
\end{itemdescr}

\rSec3[thread.thread.stack.size.class]{Class \tcode{thread_stack_size}}

\begin{codeblock}
class thread_stack_size {
public:
    constexpr thread_stack_size(std::size_t size) noexcept;
private:
    constexpr std::size_t __size; // \expos
};
\end{codeblock}

\recommended
Configure a desired stack size as if by POSIX \mbox{\tcode{pthread_attr_setstacksize()}}.\\
The stack size set by the implementation may be adjusted up or down to meet platform-specific requirements

If \tcode{__size == 0} is \tcode{true} the thread attribute should be ignored.


\begin{itemdecl}
constexpr thread_stack_size(std::size_t size) noexcept;
\end{itemdecl}

\begin{itemdescr}
\effects Initializes \tcode{__size} with \tcode{size} in an implementation-defined manner.
\end{itemdescr}

\end{addedblock}

\rSec2[thread.thread.class]{Class \tcode{thread}}

\rSec3[thread.thread.class.general]{General}

\pnum
The class \tcode{thread} provides a mechanism to create a new thread of execution, to join with
a thread (i.e., wait for a thread to complete), and to perform other operations that manage and
query the state of a thread. A \tcode{thread} object uniquely represents a particular thread of
execution. That representation may be transferred to other \tcode{thread} objects in such a way
that no two \tcode{thread} objects simultaneously represent the same thread of execution. A
thread of execution is \term{detached} when no \tcode{thread} object represents that thread.
Objects of class \tcode{thread} can be in a state that does not represent a thread of
execution.
\begin{note}
    A \tcode{thread} object does not represent a thread of execution after
    default construction, after being moved from, or after a successful call to \tcode{detach} or
    \tcode{join}.
\end{note}

\begin{codeblock}
namespace std {
class thread {
    public:
    class thread::id;
    class id;
    using native_handle_type = @\impdefnc@;         // see~\ref{thread.req.native}

    // construct/copy/destroy
    thread() noexcept;
    template<@\removed{class F,}@ class... Args> explicit thread(@\removed{F\&\& f,}@ Args&&... args);

    ~thread();
    thread(const thread&) = delete;
    thread(thread&&) noexcept;
    thread& operator=(const thread&) = delete;
    thread& operator=(thread&&) noexcept;

    // \ref{thread.thread.member}, members
    void swap(thread&) noexcept;
    bool joinable() const noexcept;
    void join();
    void detach();
    id get_id() const noexcept;
    native_handle_type native_handle();                         // see~\ref{thread.req.native}

    // static members
    static unsigned int hardware_concurrency() noexcept;
};
}
\end{codeblock}


\rSec3[thread.thread.constr]{Constructors}

\begin{itemdecl}
template<@\removed{class F,}@ class... Args>
explicit thread(@\removed{F\&\& f,}@ Args&&... args);
\end{itemdecl}

\begin{itemdescr}

\pnum
\constraints
\begin{itemize}
\begin{addedblock}
\item \tcode{sizeof...(Args) > 0} is \tcode{true}, and
\end{addedblock}
\item \tcode{remove_cvref_t<\changed{F}{Args[0]}>} is not the same type as \tcode{thread}.
\end{itemize}

\begin{addedblock}
\begin{itemize}[leftmargin=*]
\item Let $i$ be the smallest value such that \tcode{decay_t<Args...[$i$]>} is not a thread attribute type. \ednote{We need to define that}

If no such $i$ exists, the programm is ill-formed.

\item Let \tcode{F} be \tcode{Args...[$i$]}.
\item Let \tcode{f} be \tcode{args...[$i$]}.
%\item Let \tcode{Attrs} a pack of the types \tcode{Args...[$j$]} for each $j$ such that $ 0 \leq j < i $.
\item Let \tcode{attrs} a pack of the expressions \tcode{args...[$j$]} for each $j$ such that $ 0 \leq j < i $.
\item Let \tcode{FArgs} a pack of the types \tcode{Args...[$j$]} for each $j$ such that $ i \le j \le \tcode{sizeof...(args)}$.
\item Let \tcode{fargs} a pack of the expressions \tcode{args...[$j$]} for each $j$ such that $ i \le j \le \tcode{sizeof...(args)}$.
\end{itemize}
\end{addedblock}


\pnum
\mandates
The following are all \tcode{true}:
\begin{itemize}
\begin{removedblock}
    \item \tcode{is_constructible_v<decay_t<F>, F>},
\end{removedblock}
    \item \tcode{(is_constructible_v<decay_t<Args>, Args> \&\& ...)},
\begin{removedblock}
    \item \tcode{is_move_constructible_v<decay_t<F>>},
\end{removedblock}
    \item \tcode{(is_move_constructible_v<decay_t<Args>> \&\& ...)}, \removed{and}
    \item \tcode{is_invocable_v<decay_t<F>, decay_t<\added{F}Args>...>}\changed{.}{, and}
\begin{addedblock}
    \item No type is present more than once in the pack \tcode{remove_cvref_t<Attrs>}. \ednote{// is there a better way to say that ?}
\end{addedblock}
\end{itemize}

\pnum
\expects
\removed{\tcode{decay_t<F>} and} each type in \tcode{decay_t<Args>} meet the
\oldconcept{MoveConstructible} requirements

\pnum
\effects
The new thread of execution executes
\begin{codeblock}
    invoke(@\placeholdernc{decay-copy}@(std::forward<F>(f)), @\placeholdernc{decay-copy}@(std::forward<@\added{F}@Args>(@\added{f}@args))...)
\end{codeblock}
with the calls to
\tcode{\placeholder{decay-copy}} being evaluated in the constructing thread.
Any return value from this invocation is ignored.
\begin{note}
    This implies that any exceptions not thrown from the invocation of the copy
    of \tcode{f} will be thrown in the constructing thread, not the new thread.
\end{note}
If the invocation of \tcode{invoke} terminates with an uncaught exception,
\tcode{terminate} is called.

\begin{addedblock}
\tcode{attrs} can be used to tailor the thread with additional implementation-defined behaviors. (see [thread.attributes]).
\end{addedblock}

\pnum
\sync
The completion of the invocation of the constructor
synchronizes with the beginning of the invocation of the copy of \tcode{f}.

\pnum
\ensures
\tcode{get_id() != id()}. \tcode{*this} represents the newly started thread.

\pnum
\throws
\tcode{system_error} if unable to start the new thread.

\pnum
\errors
\begin{itemize}
    \item \tcode{resource_unavailable_try_again} --- the system lacked the necessary
    resources to create another thread, or the system-imposed limit on the number of
    threads in a process would be exceeded.
\end{itemize}
\end{itemdescr}

%
%\begin{addedblock}
%\rSec3[thread.thread.factory]{Factory functions}
%
%\indexlibraryctor{thread}%
%\begin{itemdecl}
%template<class F, class... Args>
%static thread make_with_attributes(F&& f, Args&&... args);
%\end{itemdecl}
%
%\begin{itemdescr}
%\pnum
%\constraints
%\pnum
%\mandates
%The following are all \tcode{true}:
%\begin{itemize}
%    \item \tcode{is_constructible_v<decay_t<F>, F>},
%    \item \tcode{(is_base_of_v<thread_attribute, Attrs>) \&\& ...)},
%    \item \tcode{(is_invocable_v<decay_t<F>>}.
%    \item \tcode{(is_base_of_v<thread_attribute, Attrs>) \&\& ...)},
%    \item \tcode{(is_invocable_v<decay_t<F>>}.
%\end{itemize}
%
%\returns \tcode{thread(\placeholder{attribute-tag}\{\}, std::forward<F>(f), std::forward<Args>(args)...));}
%
%    %The following are all \tcode{true}:
%    %\begin{itemize}
%    %    \item \tcode{is_constructible_v<decay_t<F>, F>},
%    %    \item \tcode{(is_constructible_v<decay_t<Args>, Args> \&\& ...)},
%    %    \item \tcode{is_move_constructible_v<decay_t<F>>},
%    %    \item \tcode{(is_move_constructible_v<decay_t<Args>> \&\& ...)}, and
%    %    \item \tcode{is_invocable_v<decay_t<F>, decay_t<Args>...>}.
%    %\end{itemize}
%    %\end{removedblock}
%    %\begin{addedblock}
%    %\begin{itemize}
%    %    \item \tcode{is_constructible_v<decay_t<F>, F>} is \tcode{true},
%    %    \item \tcode{is_move_constructible_v<decay_t<F>>} is \tcode{true}.
%    %
%    %    \item For the first overload, the following are all true:
%    %    \begin{itemize}
%        %    \item \tcode{(is_constructible_v<decay_t<Args>, Args> \&\& ...)},
%        %    \item \tcode{(is_move_constructible_v<decay_t<Args>> \&\& ...)}, and
%        %    \item \tcode{is_invocable_v<decay_t<F>, decay_t<Args>...>}.
%        %    \end{itemize}
%    %
%    %    \item For the second overload, the following are all true:
%    %    \begin{itemize}
%        %
%        %    \end{itemize}
%    %\end{itemize}
%\end{itemdescr}
%\end{addedblock}
\section{\tcode{jthread}}

\ednote{TODO: exact same wording as for thread}.

\section{Feature test macros}

\ednote{Add a new macro in \tcode{<version>}, \tcode{<thread>}: \tcode{__cpp_lib_thread_attributes} set to the date of adoption}.


\section{Acknowledgments}

Thanks to Martin Hořeňovský, Kamil Rytarowski, Clément Grégoire, Bruce Dawson, Patrice Roy, Ronen Friedman, Billy Baker, and others for their valuable feedback.

Thanks to Tomasz Kamiński, as well as Jonathan Wakely and Mathias Stearn for their wording suggestions.


\bibliographystyle{plain}
\bibliography{wg21}


\renewcommand{\section}[2]{}%
\begin{thebibliography}{9}
    \bibitem[P0484R1]{P0484R1}
    Patrice Roy, Billy Baker
    \emph{Enhancing Thread Constructor Attributes – An Exploration of the Design Space}\newline
    \url{https://wg21.link/P0484R1}

    \bibitem[N4885]{N4892}
    Thomas Köppe
    \emph{Working Draft, Standard for Programming Language C++}\newline
    \url{https://wg21.link/N4892}

\end{thebibliography}

\end{document}

\end{document}
