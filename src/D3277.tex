% !TeX program = luatex
% !TEX encoding = UTF-8


\documentclass{wg21}

\title{Pack Slicing}
\docnumber{D3277R0}
\audience{EWG}
\author{Corentin Jabot}{corentin.jabot@gmail.com}

\usepackage{color, colortbl}
\begin{document}
\maketitle

\section{Abstract}

This paper expands on the pack slicing feature described in \paper{P1858R2} and provides wording.


\section{Motivation}

The motivation for pack slicing is covered in \paper{P1858R2} and \paper{P2632R0}.
Slicing a pack produces an unexpaded pack, potentally excluding some of the element of the original pack.


\section{Syntax and Design}

The syntax of pack slicing extend that of pack indexing.
Mirroring languages that offer an array slicing feature (such as python) the 2 bounds required for slicing
are separated by \tcode{:}

\begin{colorblock}
Pack...[lower_bound:upper_bound]
\end{colorblock}

The lower bound is inclusive and the upper bound is exclusive.
The upper bound can be omitted (in which case it is equivalent to \tcode{sizeof...(Pack)}).

\begin{colorblock}
Pack...[0:sizeof...(Pack)] // the whole pack
Pack...[1:sizeof...(Pack)] // excludes the first element
Pack...[1:]                // excludes the first element
\end{colorblock}

\section{Wording}

\rSec2[expr.prim.id]{Names}

\rSec3[expr.prim.id.general]{General}

\begin{bnf}
	\nontermdef{id-expression}\br
	unqualified-id\br
	qualified-id\br
    pack-index-expression\br
\begin{addedblock}
    pack-slicing-expression
\end{addedblock}
\end{bnf}


\begin{addedblock}
\ednote{Add a new section after [expr.prim.pack.index]}

\rSec3[expr.prim.pack.slice]{Pack slicing expression}

\begin{bnf}
    \nontermdef{pack-slicing-expression}\br
    id-expression \terminal{...} \terminal{[} constant-expression \terminal{:} constant-expression\opt \terminal{]}  \br
\end{bnf}

The \grammarterm{id-expression} $P$  in a \grammarterm{pack-slicing-expression} shall be an \grammarterm{identifier} that denotes a pack.

Both \grammarterm{constant-expression}{s}, wehen present, shall be converted constant expressions [expr.const]\\
of type \tcode{std::size_t}.

Let $LowerBound$ (termed the lower bound) be the value of the first \grammarterm{constant-expression}.\\
Let $UpperBound$ (termed the upper bound) be the value of the second \grammarterm{constant-expression} if it is present and \tcode{sizeof...(P)} otherwise.

$LowerBound \leq UpperBound \leq \tcode{sizeof...(P)}$ shall be true.

A \grammarterm{pack-slicing-expression} is a pack expansion ([temp.variadic]).

\begin{note}
A \grammarterm{pack-slicing-expression} denotes an unexpanded pack comprising of the elements $LowerBound^\text{th} ... UpperBound-1^\text{th}$ of the pack ([temp.variadic]).
\end{note}

\end{addedblock}

\ednote{[...]}

\rSec3[dcl.type.simple]{Simple type specifiers}%
\indextext{type specifier!simple}

\pnum
The simple type specifiers are

\begin{bnf}
\nontermdef{simple-type-specifier}\br
\opt{nested-name-specifier} type-name\br
nested-name-specifier \keyword{template} simple-template-id\br\textbf{}
computed-type-specifier\br
placeholder-type-specifier\br
\opt{nested-name-specifier} template-name\br
\end{bnf}

\begin{bnf}
\nontermdef{computed-type-specifier}\br
decltype-specifier\br
pack-index-specifier\br
\begin{addedblock}
pack-slicing-specifier\br
\end{addedblock}
\end{bnf}

\textcolor{noteclr}{[...]}

\begin{simpletypetable}
    {\grammarterm{simple-type-specifier}{s} and the types they specify}
    {dcl.type.simple}
    {ll}
    \topline
    \hdstyle{Specifier(s)}            &   \hdstyle{Type}                  \\ \capsep
    \grammarterm{type-name}           & the type named                    \\
    \grammarterm{simple-template-id}  & the type as defined in [temp.names]\\
    \grammarterm{decltype-specifier}  & the type as defined in [dcl.type.decltype]\\
    \grammarterm{pack-index-specifier} & the type as defined in [dcl.type.pack.indexing] \\
    \added{\grammarterm{pack-slicing-specifier}} & \added{the type as defined in [dcl.type.pack.slicing]} \\
    \grammarterm{placeholder-type-specifier}
    & the type as defined in [dcl.spec.auto]\\
    \grammarterm{template-name}       & the type as defined in [dcl.type.class.deduct]\\
    \tcode{char}                      & ``\tcode{char}''                  \\
    \tcode{unsigned char}             & ``\tcode{unsigned char}''         \\
    \tcode{signed char}               & ``\tcode{signed char}''           \\
    \keyword{char8_t}                   & ``\tcode{char8_t}''               \\
    \keyword{char16_t}                  & ``\tcode{char16_t}''              \\
    ... & \\
\end{simpletypetable}


\pnum
When multiple \grammarterm{simple-type-specifier}{s} are allowed, they can be
freely intermixed with other \grammarterm{decl-specifier}{s} in any order.
\begin{note}
    It is \impldef{signedness of \tcode{char}} whether objects of \tcode{char} type are
    represented as signed or unsigned quantities. The \tcode{signed} specifier
    forces \tcode{char} objects to be signed; it is redundant in other contexts.
\end{note}


\ednote{Add a new section after [dcl.type.pack.indexing]}


\begin{addedblock}

\rSec3[dcl.type.pack.slicing]{Pack slicing specifier}

\begin{bnf}
    \nontermdef{pack-index-specifier}\br
    typedef-name \terminal{...} \terminal{[} constant-expression \terminal{:} constant-expression\opt \terminal{]}  \br
\end{bnf}

The \grammarterm{typedef-name} $P$ in a \grammarterm{pack-index-specifier} shall denote a pack.

Both \grammarterm{constant-expression}{s}, wehen present, shall be converted constant expressions [expr.const]\\
of type \tcode{std::size_t}.

Let $LowerBound$ (termed the lower bound) be the value of the first \grammarterm{constant-expression}.\\
Let $UpperBound$ (termed the upper bound) be the value of the second \grammarterm{constant-expression} if it is present and \tcode{sizeof...(P)} otherwise.

$LowerBound \leq UpperBound \leq \tcode{sizeof...(P)}$ shall be true.

A \grammarterm{pack-slicing-specifier} is a pack expansion ([temp.variadic]).

\begin{note}
The \grammarterm{pack-slicing-specifier} denotes an unexpanded pack comprising of the elements $LowerBound^\text{th} ... UpperBound-1^\text{th}$ of the pack ([temp.variadic]).
\end{note}

\end{addedblock}


%\rSec1[temp.type]{Type equivalence}
%
%\pnum
%If an expression $e$ is type-dependent \iref{temp.dep.expr},
%\tcode{decltype($e$)}
%denotes a unique dependent type. Two such \grammarterm{decltype-specifier}{s}
%refer to the same type only if their \grammarterm{expression}{s} are
%equivalent \iref{temp.over.link}.
%\begin{note} % lah: Thios note in the Standard draft is label "Note 1", not just "Note".
%    However, such a type might be aliased,
%    e.g., by a \grammarterm{typedef-name}.
%\end{note}
%
%For a type template parameter pack \tcode{T}, \tcode{T...[\grammarterm{constant-expression}{}]} denotes a unique dependent type.
%
%If the \grammarterm{constant-expression} of a \grammarterm{pack-index-specifier} is value-dependent, two such \grammarterm{pack-index-specifier}{s} refer to the same type only if their \grammarterm{constant-expression}{s} are equivalent \iref{temp.over.link}.
%
%Otherwise, two such \grammarterm{pack-index-specifier}{s} refer to the same type only if their indexes have the same value.
%
%\begin{addedblock}
%
%\end{addedblock}

\rSec2[temp.variadic]{Variadic templates}

\textcolor{noteclr}{[...]}

\pnum
\indextext{pattern|see{pack expansion, pattern}}%
A \defn{pack expansion}
consists of a \defnx{pattern}{pack expansion!pattern} and an ellipsis, the instantiation of which
produces zero or more instantiations of the pattern in a list (described below).
The form of the pattern
depends on the context in which the expansion occurs. Pack
expansions can occur in the following contexts:

\begin{itemize}
    \item In a function parameter pack \iref{dcl.fct}; the pattern is the
    \grammarterm{parameter-declaration} without the ellipsis.

    \item In a \grammarterm{using-declaration} \iref{namespace.udecl};
    the pattern is a \grammarterm{using-declarator}.

    \item In a template parameter pack that is a pack expansion \iref{temp.param}:
    \begin{itemize}
        \item
        if the template parameter pack is a \grammarterm{parameter-declaration};
        the pattern is the \grammarterm{parameter-declaration} without the ellipsis;

        \item
        if the template parameter pack is a \grammarterm{type-parameter};
        the pattern is the corresponding \grammarterm{type-parameter}
        without the ellipsis.
    \end{itemize}

    \item In an \grammarterm{initializer-list} \iref{dcl.init};
    the pattern is an \grammarterm{initializer-clause}.

    \item In a \grammarterm{base-specifier-list} \iref{class.derived};
    the pattern is a \grammarterm{base-specifier}.

    \item In a \grammarterm{mem-initializer-list} \iref{class.base.init} for a
    \grammarterm{mem-initializer} whose \grammarterm{mem-initializer-id} denotes a
    base class; the pattern is the \grammarterm{mem-initializer}.

    \item In a \grammarterm{template-argument-list} \iref{temp.arg};
    the pattern is a \grammarterm{template-argument}.

    \item In an \grammarterm{attribute-list} \iref{dcl.attr.grammar}; the pattern is
    an \grammarterm{attribute}.

    \item In an \grammarterm{alignment-specifier} \iref{dcl.align}; the pattern is
    the \grammarterm{alignment-specifier} without the ellipsis.

    \item In a \grammarterm{capture-list} \iref{expr.prim.lambda.capture}; the pattern is
    the \grammarterm{capture} without the ellipsis.

    \item In a \tcode{sizeof...} expression \iref{expr.sizeof}; the pattern is an
    \grammarterm{identifier}.

    \item In a \grammarterm{pack-index-expression}; the pattern is an
    \grammarterm{identifier}.
    \item In a \grammarterm{pack-index-specifier}; the pattern is a
    \grammarterm{typedef-name}.

    \begin{addedblock}
    \item In a \grammarterm{pack-slicing-expression}; the pattern is an
    \grammarterm{identifier}.
    \item In a \grammarterm{pack-slicing-specifier}; the pattern is a
    \grammarterm{typedef-name}.
    \end{addedblock}

    \item In a \grammarterm{fold-expression} \iref{expr.prim.fold};
    the pattern is the \grammarterm{cast-expression}
    that contains an unexpanded pack.
\end{itemize}

\begin{example} % lah: this is labeled Example 4 in the Standard's draft.
    \begin{codeblock}
        template<class ... Types> void f(Types ... rest);
        template<class ... Types> void g(Types ... rest) {
            f(&rest ...);     // ``\tcode{\&rest ...}'' is a pack expansion; ``\tcode{\&rest}'' is its pattern
        }
    \end{codeblock}
\end{example}

\pnum
For the purpose of determining whether a pack satisfies a rule
regarding entities other than packs, the pack is
considered to be the entity that would result from an instantiation of
the pattern in which it appears.

\pnum
A pack whose name appears within the pattern of a pack
expansion is expanded by that pack expansion. An appearance of the name of
a pack is only expanded by the innermost enclosing pack expansion.
The pattern of a pack expansion shall name one or more packs that
are not expanded by a nested pack expansion; such packs are called
\defnx{unexpanded packs}{pack!unexpanded} in the pattern. All of the packs expanded
by a pack expansion shall have the same number of arguments specified. An
appearance of a name of a pack that is not expanded is
ill-formed.
\begin{example} % lah: this is labeled Example 5 in the Standard's draft.
    \begin{codeblock}
        template<typename...> struct Tuple {};
        template<typename T1, typename T2> struct Pair {};

        template<class ... Args1> struct zip {
            template<class ... Args2> struct with {
                typedef Tuple<Pair<Args1, Args2> ... > type;
            };
        };

        typedef zip<short, int>::with<unsigned short, unsigned>::type T1;
        // \tcode{T1} is \tcode{Tuple<Pair<short, unsigned short>, Pair<int, unsigned>>}
        typedef zip<short>::with<unsigned short, unsigned>::type T2;
        // error: different number of arguments specified for \tcode{Args1} and \tcode{Args2}

        template<class ... Args>
        void g(Args ... args) {                   // OK, \tcode{Args} is expanded by the function parameter pack \tcode{args}
            f(const_cast<const Args*>(&args)...);   // OK, ``\tcode{Args}'' and ``\tcode{args}'' are expanded
            f(5 ...);                               // error: pattern does not contain any packs
            f(args);                                // error: pack ``\tcode{args}'' is not expanded
            f(h(args ...) + args ...);              // OK, first ``\tcode{args}'' expanded within \tcode{h},
            // second ``\tcode{args}'' expanded within \tcode{f}
        }
    \end{codeblock}
\end{example}

\pnum
The instantiation of a pack expansion considers
items $\tcode{E}_1, \tcode{E}_2, \dotsc, \tcode{E}_N$,
where
$N$ is the number of elements in the pack expansion parameters.
Each $\tcode{E}_i$ is generated by instantiating the pattern and
replacing each pack expansion parameter with its $i^\text{th}$ element.
Such an element, in the context of the instantiation, is interpreted as
follows:
\begin{itemize}
    \item
    if the pack is a template parameter pack, the element is
    an \grammarterm{id-expression}
    (for a non-type template parameter pack),
    a \grammarterm{typedef-name}
    (for a type template parameter pack declared without \tcode{template}), or
    a \grammarterm{template-name}
    (for a type template parameter pack declared with \tcode{template}),
    designating the $i^\text{th}$ corresponding type or value template argument;

    \item
    if the pack is a function parameter pack, the element is an
    \grammarterm{id-expression}
    designating the $i^\text{th}$ function parameter
    that resulted from instantiation of
    the function parameter pack declaration;
    otherwise

    \item
    if the pack is an \grammarterm{init-capture} pack,
    the element is an \grammarterm{id-expression}
    designating the variable introduced by
    the $i^\text{th}$ \grammarterm{init-capture}
    that resulted from instantiation of
    the \grammarterm{init-capture} pack.

    \begin{addedblock}
     \item
    if the pack is a \grammarterm{pack-slicing-expression}
    the element is an \grammarterm{id-expression}
    designating the $i^\text{th}$ element
    that resulted from instantiation of
    the \grammarterm{pack-slicing-expression} pack.

    \item
    if the pack is a \grammarterm{pack-slicing-specifier}
    the element is a \grammarterm{typedef-name}
    designating the $i^\text{th}$ element
    that resulted from instantiation of
    the \grammarterm{pack-slicing-specifier} pack.
    \end{addedblock}

\end{itemize}
When $N$ is zero, the instantiation of a pack expansion
does not alter the syntactic interpretation of the enclosing construct,
even in cases where omitting the pack expansion entirely would
otherwise be ill-formed or would result in an ambiguity in the grammar.

\pnum
The instantiation of a \tcode{sizeof...} expression \iref{expr.sizeof} produces
an integral constant with value $N$.

\pnum

When instantiating a \grammarterm{pack-index-expression} $P$,
let $K$ be the index of $P$.
The instantiation of $P$ is the \grammarterm{id-expression} \tcode{$\mathtt{E}_K$}.


When instantiating a \grammarterm{pack-index-specifier} $P$,
let $K$ be the index of $P$.
The instantiation of $P$ is the \grammarterm{typedef-name} \tcode{$\mathtt{E}_K$}.

\begin{addedblock}

When instantiating a \grammarterm{pack-slicing-expression} $P$,
let $L$ be the lower bound of $P$ and let $U$ be the upper bound of $P$.\\
The instantiation of $P$ is a pack comprised of the \grammarterm{id-expression}{s} \tcode{$\mathtt{E}_L$ ... $\mathtt{E}_U$}.

When instantiating a \grammarterm{pack-slicing-specifier} $P$,
let $L$ be the lower bound of $P$ and let $U$ be the upper bound of $P$.\\
The instantiation of $P$ is a pack comprised of the \grammarterm{typedef-name}{s} \tcode{$\mathtt{E}_L$ ... $\mathtt{E}_U$}.

\end{addedblock}


\rSec3[temp.dep.type]{Dependent types}

\ednote{Add a bullet in paragraph 7}

\pnum
A type is dependent if it is
\begin{itemize}
	\item
	a template parameter,
	\item
	denoted by a dependent (qualified) name,
	\item
	a nested class or enumeration that is a direct member of
	a class that is the current instantiation,
	\item
	a cv-qualified type where the cv-unqualified type is dependent,
	\item
	a compound type constructed from any dependent type,
	\item
	an array type whose element type is dependent or whose
	bound (if any) is value-dependent,
	\item
	a function type whose parameters include one or more function parameter packs,
	\item
	a function type whose exception specification is value-dependent,
	\item
	denoted by a \grammarterm{simple-template-id}
	in which either the template name is a template parameter or any of the
	template arguments is a dependent type or an expression that is type-dependent
	or value-dependent or is a pack expansion,
    \item a \grammarterm{pack-index-specifier},
\begin{addedblock}
    \item a \grammarterm{pack-slicing-specifier},
\end{addedblock}
	or
	\item denoted by \tcode{decltype(}\grammarterm{expression}{}\tcode{)},
	where \grammarterm{expression} is type-dependent\iref{temp.dep.expr}.
\end{itemize}

\rSec3[temp.dep.expr]{Type-dependent expressions}

\ednote{Add a paragraph at the end of temp.dep.expr}

\pnum
A \grammarterm{braced-init-list} is type-dependent if any element is
type-dependent or is a pack expansion.

\pnum
A \grammarterm{fold-expression} is type-dependent.

	\pnum
A \grammarterm{pack-index-expression} is type-dependent if its \grammarterm{id-expression} is type-dependent.

\begin{addedblock}
A \grammarterm{pack-slicing-expression} is type-dependent
\end{addedblock}

\textcolor{noteclr}{[...]}

\rSec3[temp.deduct.type]{Deducing template arguments from a type}

% An id-expression is type-dependent if it is a template-id that is not a concept-id and is dependent; OR it it a pack expression whose constant expression is a constant expression OR or if its terminal name is

% modify class.base.init.

\pnum
The non-deduced contexts are:

\indextext{context!non-deduced}%
\begin{itemize}
    \item
    The
    \grammarterm{nested-name-specifier}
    of a type that was specified using a
    \grammarterm{qualified-id}.
    \item A \grammarterm{pack-index-specifier} or a \grammarterm{pack-index-expression}.
    \begin{addedblock}
    \item A \grammarterm{pack-slicing-specifier} or a \grammarterm{pack-slicing-expression}.
    \end{addedblock}
    \item
    The \grammarterm{expression} of a \grammarterm{decltype-specifier}.
    \item
    A non-type template argument or an array bound in which a subexpression
    references a template parameter.
    \item
    A template parameter used in the parameter type of a function parameter that
    has a default argument that is being used in the call for which argument
    deduction is being done.
    \item
    A function parameter for which the associated argument is an
    overload set \iref{over.over}, and one or more of the following apply:
    \begin{itemize}
        \item
        more than one function matches the function parameter type (resulting in
        an ambiguous deduction), or
        \item
        no function matches the function parameter type, or
        \item
        the overload set supplied as an argument contains one or more function templates.
    \end{itemize}
    \item A function parameter for which the associated argument is an initializer
    list \iref{dcl.init.list} but the parameter does not have
    a type for which deduction from an initializer list is specified \iref{temp.deduct.call}.
    \begin{example}
        \begin{codeblock}
            template<class T> void g(T);
            g({1,2,3});                 // error: no argument deduced for \tcode{T}
        \end{codeblock}
    \end{example}
    \item A function parameter pack that does not occur at the end of the
    \grammarterm{parameter-declaration-list}.
\end{itemize}


\section{Feature test macros}

\ednote{Add a new macro in \tcode{[tab:cpp.predefined.ft]} : \tcode{__cpp_pack_slicing} set to the date of adoption}.


\section{Acknowledgments}

We extend our appreciation to Sean Baxter for his work on Circle and to Barry Revzin for his work on \paper{P1858R2}, both works being the foundation of the design presented here.

\section{References} % lah: Correct your citation style. Books, proceedings, conference names, and Standards use italics. Articles, presentations, and talks use quotes. Harvard style is the general UK convention. https://intranet.birmingham.ac.uk/as/libraryservices/library/referencing/icite/harvard/referencelist.aspx

\renewcommand{\section}[2]{}%
\bibliographystyle{plain}
\bibliography{wg21, extra}

\begin{thebibliography}{9}

\bibitem[P2632R0]{P2632R0}
Corentin Jabot, Pablo Halpern, John Lakos, Alisdair Meredith, Joshua Berne, and Gašper Ažman\newline
\emph{A plan for better template meta programming facilities in C++26}\newline
\url{https://wg21.link/P2632R0}\newline
October 2022

\bibitem[N4885]{N4885}
Thomas Köppe
\emph{Working Draft, Standard for Programming Language C++}\newline
\url{https://wg21.link/N4885}


\end{thebibliography}

\end{document}
