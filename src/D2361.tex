% !TeX document-id = {9322a846-f757-4574-9231-a2e85c743b21}
% !TeX program = luatex
% !TEX encoding = UTF-8


\RequirePackage{luatex85}%
\documentclass{wg21}


\RequirePackageWithOptions{fontspec}
\usepackage{newunicodechar}

\setmainfont{Noto Sans}

\title{Unevaluated strings}
\docnumber{P2361R6}
\audience{CWG}
\author{Corentin Jabot}{corentin.jabot@gmail.com}
\authortwo{Aaron Ballman}{aaron.ballman@gmail.com}

\usepackage{tikz}

\begin{document}

\maketitle

\section{Abstract}

\grammarterm{string-literal}{s} can appear in a context where they are not used to
initialize a character array, but are used at compile time for diagnostic messages,
preprocessing, and other implementation-defined behaviors.
This paper clarifies how compilers should handle these strings.


\section{Motivation}

A \grammarterm{string-literal} can appear in \tcode{_Pragma}, \tcode{asm}, \tcode{extern},
\tcode{static_assert}, \tcode{[[deprecated]]} and \tcode{[[nodiscard]]} attributes...

In all of these cases, the strings are exclusively used at compile time by the compiler, and are as such not evaluated in phase 6.
This means they should not be converted to the narrow encoding or any literal encoding specified by an encoding prefix (L, u, U, u8).

Their encoding should therefore not be constrained or otherwise specified,
except that these strings can contain any Unicode characters.

This proposal aims to identify contexts in which strings are not evaluated so that they can be handled consistently by compilers.

\section{Revisions}

\subsection{R6}
Apply core feedback:
\begin{itemize}
\item Reword [over.literal]
\item Revert the changes to [cpp]
\item Rebase the wording in [dcl.pre]
\end{itemize}

\subsection{R5}
\begin{itemize}
\item Re-specify \tcode{asm} declarations to accept any \grammarterm{balanced-token-seq} rather than trying to put constraints on the
\grammarterm{string-literal} they are currently specified to accept. This simplifies the specification and more closely matches existing practices.
\textbf{Following EWG review, additional wording has been added to make it clear that an implementation does not have to support \tcode{asm} statements or any grammar therein}.
The intent of the proposal is to allow implementer to either allow any kind of grammar they want in \tcode{asm} statements, or allow no grammar whatsoever; No implementation would have to
change its handling of \tcode{asm} statements with this proposal.
Further work to remove [decl.asm] entierly wile leaving the \tcode{asm} keyword reserved for vendor extensions could be the object of a separate paper.

\item Fix a number of typos.
\end{itemize}

\subsection{R4}
\begin{itemize}
    \item Rebase the wording and wording tweaks.
\end{itemize}

\subsection{R3}
\begin{itemize}
    \item Improve wording by not making \grammarterm{unevaluated-string} preprocessing token as preprocessing tokens should not be context dependent. Fix the wording of \tcode{\#line} and \tcode{\_Pragma} accordingly.
    \item Append null-terminator during evaluation of string-literals to make it clear that \grammarterm{unevaluated-string} are not null-terminated.
    \item Adapt the grammar of \grammarterm{literal-operator-id}
    \item Adapt the wording of \tcode{extern} to clarify that the likage specification denotes unicode characters.
    \item Allow numeric escape sequences in asm statements.
\end{itemize}
\subsection{R2}
\begin{itemize}
    \item \grammarterm{unevaluated-string-literal} to \grammarterm{unevaluated-string}.
    \item Add a note about not disallowing non-printable characters
    \item Add a note about \grammarterm{unevaluated-string} not being expressions.
    \item Fix typos.
    \item Improve wording.
\end{itemize}

\section{Proposal}

Unevaluated string literals can appear in

\begin{itemize}
\item \tcode{_Pragma}
\item \tcode{\#line} directives
%\item \tcode{header-name}
\item \tcode{[[nodiscard]]} and \tcode{[[deprecated]]} attributes
\item \tcode{extern} linkage specifications
\item \tcode{asm} statements
\item \tcode{static_assert}
\item literal operator
\end{itemize}

We propose that in all of these cases:
\begin{itemize}
\item No prefix is allowed
\item The string is not converted to the execution encoding.
\item \tcode{universal-character-name} and \tcode{simple-escape-sequence} (except \tcode{\textbackslash 0} ) are replaced by
the corresponding Unicode codepoints, and other escape sequences are ill-formed.
\end{itemize}

This last point is important. Because the encoding the compiler will convert these strings to is not known, and because UCNs can represent any Unicode characters,
numeric-escape-sequences have no use beyond forcing the compiler to contend with invalid
code units in diagnostic messages.

All of these changes are breaking changes.
However, a survey of open source projects tends to show that none of the restrictions added
impact existing code.

\textbf{This proposal does not specify how unevaluated strings are presented in diagnostic messages}.

\subsection{Non-printable characters and escape sequences}

This proposal does not attempt to restrict further the characters allowed in unevaluated strings.
In particular, they may contain all matter of space, control characters, invisible characters, and alert.
The handling of these characters in diagnostic messages is left as quality of implementation, mostly
for simplicity. The alternative would be to only allow graphic characters (General_Category L, M, N, P, S + spaces).

\section{Alternative considered}

\subsection{Allowing and ignoring any prefix}

This is arguably the status quo.
The issue is that it is hard to teach. Users should be able to expect for example that L"X"
is always in the wide execution encoding.
It could be argued that \tcode{"foo"} not being in the narrow-encoding is also confusing, however, there is precedence for that in headers names (which are already not \grammarterm{string-literals}{s}).

\subsection{Allowing prefixes and encode all strings using that prefix}

This is both implementer- and user-hostile. It would force users to use any of u, u8, U
on all of their \tcode{static_assert} which contain non-ASCII characters as it is the only way to obtain a portable encoding.
It has the advantage of being mostly consistent (all strings except those in headers names would be encoded using the encoding associated with their prefix) but would break
existing code using non-ASCII characters in \tcode{static_assert} and attributes and
litter C++ code with these prefixes, which seems to be a net negative.

\section{\tcode{asm} declarations}

Several people in SG22, as well as an implementer, raised concerns about banning numeric escape sequences
in asm statement as supposedly an implementation could do "something" here.
Given the implementation-defined, conditionally supported nature of \tcode{asm} declarations, and existing practice, we decided to modify the grammar to accept any balanced token sequence (which includes string literals).
This is consistent with GCC's "extended asm" syntax, notably.
Implementations that support a single string literal in \tcode{asm} declarations can, of course, continue to do so with no change required to their
implementation and are free to put whichever requirements they need on them.
This does not requires an implementation to be able to parse arbitrary \grammarterm{balanced-token-seq} either.
The intent is that any \grammarterm{balanced-token-seq} not recognized by an implementation that would otherwise support \tcode{asm} statements
can be ill-formed.

\section{Compilers survey}

\subsection{_Pragma}

In \tcode{_Pragma} directives, the standard specifies that the \tcode{L} prefix is ignored.
In C, all encoding prefixes are ignored. This divergence is highlighted in \paper{CWG897}.
MSVC does not support \tcode{_Pragma(L"")}.  Only Clang supports other prefixes in \tcode{_Pragma}.

Out of the 90 million lines of code of the 1300+ open source projects available on vcpkg, a
single use of that feature was found within clang’s lexer test suite, for a total of 2000 uses
of \tcode{_Pragma}. Similarly, the only uses of \tcode{_Pragma (u8"")}, \tcode{_Pragma (u"")}, \tcode{_Pragma (U"")}, etc were
found in Clang’s test suite (both because these are valid C and because neither GCC nor Clang
are conforming, only \tcode{L""} is described as valid by the C++ standard).

\subsection{Attributes}

Clang does not support strings with an encoding prefix in attributes, other compilers accept them.


\subsection{\tcode{static_assert}}

All compilers support strings with an encoding prefix in static assert.
MSVC appears to convert the string to the encoding associated with that prefix before displaying it, producing mojibake if a string
cannot be represented in the literal encoding.
The following diagnostics are emmited by MSVC with \tcode{/execution-charset:ascii}:
\begin{quoteblock}
\begin{codeblock}
static_assert(false, "Your code is on @\emoji{🔥}@");

<source>(1): warning C4566: character represented by universal-character-name
'\u00F0'  cannot be represented in the current code page (20127)
<source>(1): warning C4566: character represented by universal-character-name
'\u0178'  cannot be represented in the current code page (20127)
<source>(1): warning C4566: character represented by universal-character-name
'\u201D' cannot be represented in the current code page (20127)
<source>(1): warning C4566: character represented by universal-character-name
'\u00A5' cannot be represented in the current code page (20127)
<source>(1): error C2338: ????

static_assert(false, u8"Your code is on @\emoji{🔥}@");
<source>(1): error C2002: invalid wide-character constant
\end{codeblock}
\end{quoteblock}
\subsection{\tcode{extern} \& \tcode{asm}}

No compiler support strings with an encoding prefix in extern and asm statements.

\subsection{\tcode{\#line}}

GCC and  Clang do not support encoding prefix in \tcode{\#line} directives.

\section{Future direction}

This proposal does not prevent supporting constant expression in \tcode{static_assert} or attributes in
the future; we can imagine the following grammar:

\begin{bnf}
    \nontermdef{static_assert-declaration}\br
    \keyword{static_assert} \terminal{(} constant-expression \terminal{)} \terminal{;}\br
    \keyword{static_assert} \terminal{(} constant-expression \terminal{,} unevaluated-string \terminal{)} \terminal{;}\br
    \keyword{static_assert} \terminal{(} constant-expression \terminal{,} constant-expression \terminal{)} \terminal{;}
\end{bnf}

Those may make \tcode{static_assert(true, u8"foo");} valid again as \tcode{u8"foo"} would be a valid constant expression.


\section{Implementability}

This proposal requires implementations to keep around a non-encoded string for diagnostic purposes.
This has recently come up in a clang patch to support EBCDIC as the literal encoding.
To support diagnostics in this context, especially on a non-EBCDIC platform the original sequence of characters must be retained.
This proposal offers a well-specified, portable mechanism to solve this problem.

\section{Wording Challenges}

Strings are handled in phases 5 and 6 before the program is parsed,
which might force us to have a "reversal" of these phases.
\grammarterm{string-literal} and \grammarterm{unevaluated-string-literal} only differ by the context in which they may appear.

It is important to note that \grammarterm{unevaluated-string}, by virtue of not being evaluated, are not C++ expressions.
They are purposefully left out of the \grammarterm{literal} grammar.
Not being literal, and not being expressions, \grammarterm{unevaluated-string} do not have a value category.

\section{Previous works}

\paper{P2246R1} removes wording specific to attributes mandating that diagnostic with characters
from the basic characters are displayed in diagnostic messages, which was not implementable.

\section{Wording}

%\rSec1[lex.charset]{Character sets}

%\ednote{Modify "5.3.2" as follow}
%
%A \grammarterm{universal-character-name}
%designates the character in ISO/IEC 10646 (if any)
%whose code point is the hexadecimal number represented by
%the sequence of \grammarterm{hexadecimal-digit}s
%in the \grammarterm{universal-character-name}.
%The program is ill-formed if that number is not a code point
%or if it is a surrogate code point.
%Noncharacter code points and reserved code points
%are considered to designate separate characters distinct from
%any ISO/IEC 10646 character.
%If a \grammarterm{universal-character-name} outside
%the \grammarterm{c-char-sequence}, \grammarterm{s-char-sequence}, or
%\grammarterm{r-char-sequence} of
%a \grammarterm{character-literal}\changed{ or}{,} \grammarterm{string-literal}\added{,\grammarterm{unevaluated-string}}
%(in either case, including within a \grammarterm{user-defined-literal})
%corresponds to a control character or
%to a character in the basic
%source character set, the program is ill-formed.


%\rSec1[lex.pptoken]{Preprocessing tokens}
%
%\ednote{Modify "5.4 Preprocessing tokens" as follow}
%
%\indextext{token!preprocessing|(}%
%\begin{bnf}
%    \nontermdef{preprocessing-token}\br
%    header-name\br
%    import-keyword\br
%    module-keyword\br
%    export-keyword\br
%    identifier\br
%    pp-number\br
%    character-literal\br
%    user-defined-character-literal\br
%    string-literal\br
%    \added{unevaluated-string}\br
%    user-defined-string-literal\br
%    preprocessing-op-or-punc\br
%    \textnormal{each non-whitespace character that cannot be one of the above}
%\end{bnf}
%
%\pnum
%Each preprocessing token that is converted to a token\iref{lex.token}
%shall have the lexical form of a keyword, an identifier, a literal,
%or an operator or punctuator.
%
%\pnum
%A preprocessing token is the minimal lexical element of the language in translation
%phases 3 through 6. The categories of preprocessing token are: header names,
%placeholder tokens produced by preprocessing \tcode{import} and \tcode{module} directives
%(\grammarterm{import-keyword}, \grammarterm{module-keyword}, and \grammarterm{export-keyword}),
%identifiers, preprocessing numbers, character literals (including user-defined character
%literals), string literals (including user-defined string literals \added{and unevaluated string}), preprocessing
%operators and punctuators, and single non-whitespace characters that do not lexically
%match the other preprocessing token categories. If a \tcode{'} or a \tcode{"} character
%matches the last category, the behavior is undefined. Preprocessing tokens can be
%separated by
%\indextext{whitespace}%
%whitespace;
%\indextext{comment}%
%this consists of comments\iref{lex.comment}, or whitespace
%characters (space, horizontal tab, new-line, vertical tab, and
%form-feed), or both. As described in \ref{cpp}, in certain
%circumstances during translation phase 4, whitespace (or the absence
%thereof) serves as more than preprocessing token separation. Whitespace
%can appear within a preprocessing token only as part of a header name or
%between the quotation characters in a character literal or
%string literal..

\ednote{Add after "[lex.string]/p10"}

\begin{addedblock}

\rSec3[lex.string.unevaluated]{Unevaluated strings}

\begin{bnf}
    \nontermdef{unevaluated-string}\br
    string-literal
\end{bnf}

An \grammarterm{unevaluated-string} shall have no  \grammarterm{encoding-prefix}.

Each \grammarterm{universal-character-name} and each \grammarterm{simple-escape-sequence} in an \grammarterm{unevaluated-string} is replaced by the member of the translation character set it denotes.
An \grammarterm{unevaluated-string} that contains a \grammarterm{numeric-escape-sequence}
or a \grammarterm{conditional-escape-sequence} is ill-formed.

An \grammarterm{unevaluated-string} is never evaluated and its interpretation depends on the context in which it appears.
\end{addedblock}

\ednote{"translation set" is defined in \paper{P2314R2} in [lex.phases]}


\rSec1[dcl.dcl]{Declarations}%

\rSec1[dcl.pre]{Preamble}

\begin{bnf}
    \nontermdef{simple-declaration}\br
    decl-specifier-seq \opt{init-declarator-list} \terminal{;}\br
    attribute-specifier-seq decl-specifier-seq init-declarator-list \terminal{;}\br
    \opt{attribute-specifier-seq} decl-specifier-seq \opt{ref-qualifier} \terminal{[} identifier-list \terminal{]} initializer \terminal{;}
\end{bnf}

\begin{bnf}
    \nontermdef{static_assert-declaration}\br
    \keyword{static_assert} \terminal{(} constant-expression \terminal{)} \terminal{;}\br
    \keyword{static_assert} \terminal{(} constant-expression \terminal{,} \added{unevaluated-}string\removed{-literal} \terminal{)} \terminal{;}
\end{bnf}

[...]

\pnum
\indextext{\idxgram{static_assert}}%
In a \grammarterm{static_assert-declaration}, the
is contextually converted to \tcode{bool} and the converted expression shall be a constant expression ([expr.const]).
If the value of the expression when
so converted is \tcode{true}, the declaration has no
effect. Otherwise, the program is ill-formed, and the resulting
diagnostic message\iref{intro.compliance} should include the text of
the \grammarterm{\added{unevaluated-}string\removed{-literal}}, if one is supplied.

\begin{example}
\begin{codeblock}
    static_assert(sizeof(int) == sizeof(void*), "wrong pointer size");
\end{codeblock}\end{example}

\rSec1[dcl.asm]{The \tcode{asm} declaration}%
\indextext{declaration!\idxcode{asm}}%
\indextext{assembler}%
\indextext{\idxcode{asm}!implementation-defined}

\pnum
An \tcode{asm} declaration has the form

\begin{bnf}
    \nontermdef{asm-declaration}\br
    \opt{attribute-specifier-seq} \keyword{asm} \terminal{(} \changed{string-literal}{balanced-token-seq} \terminal{)} \terminal{;}
\end{bnf}

The \tcode{asm} declaration is conditionally-supported; \added{any restrictions on the \grammarterm{balanced-token-seq} and} its meaning \changed{is}{are}
\impldef{meaning of \tcode{asm} declaration}.
%\begin{addedblock}
%The \grammarterm{encoding-prefix} of the \grammarterm{string-literal} shall be absent.
%\end{addedblock}
The optional \grammarterm{attribute-specifier-seq} in
an \grammarterm{asm-declaration} appertains to the \tcode{asm} declaration.
\begin{note}
    Typically it is used to pass information through the implementation to
    an assembler.
\end{note}

\rSec1[dcl.link]{Linkage specifications}%
\indextext{specification!linkage|(}

\pnum
All functions and variables whose names have external linkage
and all function types
have a \defn{language linkage}.
\begin{note}
    Some of the properties associated with an entity with language linkage
    are specific to each implementation and are not described here. For
    example, a particular language linkage might be associated with a
    particular form of representing names of objects and functions with
    external linkage, or with a particular calling convention, etc.
\end{note}
The default language linkage of all function types, functions, and
variables is \Cpp{} language linkage. Two function types with
different language linkages are distinct types even if they are
otherwise identical.

\pnum
Linkage\iref{basic.link} between \Cpp{} and  non-\Cpp{} code fragments can
be achieved using a \grammarterm{linkage-specification}:

\indextext{\idxgram{linkage-specification}}%
\indextext{specification!linkage!\idxcode{extern}}%
%
\begin{bnf}
    \nontermdef{linkage-specification}\br
    \keyword{extern} \added{unevaluated-}string\removed{-literal} \terminal{\{} \opt{declaration-seq} \terminal{\}}\br
    \keyword{extern} \added{unevaluated-}string\removed{-literal} declaration
\end{bnf}

The \grammarterm{\added{unevaluated-}string\removed{-literal}} indicates the required language linkage
\begin{addedblock}
\begin{note}
Escape sequences and \grammarterm{universal-character-name}{s} have been replaced in [lex.string.unevaluated]
\end{note}
\end{addedblock}
.
%\added{as a sequence of translation set characters}.

This document specifies the semantics for the
\added{unevaluated-}string\removed{-literal} \tcode{"C"} and \tcode{"C++"}. Use of a
\added{unevaluated-}string\removed{-literal} other than \tcode{"C"} or \tcode{"C++"} is
conditionally-supported, with \impldef{semantics of linkage specifiers} semantics.
\begin{note}
    Therefore, a linkage-specification with a \changed{\grammarterm{string-literal}}{language linkage} that
    is unknown to the implementation requires a diagnostic.
\end{note}
\begin{note}
    It is recommended that the spelling of the \changed{\grammarterm{string-literal}}{language linkage} be
    taken from the document defining that language. For example, \tcode{Ada}
    (not \tcode{ADA}) and \tcode{Fortran} or \tcode{FORTRAN}, depending on
    the vintage.
\end{note}


\pnum
\indextext{specification!linkage!implementation-defined}%
Every implementation shall provide for linkage to the C programming language,
\indextext{C!linkage to}%
\tcode{"C"}, and \Cpp{}, \tcode{"C++"}.
\begin{example}
    \begin{codeblock}
        complex sqrt(complex);          // \Cpp{} language linkage by default
        extern "C" {
            double sqrt(double);          // C language linkage
        }
    \end{codeblock}
\end{example}

// [...]

\rSec1[dcl.attr]{Attributes}%
\rSec2[dcl.attr.deprecated]{Deprecated attribute}%

\pnum
The \grammarterm{attribute-token} \tcode{deprecated} can be used to mark names and entities
whose use is still allowed, but is discouraged for some reason.
\begin{note}
    In particular,
    \tcode{deprecated} is appropriate for names and entities that are deemed obsolescent or
    unsafe.
\end{note}
It shall appear at most once in each \grammarterm{attribute-list}. An
\grammarterm{attribute-argument-clause} may be present and, if present, it shall have the form:

\begin{ncbnf}
    \terminal{(} \added{unevaluated-}string\removed{-literal} \terminal{)}
\end{ncbnf}

\begin{note}
    The \grammarterm{\added{unevaluated-}string\removed{-literal}} in the \grammarterm{attribute-argument-clause}
    can be used to explain the rationale for deprecation and/or to suggest a replacing entity.
\end{note}

\rSec2[dcl.attr.nodiscard]{Nodiscard attribute}%
\indextext{attribute!nodiscard}

\pnum
The \grammarterm{attribute-token} \tcode{nodiscard}
may be applied to the \grammarterm{declarator-id}
in a function declaration or to the declaration of a class or enumeration.
It shall appear at most once in each \grammarterm{attribute-list}.
An \grammarterm{attribute-argument-clause} may be present
and, if present, shall have the form:

\begin{ncbnf}
    \terminal{(} \added{unevaluated-}string\removed{-literal} \terminal{)}
\end{ncbnf}

\pnum
A name or entity declared without the \tcode{nodiscard} attribute
can later be redeclared with the attribute and vice-versa.
\begin{note}
    Thus, an entity initially declared without the attribute
    can be marked as \tcode{nodiscard}
    by a subsequent redeclaration.
    However, after an entity is marked as \tcode{nodiscard},
    later redeclarations do not remove the \tcode{nodiscard}
    from the entity.
\end{note}
Redeclarations using different forms of the attribute
(with or without the \grammarterm{attribute-argument-clause}
or with different \grammarterm{attribute-argument-clause}s)
are allowed.

\pnum
A \defnadj{nodiscard}{type} is
a (possibly cv-qualified) class or enumeration type
marked \tcode{nodiscard} in a reachable declaration.
A \defnadj{nodiscard}{call} is either
\begin{itemize}
    \item
    a function call expression\iref{expr.call}
    that calls a function declared \tcode{nodiscard} in a reachable declaration or
    whose return type is a nodiscard type, or
    \item
    an explicit type
    conversion~(\ref{expr.type.conv}, \ref{expr.static.cast}, \ref{expr.cast})
    that constructs an object through
    a constructor declared \tcode{nodiscard} in a reachable declaration, or
    that initializes an object of a nodiscard type.
\end{itemize}

Recommended Practice:
Appearance of a nodiscard call as
a potentially-evaluated discarded-value expression\iref{expr.prop}
is discouraged unless explicitly cast to \tcode{void}.
Implementations should issue a warning in such cases.
\begin{note}
    This is typically because discarding the return value
    of a nodiscard call has surprising consequences.
\end{note}
The \grammarterm{\added{unevaluated-}string\removed{-literal}}
in a \tcode{nodiscard} \grammarterm{attribute-argument-clause}
should be used in the message of the warning
as the rationale for why the result should not be discarded.



\rSec1[over.literal]{User-defined literals}%
\indextext{user-defined literal!overloaded}%
\indextext{overloading!user-defined literal}

\begin{bnf}
    \nontermdef{literal-operator-id}\br
    \keyword{operator} \changed{string-literal}{unevaluated-string} identifier\br
    \keyword{operator} user-defined-string-literal
\end{bnf}

\pnum
The \removed{\grammarterm{string-literal} or} \grammarterm{user-defined-string-literal}
in a \grammarterm{literal-operator-id} shall have no
\grammarterm{encoding-prefix}\changed{and}{. The \grammarterm{unevaluated-string} or \grammarterm{user-defined-string-literal}} shall \changed{contain no characters other than the
implicit terminating \tcode{'\textbackslash 0'}}{be empty}.
The \grammarterm{ud-suffix} of the \grammarterm{user-defined-string-literal} or
the \grammarterm{identifier} in a \grammarterm{literal-operator-id} is called a
\defnx{literal suffix identifier}{literal!suffix identifier}.
Some literal suffix identifiers are reserved for future standardization;
see~\ref{usrlit.suffix}.  A declaration whose \grammarterm{literal-operator-id} uses
such a literal suffix identifier is ill-formed, no diagnostic required.




%%\rSec1[cpp]{Preprocessing directives}%

[...]


%\rSec1[cpp.line]{Line control}%
%\indextext{preprocessing directive!line control}%
%\indextext{\idxcode{\#line}|see{preprocessing directive, line control}}
%
%\pnum
%The \grammarterm{string-literal} of a
%\tcode{\#line}
%directive, if present,
%shall \changed{be a character string literal}{satisfy the semantic constraints of an \grammarterm{unevaluated-string} [lex.string.unevaluated]}.
%
%\ednote {
%\grammarterm{string-literal} is prexisting wording, but the grammar refers to a quoted sequence of s-char. This is arguably a bug that CWG may want to address
%as a drive-by when processing this paper.
%}
%
%\pnum
%The
%\defn{line number}
%of the current source line is one greater than
%the number of new-line characters read or introduced
%in translation phase 1\iref{lex.phases}
%while processing the source file to the current token.
%
%\pnum
%A preprocessing directive of the form
%\begin{ncsimplebnf}
%    \terminal{\# line} digit-sequence new-line
%\end{ncsimplebnf}
%causes the implementation to behave as if
%the following sequence of source lines begins with a
%source line that has a line number as specified
%by the digit sequence (interpreted as a decimal integer).
%If the digit sequence specifies zero
%or a number greater than 2147483647,
%the behavior is undefined.
%
%\pnum
%A preprocessing directive of the form
%
%\begin{ncsimplebnf}
%    \terminal{\# line} digit-sequence \terminal{"} \opt{s-char-sequence} \terminal{"} new-line
%\end{ncsimplebnf}
%
%sets the presumed line number similarly and changes the
%presumed name of the source file to be the contents
%of the \removed{character} string literal.
%
%\pnum
%A preprocessing directive of the form
%
%\begin{ncsimplebnf}
%    \terminal{\# line} pp-tokens new-line
%\end{ncsimplebnf}
%
%(that does not match one of the two previous forms)
%is permitted.
%The preprocessing tokens after
%\tcode{line}
%on the directive are processed just as in normal text
%(each identifier currently defined as a macro name is replaced by its
%replacement list of preprocessing tokens).
%If the directive resulting after all replacements does not match
%one of the two previous forms, the behavior is undefined;
%otherwise, the result is processed as appropriate.

%\rSec1[cpp.pragma.op]{Pragma operator}%
%\indextext{macro!pragma operator}%
%\indextext{operator!pragma|see{macro, pragma operator}}
%
%\begin{addedblock}
%The \grammarterm{string-literal} of a Pragma operator shall satisfy the semantic constraints of an \grammarterm{unevaluated-string} [lex.string.unevaluated].
%\end{addedblock}
%
%\pnum
%A unary operator expression of the form:
%
%\begin{ncbnf}
%    \terminal{_Pragma} \terminal{(} string-literal \terminal{)}
%\end{ncbnf}
%
%is processed as follows: The \grammarterm{string-literal}
%is \defnx{destringized}{destringization} by
%\removed{deleting the \tcode{L} prefix, if present,} deleting the leading and trailing
%double-quotes,replacing each escape sequence \tcode{\textbackslash"} by a double-quote, and
%replacing each escape sequence \tcode{\textbackslash\textbackslash} by a single
%backslash. The resulting sequence of characters is processed through translation phase 3
%to produce preprocessing tokens that are executed as if they were the
%\grammarterm{pp-tokens} in a pragma directive. The original four preprocessing
%tokens in the unary operator expression are removed.
%
%\pnum
%\begin{example}
%\begin{codeblock}
%    #pragma listing on "..\listing.dir"
%\end{codeblock}
%can also be expressed as:
%\begin{codeblock}
%    _Pragma ( "listing on \"..\\listing.dir\"" )
%\end{codeblock}
%The latter form is processed in the same way whether it appears literally
%as shown, or results from macro replacement, as in:
%\begin{codeblock}
%    #define LISTING(x) PRAGMA(listing on #x)
%    #define PRAGMA(x) _Pragma(#x)
%
%    LISTING( ..\listing.dir )
%\end{codeblock}
%\end{example}


\section{Acknowledgments}

Thank you to Masayoshi Kanke and Peter Brett who offered valuable feedback on this paper!


\bibliographystyle{plain}
\bibliography{wg21}


\renewcommand{\section}[2]{}%
\begin{thebibliography}{9}

\bibitem[N4885]{N4885}
Thomas Köppe
\emph{Working Draft, Standard for Programming Language C++}\newline
\url{https://wg21.link/N4885}

\end{thebibliography}

\end{document}
