% !TeX program = luatex
% !TEX encoding = UTF-8

\documentclass{wg21}

\title{Relax requirements on \tcode{wchar\_t} to match existing practices}
\docnumber{D2460R2}
\audience{SG-22, LWG}
\author{Corentin Jabot}{corentin.jabot@gmail.com}

\begin{document}
\maketitle

\section{Target}

C++23

\section{Abstract}

We propose to remove the constraints put on the encoding associated with \tcode{wchar_t} in the core wording.


\section{Revisions}

\subsection{R2}
\begin{itemize}
\item Reformulation of the library section.
\end{itemize}

\subsection{R1}
\begin{itemize}
\item Rebase and cleanup the wording.
\end{itemize}

\section{Motivation}

The standard claims that \tcode{wchar_t} should encode all characters of all wide encoding
as a single code unit.
This does not match existing practices, as \tcode{wchar_t} denotes UTF-16 on Windows.
The \href{https://docs.microsoft.com/en-us/windows/win32/learnwin32/working-with-strings}{Windows Documentation} states:

\begin{quoteblock}
Windows represents Unicode characters using UTF-16 encoding, in which each character is encoded as a 16-bit value. UTF-16 characters are called wide characters, to distinguish them from 8-bit ANSI characters. The Visual C++ compiler supports the built-in data type wchar_t for wide characters.
\end{quoteblock}

This is not merely an issue of MSVC being none conforming.
It makes C++ unsuitable for development on a widely deployed operating system.


ISO 10646 also mentions:

\begin{quoteblock}
NOTE – Former editions of this document included references to a two-octet BMP form called UCS-2 which would be a subset of the
UTF-16  encoding form restricted to the BMP UCS scalar values. The UCS-2 form is deprecated.
\end{quoteblock}

Moreover, the requirement that "the values of type wchar_t can represent distinct codes for all
members of the largest extended character set specified among the supported locales" also precludes any 2 bytes encodings (including UCS2),
if (one of) the execution character set is UTF-8, as not all Unicode codepoints (21 bits) are representable in a single 2 bytes \tcode{wchar_t}.


Instead of stating Windows, and environments where UTF-8 is used are non-conforming, which is not useful to users, we propose to remove the constraint
from the core wording.

However, we cannot change the wide functions, both for API/ABI reasons, because they are controlled by C,
and at best, this requires complex surgery.

Instead, we move the constraints from the type of \tcode{wchar_t} to the constraints of the execution encoding,
as defined by \paper{P2314R3}.

Previous discussions can be found in this \href{https://github.com/sg16-unicode/sg16/issues/9}{SG-16 issue}.

\section{Behavior changes}

This paper makes UTF-16 in wide literals well-formed. This does not affect implementations that were already accepting them \href{https://godbolt.org/z/cPe69bshM}{[Compiler Explorer]}.
This paper is therefore standardizing standard practices.

\subsection{What about the library?}

Still the status quo. Further work is needed there.

\subsection{C compat}

C has the same wording.

\begin{quoteblock}
wide character

value representable by an object of type wchar_t, capable of representing any character in the current locale
\end{quoteblock}

C should consider adopting a similar resolution, however, the proposed change has no impact on C compatibility.
(we are removing a constraint).


\section{Previous polls}


\begin{quoteblock}
SG16 POLL: Add expanded motivation to D2460R0 and forward the paper so
revised to EWG with a recommended ship vehicle of C++23.
    
\begin{tabular}{|c|c|c|c|c|}
    \hline
    SF & F & N & A & SA\\
    \hline
     5 & 3 & 1 & 0 & 0 \\
    \hline
\end{tabular}
\end{quoteblock}



\section{Wording}

\ednote{Modify [basic.fundamental] p8 as follow:}

\pnum
\indextext{\idxcode{wchar_t}|see{type, \tcode{wchar_t}}}%
\indextext{type!\idxcode{wchar_t}}%
\indextext{type!underlying!\idxcode{wchar_t}}%
Type \keyword{wchar_t} is a distinct type that has
an \impldef{underlying type of \tcode{wchar_t}}
signed or unsigned integer type as its underlying type.
\removed{The values of type \keyword{wchar_t} can represent
distinct codes for all members of the largest extended character set
specified among the supported locales\iref{locale}.}


\ednote{Change 16.3.3.3.5.1 [character.seq.general] paragraph 1.}

The C standard library makes widespread use of characters and character sequences that follow a few uniform conventions:

\begin{itemize}
\item Properties specified as \defn{locale-specific} may change during program execution by a call to \tcode{setlocale(int, const char*)} (28.5.1 [clocale.syn]), or by a change to a \tcode{locale} object, as described in 28.3 [locales] and Clause 29 [input.output].
\item The \defn{execution character set} and the \defn{execution wide-character set} are supersets of the basic literal character set (5.3 [lex.charset]). The encodings of the execution character sets and the sets of additional elements (if any) are locale-specific. \added{Each element of the execution wide-character set is encoded as a single code unit representable by a value of type \tcode{wchar_t}.} 

\begin{note}
The encoding of the execution character sets can be unrelated to any literal encoding.
\end{note}

\item A letter is any of the 26 lowercase or 26 uppercase letters in the basic execution character set.

\item The decimal-point character is the locale-specific (single-byte) character used by functions that convert between a (single-byte) character sequence and a value of one of the floating-point types. It is used in the character sequence to denote the beginning of a fractional part.It is represented in [support] through [thread] and [depr], ’.’, which is also its value in the "C" locale.
\end{itemize}

\section{Acknowledgment}

Thanks to SG-16 for their feedback on this paper, notably Hubert Tong for mentioning that even UCS-2 does not always satisfy the core wording requirements.

\bibliographystyle{plain}
\bibliography{wg21}


\renewcommand{\section}[2]{}%
\begin{thebibliography}{9}
    \bibitem[N4885]{N4885}
    Thomas Köppe
    \emph{Working Draft, Standard for Programming Language C++}\newline
    \url{https://wg21.link/N4885}
\end{thebibliography}

\end{document}
