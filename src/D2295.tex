% !TeX program = luatex
% !TEX encoding = UTF-8


\RequirePackage{luatex85}


\documentclass{wg21}



\newcommand{\UnicodeLetter}[1]{\textbf{\textcolor{BrickRed}{\Large\tcode{#1}}}}

\title{Correct UTF-8 handling during phase 1 of translation}
\docnumber{P2290R0}
\audience{EWG, SG-22}
\author{Corentin Jabot}{corentin.jabot@gmail.com}

\begin{document}
\maketitle

\paperquote{We should have some notion of a portable C++ source file, and without a known fixed encoding it's hard to argue that such a thing exists - Richard Smith}

\section{Credits}

While many people, notably many people from SG-16 provided feedbacks on the design presented in this paper,
the general direction was originally proposed in 2012 by Beman Dawes in \paper{N3463}.

Thanks Mr Dawes!

\section{Abstract}

We propose that UTF-8 source files should be supported by all C++ compilers.

\section{Motivation}

\begin{quoteblock}
    Even this simple program cannot be written portably in C++ and may fail to compile:
\begin{codeblock}
    int main() {}
\end{codeblock}
 - Beman Dawes
\end{quoteblock}

The set of source file character sets is implementation-defined, which makes writing portable C++ code impossible.
We proposed to mandate that C++ compilers must accept UTF-8 as an input format. Both to increase portability and
to ensure that Unicode related features (ex \paper{P1949R3}) can be used widely.
This would also allow us to better specify how Unicode encoded files are handled.

How the source file encoding is detected, and which other input formats are accepted would remain implementation-defined.
Supporting UTF-8 would also not require mentioning files in the wording, the media providing the stream of UTF-8 data is not important.

Most C++ compilers ( GCC, EDG, MSVC, Clang) support utf8 as one of their input format - Clang only supports utf8.

\section{Previous works}

Original Proposal: \paper{N3463} (Beman Dawes, 2012).
More recent discussions about Normalizxation preservation, wghitepaces ansd UTF in \paper{P2178R1} (Points 1, 2, 3 and 8)

\section{Design}

\subsection{Codepoint preservations}

An important reason to mandate support for UTF-8 support is that it allows us to mandate codepoint preservation during lexing:
The compiler should not try to normalize utf encoded strings, remove characters, and so forth so that users can expect the content
of strings to be preserved through translation.
(This does not negate that neither the source encoding nor the encoding used internally by the compiler are unobservable by the program).

\subsection{Whitespaces}

Unicode recommends that codepoints with the \tcode{Pattern_White_Space} property are recognized by
whitespaces by conforming parsers.
Some sequences of whitespace characters are further described as line breaks in UAX14.

For compatibility with Unicode, UTF-EBCDIC, and a variety of existing systems, implementations, and text editors,
we recommend specifying exactly what constitutes a new-line and a whitespace in C++.
The wording provides the complete list.

New lines affect diagnostics, \tcode{__LINE__} and \tcode{source_location::line()}.
They also affect line-endings in raw string literals, which is the subject of \paper{CWG1655}, which this paper resolves.

\subsection{Invalid code units}

Not all code units or code unit sequences represent valid code points.
For example, in UTF8, some sequences can be overlong(\tcode{0xC0 0xAF}).

In other encodings, some code unit sequences may encode an unassigned code point, which therefore cannot be represented in Unicode.
For example \tcode{0x8 0x00} is an unallocated JIS X 0208:1997 sequence.

In both cases, we believe the program should be ill-formed.
Both these scenarios seldom occur in practice except when the implementation (or the user) infers the wrong encoding.
For example, in \tcode{windows-1251}, \tcode{0xC0} represents the cyrillic letter А (U+0410), which when interpreted
as UTF-8 results in an invalid code unit sequence.

As such, the compiler is interpreting the source file incorrectly and the result of the compilation cannot be trusted,
resulting in compilation failures or bad interpretation of string literals further down the line.

The issue can sometimes be innocuous:
Frequently, non-ASCII characters appear in comments of files that otherwise only contain ASCII, usually when the name of
a maintainer appears in a source file.
However, most mojibake cannot be detected. If the compiler detects invalid characters in comments, there may be undetectable
mojibake in string literals, which would lead to runtime bugs.

Therefore, we think invalid code unit sequences in phase 1 should always be diagnosed.
Our proposed wording makes invalid code unit sequences ill-formed for the UTF-8 encoding.

However this does not necessarilly constitute a breaking change (see the "Impact on implementations" section)

\subsection{A note on unassigned Unicode codepoints}

When mapping from UTF-8 to any representation of a Unicode code point sequence, it is neither necessary to know
whether a codepoint is assigned or not.
The only requirement on codepoints is that they are scalar values.
As such, the proposed change is completely agnostic of the Unicode version.
The only time the Unicode is relevant during lexing is for checking whether an identifier is a valid identifier.

\subsection{BOM}

Unlike \paper{N3463}, we do not intend to mandate or say anything about BOMs (Byte Order Mark), following Unicode recommendations.
BOMs should be ignored (but using BOMs as a means to detect UTF-8 files is a valid implementation strategy, which is notably used by MSVC).
Indeed, we should give enough freedom to implementers to handle the cases where a BOM contradicts a compiler flag.
Web browsers for example found a BOM to not be a reliable mechanism.
There are further issues with BOMs and toolings, such that they may be removed by IDEs or tooling.
They are also widely used, and different companies have different policies.

In any case, mandating BOM recognition goes against Unicode recommendations\footnote{Unicode 13 Standard, and confirmed in a \href{https://corp.unicode.org/mailman/private/unicode/2020-June/008716.html}{mailing list discussion} with members of the Unicode consortium}. But it is a widespread practice, and as such we recommend neither mandating nor precluding any behavior.

For the purpose of translation, a leading BOM is therefore ignored (and doesn't affect the column count of \tcode{source_location::column()}).

\section{Non goals}

Other existing and upcoming papers try to improve various aspects of lexing related to Unicode, and text encoding, some of them described in \paper{P2178R1}
and \paper{P2194R0}.
This paper only focuses on the handling of UTF-8 physical source files.

In particular, it doesn't try to provide a standard  way to specify or detect encoding for the current translation unit or current source file:
This proposal conserves the status quo: The mechanism by which encoded is inferred by the compiler for each source file remains implementation-defined.

We further do not propose to restrict in any way the set of input encodings or "physical source character set"
accepted by compiler beyond mandating that this implementation-defined set contains at least the UTF-8 encoding.

We do not propose a standard mechanism to specify a different encoding per header. This may be explored in a separate paper.

Finally, conservation of code point sequences in phase 5 of translation when encoding a narrow literal character or string in the utf-8
encoding is not proposed in this paper which focuses on phase 1 of translation.




\section{Impact on implementations}

\subsection{UTF-8 support}
MSVC, EDG, Clang, GCC support compiling UTF-8 source files.

\begin{itemize}
\item Currently (Clang 11), Clang only support UTF-8 and assume all files are UTF-8. BOMs are ignored.
\item GCC supports UTF-8 through iconv and the command line flag \tcode{-finput-charset=UTF-8} can be used to interpret source files as UTF-8.
The default encoding is inferred from the environment and fallbacks to UTF-8 when not possible. BOMs are ignored.
\item MSVC supports UTF-8 source files with the \tcode{/source-charset:utf-8} command line flag. MSVC uses UTF-8 by default when a BOM is present.
\end{itemize}


\subsection{Input validation}

Compilers currently have very different strategies for handling invalid input:

\begin{itemize}
\item GCC will ensure that a non-UTF-8 input decodes cleanly and will emit an error otherwise. However, when the input is UTF-8 it is
not decoded at all (GCC uses internally) and so the input is not validated. The handling of UTF-8 is then inconsistent with other encodings.
We don't know if this is intentional.
\item Clang does not check invalid comments. By reading the source code this is very intentional. However invalid Unicode is diagnosed (error) in string literals.
\item By default, MSVC will emit a warning for invalid UTF-8, even in comments
\end{itemize}

\begin{quoteblock}
main.cpp(1): warning \href{https://docs.microsoft.com/en-us/cpp/error-messages/compiler-warnings/compiler-warnings-by-compiler-version?view=msvc-160#warnings-introduced-in-visual-studio-2015-update-2-compiler-version-1900239180}{C4828}: The file contains a character starting at offset 0x2 that is illegal in the current source character set (codepage 65001).
\end{quoteblock}

As such, implementers have two strategies for the implementation of this proposal:

\begin{itemize}
\item Always diagnose invalid code units sequences when interpreting a UTF-8 input.
\item Provide a conforming support for UTF-8 inputs, along with an implementation-defined "UTF-8 like" encoding that would behave like UTF-8 but maybe discard
invalid code units sequences in comments as part of the "implementation-defined mapping" prescribed in phase 1 for non UTF-8 encodings.
\end{itemize}

And so this proposal guarantees that users can have a way to ensure their source code is properly decoded, while giving implementers
the ability to offer more leniant options.

For example, for MSVC, the flags \tcode{/source-charset:utf-8 /we4828} are sufficiant to be conforming with the current proposal.


\subsection{UTF-8 source files is existing practice}

\begin{itemize}
    \item By default, VCPKG compiles all the packages in its repository with \tcode{/utf-8} - which sets utf-8 as the source AND execution encoding (\href{https://github.com/vicroms/vcpkg/blob/master/scripts/toolchains/windows.cmake#L16}{Reference})
    \item Qt source files are UTF-8 - Users of Qt are recommemded to use UTF-8 source files (\href{https://wiki.qt.io/Strings_and_encodings_in_Qt}{Reference})
    \item Chromium is built with UTF-8 (by virtue of being compiled with Clang)
\end{itemize}

\section{SG16 Polls}

See \href{https://github.com/sg16-unicode/sg16-meetings/blob/master/README-2020.md}{Minutes for P2178}

\begin{codeblock}


It should be implementation-defined whether a UTF-8 BOM is used to inform the encoding of a source file
SF	F	N	A	SA
4	3	1	2	0
Consensus is in favor

The presence or absence of a BOM is a reasonable portable mechanism
for detecting UTF-8 source file encoding.

SF	F	N	A	SA
0	1	0	3	6

No consensus;  or rather,
consensus is that a BOM is not a reasonable portable mechanism for detection of source file encoding.


Poll: We agree that, for Unicode source files, that normalization is preserved
through translation phases 1 and 5.

No objection to unanimous consent.

\end{codeblock}

SG-16 was generally in favor of specifying what the whitespace characters are, but no polls were taken.

\section{Wording}

\subsection{A note on wording}

As SG-16 and Core try to improve the terminology and wording used to describe lexing in the standard,
the proposed wording likely collides with other work happening in parallel.

As such, this wording is based on the current standard.
Once EWG accepts the design, a CWG-targeting paper merging the different lexing related design changes may be brought forward.
As such, the wording reflects the intent but not necessarily the terminology that SG-16 wants to use, nor is it intended to be merged in its current form in the working draft.

Notably, SG-16 and CWG are working on removing the mapping of characters to universal-character-name in phase 1, which has no observable impact on the behavior of C++ programs.

The proposed wording assumes P2223R1 has previously been applied to the wording draft


\rSec1[lex.phases]{Phases of translation}%

\pnum
\indextext{translation!phases|(}%
The precedence among the syntax rules of translation is specified by the
following phases.

\begin{enumerate}
    \item

\begin{addedblock}
    A UTF-8 source file is a physical source file encoded as a sequence of UTF-8 code units representing a sequence of Unicode scalar values as defined in ISO/IEC 10646.

    An implementation accepts UTF-8 source files; The set of additional physical source file character sets accepted is implementation-defined.

    A source file shall contain only one physical source file character set; how that character set is determined is implementation-defined.
\end{addedblock}

Physical source file characters are mapped, in an
\impldef{mapping physical source file characters to basic source character set} manner,
to the basic source character set \removed{(introducing new-line characters for end-of-line
indicators) if necessary}.
\removed{The set of physical source file characters accepted is \impldef{physical source file characters}}.
Any source file character not in the basic source character set\iref{lex.charset} is replaced by the \indextext{universal character name}\grammarterm{universal-character-name} that
designates that character.
\added{If the physical source file is a UTF-8 source file, the scalar value of each physical source character shall be preserved.}
An implementation may use any internal
encoding, so long as an actual extended character encountered in the
source file, and the same extended character expressed in the source
file as a \grammarterm{universal-character-name} (e.g., using the \tcode{\textbackslash uXXXX} notation), are handled equivalently except where this replacement is reverted\iref{lex.pptoken} in a raw string literal.

\item
\indextext{line splicing}%
\added{If the first character is U+FEFF BYTE ORDER MARK, it is deleted.}

Each instance of a backslash character (\textbackslash)
immediately followed by zero or more \changed{whitespace characters}{\grammarterm{whitespace}s} (other than \changed{new-line character}{\grammarterm{new-line}}) followed by a \changed{new-line character}{\grammarterm{new-line}} is deleted, splicing
physical source lines to form logical source lines. Only the last
backslash on any physical source line shall be eligible for being part
of such a splice.
Except for splices reverted in a raw string literal, if a splice results in
a character sequence that matches the
syntax of a \grammarterm{universal-character-name}, the behavior is
undefined. A source file that is not empty and that does not end in a \changed{new-line character}{\grammarterm{new-line}}, or that ends in a \changed{new-line character}{\grammarterm{new-line}} immediately preceded by a
backslash character before any such splicing takes place,
shall be processed as if an additional \changed{new-line character}{LINE FEED} were appended
to the file.

\item The source file is decomposed into preprocessing
tokens\iref{lex.pptoken} and sequences of \changed{whitespace characters
(including comments)}{\grammarterm{whitespace} and comments}. A source file shall not end in a partial
preprocessing token or in a partial comment.

Each comment is replaced by one \changed{space}{SPACE} character. \changed{New-line characters}{\grammarterm{new-line}} are
retained. Whether each nonempty sequence of \changed{whitespace characters other
than new-line}{\grammarterm{whitespace} other than \grammarterm{new-line}} is retained or replaced by one \changed{space}{SPACE}  character is
unspecified. The process of dividing a source file's
characters into preprocessing tokens is context-dependent.
\begin{example}
    See the handling of \tcode{<} within a \tcode{\#include} preprocessing
    directive.
\end{example}

\item Preprocessing directives are executed, macro invocations are
expanded, and \tcode{_Pragma} unary operator expressions are executed.
If a character sequence that matches the syntax of a
\grammarterm{universal-character-name} is produced by token
concatenation\iref{cpp.concat}, the behavior is undefined. A
\tcode{\#include} preprocessing directive causes the named header or
source file to be processed from phase 1 through phase 4, recursively.
All preprocessing directives are then deleted.

\item
Each
\grammarterm{basic-c-char},
\grammarterm{basic-s-char}, and
\grammarterm{r-char}
in a \grammarterm{character-literal} or a \grammarterm{string-literal},
as well as each
\grammarterm{escape-sequence} and \grammarterm{universal-character-name}
in a \grammarterm{character-literal} or a non-raw string literal,
is encoded in the literal's associated character encoding as specified in
\ref{lex.ccon} and \ref{lex.string}.

\item
Adjacent \grammarterm{string-literal}s are concatenated
and a null character is appended to the result
as specified in \ref{lex.string}.

\item \changed{White-space characters}{Sequences of \grammarterm{whitespace}} separating tokens are no longer
significant. Each preprocessing token is converted into a
token\iref{lex.token}. The resulting tokens are syntactically and
semantically analyzed and translated as a translation unit.
\begin{note}
    The process of analyzing and translating the tokens can occasionally
    result in one token being replaced by a sequence of other
    tokens\iref{temp.names}.
\end{note}
It is
\impldef{whether the sources for
    module units and header units
    on which the current translation unit has an interface
    dependency are required to be available during translation}
whether the sources for
module units and header units
on which the current translation unit has an interface
dependency (\ref{module.unit}, \ref{module.import})
are required to be available.
\begin{note}
    Source files, translation
    units and translated translation units need not necessarily be stored as
    files, nor need there be any one-to-one correspondence between these
    entities and any external representation. The description is conceptual
    only, and does not specify any particular implementation.
\end{note}
\end{enumerate}

\rSec1[lex.pptoken]{Preprocessing tokens}

\begin{addedblock}
\begin{bnf}
\nontermdef{new-line}\br
    \textnormal{U+000A LINE FEED}\br
    \textnormal{U+000B VERTICAL TAB}\br
    \textnormal{U+000C FORM FEED}\br
    \textnormal{U+000D CARRIAGE RETURN}\br
    \textnormal{U+0085 NEXT LINE}\br
    \textnormal{U+2028 LINE SEPARATOR}\br
    \textnormal{U+2029 PARAGRAPH SEPARATOR}\br
    \textnormal{U+000D U+000A}
\end{bnf}

\begin{bnf}
    \nontermdef{whitespace}\br
    new-line\br
    \textnormal{U+0009 HORIZONTAL TAB}\br
    \textnormal{U+0009 SPACE}\br
  %  \textnormal{U+200E LEFT-TO-RIGHT MARK}\br
  %  \textnormal{U+200F RIGHT-TO-LEFT MARK}
\end{bnf}

Unless otherwise specified, whitespaces and whitespace characters refer to \grammarterm{whitespace}.

\end{addedblock}

\begin{bnf}
    \nontermdef{preprocessing-token}\br
    header-name\br
    import-keyword\br
    module-keyword\br
    export-keyword\br
    identifier\br
    pp-number\br
    character-literal\br
    user-defined-character-literal\br
    string-literal\br
    user-defined-string-literal\br
    preprocessing-op-or-punc\br
    \textnormal{each non-whitespace character that cannot be one of the above}
\end{bnf}

\pnum
Each preprocessing token that is converted to a token\iref{lex.token}
shall have the lexical form of a keyword, an identifier, a literal,
or an operator or punctuator.

\pnum
A preprocessing token is the minimal lexical element of the language in translation
phases 3 through 6. The categories of preprocessing token are: header names,
placeholder tokens produced by preprocessing \tcode{import} and \tcode{module} directives
(\grammarterm{import-keyword}, \grammarterm{module-keyword}, and \grammarterm{export-keyword}),
identifiers, preprocessing numbers, character literals (including user-defined character
literals), string literals (including user-defined string literals), preprocessing
operators and punctuators, and single non-whitespace characters that do not lexically
match the other preprocessing token categories. If a \tcode{'} or a \tcode{"} character
matches the last category, the behavior is undefined. Preprocessing tokens can be
separated by
\indextext{whitespace}%
whitespace;
\indextext{comment}
this consists of comments\iref{lex.comment}, or \changed{whitespace characters}{\grammarterm{whitespace}}
\removed{(space, horizontal tab, new-line, vertical tab, and
form-feed)}, or both. As described in \ref{cpp}, in certain
circumstances during translation phase 4, \changed{whitespace}{\grammarterm{whitespace}} (or the absence
thereof) serves as more than preprocessing token separation. \changed{White space}{\grammarterm{whitespace}}
can appear within a preprocessing token only as part of a header name or
between the quotation characters in a character literal or
string literal.

[..]

\rSec1[lex.comment]{Comments}

\pnum
\indextext{comment|(}%
\indextext{comment!\tcode{/*} \tcode{*/}}%
\indextext{comment!\tcode{//}}%
The characters \tcode{/*} start a comment, which terminates with the
characters \tcode{*/}. These comments do not nest.
\indextext{comment!\tcode{//}}%
The characters \tcode{//} start a comment, which terminates immediately before the
next \changed{new-line character}{\grammarterm{new-line}}. \removed{If there is a form-feed or a vertical-tab
character in such a comment, only whitespace characters shall appear
between it and the new-line that terminates the comment; no diagnostic
is required}.
\begin{note}
    The comment characters \tcode{//}, \tcode{/*},
    and \tcode{*/} have no special meaning within a \tcode{//} comment and
    are treated just like other characters. Similarly, the comment
    characters \tcode{//} and \tcode{/*} have no special meaning within a
    \tcode{/*} comment.
\end{note}
\indextext{comment|)}

\rSec1[lex.header]{Header names}

\indextext{header!name|(}%
\begin{bnf}
    \nontermdef{header-name}\br
    \terminal{<} h-char-sequence \terminal{>}\br
    \terminal{"} q-char-sequence \terminal{"}
\end{bnf}

\begin{bnf}
    \nontermdef{h-char-sequence}\br
    h-char\br
    h-char-sequence h-char
\end{bnf}

\begin{bnf}
    \nontermdef{h-char}\br
    \textnormal{any member of the source character set except \changed{new-line}{\grammarterm{new-line}} and \terminal{>}}
\end{bnf}

\begin{bnf}
    \nontermdef{q-char-sequence}\br
    q-char\br
    q-char-sequence q-char
\end{bnf}

\begin{bnf}
    \nontermdef{q-char}\br
    \textnormal{any member of the source character set except \changed{new-line}{\grammarterm{new-line}} and \terminal{"}}
\end{bnf}


\rSec2[lex.ccon]{Character literals}

[..]

\begin{bnf}
    \nontermdef{c-char}\br
    basic-c-char\br
    escape-sequence\br
    universal-character-name
\end{bnf}

\begin{bnf}
    \nontermdef{basic-c-char}\br
    \textnormal{any member of the basic source character set except the single-quote \terminal{'}, backslash \terminal{\textbackslash}, or \changed{new-line character}{\grammarterm{new-line}}}
\end{bnf}

[...]

\begin{floattable}{Simple escape sequences}{lex.ccon.esc}
    {lll}
\topline
\changed{new-line}{LINE FEED}       &   NL(LF)          &   \tcode{\textbackslash n}                \\
horizontal tab  &   HT              &   \tcode{\textbackslash t}                \\
vertical tab    &   VT              &   \tcode{\textbackslash v}                \\
backspace       &   BS              &   \tcode{\textbackslash b}                \\
carriage return &   CR              &   \tcode{\textbackslash r}                \\
form feed       &   FF              &   \tcode{\textbackslash f}                \\
alert           &   BEL             &   \tcode{\textbackslash a}                \\
backslash       &   \textbackslash  &   \tcode{\textbackslash\textbackslash}    \\
question mark   &   ?               &   \tcode{\textbackslash ?}                \\
single quote    &   \tcode{'}       &   \tcode{\textbackslash '}                \\
double quote    &   \tcode{"}       &   \tcode{\textbackslash "}                \\
\end{floattable}

\rSec2[lex.string]{String literals}

[...]

\begin{bnf}
    \nontermdef{s-char}\br
    basic-s-char\br
    escape-sequence\br
    universal-character-name
\end{bnf}

\begin{bnf}
    \nontermdef{basic-s-char}\br
    \textnormal{any member of the basic source character set except the double-quote \terminal{"}, backslash \terminal{\textbackslash}, or \changed{new-line character}{\grammarterm{new-line}}}
\end{bnf}

[..]

A \grammarterm{string-literal} that has an \tcode{R}
\indextext{prefix!\idxcode{R}}%
in the prefix is a \defn{raw string literal}. The
\grammarterm{d-char-sequence} serves as a delimiter. The terminating
\grammarterm{d-char-sequence} of a \grammarterm{raw-string} is the same sequence of
characters as the initial \grammarterm{d-char-sequence}. A \grammarterm{d-char-sequence}
shall consist of at most 16 characters.

\begin{addedblock}
Each instance of a \grammarterm{new-line} in the \defn{raw string literal} is replaced by a \tcode{LINE FEED (U+000A).}
\end{addedblock}

\begin{quoteblock}
Is this the behavior that we want?
\end{quoteblock}

\pnum
\begin{note}
    The characters \tcode{'('} and \tcode{')'} are permitted in a
    \grammarterm{raw-string}. Thus, \tcode{R"delimiter((a|b))delimiter"} is equivalent to
    \tcode{"(a|b)"}.
\end{note}

\pnum
\begin{note}
    \begin{removedblock}
    A source-file new-line in a raw string literal results in a new-line in the
    resulting execution string literal.
   \end{removedblock}
 Assuming no
    whitespace at the beginning of lines in the following example, the assert will succeed:
    \begin{codeblock}
        const char* p = R"(a\
        b
        c)";
        assert(std::strcmp(p, "a\\\nb\nc") == 0);
    \end{codeblock}
\end{note}


\subsection{Preprocessing directives}%
\indextext{preprocessing directive|(}

\indextext{compiler control line|see{preprocessing directive}}%
\indextext{control line|see{preprocessing directive}}%
\indextext{directive, preprocessing|see{preprocessing directive}}

\gramSec[gram.cpp]{Preprocessing directives}

\rSec1[cpp.pre]{Preamble}

\begin{bnf}
    \nontermdef{preprocessing-file}\br
    \opt{group}\br
    module-file
\end{bnf}

\begin{bnf}
    \nontermdef{module-file}\br
    \opt{pp-global-module-fragment} pp-module \opt{group} \opt{pp-private-module-fragment}
\end{bnf}

\begin{bnf}
    \nontermdef{pp-global-module-fragment}\br
    \keyword{module} \terminal{;} new-line \opt{group}
\end{bnf}

\begin{bnf}
    \nontermdef{pp-private-module-fragment}\br
    \keyword{module} \terminal{:} \keyword{private} \terminal{;} new-line \opt{group}
\end{bnf}

\begin{bnf}
    \nontermdef{group}\br
    group-part\br
    group group-part
\end{bnf}

\begin{bnf}
    \nontermdef{group-part}\br
    control-line\br
    if-section\br
    text-line\br
    \terminal{\#} conditionally-supported-directive
\end{bnf}

\begin{bnf}\obeyspaces
    \nontermdef{control-line}\br
    \terminal{\# include} pp-tokens new-line\br
    pp-import\br
    \terminal{\# define } identifier replacement-list new-line\br
    \terminal{\# define } identifier lparen \opt{identifier-list} \terminal{)} replacement-list new-line\br
    \terminal{\# define } identifier lparen \terminal{... )} replacement-list new-line\br
    \terminal{\# define } identifier lparen identifier-list \terminal{, ... )} replacement-list new-line\br
    \terminal{\# undef  } identifier new-line\br
    \terminal{\# line   } pp-tokens new-line\br
    \terminal{\# error  } \opt{pp-tokens} new-line\br
    \terminal{\# pragma } \opt{pp-tokens} new-line\br
    \terminal{\# }new-line
\end{bnf}

\begin{bnf}
    \nontermdef{if-section}\br
    if-group \opt{elif-groups} \opt{else-group} endif-line
\end{bnf}

\begin{bnf}\obeyspaces
    \nontermdef{if-group}\br
    \terminal{\# if     } constant-expression new-line \opt{group}\br
    \terminal{\# ifdef  } identifier new-line \opt{group}\br
    \terminal{\# ifndef } identifier new-line \opt{group}
\end{bnf}

\begin{bnf}
    \nontermdef{elif-groups}\br
    elif-group\br
    elif-groups elif-group
\end{bnf}

\begin{bnf}\obeyspaces
    \nontermdef{elif-group}\br
    \terminal{\# elif   } constant-expression new-line \opt{group}
\end{bnf}

\begin{bnf}\obeyspaces
    \nontermdef{else-group}\br
    \terminal{\# else   } new-line \opt{group}
\end{bnf}

\begin{bnf}\obeyspaces
    \nontermdef{endif-line}\br
    \terminal{\# endif  } new-line
\end{bnf}

\begin{bnf}
    \nontermdef{text-line}\br
    \opt{pp-tokens} new-line
\end{bnf}

\begin{bnf}
    \nontermdef{conditionally-supported-directive}\br
    pp-tokens new-line
\end{bnf}

\begin{bnf}
    \nontermdef{lparen}\br
    \descr{a \terminal{(} character not immediately preceded by whitespace}
\end{bnf}

\begin{bnf}
    \nontermdef{identifier-list}\br
    identifier\br
    identifier-list \terminal{,} identifier
\end{bnf}

\begin{bnf}
    \nontermdef{replacement-list}\br
    \opt{pp-tokens}
\end{bnf}

\begin{bnf}
    \nontermdef{pp-tokens}\br
    preprocessing-token\br
    pp-tokens preprocessing-token
\end{bnf}

\begin{removedblock}
\begin{bnf}
    \nontermdef{new-line}\br
    \descr{the new-line character}
\end{bnf}
\end{removedblock}

\pnum
A \defn{preprocessing directive} consists of a sequence of preprocessing tokens
that satisfies the following constraints:
At the start of translation phase 4,
the first token in the sequence,
referred to as a \defnadj{directive-introducing}{token},
begins with the first character in the source file
(optionally after whitespace containing no \changed{new-line characters}{\grammarterm{new-line}}) or
follows whitespace containing at least one \changed{new-line character}{\grammarterm{new-line}},
and is

\begin{itemize}
    \item
    a \tcode{\#} preprocessing token, or

    \item
    an \keyword{import} preprocessing token
    immediately followed on the same logical line by a
    \grammarterm{header-name},
    \tcode{<},
    \grammarterm{identifier},
    \grammarterm{string-literal}, or
    \tcode{:}
    preprocessing token, or

    \item
    a \keyword{module} preprocessing token
    immediately followed on the same logical line by an
    \grammarterm{identifier},
    \tcode{:}, or
    \tcode{;}
    preprocessing token, or

    \item
    an \keyword{export} preprocessing token
    immediately followed on the same logical line by
    one of the two preceding forms.
\end{itemize}

The last token in the sequence is the first token within the sequence that
is immediately followed by whitespace containing a \changed{new-line character}{\grammarterm{new-line}}.
\begin{note}
    A \changed{new-line character}{\grammarterm{new-line}} ends the preprocessing directive even if it occurs
    within what would otherwise be an invocation of a function-like macro.
\end{note}

\begin{example}
    \begin{codeblock}
        #                       // preprocessing directive
        module ;                // preprocessing directive
        export module leftpad;  // preprocessing directive
        import <string>;        // preprocessing directive
        export import "squee";  // preprocessing directive
        import rightpad;        // preprocessing directive
        import :part;           // preprocessing directive

        module                  // not a preprocessing directive
        ;                       // not a preprocessing directive

        export                  // not a preprocessing directive
        import                  // not a preprocessing directive
        foo;                    // not a preprocessing directive

        export                  // not a preprocessing directive
        import foo;             // preprocessing directive (ill-formed at phase 7)

        import ::               // not a preprocessing directive
        import ->               // not a preprocessing directive
    \end{codeblock}
\end{example}

\pnum
A sequence of preprocessing tokens is only a \grammarterm{text-line}
if it does not begin with a directive-introducing token.
A sequence of preprocessing tokens is only a \grammarterm{conditionally-supported-directive}
if it does not begin with any of the directive names
appearing after a \tcode{\#} in the syntax.
A \grammarterm{conditionally-supported-directive} is
conditionally-supported with
\impldef{additional supported forms of preprocessing directive}
semantics.

\pnum
At the start of phase 4 of translation,
the \grammarterm{group} of a \grammarterm{pp-global-module-fragment} shall
contain neither a \grammarterm{text-line} nor a \grammarterm{pp-import}.

\pnum
When in a group that is skipped\iref{cpp.cond}, the directive
syntax is relaxed to allow any sequence of preprocessing tokens to occur between
the directive name and the following \changed{new-line character}{\grammarterm{new-line}}.

\pnum
The only whitespace characters that shall appear
between preprocessing tokens
within a preprocessing directive
(from just after the directive-introducing token
through just before the terminating \changed{new-line character}{\grammarterm{new-line}})
are \changed{space}{SPACE} and \changed{horizontal-tab}{HOROZONTAL TAB}
(including \changed{spaces}{SPACE} that have replaced comments
or possibly other whitespace characters
in translation phase 3).

[..]

\pnum
A preprocessing directive of the form
\begin{ncsimplebnf}
    \terminal{\# define} identifier replacement-list new-line
    \indextext{\idxcode{\#define}}%
\end{ncsimplebnf}
defines an
\defnadj{object-like}{macro} that
causes each subsequent instance of the macro name
to be replaced by the replacement list of preprocessing tokens
that constitute the remainder of the directive.
The replacement list is then rescanned for more macro names as
specified below.

\pnum
\begin{example}
    The simplest use of this facility is to define a ``manifest constant'',
    as in
    \begin{codeblock}
        #define TABSIZE 100
        int table[TABSIZE];
    \end{codeblock}
\end{example}

\pnum
A preprocessing directive of the form
\begin{ncsimplebnf}
    \terminal{\# define} identifier lparen \opt{identifier-list} \terminal{)} replacement-list new-line\br
    \terminal{\# define} identifier lparen \terminal{...} \terminal{)} replacement-list new-line\br
    \terminal{\# define} identifier lparen identifier-list \terminal{, ...} \terminal{)} replacement-list new-line
\end{ncsimplebnf}
defines a \defnadj{function-like}{macro}
with parameters, whose use is
similar syntactically to a function call.
The parameters
\indextext{parameter!macro}%
are specified by the optional list of identifiers.
Each subsequent instance of the function-like macro name followed by a
\tcode{(}
as the next preprocessing token
introduces the sequence of preprocessing tokens that is replaced
by the replacement list in the definition
(an invocation of the macro).
\indextext{invocation!macro}%
The replaced sequence of preprocessing tokens is terminated by the matching
\tcode{)}
preprocessing token, skipping intervening matched pairs of left and
right parenthesis preprocessing tokens.
Within the sequence of preprocessing tokens making up an invocation
of a function-like macro, \changed{new-line is considered a normal whitespace character}{\grammarterm{new-line} are considered like other \grammarterm{whitespace}}.

\pnum
\indextext{macro!function-like!arguments}%
The sequence of preprocessing tokens
bounded by the outside-most matching parentheses
forms the list of arguments for the function-like macro.
The individual arguments within the list
are separated by comma preprocessing tokens,
but comma preprocessing tokens between matching
inner parentheses do not separate arguments.
If there are sequences of preprocessing tokens within the list of
arguments that would otherwise act as preprocessing directives,
the behavior is undefined.

[...]

\pnum
The
\defn{line number}
of the current source line is one greater than
the number of \changed{new-line}{\grammarterm{new-line}} characters read or introduced
in translation phase 1\iref{lex.phases}
while processing the source file to the current token.

[...]


\subsection{Time library}

\rSec1[time.format]{Formatting}


\begin{LongTable}{Meaning of conversion specifiers}{time.format.spec}{lx{.8\hsize}}
    \\ \topline
    \lhdr{Specifier} & \rhdr{Replacement} \\ \capsep
    \endfirsthead
    \continuedcaption\\
    \hline
    \lhdr{Specifier} & \rhdr{Replacement} \\ \capsep
    \endhead
    \\ \rowsep
    \tcode{\%M} &
    The minute as a decimal number.
    If the result is a single digit, it is prefixed with \tcode{0}.
    The modified command \tcode{\%OM} produces
    the locale's alternative representation.
    \\ \rowsep
    \tcode{\%n} &
    A \changed{new-line}{LINE FEED} character.
    \\ \rowsep
    \tcode{\%p} &
    The locale's equivalent of the AM/PM designations associated with a 12-hour clock.
    \\
\end{LongTable}




\section{References}
\renewcommand{\section}[2]{}%
\bibliographystyle{plain}
\bibliography{wg21}

\begin{thebibliography}{9}


    \bibitem[Unicode]{Unicode}
    Unicode 13\newline
    \url{http://www.unicode.org/versions/Unicode13.0.0/}


    \bibitem[N4861]{N4861}
    Richard Smith
    \emph{Working Draft, Standard for Programming Language C++}\newline
    \url{https://wg21.link/N4861}


    \bibitem[UAX-14]{UAX-14}
    \emph{UNICODE LINE BREAKING ALGORITHM}\newline
    \url{https://www.unicode.org/reports/tr14/}

     \bibitem[UAX-31]{UAX-31}
    \emph{UNICODE IDENTIFIER AND PATTERN SYNTAX}\newline
    \url{https://www.unicode.org/reports/tr31/}

\end{thebibliography}
\end{document}
