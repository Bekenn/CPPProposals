% !TeX document-id = {9322a846-f757-4574-9231-a2e85c743b21}
% !TeX program = luatex
% !TEX encoding = UTF-8


\RequirePackage{luatex85}%
\documentclass{wg21}

\usepackage{luatexja-fontspec}
\RequirePackageWithOptions{fontspec}
\usepackage{newunicodechar}

\setmainfont{Noto Sans}

\title{Whitespaces Wording Revamp}
\docnumber{P2348R2}
\audience{EWG, CWG}
\author{Corentin Jabot}{corentin.jabot@gmail.com}

\newcommand{\cwhitespace}[1]{\removed{#1} \added{\grammarterm{whitespace}}}
\newcommand{\cwhitespaces}[1]{\removed{#1} \added{\grammarterm{whitespace}{s}}}
\newcommand{\chwhitespace}[1]{\removed{#1} \added{\grammarterm{horizontal-whitespace}}}
\newcommand{\chwhitespaces}[1]{\removed{#1} \added{\grammarterm{horizontal-whitespace}{s}}}
\newcommand{\clb}[1]{\removed{#1} \added{\grammarterm{line-break}}}
\newcommand{\clbs}[1]{\removed{#1} \added{\grammarterm{line-break}{s}}}
\newcommand{\clbc}[1]{\removed{#1} \added{\grammarterm{line-break-character}}}

\begin{document}

\maketitle

\section{Revisions}

\subsection*{Revision 2}
\begin{itemize}
    \item Wording fixes
\end{itemize}

\subsection*{Revision 1}
\begin{itemize}
    \item The previous revisions classified Vertical Tab and Form Feed as vertical spaces. This was consistent with Unicode, but not with the current wording, nor with any of the existing implementations. As this paper is trying not to modify the status quo, and because the author cannot find a reason to challenge the status quo, this version treats these codepoints as horizontal whitespaces.
    \item Add a note about \tcode{\textbackslash n\textbackslash r} and other not spelled-out sequences.
    \item Align the grammar in [lex.whitespaces] with terminology used in \paper{P2314R2} and rebase the rest of the wording on top of  \paper{P2314R2}. 
\end{itemize}

\section{Design}

This paper aims to clarify what constitutes a new-line and a whitespace, using Unicode terminology (which is consistent with \paper{P2314R1}).
No breaking behavior change is intended.

\textbf{This paper resolves \paper{CWG2002} and  \paper{CWG1655}}.

\subsection{Make whitespace grammar elements}

For clarity, this paper introduces a grammar for whitespaces, including comments, and then refers to whitespace as grammar terms.
This also makes it easier to extend the list of whitespaces later.
However, the question of extending the list of Unicode characters
treated as whitespace is not explored in this paper.
Non-Unicode source encoding would continue to map their set of whitespaces and new-line to what is currently designated in the standard: CR, FF, LF, VT, SPACE, TAB.

\subsection{Comments}

The proposed wording change makes vertical tabs and form-feed allowed in // comments, rather than make them ill-formed no diagnostic required.

\subsection{VT and FF are treated as horizontal whitespaces}

In R0, \tcode{VT} \tcode{FF} were treated as vertical, line-breaking whitespaces. This was consistent with Unicode.
However, I failed to realize that all implementations treat them as non-breaking whitespaces.
As this paper is not trying to challenge the status quo, which frankly would have little value, it classifies \tcode{VT} and \tcode{FF}
as horizontal whitespaces. As such, an implementation must support them in comments and string/character literals, which is consistent with implementations.

\href{https://compiler-explorer.com/z/9ahz7W7xc}{Compiler explorer}.

Despite the contradition with Unicode, further research showed that treating vertical space and form feed is standard practice in many languages
including C\#, Rust, Java, JavaScript and others.


\subsection{\textbackslash n\textbackslash r}

On BBC Micro and other Acorn systems, \tcode{\textbackslash n\textbackslash r} was used to delimited new lines.
Similarly, some systems may not use any of the sequences for new lines outlined in this paper.
This is consistent with both the status quo and Unicode.
Assuming once could run a modern compiler on a BBC micro (note that this proposal only affects translation, not execution),
the implementation-defined mapping provision allows an implementation to map platform/encoding specific line breaks to \textbackslash n in phase 1
of translation.

Similarly, while this paper tries to preserve sequences of line breaks through phases 1-6 of translation, it is not observable
whether a sequence of line breaks is treated as one or several line-breaks except in the value of \tcode{__LINE__} and \tcode{source_location} whose values are implementation-defined

\subsection{Replace the term \grammarterm{new-line} by \grammarterm{line-break}}

\tcode{new-line} in the wording refers to both an unspecified character to which all source line terminator map to and
a specific, implementation-defined character in the literal character set.
Using different terminology makes it clearer which is which.

\subsection{Do not perform whitespace replacement}

All whitespaces (including comments) are conserved by the proposed wording through phases 1-7.
In practice, this is not observable (and we may want a note stating that somewhere).

Note that some compilers (namely GCC) will emit a diagnostic for the presence of vertical tabs
in preprocessing directives.
With this, such diagnostic in pedantic mode is not necessary for corformance, but might still
be useful [\href{https://godbolt.org/z/r8hdzM6Gx}{Compiler Explorer}].


This fixes \paper{CWG2002}.

\begin{quoteblock}
According to Clause 15 [cpp] paragraphg 4,

\begin{quoteblock}
The only whitespace characters that shall appear between preprocessing tokens within a preprocessing directive (from just after the introducing \# preprocessing token through just before the terminating new-line character) are space and horizontal-tab (including spaces that have replaced comments or possibly other whitespace characters in translation phase 3).
\end{quoteblock}
The effect of this restriction is unclear, however, since translation phase 3 is permitted to transform all white space characters and comments into spaces. The relationship between these two rules should be clarified.
\end{quoteblock}

The list of whitespaces is sufficient to address the needs of
codepoints conservations through phases 1-6 of \paper{P2295R2}.
The clause mentioned in CWG2002 is deleted.

\subsection{The set of character is not expanded}

We do not propose new characters as whitespaces.
However, following Unicode guidelines, we clarify that CRLF is to be considered a single line break.

\subsection{Raw-string literals}

For clarity, that line-breaks are mapped to a new-line in raw-string-literal is made normative.
We intend that this paper addresses \paper{CWG1655}: by using different terms for \grammarterm{line-break} and \grammarterm{new-line}, we make clear that these things are not related.

\section{Wording}
\rSec1[lex.phases]{Phases of translation}%

\ednote{The wording is based on \paper{P2314R2}}

\pnum
\indextext{translation!phases|(}%
The precedence among the syntax rules of translation is specified by the
following phases.
\begin{enumerate}
    \item
    \indextext{character set!basic source}%
    \changed{Physical source file characters are mapped, in an
    \impldef{mapping physical source file characters to basic source character set} manner,
    to the translation character set (introducing new-line characters for end-of-line
    indicators)}{The physical source file is mapped, in an implementation-defined manner, to a sequence of translation character set elements}.

    \ednote{The intent of this reformulation is to get rid of the term "end-of-line-indicator" which is not defined,
    while supporting source files stored as records.}

    The set of physical source file characters accepted is \impldef{physical source file
        characters}.

    \item
    \indextext{line splicing}%
    Each instance of a backslash character (\textbackslash)
    immediately followed by a \clb{new-line character} is deleted, splicing
    physical source lines to form logical source lines. Only the last
    backslash on any physical source line shall be eligible for being part
    of such a splice.
    Except for splices reverted in a raw string literal, if a splice results in
    a character sequence that matches the
    syntax of a \grammarterm{universal-character-name}, the behavior is
    undefined. A source file that is not empty and that does not end in a \clb{new-line
    character}, or that ends in a \clb{new-line character} immediately preceded by a
    backslash character before any such splicing takes place,
    shall be processed as if an additional \clb{new-line character} were appended
    to the file.

    \item The source file is decomposed into preprocessing
    tokens\iref{lex.pptoken} and \cwhitespaces{sequences of whitespace characters
    (including comments)}. A source file shall not end in a partial
    preprocessing token or in a partial comment.
    \removed{Each comment is replaced by one space character.} 
    %\changed{New-line characters are}{Each \grammarterm{line-break} is} retained. 
    \removed{New-line characters are retained.} 
    \removed{Whether each nonempty sequence of whitespace characters other
    than new-line is retained or replaced by one space character is
    unspecified.}  Each \grammarterm{universal-character-name} outside of a header-name or a character or string literal is replaced by the designated element of the translation character set ([lex.charset]). The process of dividing a source file's
    characters into preprocessing tokens is context-dependent.
    \begin{example}
        See the handling of \tcode{<} within a \tcode{\#include} preprocessing
        directive.
    \end{example}

    \item Preprocessing directives are executed, macro invocations are
    expanded, and \tcode{_Pragma} unary operator expressions are executed.
    If a character sequence that matches the syntax of a
    \grammarterm{universal-character-name} is produced by token
    concatenation\iref{cpp.concat}, the behavior is undefined. A
    \tcode{\#include} preprocessing directive causes the named header or
    source file to be processed from phase 1 through phase 4, recursively.
    All preprocessing directives are then deleted.

    \item
    Each
    \grammarterm{basic-c-char},
    \grammarterm{basic-s-char}, and
    \grammarterm{r-char}
    in a \grammarterm{character-literal} or a \grammarterm{string-literal},
    as well as each
    \grammarterm{escape-sequence} and \grammarterm{universal-character-name}
    in a \grammarterm{character-literal} or a non-raw string literal,
    is encoded in the literal's associated character encoding as specified in
    \ref{lex.ccon} and \ref{lex.string}.

    \item
    Adjacent \grammarterm{string-literal}s are concatenated
    and a null character is appended to the result
    as specified in \ref{lex.string}.

    \item \added{The} \cwhitespaces{Whitespace characters} separating tokens are no longer
    significant. Each preprocessing token is converted into a
    token\iref{lex.token}. The resulting tokens are syntactically and
    semantically analyzed and translated as a translation unit.
    \begin{note}
        The process of analyzing and translating the tokens can occasionally
        result in one token being replaced by a sequence of other
        tokens\iref{temp.names}.
    \end{note}
    It is
    \impldef{whether the sources for
        module units and header units
        on which the current translation unit has an interface
        dependency are required to be available during translation}
    whether the sources for
    module units and header units
    on which the current translation unit has an interface
    dependency (\ref{module.unit}, \ref{module.import})
    are required to be available.
    \begin{note}
        Source files, translation
        units and translated translation units need not necessarily be stored as
        files, nor need there be any one-to-one correspondence between these
        entities and any external representation. The description is conceptual
        only, and does not specify any particular implementation.
    \end{note}
\end{enumerate}

\rSec1[lex.pptoken]{Preprocessing tokens}

\indextext{token!preprocessing|(}%
\begin{bnf}
    \nontermdef{preprocessing-token}\br
    header-name\br
    import-keyword\br
    module-keyword\br
    export-keyword\br
    identifier\br
    pp-number\br
    character-literal\br
    user-defined-character-literal\br
    string-literal\br
    user-defined-string-literal\br
    preprocessing-op-or-punc\br
    \textnormal{each \changed{non-whitespace character}{character that cannot be part of a \grammarterm{whitespace} and} cannot be one of the above}
\end{bnf}

\pnum
Each preprocessing token that is converted to a token\iref{lex.token}
shall have the lexical form of a keyword, an identifier, a literal,
or an operator or punctuator.

\pnum
A preprocessing token is the minimal lexical element of the language in translation
phases 3 through 6. In this document, glyphs are used to identify elements of the basic character set ([lex.charset]) The categories of preprocessing token are: header names,
placeholder tokens produced by preprocessing \tcode{import} and \tcode{module} directives
(\grammarterm{import-keyword}, \grammarterm{module-keyword}, and \grammarterm{export-keyword}),
identifiers, preprocessing numbers, character literals (including user-defined character
literals), string literals (including user-defined string literals), preprocessing
operators and punctuators, and single non-whitespace characters that do not lexically
match the other preprocessing token categories. If a U+0027 APOSTROPHE or a U+0022 QUOTATION MARK character
matches the last category, the behavior is undefined. Preprocessing tokens can be
separated by \cwhitespaces{whitespace;}\added{.}
\indextext{comment}%
\removed{this consists of comments\iref{lex.comment}, or whitespace
characters (U+0020 SPACE, U+0009 CHARACTER TABULATION, new-line, U+000B LINE TABULATION, and U+000C FORM FEED), or both}. As described in [cpp], in certain
circumstances during translation phase 4, \cwhitespaces{whitespace} (or the absence
thereof) serves as more than preprocessing token separation. \changed{Whitespace}{Whitespace characters}
can appear within a preprocessing token only as part of a header name or
between the quotation characters in a character literal or
string literal.

\ednote{"Whitespace characters" is intentional here: this may refer to unicode characters not considered whitespaces for the purpose of lexing, and do not refer to elements used to separate tokens.}

\rSec1[lex.token]{Tokens}

\indextext{token|(}%
\begin{bnf}
    \nontermdef{token}\br
    identifier\br
    keyword\br
    literal\br
    operator-or-punctuator
\end{bnf}

\pnum
\indextext{\idxgram{token}}%
\removed{There are five kinds of tokens: identifiers, keywords, literals,%
operators, and other separators}.

\ednote{This is somewhat of a drive-by partial fix of \paper{CWG1901}.
However the use of "separator" is somewhat missleading here. Was it meant to be mean whitespace?
The following sentence is explicitely stating that whitespace are not tokens.
}

\begin{removedblock}
\indextext{whitespace}%
Blanks, horizontal and vertical tabs, new-lines, form-feeds, and comments
(collectively, ``whitespace''), as described below, are ignored except
as they serve to separate tokens.
\end{removedblock}
\begin{note}
    Some \cwhitespace{whitespace} is
    required to separate otherwise adjacent identifiers, keywords, numeric
    literals, and alternative tokens containing alphabetic characters.
\end{note}
\indextext{token|)}



\begin{addedblock}

\rSec1[lex.whitespaces]{Whitespaces}

\begin{bnf}
    \nontermdef{whitespace}\br
    horizontal-whitespace\br
    line-break\br
\end{bnf}

\begin{bnf}
    \nontermdef{horizontal-whitespace}\br
    horizontal-whitespace-character\br
    comment\br
\end{bnf}

\begin{bnf}
    \nontermdef{comment}\br
    single-line-comment\br
    multi-line-comment
\end{bnf}

\begin{bnf}
    \nontermdef{single-line-comment}\br
    \textnormal{//}\br
    single-line-comment single-line-comment-elem\br
\end{bnf}

\begin{bnf}
    \nontermdef{single-line-comment-elem}\br
    \textnormal{any member of the translation character set except any character that matches} line-break-character\br
\end{bnf}

\begin{bnf}
    \nontermdef{multi-line-comment}\br
    \textnormal{/*}  \opt{multi-line-comment-elem-seq} \textnormal{*/}\br
\end{bnf}

\begin{bnf}
    \nontermdef{multi-line-comment-elem-seq}\br
    multi-line-comment-elem\br
    multi-line-comment-elem-seq multi-line-comment-elem    
\end{bnf}

\begin{bnf}
    \nontermdef{multi-line-comment-elem}\br
    \textnormal{any member of the translation character set except * followed by /}\br
\end{bnf}

\begin{bnf}
    \nontermdef{line-break}\br
    line-break-character\br
    \textnormal{U+000D CARRIAGE RETURN followed by U+000A LINE FEED}\br
\end{bnf}

\begin{bnf}
    \nontermdef{line-break-character}\br
    \textnormal{U+000A LINE FEED}\br
    \textnormal{U+000D CARRIAGE RETURN}\br
\end{bnf}

\begin{bnf}
    \nontermdef{horizontal-whitespace-character}\br
    \textnormal{U+0009 HORIZONTAL TAB}\br
    \textnormal{U+000C FORM FEED}\br
    \textnormal{U+000B VERTICAL TAB}\br
    \textnormal{U+0020 SPACE}\br
\end{bnf}

A \tcode{whitespace} is the longest sequence of characters that could constitute a \tcode{whitespace}.

\end{addedblock}

\begin{removedblock}
\rSec2[lex.comment]{Comments}
\end{removedblock}
\pnum
\removed{The characters \tcode{/*} start a comment, which terminates with the
characters \tcode{*/}.} 
%\ednote{This is retained to explain that a ine comment cannot contain the sequence \tcode{*/}}  These comments do not nest.
\indextext{comment!\tcode{//}}%
\removed{The characters \tcode{//} start a comment, which terminates immediately before the
next new-line character. If there is a form-feed or a vertical-tab
character in such a comment, only whitespace characters shall appear
between it and the new-line that terminates the comment; no diagnostic
is required.}
\begin{note}
    The comment characters \tcode{//}, \tcode{/*},
    and \tcode{*/} have no special meaning within a \tcode{//} comment and
    are treated just like other characters. Similarly, the comment
    characters \tcode{//} and \tcode{/*} have no special meaning within a
    \tcode{/*} comment.
\end{note}

%\begin{addedblock}
%\grammarterm{whitespace}{s} are ignored except as they serve to separate tokens.
%\end{addedblock}

\indextext{comment|)}

\rSec1[lex.header]{Header names}

\indextext{header!name|(}%
\begin{bnf}
    \nontermdef{header-name}\br
    \terminal{<} h-char-sequence \terminal{>}\br
    \terminal{"} q-char-sequence \terminal{"}
\end{bnf}

\begin{bnf}
    \nontermdef{h-char-sequence}\br
    h-char\br
    h-char-sequence h-char
\end{bnf}

\begin{bnf}
    \nontermdef{h-char}\br
    \textnormal{any member of the source character set except \added{any character that matches} \clbc{new-line} and \terminal{>}}
\end{bnf}

\begin{bnf}
    \nontermdef{q-char-sequence}\br
    q-char\br
    q-char-sequence q-char
\end{bnf}

\begin{bnf}
    \nontermdef{q-char}\br
    \textnormal{any member of the source character set except \added{any character that matches} \clbc{new-line} and \terminal{"}}
\end{bnf}

\pnum
\begin{note}
    Header name preprocessing tokens only appear within
    a \tcode{\#include} preprocessing directive,
    a \tcode{__has_include} preprocessing expression, or
    after certain occurrences of an \tcode{import} token
    (see~\ref{lex.pptoken}).
\end{note}
The sequences in both forms of \grammarterm{header-name}{s} are mapped in an
\impldef{mapping header name to header or external source file} manner to headers or to
external source file names as specified in~\ref{cpp.include}.

\pnum
The appearance of either of the characters \tcode{'} or \tcode{\textbackslash} or of
either of the character sequences \tcode{/*} or \tcode{//} in a
\grammarterm{q-char-sequence} or an \grammarterm{h-char-sequence}
is conditionally-supported with \impldef{meaning of \tcode{'}, \tcode{\textbackslash},
    \tcode{/*}, or \tcode{//} in a \grammarterm{q-char-sequence} or an
    \grammarterm{h-char-sequence}} semantics, as is the appearance of the character
\tcode{"} in an \grammarterm{h-char-sequence}.

\rSec2[lex.ccon]{Character literals}

\indextext{literal!character}%
\begin{bnf}
    \nontermdef{character-literal}\br
    \opt{encoding-prefix} \terminal{'} c-char-sequence \terminal{'}
\end{bnf}

\begin{bnf}
    \nontermdef{encoding-prefix} \textnormal{one of}\br
    \terminal{u8}\quad\terminal{u}\quad\terminal{U}\quad\terminal{L}
\end{bnf}

\begin{bnf}
    \nontermdef{c-char-sequence}\br
    c-char\br
    c-char-sequence c-char
\end{bnf}

\begin{bnf}
    \nontermdef{c-char}\br
    basic-c-char\br
    escape-sequence\br
    universal-character-name
\end{bnf}

\begin{bnf}
    \nontermdef{basic-c-char}\br
    \textnormal{any member of the basic source character set except the single-quote \terminal{'}, backslash \terminal{\textbackslash}, or \added{any character that matches} \clbc{new-line character}}
\end{bnf}

\begin{bnf}
    \nontermdef{escape-sequence}\br
    simple-escape-sequence\br
    numeric-escape-sequence\br
    conditional-escape-sequence
\end{bnf}

\begin{bnf}
    \nontermdef{simple-escape-sequence}\br
    \terminal{\textbackslash} simple-escape-sequence-char
\end{bnf}

\begin{bnf}
    \nontermdef{simple-escape-sequence-char} \textnormal{one of}\br
    \terminal{'  "  ?  \textbackslash{} a  b  f  n  r  t  v}
\end{bnf}

\begin{bnf}
    \nontermdef{numeric-escape-sequence}\br
    octal-escape-sequence\br
    hexadecimal-escape-sequence
\end{bnf}

\begin{bnf}
    \nontermdef{octal-escape-sequence}\br
    \terminal{\textbackslash} octal-digit\br
    \terminal{\textbackslash} octal-digit octal-digit\br
    \terminal{\textbackslash} octal-digit octal-digit octal-digit
\end{bnf}

\begin{bnf}
    \nontermdef{hexadecimal-escape-sequence}\br
    \terminal{\textbackslash x} hexadecimal-digit\br
    hexadecimal-escape-sequence hexadecimal-digit
\end{bnf}

\begin{bnf}
    \nontermdef{conditional-escape-sequence}\br
    \terminal{\textbackslash} conditional-escape-sequence-char
\end{bnf}

\begin{bnf}
    \nontermdef{conditional-escape-sequence-char}\br
    \textnormal{any member of the basic source character set that is not an} octal-digit\textnormal{, a} simple-escape-sequence-char\textnormal{, or the characters \terminal{u}, \terminal{U}, or \terminal{x}}
\end{bnf}

//...

\pnum
\indextext{backslash character}%
\indextext{\idxcode{\textbackslash}|see{backslash character}}%
\indextext{escape character|see{backslash character}}%
The character specified by a \grammarterm{simple-escape-sequence}
is specified in \tref{lex.ccon.esc}.
\begin{note}
    Using an escape sequence for a question mark
    is supported for compatibility with ISO C++ 2014 and ISO C.
\end{note}

\begin{floattable}{Simple escape sequences}{lex.ccon.esc}
    {lll}
    \topline
    new-line        &   NL(LF)          &   \tcode{\textbackslash n}                \\
    horizontal tab  &   HT              &   \tcode{\textbackslash t}                \\
    vertical tab    &   VT              &   \tcode{\textbackslash v}                \\
    backspace       &   BS              &   \tcode{\textbackslash b}                \\
    carriage return &   CR              &   \tcode{\textbackslash r}                \\
    form feed       &   FF              &   \tcode{\textbackslash f}                \\
    alert           &   BEL             &   \tcode{\textbackslash a}                \\
    backslash       &   \textbackslash  &   \tcode{\textbackslash\textbackslash}    \\
    question mark   &   ?               &   \tcode{\textbackslash ?}                \\
    single quote    &   \tcode{'}       &   \tcode{\textbackslash '}                \\
    double quote    &   \tcode{"}       &   \tcode{\textbackslash "}                \\
\end{floattable}

\rSec2[lex.string]{String literals}

\indextext{literal!string}%
\begin{bnf}
    \nontermdef{string-literal}\br
    \opt{encoding-prefix} \terminal{"} \opt{s-char-sequence} \terminal{"}\br
    \opt{encoding-prefix} \terminal{R} raw-string
\end{bnf}

\begin{bnf}
    \nontermdef{s-char-sequence}\br
    s-char\br
    s-char-sequence s-char
\end{bnf}

\begin{bnf}
    \nontermdef{s-char}\br
    basic-s-char\br
    escape-sequence\br
    universal-character-name
\end{bnf}

\begin{bnf}
    \nontermdef{basic-s-char}\br
    \textnormal{any member of the translation character set except the double-quote \terminal{"}, backslash \terminal{\textbackslash}, or \added{any character that matches} \clbc{new-line character}}
\end{bnf}

\begin{bnf}
    \nontermdef{raw-string}\br
    \terminal{"} \opt{d-char-sequence} \terminal{(} \opt{r-char-sequence} \terminal{)} \opt{d-char-sequence} \terminal{"}
\end{bnf}

\begin{bnf}
    \nontermdef{r-char-sequence}\br
    r-char\br
    r-char-sequence r-char
\end{bnf}

\begin{bnf}
    \nontermdef{r-char}\br
    \textnormal{any member of the source character set, except a right parenthesis \terminal{)} followed by}\br
    \bnfindent\textnormal{the initial \grammarterm{d-char-sequence} (which may be empty) followed by a double quote \terminal{"}.}
\end{bnf}

\begin{bnf}
    \nontermdef{d-char-sequence}\br
    d-char\br
    d-char-sequence d-char
\end{bnf}

\begin{bnf}
\nontermdef{d-char}\br
\textnormal{any member of the basic source character set except:}\br
\changed{\textnormal{space}}{any charachter that matches horizontal-whitespace-character, \\ or \clbc{new-line character}}, \textnormal{the left parenthesis \terminal{(}, the right parenthesis \terminal{)}, the backslash \terminal{\textbackslash}\removed{, and the control characters\br
\textnormal{representing horizontal tab, vertical tab, form feed, and newline.}}}
\end{bnf}

\pnum
\indextext{literal!string!raw}%
A \grammarterm{string-literal} that has an \tcode{R}
\indextext{prefix!\idxcode{R}}%
in the prefix is a \defn{raw string literal}. The
\grammarterm{d-char-sequence} serves as a delimiter. The terminating
\grammarterm{d-char-sequence} of a \grammarterm{raw-string} is the same sequence of
characters as the initial \grammarterm{d-char-sequence}. A \grammarterm{d-char-sequence}
shall consist of at most 16 characters.

\pnum
\begin{note}
    The characters \tcode{'('} and \tcode{')'} are permitted in a
    \grammarterm{raw-string}. Thus, \tcode{R"delimiter((a|b))delimiter"} is equivalent to
    \tcode{"(a|b)"}.
\end{note}

\pnum
\removed{[\textit{Note:}}\space \removed{A} \added{Each longest sequence of characters that matches the grammar of a} \clb{source-file new-line} in a raw string literal \changed{results in a new-line in the
    resulting execution string literal}{denotes a new-line}. Assuming no
    whitespace at the beginning of lines in the following example, the assert will succeed:
    \begin{codeblock}
        const char* p = R"(a\
        b
        c)";
        assert(std::strcmp(p, "a\\\nb\nc") == 0);
    \end{codeblock}
\removed{\textit{--- end note}]}

\pnum
\begin{example}
    The raw string
    \begin{codeblock}
        R"a(
        )\
        a"
        )a"
    \end{codeblock}
    is equivalent to \tcode{"\textbackslash n)\textbackslash \textbackslash \textbackslash na\textbackslash"\textbackslash n"}. The raw string
    \begin{codeblock}
        R"(x = "\"y\"")"
    \end{codeblock}
    is equivalent to \tcode{"x = \textbackslash "\textbackslash\textbackslash\textbackslash "y\textbackslash\textbackslash\textbackslash "\textbackslash ""}.
\end{example}


\section{[cpp]}

\gramSec[gram.cpp]{Preprocessing directives}

\rSec1[cpp.pre]{Preamble}

\begin{bnf}
    \nontermdef{preprocessing-file}\br
    \opt{group}\br
    module-file
\end{bnf}

\begin{bnf}
    \nontermdef{module-file}\br
    \opt{pp-global-module-fragment} pp-module \opt{group} \opt{pp-private-module-fragment}
\end{bnf}

\begin{bnf}
    \nontermdef{pp-global-module-fragment}\br
    \keyword{module} \terminal{;} \clb{new-line} \opt{group}
\end{bnf}

\begin{bnf}
    \nontermdef{pp-private-module-fragment}\br
    \keyword{module} \terminal{:} \keyword{private} \terminal{;} \clb{new-line} \opt{group}
\end{bnf}

\begin{bnf}
    \nontermdef{group}\br
    group-part\br
    group group-part
\end{bnf}

\begin{bnf}
    \nontermdef{group-part}\br
    control-line\br
    if-section\br
    text-line\br
    \terminal{\#} conditionally-supported-directive
\end{bnf}

\begin{bnf}\obeyspaces
    \nontermdef{control-line}\br
    \terminal{\# include} pp-tokens \clb{new-line}\br
    pp-import\br
    \terminal{\# define } identifier replacement-list \clb{new-line}\br
    \terminal{\# define } identifier lparen \opt{identifier-list} \terminal{)} replacement-list \clb{new-line}\br
    \terminal{\# define } identifier lparen \terminal{... )} replacement-list \clb{new-line}\br
    \terminal{\# define } identifier lparen identifier-list \terminal{, ... )} replacement-list \clb{new-line}\br
    \terminal{\# undef  } identifier \clb{new-line}\br
    \terminal{\# line   } pp-tokens \clb{new-line}\br
    \terminal{\# error  } \opt{pp-tokens} \clb{new-line}\br
    \terminal{\# pragma } \opt{pp-tokens} \clb{new-line}\br
    \terminal{\# }\clb{new-line}
\end{bnf}

\begin{bnf}
    \nontermdef{if-section}\br
    if-group \opt{elif-groups} \opt{else-group} endif-line
\end{bnf}

\begin{bnf}\obeyspaces
    \nontermdef{if-group}\br
    \terminal{\# if     } constant-expression \clb{new-line} \opt{group}\br
    \terminal{\# ifdef  } identifier \clb{new-line} \opt{group}\br
    \terminal{\# ifndef } identifier \clb{new-line} \opt{group}
\end{bnf}

\begin{bnf}
    \nontermdef{elif-groups}\br
    elif-group\br
    elif-groups elif-group
\end{bnf}

\begin{bnf}\obeyspaces
    \nontermdef{elif-group}\br
    \terminal{\# elif   } constant-expression \clb{new-line} \opt{group}
\end{bnf}

\begin{bnf}\obeyspaces
    \nontermdef{else-group}\br
    \terminal{\# else   } \clb{new-line} \opt{group}
\end{bnf}

\begin{bnf}\obeyspaces
    \nontermdef{endif-line}\br
    \terminal{\# endif  } \clb{new-line}
\end{bnf}

\begin{bnf}
    \nontermdef{text-line}\br
    \opt{pp-tokens} \clb{new-line}
\end{bnf}

\begin{bnf}
    \nontermdef{conditionally-supported-directive}\br
    pp-tokens \clb{new-line}
\end{bnf}

\begin{bnf}
    \nontermdef{lparen}\br
    \descr{a \terminal{(} character not immediately preceded by \cwhitespace{whitespace}}
\end{bnf}

\begin{bnf}
    \nontermdef{identifier-list}\br
    identifier\br
    identifier-list \terminal{,} identifier
\end{bnf}

\begin{bnf}
    \nontermdef{replacement-list}\br
    \opt{pp-tokens}
\end{bnf}

\begin{bnf}
    \nontermdef{pp-tokens}\br
    preprocessing-token\br
    pp-tokens preprocessing-token
\end{bnf}

\begin{removedblock}
\begin{bnf}
    \nontermdef{new-line}\br
    \descr{the new-line character}
\end{bnf}
\end{removedblock}

\pnum
A \defn{preprocessing directive} consists of a sequence of preprocessing tokens
that satisfies the following constraints:
At the start of translation phase 4,
the first token in the sequence,
referred to as a \defnadj{directive-introducing}{token},
begins with the first \changed{character in the source file
(optionally after whitespace containing no new-line characters)}{non-whitespace} or
follows \added{a sequence of} \cwhitespaces{whitespace} containing at least one \clb{new-line character},
and is

\begin{itemize}
    \item
    a \tcode{\#} preprocessing token, or

    \item
    an \keyword{import} preprocessing token
    immediately followed on the same logical line by a
    \grammarterm{header-name},
    \tcode{<},
    \grammarterm{identifier},
    \grammarterm{string-literal}, or
    \tcode{:}
    preprocessing token, or

    \item
    a \keyword{module} preprocessing token
    immediately followed on the same logical line by an
    \grammarterm{identifier},
    \tcode{:}, or
    \tcode{;}
    preprocessing token, or

    \item
    an \keyword{export} preprocessing token
    immediately followed on the same logical line by
    one of the two preceding forms.
\end{itemize}

The last token in the sequence is the first token within the sequence that
is immediately followed by \cwhitespaces{whitespace} containing a \clb{new-line character}.
\begin{note}
    Thus, preprocessing directives are commonly called ``lines''.
    These ``lines'' have no other syntactic significance,
    as all \cwhitespace{whitespace} is equivalent except in certain situations
    during preprocessing (see the
    \tcode{\#}
    character string literal creation operator in~\ref{cpp.stringize}, for example).
\end{note}
\begin{note}
    A \clb{new-line character} ends the preprocessing directive even if it occurs
    within what would otherwise be an invocation of a function-like macro.
\end{note}

\begin{example}
    \begin{codeblock}
        #                       // preprocessing directive
        module ;                // preprocessing directive
        export module leftpad;  // preprocessing directive
        import <string>;        // preprocessing directive
        export import "squee";  // preprocessing directive
        import rightpad;        // preprocessing directive
        import :part;           // preprocessing directive

        module                  // not a preprocessing directive
        ;                       // not a preprocessing directive

        export                  // not a preprocessing directive
        import                  // not a preprocessing directive
        foo;                    // not a preprocessing directive

        export                  // not a preprocessing directive
        import foo;             // preprocessing directive (ill-formed at phase 7)

        import ::               // not a preprocessing directive
        import ->               // not a preprocessing directive
    \end{codeblock}
\end{example}

\pnum
A sequence of preprocessing tokens is only a \grammarterm{text-line}
if it does not begin with a directive-introducing token.
A sequence of preprocessing tokens is only a \grammarterm{conditionally-supported-directive}
if it does not begin with any of the directive names
appearing after a \tcode{\#} in the syntax.
A \grammarterm{conditionally-supported-directive} is
conditionally-supported with
\impldef{additional supported forms of preprocessing directive}
semantics.

\pnum
At the start of phase 4 of translation,
the \grammarterm{group} of a \grammarterm{pp-global-module-fragment} shall
contain neither a \grammarterm{text-line} nor a \grammarterm{pp-import}.

\pnum
When in a group that is skipped\iref{cpp.cond}, the directive
syntax is relaxed to allow any sequence of preprocessing tokens to occur between
the directive name and the following \clb{new-line character}.

\pnum
\begin{removedblock}
The only whitespace characters that shall appear
between preprocessing tokens
within a preprocessing directive
(from just after the directive-introducing token
through just before the terminating new-line character)
are space and horizontal-tab
(including spaces that have replaced comments
or possibly other whitespace characters
in translation phase 3).
\end{removedblock}

\begin{addedblock}
Only \grammarterm{horizontal-whitespace}{s} can appear between preprocessing tokens within a preprocessing directive.
\end{addedblock}

\ednote{multi-line comments containing line-breaks are considered atomic \grammarterm{horizontal-whitespace}{s} after phase 3}

\pnum
The implementation can
process and skip sections of source files conditionally,
include other source files,
import macros from header units,
and replace macros.
These capabilities are called
\defn{preprocessing},
because conceptually they occur
before translation of the resulting translation unit.

\pnum
The preprocessing tokens within a preprocessing directive
are not subject to macro expansion unless otherwise stated.

\begin{example}
    In:
    \begin{codeblock}
        #define EMPTY
        EMPTY   #   include <file.h>
    \end{codeblock}
    the sequence of preprocessing tokens on the second line is \textit{not}
    a preprocessing directive, because it does not begin with a \tcode{\#} at the start of
    translation phase 4, even though it will do so after the macro \tcode{EMPTY}
    has been replaced.
\end{example}

\rSec1[cpp.cond]{Conditional inclusion}%
\indextext{preprocessing directive!conditional inclusion}%
\indextext{inclusion!conditional|see{preprocessing directive, conditional inclusion}}

\indextext{\idxcode{defined}}%
\begin{bnf}
    \nontermdef{defined-macro-expression}\br
    \terminal{defined} identifier\br
    \terminal{defined (} identifier \terminal{)}
\end{bnf}

\begin{bnf}
    \nontermdef{h-preprocessing-token}\br
    \textnormal{any \grammarterm{preprocessing-token} other than \terminal{>}}
\end{bnf}

\begin{bnf}
    \nontermdef{h-pp-tokens}\br
    h-preprocessing-token\br
    h-pp-tokens h-preprocessing-token
\end{bnf}

\begin{bnf}
    \nontermdef{header-name-tokens}\br
    string-literal\br
    \terminal{<} h-pp-tokens \terminal{>}
\end{bnf}

\indextext{\idxxname{has_include}}%
\begin{bnf}
    \nontermdef{has-include-expression}\br
    \terminal{\xname{has_include}} \terminal{(} header-name \terminal{)}\br
    \terminal{\xname{has_include}} \terminal{(} header-name-tokens \terminal{)}
\end{bnf}

\indextext{\idxxname{has_cpp_attribute}}%
\begin{bnf}
    \nontermdef{has-attribute-expression}\br
    \terminal{\xname{has_cpp_attribute} (} pp-tokens \terminal{)}
\end{bnf}

\pnum
The expression that controls conditional inclusion
shall be an integral constant expression except that
identifiers
(including those lexically identical to keywords)
are interpreted as described below
\begin{footnote}
    Because the controlling constant expression is evaluated
    during translation phase 4,
    all identifiers either are or are not macro names ---
    there simply are no keywords, enumeration constants, etc.
\end{footnote}
and it may contain zero or more \grammarterm{defined-macro-expression}{s} and/or
\grammarterm{has-include-expression}{s} and/or
\grammarterm{has-attribute-expression}{s} as unary operator expressions.

\pnum
A \grammarterm{defined-macro-expression} evaluates to \tcode{1}
if the identifier is currently defined
as a macro name
(that is, if it is predefined
or if it has one or more active macro definitions\iref{cpp.import},
for example because
it has been the subject of a
\tcode{\#define}
preprocessing directive
without an intervening
\tcode{\#undef}
directive with the same subject identifier), \tcode{0} if it is not.

\pnum
The second form of \grammarterm{has-include-expression}
is considered only if the first form does not match,
in which case the preprocessing tokens are processed just as in normal text.

\pnum
The header or source file identified by
the parenthesized preprocessing token sequence
in each contained \grammarterm{has-include-expression}
is searched for as if that preprocessing token sequence
were the \grammarterm{pp-tokens} in a \tcode{\#include} directive,
except that no further macro expansion is performed.
If such a directive would not satisfy the syntactic requirements
of a \tcode{\#include} directive, the program is ill-formed.
The \grammarterm{has-include-expression} evaluates
to \tcode{1} if the search for the source file succeeds, and
to \tcode{0} if the search fails.

\pnum
Each \grammarterm{has-attribute-expression} is replaced by
a non-zero \grammarterm{pp-number}
matching the form of an \grammarterm{integer-literal}
if the implementation supports an attribute
with the name specified by interpreting
the \grammarterm{pp-tokens}, after macro expansion,
as an \grammarterm{attribute-token},
and by \tcode{0} otherwise.
The program is ill-formed if the \grammarterm{pp-tokens}
do not match the form of an \grammarterm{attribute-token}.

\pnum
For an attribute specified in this document,
the value of the \grammarterm{has-attribute-expression}
is given by \tref{cpp.cond.ha}.
For other attributes recognized by the implementation,
the value is
\impldef{value of \grammarterm{has-attribute-expression}
    for non-standard attributes}.
\begin{note}
    It is expected
    that the availability of an attribute can be detected by any non-zero result.
\end{note}

\begin{floattable}{\xname{has_cpp_attribute} values}{cpp.cond.ha}
    {ll}
    \topline
    \lhdr{Attribute} & \rhdr{Value} \\ \rowsep
    \tcode{carries_dependency}    & \tcode{200809L} \\
    \tcode{deprecated}            & \tcode{201309L} \\
    \tcode{fallthrough}           & \tcode{201603L} \\
    \tcode{likely}                & \tcode{201803L} \\
    \tcode{maybe_unused}          & \tcode{201603L} \\
    \tcode{no_unique_address}     & \tcode{201803L} \\
    \tcode{nodiscard}             & \tcode{201907L} \\
    \tcode{noreturn}              & \tcode{200809L} \\
    \tcode{unlikely}              & \tcode{201803L} \\
\end{floattable}

\pnum
The \tcode{\#ifdef} and \tcode{\#ifndef} directives, and
the \tcode{defined} conditional inclusion operator,
shall treat \xname{has_include} and \xname{has_cpp_attribute}
as if they were the names of defined macros.
The identifiers \xname{has_include} and \xname{has_cpp_attribute}
shall not appear in any context not mentioned in this subclause.

\pnum
Each preprocessing token that remains (in the list of preprocessing tokens that
will become the controlling expression)
after all macro replacements have occurred
shall be in the lexical form of a token\iref{lex.token}.

\pnum
Preprocessing directives of the forms
\begin{ncsimplebnf}\obeyspaces
    \indextext{\idxcode{\#if}}%
    \terminal{\# if     } constant-expression \clb{new-line} \opt{group}\br
    \indextext{\idxcode{\#elif}}%
    \terminal{\# elif   } constant-expression \clb{new-line} \opt{group}
\end{ncsimplebnf}
check whether the controlling constant expression evaluates to nonzero.

\pnum
Prior to evaluation,
macro invocations in the list of preprocessing tokens
that will become the controlling constant expression
are replaced
(except for those macro names modified by the
\tcode{defined}
unary operator),
just as in normal text.
If the token
\tcode{defined}
is generated as a result of this replacement process
or use of the
\tcode{defined}
unary operator does not match one of the two specified forms
prior to macro replacement,
the behavior is undefined.

\pnum
After all replacements due to macro expansion and
evaluations of
\grammarterm{defined-macro-expression}s,
\grammarterm{has-include-expression}s, and
\grammarterm{has-attribute-expression}s
have been performed,
all remaining identifiers and keywords,
except for
\tcode{true}
and
\tcode{false},
are replaced with the \grammarterm{pp-number}
\tcode{0},
and then each preprocessing token is converted into a token.
\begin{note}
    An alternative
    token\iref{lex.digraph} is not an identifier,
    even when its spelling consists entirely of letters and underscores.
    Therefore it is not subject to this replacement.
\end{note}

\pnum
The resulting tokens comprise the controlling constant expression
which is evaluated according to the rules of~\ref{expr.const}
using arithmetic that has at least the ranges specified
in~\ref{support.limits}. For the purposes of this token conversion and evaluation
all signed and unsigned integer types
act as if they have the same representation as, respectively,
\tcode{intmax_t} or \tcode{uintmax_t}\iref{cstdint}.
\begin{note}
    Thus on an
    implementation where \tcode{std::numeric_limits<int>::max()} is \tcode{0x7FFF}
    and \tcode{std::numeric_limits<unsigned int>::max()} is \tcode{0xFFFF},
    the integer literal \tcode{0x8000} is signed and positive within a \tcode{\#if}
    expression even though it is unsigned in translation phase
    7\iref{lex.phases}.
\end{note}
This includes interpreting \grammarterm{character-literal}s, which may involve
converting escape sequences into execution character set members.
Whether the numeric value for these \grammarterm{character-literal}s
matches the value obtained when an identical \grammarterm{character-literal}
occurs in an expression
(other than within a
\tcode{\#if}
or
\tcode{\#elif}
directive)
is \impldef{numeric values of \grammarterm{character-literal}s in \tcode{\#if}
    directives}.
\begin{note}
    Thus, the constant expression in the following
    \tcode{\#if}
    directive and
    \tcode{if} statement\iref{stmt.if}
    is not guaranteed to evaluate to the same value in these two
    contexts:
    \begin{codeblock}
        #if 'z' - 'a' == 25
        if ('z' - 'a' == 25)
    \end{codeblock}
\end{note}
Also, whether a single-character \grammarterm{character-literal} may have a negative
value is \impldef{negative value of \grammarterm{character-literal} in preprocessor}.
Each subexpression with type
\tcode{bool}
is subjected to integral promotion before processing continues.

\pnum
Preprocessing directives of the forms
\begin{ncsimplebnf}\obeyspaces
    \terminal{\# ifdef  } identifier \clb{new-line} \opt{group}\br
    \indextext{\idxcode{\#ifdef}}%
    \terminal{\# ifndef } identifier \clb{new-line} \opt{group}
    \indextext{\idxcode{\#ifndef}}%
\end{ncsimplebnf}
check whether the identifier is or is not currently defined as a macro name.
Their conditions are equivalent to
\tcode{\#if}
\tcode{defined}
\grammarterm{identifier}
and
\tcode{\#if}
\tcode{!defined}
\grammarterm{identifier}
respectively.

\pnum
Each directive's condition is checked in order.
If it evaluates to false (zero),
the group that it controls is skipped:
directives are processed only through the name that determines
the directive in order to keep track of the level
of nested conditionals;
the rest of the directives' preprocessing tokens are ignored,
as are the other preprocessing tokens in the group.
Only the first group
whose control condition evaluates to true (nonzero) is processed;
any following groups are skipped and their controlling directives
are processed as if they were in a group that is skipped.
If none of the conditions evaluates to true,
and there is a
\tcode{\#else}
\indextext{\idxcode{\#else}}%
directive,
the group controlled by the
\tcode{\#else}
is processed; lacking a
\tcode{\#else}
directive, all the groups until the
\tcode{\#endif}
\indextext{\idxcode{\#endif}}%
are skipped.%
\begin{footnote}
    As indicated by the syntax,
    a preprocessing token cannot follow a
    \tcode{\#else}
    or
    \tcode{\#endif}
    directive before the terminating \clb{new-line character}.
    However,
    comments can appear anywhere in a source file,
    including within a preprocessing directive.
\end{footnote}

\pnum
\begin{example}
    This demonstrates a way to include a library \tcode{optional} facility
    only if it is available:

    \begin{codeblock}
        #if __has_include(<optional>)
        #  include <optional>
        #  if __cpp_lib_optional >= 201603
        #    define have_optional 1
        #  endif
        #elif __has_include(<experimental/optional>)
        #  include <experimental/optional>
        #  if __cpp_lib_experimental_optional >= 201411
        #    define have_optional 1
        #    define experimental_optional 1
        #  endif
        #endif
        #ifndef have_optional
        #  define have_optional 0
        #endif
    \end{codeblock}
\end{example}

\pnum
\begin{example}
    This demonstrates a way to use the attribute \tcode{[[acme::deprecated]]}
    only if it is available.
    \begin{codeblock}
        #if __has_cpp_attribute(acme::deprecated)
        #  define ATTR_DEPRECATED(msg) [[acme::deprecated(msg)]]
        #else
        #  define ATTR_DEPRECATED(msg) [[deprecated(msg)]]
        #endif
        ATTR_DEPRECATED("This function is deprecated") void anvil();
    \end{codeblock}
\end{example}

\rSec1[cpp.include]{Source file inclusion}
\indextext{preprocessing directive!header inclusion}
\indextext{preprocessing directive!source-file inclusion}
\indextext{inclusion!source file|see{preprocessing directive, source-file inclusion}}%
\indextext{\idxcode{\#include}}%

\pnum
A
\tcode{\#include}
directive shall identify a header or source file
that can be processed by the implementation.

\pnum
A preprocessing directive of the form
\begin{ncsimplebnf}
    \terminal{\# include <} h-char-sequence \terminal{>} \clb{new-line}
\end{ncsimplebnf}
searches a sequence of
\impldef{sequence of places searched for a header}
places
for a header identified uniquely by the specified sequence
between the
\tcode{<}
and
\tcode{>}
delimiters,
and causes the replacement of that
directive by the entire contents of the header.
How the places are specified
or the header identified
is \impldef{search locations for \tcode{<>} header}.

\pnum
A preprocessing directive of the form
\begin{ncsimplebnf}
    \terminal{\# include "} q-char-sequence \terminal{"} \clb{new-line}
\end{ncsimplebnf}
causes the replacement of that
directive by the entire contents of the
source file identified by the specified sequence between the
\tcode{"}
delimiters.
The named source file is searched for in an
\impldef{manner of search for included source file}
manner.
If this search is not supported,
or if the search fails,
the directive is reprocessed as if it read
\begin{ncsimplebnf}
    \terminal{\# include <} h-char-sequence \terminal{>} \clb{new-line}
\end{ncsimplebnf}
with the identical contained sequence (including
\tcode{>}
characters, if any) from the original directive.

\pnum
A preprocessing directive of the form
\begin{ncsimplebnf}
    \terminal{\# include} pp-tokens \clb{new-line}
\end{ncsimplebnf}
(that does not match one of the two previous forms) is permitted.
The preprocessing tokens after
\tcode{include}
in the directive are processed just as in normal text
(i.e., each identifier currently defined as a macro name is replaced by its
replacement list of preprocessing tokens).
If the directive resulting after all replacements does not match
one of the two previous forms, the behavior is
undefined.
\begin{footnote}
    Note that adjacent \grammarterm{string-literal}s are not concatenated into
    a single \grammarterm{string-literal}
    (see the translation phases in~\ref{lex.phases});
    thus, an expansion that results in two \grammarterm{string-literal}s is an
    invalid directive.
\end{footnote}
The method by which a sequence of preprocessing tokens between a
\tcode{<}
and a
\tcode{>}
preprocessing token pair or a pair of
\tcode{"}
characters is combined into a single header name
preprocessing token is \impldef{search locations for \tcode{""""} header}.

\pnum
The implementation shall provide unique mappings for
sequences consisting of one or more
\grammarterm{nondigit}{s} or \grammarterm{digit}{s}\iref{lex.name}
followed by a period
(\tcode{.})
and a single
\grammarterm{nondigit}.
The first character shall not be a \grammarterm{digit}.
The implementation may ignore distinctions of alphabetical case.

\pnum
A
\tcode{\#include}
preprocessing directive may appear
in a source file that has been read because of a
\tcode{\#include}
directive in another file,
up to an \impldef{nesting limit for \tcode{\#include} directives} nesting limit.

\pnum
If the header identified by the \grammarterm{header-name}
denotes an importable header\iref{module.import},
it is
\impldef{whether source file inclusion of importable header
    is replaced with \tcode{import} directive}
whether the \tcode{\#include} preprocessing directive
is instead replaced by an \tcode{import} directive\iref{cpp.import} of the form
\begin{ncbnf}
    \terminal{import} header-name \terminal{;} \clb{new-line}
\end{ncbnf}

\pnum
\begin{note}
    An implementation can provide a mechanism for making arbitrary
    source files available to the \tcode{< >} search.
    However, using the \tcode{< >} form for headers provided
    with the implementation and the \tcode{" "} form for sources
    outside the control of the implementation
    achieves wider portability. For instance:

    \begin{codeblock}
        #include <stdio.h>
        #include <unistd.h>
        #include "usefullib.h"
        #include "myprog.h"
    \end{codeblock}

\end{note}

\pnum
\begin{example}
    This illustrates macro-replaced
    \tcode{\#include}
    directives:

    \begin{codeblock}
        #if VERSION == 1
        #define INCFILE  "vers1.h"
        #elif VERSION == 2
        #define INCFILE  "vers2.h"  // and so on
        #else
        #define INCFILE  "versN.h"
        #endif
        #include INCFILE
    \end{codeblock}
\end{example}

\rSec1[cpp.module]{Module directive}
\indextext{preprocessing directive!module}%

\begin{bnf}
    \nontermdef{pp-module}\br
    \opt{\keyword{export}} \keyword{module} \opt{pp-tokens} \terminal{;} \clb{new-line}
\end{bnf}

\pnum
A \grammarterm{pp-module} shall not
appear in a context where \tcode{module}
or (if it is the first token of the \grammarterm{pp-module}) \tcode{export}
is an identifier defined as an object-like macro.

\pnum
Any preprocessing tokens after the \tcode{module} preprocessing token
in the \tcode{module} directive are processed just as in normal text.
\begin{note}
    Each identifier currently defined as a macro name
    is replaced by its replacement list of preprocessing tokens.
\end{note}

\pnum
The \tcode{module} and \tcode{export} (if it exists) preprocessing tokens
are replaced by the \grammarterm{module-keyword} and
\grammarterm{export-keyword} preprocessing tokens respectively.
\begin{note}
    This makes the line no longer a directive
    so it is not removed at the end of phase 4.
\end{note}

\rSec1[cpp.import]{Header unit importation}
\indextext{header unit!preprocessing}%
\indextext{preprocessing directive!import}%
\indextext{macro!import|(}%

\begin{bnf}
    \nontermdef{pp-import}\br
    \opt{\keyword{export}} \keyword{import} header-name \opt{pp-tokens} \terminal{;} \clb{new-line}\br
    \opt{\keyword{export}} \keyword{import} header-name-tokens \opt{pp-tokens} \terminal{;} \clb{new-line}\br
    \opt{\keyword{export}} \keyword{import} pp-tokens \terminal{;} \clb{new-line}
\end{bnf}

\pnum
A \grammarterm{pp-import} shall not
appear in a context where \tcode{import}
or (if it is the first token of the \grammarterm{pp-import}) \tcode{export}
is an identifier defined as an object-like macro.

\pnum
The preprocessing tokens after the \tcode{import} preprocessing token
in the \tcode{import} \grammarterm{control-line}
are processed just as in normal text
(i.e., each identifier currently defined as a macro name
is replaced by its replacement list of preprocessing tokens).
\begin{note}
    An \tcode{import} directive
    matching the first two forms of a \grammarterm{pp-import}
    instructs the preprocessor to import macros
    from the header unit\iref{module.import}
    denoted by the \grammarterm{header-name},
    as described below.
\end{note}
\indextext{point of!macro import|see{macro, point of import}}%
The \defnx{point of macro import}{macro!point of import} for the
first two forms of \grammarterm{pp-import} is
immediately after the \grammarterm{\clb{new-line}} terminating
the \grammarterm{pp-import}.
The last form of \grammarterm{pp-import} is only considered
if the first two forms did not match, and
does not have a point of macro import.

//...

\rSec1[cpp.replace]{Macro replacement}%

\rSec2[cpp.replace.general]{General}%
\indextext{macro!replacement|(}%
\indextext{replacement!macro|see{macro, replacement}}%
\indextext{preprocessing directive!macro replacement|see{macro, replacement}}

\pnum
\indextext{macro!replacement list}%
Two replacement lists are identical if and only if
the preprocessing tokens in both have
the same number, ordering, spelling, and \cwhitespace{whitespace} separation,
where all \changed{whitespace separations}{sequences of one or more \grammarterm{whitespace}{s}} are considered identical.

\pnum
An identifier currently defined as an
\indextext{macro!object-like}%
object-like macro (see below) may be redefined by another
\tcode{\#define}
preprocessing directive provided that the second definition is an
object-like macro definition and the two replacement lists
are identical, otherwise the program is ill-formed.
Likewise, an identifier currently defined as a
\indextext{macro!function-like}%
function-like macro (see below) may be redefined by another
\tcode{\#define}
preprocessing directive provided that the second definition is a
function-like macro definition that has the same number and spelling
of parameters,
and the two replacement lists are identical,
otherwise the program is ill-formed.

\pnum
\begin{example}
    The following sequence is valid:
    \begin{codeblock}
        #define OBJ_LIKE      (1-1)
        #define OBJ_LIKE      @\tcode{/* whitespace */ (1-1) /* other */}@
        #define FUNC_LIKE(a)   ( a )
        #define FUNC_LIKE( a )(     @\tcode{/* note the whitespace */ \textbackslash}@
        a @\tcode{/* other stuff on this line}@
        @\tcode{*/}@ )
    \end{codeblock}
    But the following redefinitions are invalid:
    \begin{codeblock}
        #define OBJ_LIKE    (0)         // different token sequence
        #define OBJ_LIKE    (1 - 1)     // different whitespace
        #define FUNC_LIKE(b) ( a )      // different parameter usage
        #define FUNC_LIKE(b) ( b )      // different parameter spelling
    \end{codeblock}
\end{example}

\pnum
\indextext{macro!replacement list}%
There shall be \changed{whitespace}{one or more \grammarterm{whitespace}{s}} between the identifier and the replacement list in the definition of an object-like macro.

\pnum
If the \grammarterm{identifier-list} in the macro definition does not end with
an ellipsis, the number of arguments (including those arguments consisting
of no preprocessing tokens)
in an invocation of a function-like macro shall
equal the number of parameters in the macro definition.
Otherwise, there shall be at least as many arguments in the invocation as there are
parameters in the macro definition (excluding the \tcode{...}). There
shall exist a
\tcode{)}
preprocessing token that terminates the invocation.

\pnum
\indextext{__va_args__@\mname{VA_ARGS}}%
\indextext{__va_opt__@\mname{VA_OPT}}%
The identifiers \mname{VA_ARGS} and \mname{VA_OPT}
shall occur only in the \grammarterm{replacement-list}
of a function-like macro that uses the ellipsis notation in the parameters.

\pnum
A parameter identifier in a function-like macro
shall be uniquely declared within its scope.

\pnum
The identifier immediately following the
\tcode{define}
is called the
\indextext{name!macro|see{macro, name}}%
\defnx{macro name}{macro!name}.
There is one name space for macro names.
Any \cwhitespaces{whitespace characters} preceding or following the
replacement list of preprocessing tokens are not considered
part of the replacement list for either form of macro.

\pnum
If a
\indextext{\#\#0 operator@\tcode{\#} operator}
\tcode{\#}
preprocessing token,
followed by an identifier,
occurs lexically
at the point at which a preprocessing directive can begin,
the identifier is not subject to macro replacement.

\pnum
A preprocessing directive of the form
\begin{ncsimplebnf}
    \terminal{\# define} identifier replacement-list \clb{new-line}
    \indextext{\idxcode{\#define}}%
\end{ncsimplebnf}
defines an
\defnadj{object-like}{macro} that
causes each subsequent instance of the macro name
\begin{footnote}
    Since, by macro-replacement time,
    all \grammarterm{character-literal}s and \grammarterm{string-literal}s are preprocessing tokens,
    not sequences possibly containing identifier-like subsequences
    (see \ref{lex.phases}, translation phases),
    they are never scanned for macro names or parameters.
\end{footnote}
to be replaced by the replacement list of preprocessing tokens
that constitute the remainder of the directive.
\begin{footnote}
    An alternative token\iref{lex.digraph} is not an identifier,
    even when its spelling consists entirely of letters and underscores.
    Therefore it is not possible to define a macro
    whose name is the same as that of an alternative token.
\end{footnote}
The replacement list is then rescanned for more macro names as
specified below.

\pnum
\begin{example}
    The simplest use of this facility is to define a ``manifest constant'',
    as in
    \begin{codeblock}
        #define TABSIZE 100
        int table[TABSIZE];
    \end{codeblock}
\end{example}

\pnum
A preprocessing directive of the form

\begin{ncsimplebnf}
    \terminal{\# define} identifier lparen \opt{identifier-list} \terminal{)} replacement-list \clb{new-line}\br
    \terminal{\# define} identifier lparen \terminal{...} \terminal{)} replacement-list \clb{new-line}\br
    \terminal{\# define} identifier lparen identifier-list \terminal{, ...} \terminal{)} replacement-list \clb{new-line}
\end{ncsimplebnf}

defines a \defnadj{function-like}{macro}
with parameters, whose use is
similar syntactically to a function call.
The parameters
\indextext{parameter!macro}%
are specified by the optional list of identifiers.
Each subsequent instance of the function-like macro name followed by a
\tcode{(}
as the next preprocessing token
introduces the sequence of preprocessing tokens that is replaced
by the replacement list in the definition
(an invocation of the macro).
\indextext{invocation!macro}%
The replaced sequence of preprocessing tokens is terminated by the matching
\tcode{)}
preprocessing token, skipping intervening matched pairs of left and
right parenthesis preprocessing tokens.
Within the sequence of preprocessing tokens making up an invocation
of a function-like macro,
\changed{new-line is considered a normal whitespace character}{\grammarterm{line-break} is considered like other \grammarterm{whitespace}{s}}.

\pnum
\indextext{macro!function-like!arguments}%
The sequence of preprocessing tokens
bounded by the outside-most matching parentheses
forms the list of arguments for the function-like macro.
The individual arguments within the list
are separated by comma preprocessing tokens,
but comma preprocessing tokens between matching
inner parentheses do not separate arguments.
If there are sequences of preprocessing tokens within the list of
arguments that would otherwise act as preprocessing directives,
\begin{footnote}
    A \grammarterm{conditionally-supported-directive} is a preprocessing directive regardless of whether the implementation supports it.
\end{footnote}
the behavior is undefined.

\pnum
\begin{example}
    The following defines a function-like
    macro whose value is the maximum of its arguments.
    It has the disadvantages of evaluating one or the other of its arguments
    a second time
    (including
    \indextext{side effects}%
    side effects)
    and generating more code than a function if invoked several times.
    It also cannot have its address taken,
    as it has none.

    \begin{codeblock}
        #define max(a, b) ((a) > (b) ? (a) : (b))
    \end{codeblock}

    The parentheses ensure that the arguments and
    the resulting expression are bound properly.
\end{example}

\pnum
\indextext{macro!function-like!arguments}%
If there is a \tcode{...} immediately preceding the \tcode{)} in the
function-like macro
definition, then the trailing arguments (if any), including any separating comma preprocessing
tokens, are merged to form a single item: the \defn{variable arguments}. The number of
arguments so combined is such that, following merger, the number of arguments is
either equal to or
one more than the number of parameters in the macro definition (excluding the
\tcode{...}).

\rSec2[cpp.stringize]{The \tcode{\#} operator}%
\indextext{\#\#0 operator@\tcode{\#} operator}%
\indextext{stringize|see{\tcode{\#} operator}}

\pnum
Each
\tcode{\#}
preprocessing token in the replacement list for a function-like
macro shall be followed by a parameter as the next preprocessing
token in the replacement list.

\pnum
A \defn{character string literal} is a \grammarterm{string-literal} with no prefix.
If, in the replacement list, a parameter is immediately
preceded by a
\tcode{\#}
preprocessing token,
both are replaced by a single character string literal preprocessing token that
contains the spelling of the preprocessing token sequence for the
corresponding argument (excluding placemarker tokens).
Let the \defn{stringizing argument} be the preprocessing token sequence
for the corresponding argument with placemarker tokens removed.
Each \changed{occurrence of whitespace}{sequence of one or more \grammarterm{whitespace}{s}} between the stringizing argument's preprocessing
tokens becomes a single space character in the character string literal.
\added{All} \cwhitespaces{whitespace} before the first preprocessing token and after the last
preprocessing token comprising the stringizing argument \changed{is}{are} deleted.
Otherwise, the original spelling of each preprocessing token in the
stringizing argument is retained in the character string literal,
except for special handling for producing the spelling of
\grammarterm{string-literal}s and \grammarterm{character-literal}s:
a
\tcode{\textbackslash}
character is inserted before each
\tcode{"}
and
\tcode{\textbackslash}
character of a \grammarterm{character-literal} or \grammarterm{string-literal}
(including the delimiting
\tcode{"}
characters).
If the replacement that results is not a valid character string literal,
the behavior is undefined. The character string literal corresponding to
an empty stringizing argument is \tcode{""}.
The order of evaluation of
\tcode{\#}
and
\tcode{\#\#}
operators is unspecified.

\rSec2[cpp.concat]{The \tcode{\#\#} operator}%
\indextext{\#\#1 operator@\tcode{\#\#} operator}%
\indextext{concatenation!macro argument|see{\tcode{\#\#} operator}}

\pnum
A
\tcode{\#\#}
preprocessing token shall not occur at the beginning or
at the end of a replacement list for either form
of macro definition.

\pnum
If, in the replacement list of a function-like macro, a parameter is
immediately preceded or followed by a
\tcode{\#\#}
preprocessing token, the parameter is replaced by the
corresponding argument's preprocessing token sequence; however, if an argument consists of no preprocessing tokens, the parameter is
replaced by a placemarker preprocessing token instead.
\begin{footnote}
    Placemarker preprocessing tokens do not appear in the syntax
    because they are temporary entities that exist only within translation phase 4.
\end{footnote}

\pnum
For both object-like and function-like macro invocations, before the
replacement list is reexamined for more macro names to replace,
each instance of a
\tcode{\#\#}
preprocessing token in the replacement list
(not from an argument) is deleted and the
preceding preprocessing token is concatenated
with the following preprocessing token.
Placemarker preprocessing tokens are handled specially: concatenation
of two placemarkers results in a single placemarker preprocessing token, and
concatenation of a placemarker with a non-placemarker preprocessing token results
in the non-placemarker preprocessing token.
If the result is not a valid preprocessing token,
the behavior is undefined.
The resulting token is available for further macro replacement.
The order of evaluation of
\tcode{\#\#}
operators is unspecified.

\pnum
\begin{example}
    The sequence
    \begin{codeblock}
        #define str(s)      # s
        #define xstr(s)     str(s)
        #define debug(s, t) printf("x" # s "= %d, x" # t "= %s", @\textbackslash@
        x ## s, x ## t)
        #define INCFILE(n)  vers ## n
        #define glue(a, b)  a ## b
        #define xglue(a, b) glue(a, b)
        #define HIGHLOW     "hello"
        #define LOW         LOW ", world"

        debug(1, 2);
        fputs(str(strncmp("abc@\textbackslash@0d", "abc", '@\textbackslash@4')        // this goes away
        == 0) str(: @\atsign\textbackslash@n), s);
        #include xstr(INCFILE(2).h)
        glue(HIGH, LOW);
        xglue(HIGH, LOW)
    \end{codeblock}
    results in
    \begin{codeblock}
        printf("x" "1" "= %d, x" "2" "= %s", x1, x2);
        fputs("strncmp(@\textbackslash@"abc@\textbackslash\textbackslash@0d@\textbackslash@", @\textbackslash@"abc@\textbackslash@", '@\textbackslash\textbackslash@4') == 0" ": @\atsign\textbackslash@n", s);
        #include "vers2.h"      @\textrm{(\textit{after macro replacement, before file access})}@
        "hello";
        "hello" ", world"
    \end{codeblock}
    or, after concatenation of the character string literals,
    \begin{codeblock}
        printf("x1= %d, x2= %s", x1, x2);
        fputs("strncmp(@\textbackslash@"abc@\textbackslash\textbackslash@0d@\textbackslash@", @\textbackslash@"abc@\textbackslash@", '@\textbackslash\textbackslash@4') == 0: @\atsign\textbackslash@n", s);
        #include "vers2.h"      @\textrm{(\textit{after macro replacement, before file access})}@
        "hello";
        "hello, world"
    \end{codeblock}

    \added{The presence of} \cwhitespace{Space} around the \tcode{\#} and \tcode{\#\#} tokens in the macro definition
    is optional.
\end{example}

\pnum
\begin{example}
    In the following fragment:

    \begin{codeblock}
        #define hash_hash # ## #
        #define mkstr(a) # a
        #define in_between(a) mkstr(a)
        #define join(c, d) in_between(c hash_hash d)
        char p[] = join(x, y);          // equivalent to \tcode{char p[] = "x \#\# y";}
    \end{codeblock}

    The expansion produces, at various stages:

    \begin{codeblock}
        join(x, y)
        in_between(x hash_hash y)
        in_between(x ## y)
        mkstr(x ## y)
        "x ## y"
    \end{codeblock}

    In other words, expanding \tcode{hash_hash} produces a new token,
    consisting of two adjacent sharp signs, but this new token is not the
    \tcode{\#\#} operator.
\end{example}

\pnum
\begin{example}
    To illustrate the rules for placemarker preprocessing tokens, the sequence
    \begin{codeblock}
        #define t(x,y,z) x ## y ## z
        int j[] = { t(1,2,3), t(,4,5), t(6,,7), t(8,9,),
            t(10,,), t(,11,), t(,,12), t(,,) };
    \end{codeblock}
    results in
    \begin{codeblock}
        int j[] = { 123, 45, 67, 89,
            10, 11, 12, };
    \end{codeblock}
\end{example}

\rSec2[cpp.scope]{Scope of macro definitions}%
\indextext{macro!scope of definition}%
\indextext{scope!macro definition|see{macro, scope of definition}}

\pnum
A macro definition lasts
(independent of block structure)
until a corresponding
\tcode{\#undef}
directive is encountered or
(if none is encountered)
until the end of the translation unit.
Macro definitions have no significance after translation phase 4.

\pnum
A preprocessing directive of the form
\begin{ncsimplebnf}
    \terminal{\# undef} identifier \clb{new-line}
    \indextext{\idxcode{\#undef}}%
\end{ncsimplebnf}
causes the specified identifier no longer to be defined as a macro name.
It is ignored if the specified identifier is not currently defined as
a macro name.

\indextext{macro!replacement|)}

\rSec1[cpp.line]{Line control}%
\indextext{preprocessing directive!line control}%
\indextext{\idxcode{\#line}|see{preprocessing directive, line control}}

\pnum
The \grammarterm{string-literal} of a
\tcode{\#line}
directive, if present,
shall be a character string literal.

\pnum
The
\defn{line number}
of the current source line is one greater than
the number of \clbs{new-line characters} \changed{read or introduced
in}{resulting from} translation phase 1\iref{lex.phases}
while processing the source file to the current token.

%\ednote{Do we want to keep "introduced" here?}

\pnum
A preprocessing directive of the form
\begin{ncsimplebnf}
    \terminal{\# line} digit-sequence \clb{new-line}
\end{ncsimplebnf}
causes the implementation to behave as if
the following sequence of source lines begins with a
source line that has a line number as specified
by the digit sequence (interpreted as a decimal integer).
If the digit sequence specifies zero
or a number greater than 2147483647,
the behavior is undefined.

\pnum
A preprocessing directive of the form
\begin{ncsimplebnf}
    \terminal{\# line} digit-sequence \terminal{"} \opt{s-char-sequence} \terminal{"} \clb{new-line}
\end{ncsimplebnf}
sets the presumed line number similarly and changes the
presumed name of the source file to be the contents
of the character string literal.

\pnum
A preprocessing directive of the form
\begin{ncsimplebnf}
    \terminal{\# line} pp-tokens \clb{new-line}
\end{ncsimplebnf}
(that does not match one of the two previous forms)
is permitted.
The preprocessing tokens after
\tcode{line}
on the directive are processed just as in normal text
(each identifier currently defined as a macro name is replaced by its
replacement list of preprocessing tokens).
If the directive resulting after all replacements does not match
one of the two previous forms, the behavior is undefined;
otherwise, the result is processed as appropriate.

\rSec1[cpp.error]{Error directive}%
\indextext{preprocessing directive!error}%
\indextext{\idxcode{\#error}|see{preprocessing directive, error}}

\pnum
A preprocessing directive of the form
\begin{ncsimplebnf}
    \terminal{\# error} \opt{pp-tokens} \clb{new-line}
\end{ncsimplebnf}
causes the implementation to produce
a diagnostic message that includes the specified sequence of preprocessing tokens,
and renders the program ill-formed.

\rSec1[cpp.pragma]{Pragma directive}%
\indextext{preprocessing directive!pragma}%
\indextext{\idxcode{\#pragma}|see{preprocessing directive, pragma}}

\pnum
A preprocessing directive of the form
\begin{ncsimplebnf}
    \terminal{\# pragma} \opt{pp-tokens} \clb{new-line}
\end{ncsimplebnf}
causes the implementation to behave
in an \impldef{\tcode{\#pragma}} manner.
The behavior may cause translation to fail or cause the translator or
the resulting program to behave in a non-conforming manner.
Any pragma that is not recognized by the implementation is ignored.

\rSec1[cpp.null]{Null directive}%
\indextext{preprocessing directive!null}

\pnum
A preprocessing directive of the form
\begin{ncsimplebnf}
    \terminal{\#} \clb{new-line}
\end{ncsimplebnf}
has no effect.

\section{Acknowledgments}

Thanks to Peter Brett, Hubert Tong, Jens Maurer and the entierty of SG-16 who provided valuable feedback!

\bibliographystyle{plain}
\bibliography{wg21}


\renewcommand{\section}[2]{}%
\begin{thebibliography}{9}

\bibitem[N4885]{N4885}
Thomas Köppe
\emph{Working Draft, Standard for Programming Language C++}\ new-line
\url{https://wg21.link/N4885}

\end{thebibliography}

\end{document}
