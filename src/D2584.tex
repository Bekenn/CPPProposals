% !TeX program = luatex
% !TEX encoding = UTF-8


\documentclass{wg21}

\title{A More Composable \tcode{from\_chars}}
\docnumber{D2584R1}
\audience{LEWG}
\author{Corentin Jabot}{corentin.jabot@gmail.com}

\begin{document}
\maketitle

\section{Abstract}

We propose an easier way to convert a sequence of characters to a number using \tcode{std::from_chars}.
This paper is a follow-up to \paper{P2007R0}.

\section{Tony table}

\begin{center}
\begin{tabular}{l|l}
Before & After\\ \hline

\begin{minipage}[t]{0.5\textwidth}

\begin{colorblock}
std::string s = "1.2.3.4";

auto ints =
s | std::views::split('.')
  | std::views::transform([](const auto & v){
    int i = 0;
    std::from_chars(std::to_address(v.begin()),
               std::to_address(v.end(), i);
    return i;
});
\end{colorblock}
\end{minipage}
&
\begin{minipage}[t]{0.5\textwidth}
\begin{colorblock}
std::string s = "1.2.3.4";

auto ints =
s | std::views::split('.')
  | std::views::transform([](const auto & v) {
      return std::from_chars<int>(v).value_or(0);
});
\end{colorblock}
\end{minipage}
\\\\ \hline

\end{tabular}
\end{center}

This example was taken from Barry's Revzin \href{https://brevzin.github.io/c++/2020/07/06/split-view/}{blog post} on the deficiencies of the old split view.

\pagebreak


\section{Revisions}

\subsection{R1}
\begin{itemize}
\item Modify the \tcode{expected} base interface to inherit from \tcode{expected} such that the unparsed information is always preserved.
\item Present the \tcode{expected} interface as the primary option preferred by the author.
\item Add wording for the \tcode{expected} based option
\item Fix typos and wording issues
\end{itemize}

\section{Example}


\section{Motivation and design}

We propose to add new \tcode{from_chars} overloads with the aim of simplifying the use of the interface and making it more composable.


\section{Design using \tcode{std::expected}}

We propose an interface returning an object inheriting from expected
to make it easier to access the value, and to check for errors:

\begin{colorblock}
    template <typename T>
    struct from_chars_result_range : std::expected<T, std::errc> {
        std::span<const char> unparsed = {};
    };

    template <std::integral T>
    constexpr from_chars_result_range<T> from_chars(span<const char> rng, int base = 10);

    template <std::floating_point T>
    from_chars_result_range<T>
    from_chars(std::span<const char> rng, std::chars_format fmt = std::chars_format::general);
\end{colorblock}

This interface is hard to misuse and would encourage checking for errors.
It turns out to be pretty nice to use too.
The one drawback is the reliance on the \tcode{expected} header.

Contrary to R0 and what was claimed previously, if one or more characters are matched, but the value is outside of the type
bound, there is both an error and some characters parsed.
So we need to inherit from \tcode{expected} to add the unparsed member in both the value and error case.


\begin{colorblock}
    int main() {
        assert(from_chars<int>("123").value_or(0) == 123);
        assert(from_chars<int>("cafe", 16).value_or(0) == 0xcafe);
        assert(from_chars<int>("cafe").value_or(42) == 42);

        if(auto parsed = std::from_chars<int>("123!!"); parsed) {
            assert(*parsed == 123);
            assert(std::ranges::equal(parsed.unparsed, "!!"));
        }
    }
\end{colorblock}

\subsection{\tcode{from_chars} should take a range rather than a pair of pointers}

As explained in \paper{P2007R0}, a correct use of \tcode{from_chars} with any kind of range call for

\begin{colorblock}
std::from_chars(std::to_address(std::ranges::begin(rng)), std::to_address(std::ranges::end(rng)), out);
\end{colorblock}

This is because:
\begin{itemize}
\item The iterators may not be pointers
\item The range may be contiguous but not sized (so \tcode{data(), data()+size()} isn't an option).
\end{itemize}

It's a lot of subtleties and verbosity for a relatively common interface.

\subsection{\tcode{from_chars} should return its result by value}

Having the converted value as part of the return type gives more opportunity for composition.
For example, it allows patterns such as:

\begin{colorblock}
if(auto [value, ec, _]  = std::from_chars<int>(range); ec == std::errc()) {}
\end{colorblock}

To achieve that, the proposed \tcode{from_chars} overloads take the desired output type as a template parameter
and return a \tcode{from_chars_result_range} object.

\subsection{\tcode{span} vs \tcode{string_view} vs \tcode{contiguous_range}}
This proposal uses \tcode{span<const char>}.
This is because \paper{P2499R0}, by making \tcode{string_view}'s string_view range constructor explicit, makes using it in contexts where we want to accept any range of char more tedious than it needs to be and less composable.

Ultimately, whether we choose \tcode{span<const char>} or \tcode{string_view} depends on whether we think the range case is more commone than the \tcode{const char*} use case.

Using \tcode{contiguous_range} over \tcode{span} has very little benefits.
The proposed design uses span in its returned object anyway (to store the remaining range), so it would not save on headers inclusion, and \tcode{<span>} is a very small header anyway,

\subsection{Header}

During previous discussions, there were some concerns that this would impact compile times.
In the meantime we:

\begin{itemize}
\item Made \tcode{from_chars} constexpr, leading to potentially bigger header
\item Standardized header units and a \tcode{std} module.
\end{itemize}

\subsection{\tcode{from_chars_result_range} is not comparable}

The rationale to make \tcode{from_chars_result} (\paper{P1191R0}) comparable is unclear, and \href{https://lists.isocpp.org/lib/2021/11/21141.php}{it has been regarded as a bad move}.
Indeed, it is unclear what the invariant of \tcode{from_chars_result} is.
We do, therefore, not propose to make the new \tcode{from_chars_result_range} type comparable, especially in the absence of good rationale.

\subsection{But \tcode{from_chars} is intended as a low level interface!}

\tcode{from_chars} is efficient, correct and usable portably. That doesn't mean it should be hard to use.
The proposed interface doesn't make \tcode{from_chars} less usable, quite the contrary, and that's a good thing.
It's not because a facility is "low-level" that it should be gratuitously expert-friendly.


\subsection{Alternative interface}

The following design, which was presented as the primary option in R0 does not use expected and can be more easily used
with structured binding. But, it encourages ignoring errors, and does not offer the same ergonomic benefits as the monadic interfaces
of \tcode{expected}.

\begin{colorblock}
template <typename T>
struct from_chars_result_range {
    T value;
    std::errc ec;
    std::span<const char> unparsed;
};
template <integral T>
requires (!std::same_as<bool, T>)
constexpr from_chars_result_range<T> from_chars(std::span<const char> rng, int base = 10);

template <floating_point T>
from_chars_result_range<T> from_chars(std::span<const char> rng, chars_format fmt = chars_format::general);
\end{colorblock}



\section{Question for LEWG}

\begin{itemize}
    \item Do we like the general direction?
    \item Do we prefer the version with expected or the one without?
\end{itemize}

\section{Implementation experience}

The new overloads are specified to wrap the existing one, so this proposal presents
no particular implementation complexity.
The design using \tcode{std::expected} is demoed \href{https://godbolt.org/z/P65f3bY9c}{here}.

The alternative design (without expected) is also on \href{https://godbolt.org/z/rnxE1o9Ma}{Compiler Explorer}.

\section{Wording (for the design with expected)}

\rSec2[charconv.syn]{Header \tcode{<charconv>} synopsis}

\begin{codeblock}
@%
\indexlibraryglobal{chars_format}%
\indexlibrarymember{scientific}{chars_format}%
\indexlibrarymember{fixed}{chars_format}%
\indexlibrarymember{hex}{chars_format}%
\indexlibrarymember{general}{chars_format}%\textbf{}
@namespace std {
    // \ref{charconv.to.chars}, primitive numerical output conversion
    struct to_chars_result {
        char* ptr;
        errc ec;
        friend bool operator==(const to_chars_result&, const to_chars_result&) = default;
    };

    @%
    \indexlibraryglobal{from_chars_result}%
    \indexlibrarymember{ptr}{from_chars_result}%
    \indexlibrarymember{ec}{from_chars_result}
    @
    // \ref{charconv.from.chars}, primitive numerical input conversion
    struct from_chars_result {
        const char* ptr;
        errc ec;
        friend bool operator==(const from_chars_result&, const from_chars_result&) = default;
    };

\end{codeblock}
\begin{addedblock}
\begin{codeblock}
    template <typename T>
    struct from_chars_result_range : expected<T, errc> {
        std::span<const char> unparsed = {};

        constexpr from_chars_result_range(T value, span<const char> unparsed)  // \expos
        noexcept
            : expected<T, errc>(value), unparsed(unparsed) {};
        constexpr from_chars_result_range(errc err, span<const char> unparsed)  // \expos
        noexcept
            : expected<T, errc>(std::unexpect, err), unparsed(unparsed) {};

        template <typename U>
        bool operator==(const from_chars_result_range<U>&) = delete;
    };
\end{codeblock}
\end{addedblock}
\begin{codeblock}


    from_chars_result from_chars(const char* first, const char* last,
    @\seebelow@& value, int base = 10);
    @\added{template <typename T>}@
    @\added{constexpr from_chars_result_range<T> from_chars(span<const char> rng, int base = 10)}@

    from_chars_result from_chars(const char* first, const char* last, float& value,
    chars_format fmt = chars_format::general);
    from_chars_result from_chars(const char* first, const char* last, double& value,
    chars_format fmt = chars_format::general);
    from_chars_result from_chars(const char* first, const char* last, long double& value,
    chars_format fmt = chars_format::general);
    @\added{template <typename T>}@
    @\added{from_chars_result_range<T> from_chars(span<const char> rng, }@
    @\added{               chars_format fmt = chars_format::general);}@
}
\end{codeblock}

\pnum
The type \tcode{chars_format} is a bitmask type\iref{bitmask.types}
with elements \tcode{scientific}, \tcode{fixed}, and \tcode{hex}.

\pnum
The types \tcode{to_chars_result}\added{, \tcode{from\char`_chars\char`_result\char`_range}, }and \tcode{from_chars_result}
have the data members and special members specified above.
They have no base classes or members other than those specified.

\rSec2[charconv.from.chars]{Primitive numeric input conversion}

\pnum
All functions named \tcode{from_chars}
analyze the string \range{first}{last}
for a pattern,
where \range{first}{last} is required to be a valid range.
If no characters match the pattern,
\tcode{value} is unmodified,
the member \tcode{ptr} of the return value is \tcode{first} and
the member \tcode{ec} is equal to \tcode{errc::invalid_argument}.
\begin{note}
    If the pattern allows for an optional sign,
    but the string has no digit characters following the sign,
    no characters match the pattern.
\end{note}
Otherwise,
the characters matching the pattern
are interpreted as a representation
of a value of the type of \tcode{value}.
The member \tcode{ptr}
of the return value
points to the first character
not matching the pattern,
or has the value \tcode{last}
if all characters match.
If the parsed value
is not in the range
representable by the type of \tcode{value},
\tcode{value} is unmodified and
the member \tcode{ec} of the return value
is equal to \tcode{errc::result_out_of_range}.
Otherwise,
\tcode{value} is set to the parsed value,
after rounding according to \tcode{round_to_nearest}\iref{round.style}, and
the member \tcode{ec} is value-initialized.

\indexlibraryglobal{from_chars}%
\begin{itemdecl}
    from_chars_result from_chars(const char* first, const char* last,
    @\seebelow@&@\itcorr[-1]@ value, int base = 10);
\end{itemdecl}

\begin{itemdescr}
    \pnum
    \expects
    \tcode{base} has a value between 2 and 36 (inclusive).

    \pnum
    \effects
    The pattern is the expected form of the subject sequence
    in the \tcode{"C"} locale
    for the given nonzero base,
    as described for \tcode{strtol},
    except that no \tcode{"0x"} or \tcode{"0X"} prefix shall appear
    if the value of \tcode{base} is 16,
    and except that \tcode{'-'}
    is the only sign that may appear,
    and only if \tcode{value} has a signed type.

    \pnum
    \throws
    Nothing.

    \pnum
    \remarks
    The implementation shall provide overloads
    for all signed and unsigned integer types
    and \tcode{char}
    as the referenced type of the parameter \tcode{value}.
\end{itemdescr}

\begin{addedblock}
\indexlibraryglobal{from_chars}%
\begin{itemdecl}
    template <typename T>
    constexpr from_chars_result_range<T>
    from_chars(span<const char> rng, int base = 10)
\end{itemdecl}

\begin{itemdescr}

    \constraints \tcode{T} models \tcode{integral} and \tcode{same_as<T, bool>} is \tcode{false}.

    \pnum
    \expects
    \tcode{base} has a value between 2 and 36 (inclusive).

    \pnum
    \effects Equivalent to
    \begin{codeblock}
        T out;
        auto [ptr, ec] = from_chars(to_address(begin(rng)),
                                    to_address(end(rng)), out, base);
        auto subspan = rng.subspan(ptr - rng.data());
        return ec != errc()
            ? from_chars_result_range<T>{ec,  subspan}
            : from_chars_result_range<T>{out, subspan};
    \end{codeblock}
    \pnum
    \throws
    Nothing.
    \pnum
\end{itemdescr}

\end{addedblock}

\indexlibraryglobal{from_chars}%
\begin{itemdecl}
    from_chars_result from_chars(const char* first, const char* last, float& value,
    chars_format fmt = chars_format::general);
    from_chars_result from_chars(const char* first, const char* last, double& value,
    chars_format fmt = chars_format::general);
    from_chars_result from_chars(const char* first, const char* last, long double& value,
    chars_format fmt = chars_format::general);
\end{itemdecl}

\begin{itemdescr}
    \pnum
    \expects
    \tcode{fmt} has the value of
    one of the enumerators of \tcode{chars_format}.

    \pnum
    \effects
    The pattern is the expected form of the subject sequence
    in the \tcode{"C"} locale,
    as described for \tcode{strtod},
    except that
    \begin{itemize}
        \item
        the sign \tcode{'+'} may only appear in the exponent part;
        \item
        if \tcode{fmt} has \tcode{chars_format::scientific} set
        but not \tcode{chars_format::fixed},
        the otherwise optional exponent part shall appear;
        \item
        if \tcode{fmt} has \tcode{chars_format::fixed} set
        but not \tcode{chars_format::scientific},
        the optional exponent part shall not appear; and
        \item
        if \tcode{fmt} is \tcode{chars_format::hex},
        the prefix \tcode{"0x"} or \tcode{"0X"} is assumed.
        \begin{example}
            The string \tcode{0x123}
            is parsed to have the value
            \tcode{0}
            with remaining characters \tcode{x123}.
        \end{example}
    \end{itemize}
    In any case, the resulting \tcode{value} is one of
    at most two floating-point values
    closest to the value of the string matching the pattern.

    \pnum
    \throws
    Nothing.
\end{itemdescr}

\begin{addedblock}
\begin{itemdecl}
template <typename T>
from_chars_result_range<T> from_chars(span<const char> rng, chars_format fmt = chars_format::general);
\end{itemdecl}

\begin{itemdescr}
\pnum

\constraints \tcode{T} models \tcode{floating_point}.

\expects
\tcode{fmt} has the value of
one of the enumerators of \tcode{chars_format}.


\effects Equivalent to
\begin{codeblock}
    T out;
    auto [ptr, ec] = from_chars(to_address(begin(rng)),
                                to_address(end(rng)), out, fmt);
    auto subspan = rng.subspan(ptr - rng.data());
    return ec != errc()
        ? from_chars_result_range<T>{ec,  subspan}
        : from_chars_result_range<T>{out, subspan};
\end{codeblock}

\end{itemdescr}

\end{addedblock}

\section{Wording (for the design without expected)}

\rSec2[charconv.syn]{Header \tcode{<charconv>} synopsis}

\begin{codeblock}
@%
\indexlibraryglobal{chars_format}%
\indexlibrarymember{scientific}{chars_format}%
\indexlibrarymember{fixed}{chars_format}%
\indexlibrarymember{hex}{chars_format}%
\indexlibrarymember{general}{chars_format}%\textbf{}
@namespace std {
    // floating-point format for primitive numerical conversion
    enum class chars_format {
        scientific = @\unspec@,
        fixed = @\unspec@,
        hex = @\unspec@,
        general = fixed | scientific\textbf{}
    };
    @%
    \indexlibraryglobal{to_chars_result}%
    \indexlibrarymember{ptr}{to_chars_result}%
    \indexlibrarymember{ec}{to_chars_result}
    @
    // \ref{charconv.to.chars}, primitive numerical output conversion
    struct to_chars_result {
        char* ptr;
        errc ec;
        friend bool operator==(const to_chars_result&, const to_chars_result&) = default;
    };

    to_chars_result to_chars(char* first, char* last, @\seebelow@ value, int base = 10);
    to_chars_result to_chars(char* first, char* last, bool value, int base = 10) = delete;

    to_chars_result to_chars(char* first, char* last, float value);
    to_chars_result to_chars(char* first, char* last, double value);
    to_chars_result to_chars(char* first, char* last, long double value);

    to_chars_result to_chars(char* first, char* last, float value, chars_format fmt);
    to_chars_result to_chars(char* first, char* last, double value, chars_format fmt);
    to_chars_result to_chars(char* first, char* last, long double value, chars_format fmt);

    to_chars_result to_chars(char* first, char* last, float value,
    chars_format fmt, int precision);
    to_chars_result to_chars(char* first, char* last, double value,
    chars_format fmt, int precision);
    to_chars_result to_chars(char* first, char* last, long double value,
    chars_format fmt, int precision);
    @%
    \indexlibraryglobal{from_chars_result}%
    \indexlibrarymember{ptr}{from_chars_result}%
    \indexlibrarymember{ec}{from_chars_result}
    @
    // \ref{charconv.from.chars}, primitive numerical input conversion
    struct from_chars_result {
        const char* ptr;
        errc ec;
        friend bool operator==(const from_chars_result&, const from_chars_result&) = default;
    };

    @\added{template <integral T>}@
    @\added{struct from_chars_result_range \{}@
        @\added{T value;}@
        @\added{errc ec;}@
        @\added{span<const char> unparsed;}@
    @\added{ \};}@


    from_chars_result from_chars(const char* first, const char* last,
    @\seebelow@& value, int base = 10);
    @\added{template <typename T>}@
    @\added{constexpr from_chars_result_range<T> from_chars(span<const char> rng, int base = 10)}@

    from_chars_result from_chars(const char* first, const char* last, float& value,
    chars_format fmt = chars_format::general);
    from_chars_result from_chars(const char* first, const char* last, double& value,
    chars_format fmt = chars_format::general);
    from_chars_result from_chars(const char* first, const char* last, long double& value,
    chars_format fmt = chars_format::general);
    @\added{template <typename T>}@
    @\added{from_chars_result_range<T> from_chars(span<const char> rng, }@
                                          @\added{chars_format fmt = chars_format::general);}@
}
\end{codeblock}

\pnum
The type \tcode{chars_format} is a bitmask type\iref{bitmask.types}
with elements \tcode{scientific}, \tcode{fixed}, and \tcode{hex}.

\pnum
The types \tcode{to_chars_result}\added{, \tcode{from\char`_chars\char`_result\char`_range}, }and \tcode{from_chars_result}
have the data members and special members specified above.
They have no base classes or members other than those specified.

\rSec2[charconv.from.chars]{Primitive numeric input conversion}

\pnum
All functions named \tcode{from_chars}
analyze the string \range{first}{last}
for a pattern,
where \range{first}{last} is required to be a valid range.
If no characters match the pattern,
\tcode{value} is unmodified,
the member \tcode{ptr} of the return value is \tcode{first} and
the member \tcode{ec} is equal to \tcode{errc::invalid_argument}.
\begin{note}
    If the pattern allows for an optional sign,
    but the string has no digit characters following the sign,
    no characters match the pattern.
\end{note}
Otherwise,
the characters matching the pattern
are interpreted as a representation
of a value of the type of \tcode{value}.
The member \tcode{ptr}
of the return value
points to the first character
not matching the pattern,
or has the value \tcode{last}
if all characters match.
If the parsed value
is not in the range
representable by the type of \tcode{value},
\tcode{value} is unmodified and
the member \tcode{ec} of the return value
is equal to \tcode{errc::result_out_of_range}.
Otherwise,
\tcode{value} is set to the parsed value,
after rounding according to \tcode{round_to_nearest}\iref{round.style}, and
the member \tcode{ec} is value-initialized.

\indexlibraryglobal{from_chars}%
\begin{itemdecl}
from_chars_result from_chars(const char* first, const char* last,
@\seebelow@&@\itcorr[-1]@ value, int base = 10);
\end{itemdecl}

\begin{itemdescr}
\pnum
\expects
\tcode{base} has a value between 2 and 36 (inclusive).

\pnum
\effects
The pattern is the expected form of the subject sequence
in the \tcode{"C"} locale
for the given nonzero base,
as described for \tcode{strtol},
except that no \tcode{"0x"} or \tcode{"0X"} prefix shall appear
if the value of \tcode{base} is 16,
and except that \tcode{'-'}
is the only sign that may appear,
and only if \tcode{value} has a signed type.

\pnum
\throws
Nothing.

\pnum
\remarks
The implementation shall provide overloads
for all signed and unsigned integer types
and \tcode{char}
as the referenced type of the parameter \tcode{value}.
\end{itemdescr}

\begin{addedblock}
\indexlibraryglobal{from_chars}%
\begin{itemdecl}
template <typename T>
constexpr from_chars_result_range<T> from_chars(span<const char> rng, int base = 10);
\end{itemdecl}

\begin{itemdescr}

\constraints \tcode{T} models \tcode{integral} and \tcode{same_as<T, bool>} is \tcode{false}.

\pnum
\expects
\tcode{base} has a value between 2 and 36 (inclusive).

\pnum
\effects Equivalent to
\begin{codeblock}
T out;
auto res = from_chars(to_address(rng.begin()), to_address(rng.end()), out, base);
return {out, res.ec, rng.subspan(res.ptr - rng.data())};
\end{codeblock}

\pnum
\throws
Nothing.

\pnum
\end{itemdescr}

\end{addedblock}

\indexlibraryglobal{from_chars}%
\begin{itemdecl}
from_chars_result from_chars(const char* first, const char* last, float& value,
chars_format fmt = chars_format::general);
from_chars_result from_chars(const char* first, const char* last, double& value,
chars_format fmt = chars_format::general);
from_chars_result from_chars(const char* first, const char* last, long double& value,
chars_format fmt = chars_format::general);
\end{itemdecl}

\begin{itemdescr}
\pnum
\expects
\tcode{fmt} has the value of
one of the enumerators of \tcode{chars_format}.

\pnum
\effects
The pattern is the expected form of the subject sequence
in the \tcode{"C"} locale,
as described for \tcode{strtod},
except that
\begin{itemize}
    \item
    the sign \tcode{'+'} may only appear in the exponent part;
    \item
    if \tcode{fmt} has \tcode{chars_format::scientific} set
    but not \tcode{chars_format::fixed},
    the otherwise optional exponent part shall appear;
    \item
    if \tcode{fmt} has \tcode{chars_format::fixed} set
    but not \tcode{chars_format::scientific},
    the optional exponent part shall not appear; and
    \item
    if \tcode{fmt} is \tcode{chars_format::hex},
    the prefix \tcode{"0x"} or \tcode{"0X"} is assumed.
    \begin{example}
        The string \tcode{0x123}
        is parsed to have the value
        \tcode{0}
        with remaining characters \tcode{x123}.
    \end{example}
\end{itemize}
In any case, the resulting \tcode{value} is one of
at most two floating-point values
closest to the value of the string matching the pattern.

\pnum
\throws
Nothing.
\end{itemdescr}

\begin{addedblock}
\begin{itemdecl}
template <typename T>
from_chars_result_range<T> from_chars(span<const char> rng, chars_format fmt = chars_format::general);
\end{itemdecl}

\begin{itemdescr}
\pnum

\constraints \tcode{T} models \tcode{floating_point}.

\expects
\tcode{fmt} has the value of
one of the enumerators of \tcode{chars_format}.


\effects Equivalent to
\begin{codeblock}
    T res;
    auto [ptr, ec] = from_chars(to_address(rng.begin()), to_address(rng.end()), res, base);
    return {res, ec, rng.subspan(ptr - rng.data())};
\end{codeblock}

\end{itemdescr}

\end{addedblock}




\section{Feature test macro}

\ednote{
Bump the value of \tcode{__cpp_lib_to_chars} to the date of adoption in \tcode{charconv} and \tcode{version}
}
\section{Acknowledgments}

Thanks to Mateusz Pusz for writing \paper{P2007R0} which this paper is derived from.
Thanks to Zhihao Yuan, Jeff Garland and others for helping me brainstorm these interfaces.

\bibliographystyle{plain}
\bibliography{wg21}

\end{document}
