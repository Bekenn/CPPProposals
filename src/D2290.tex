% !TeX program = luatex
% !TEX encoding = UTF-8


\RequirePackage{luatex85}


\documentclass{wg21}



\newcommand{\UnicodeLetter}[1]{\textbf{\textcolor{BrickRed}{\Large\tcode{#1}}}}

\title{Delimited escapes sequences}
\docnumber{P2290R0}
\audience{EWG, SG-22}
\author{Corentin Jabot}{corentin.jabot@gmail.com}

\begin{document}
    \maketitle
    
    \paperquote{}
    
    \section{Abstract}
    
    We propose an additional, clearly delimited syntax for octal, hexadecimal and universal character name escape sequences.
    
    
    \section{Motivation}
    
    \subsection{universal-character-name escape sequences}
    
    As their name does not indicate, universal-character-name escape sequences represent Unicode scalar values,
    using either 4, or 8 hexadecimal digits, which is either 16 or 32 bits.
    However, the Unicode codespace is limited to 0-0x10FFFF, and all currently assigned codepoints can be written with 5 or less
    hexadecimal digits (Supplementary Private Use Area-B non-withstanding).
    As such, the \~50\% of codepoints that needs 5 hexadecimal digits to be expressed are currently a bit awkward to write: \tcode{\textbackslash U0001F1F8}.
    
    \subsection{Octal and hexadecimal escape sequences have variable lenght}
    
    \tcode {\textbackslash 1}, {\textbackslash 01}, {\textbackslash 001} are all valid escape sequences.
    \tcode {\textbackslash 17} is equivalent to 0x0F while \tcode {\textbackslash 18} is equivalent to \tcode {"0x01" "8"}
    
    While octal escape sequences accept 1 to 3 digits as arguments, hexadecimal sequences accept an arbitrary number of digits
    applying the maximal much principle.
    
    
    This is how the \href{https://docs.microsoft.com/en-us/cpp/c-language/octal-and-hexadecimal-character-specifications?view=msvc-160}{Microsoft documentation} describes this problem:
    
    \begin{quoteblock}
        
        Unlike octal escape constants, the number of hexadecimal digits in an escape sequence is unlimited. A hexadecimal escape sequence terminates at the first character that is not a hexadecimal digit. Because hexadecimal digits include the letters a through f, care must be exercised to make sure the escape sequence terminates at the intended digit. To avoid confusion, you can place octal or hexadecimal character definitions in a macro definition:
        
        \begin{colorblock}
            #define Bell '\x07'
        \end{colorblock}
        
        For hexadecimal values, you can break the string to show the correct value clearly:
        
        \begin{colorblock}
            "\xabc"    /* one character  */
            "\xab" "c" /* two characters */
        \end{colorblock}
        
    \end{quoteblock}
    
    While, as this documentation suggests, there are workarounds, to this problem, neither solution is really appealing, nor do
    they completely solve the maintenance issue. It might, for example not be clear why a string is split in 2, and that split may
    be refactored away by zealous tooling or contributors.
    
    \section{Proposed solution}
    
    We propose new syntaxes \tcode{\textbackslash u\{\}, \textbackslash o\{\}, \textbackslash x\{\}} usable in places where
    \textbackslash u, \textbackslash x, \textbackslash nnn currently are.
    \textbackslash o\{\} accepts an arbitrary number of octal digits while \textbackslash u\{\} and \textbackslash x\{\} accept
    an arbitrary number of hexadecimal digit.
    
    The values represented by these new syntaxes would of course have the same requirements as existing escape sequences:
    
    \textbackslash u\{nnnn\} must represent a valid Unicode scalar value.
    
    \textbackslash x\{nnnn\} and \textbackslash o\{nnnn\} must represent a value that can be represented in a single code unit of the encoding of string or character literal they are a part of.
    
    Note that \tcode{"\textbackslash x\{4" "2\}"} would not be valid as escape sequences are replaced before string concatenation, which we think is the right design.
    
    
    \subsection{Is it worth it?}
    
    It is certainly not an important feature. Low cost, mild benefits.
    However, it should be relatively simple to write refactoring tools to migrate the old syntax to the new one for codebases interested in the added visibility and safety.
    
    
    \subsection{Should existing forms be deprecated?}
    
    No (we are not in the business of breakings everyone's code)!
    
    \subsection{Impact on existing implementations}
    
    No compiler currently accept \tcode{\textbackslash x\{} or \tcode{\textbackslash u\{} as valid syntax.
    Furthermore, while \tcode{\textbackslash o} is currently reserved for implementations, no tested implementation (GCC, Clang, MSVC, ICC) makes use of it.
    
    
    \subsection{Impact on C}
    
    This proposal does not impact C, however, the C committee could find that proposal interesting.
    
    
    \subsection{Impact on wording}
    
    While this does not constitute motivation, this proposal may simplify slightly CWG/SG-16 efforts to separate string concatenation/escape sequences replacement from
    conversion to execution encoding.
    
    \section{Prior arts and alternative considered}
    
    \tcode{\textbackslash u\{\}} is a valid syntax in rust and javascript.
    The syntax is also similar to that of \paper{P2071R0}
    
    \tcode{\textbackslash x\{\}} is a bit more novel - It is present in Perl and some regex syntaxes.
    However, most languages (python, D, Perl, javascript, rust, PHP) specify hexadecimal sequences to be exactly 2 hexadecimal digits long (\tcode{\textbackslash xFF}) which sidestep the issues described in this paper.
    
    Most languages surveyed follow in C and C++ footstep for the syntax of octal numbers (no braces, 1-3 digits), so this would novel indeed.
    
    As such, for consistency with other C++ proposal and existing art, we have not considered other syntaxes.
    
    \section{Wording}
    
    \begin{quote}
        
        \rSec1[lex.charset]{Character sets}
        
        [...]    
        
        \pnum
        The \grammarterm{universal-character-name} construct provides a way to name
        other characters.
        
        \begin{bnf}
            \nontermdef{hex-quad}\br
            hexadecimal-digit hexadecimal-digit hexadecimal-digit hexadecimal-digit
        \end{bnf}
        
        \begin{bnf}
            \nontermdef{universal-character-name}\br
            \terminal{\textbackslash u} hex-quad\br
            \terminal{\textbackslash U} hex-quad hex-quad\br
            \added{\terminal{\textbackslash u\{}hexadecimal-digit...  \terminal{\}}}
        \end{bnf}
        
        
        A \grammarterm{universal-character-name}
        designates the character in ISO/IEC 10646 (if any)
        whose code point is the hexadecimal number represented by
        the sequence of \grammarterm{hexadecimal-digit}s
        in the \grammarterm{universal-character-name}.
        The program is ill-formed if that number is not a code point
        or if it is a surrogate code point.
        
        [...]
        
        \rSec2[lex.ccon]{Character literals}
        
        [...]
        
        \begin{bnf}
            \nontermdef{numeric-escape-sequence}\br
            octal-escape-sequence\br
            hexadecimal-escape-sequence
        \end{bnf}
        
        \begin{bnf}
            \nontermdef{octal-escape-sequence}\br
            \terminal{\textbackslash} octal-digit\br
            \terminal{\textbackslash} octal-digit octal-digit\br
            \terminal{\textbackslash} octal-digit octal-digit octal-digit\br
            \added{\terminal{\textbackslash o\{}octal-digit...  \terminal{\}}}
        \end{bnf}
        
        \begin{bnf}
            \nontermdef{hexadecimal-escape-sequence}\br
            \terminal{\textbackslash x} hexadecimal-digit\br
            hexadecimal-escape-sequence hexadecimal-digit\br
            \added{\terminal{\textbackslash x\{}hexadecimal-digit...  \terminal{\}}}
        \end{bnf}
        
        \begin{bnf}
            \nontermdef{conditional-escape-sequence}\br
            \terminal{\textbackslash} conditional-escape-sequence-char
        \end{bnf}
        
    \end{quote}
    
    \section{Acknowledgments}
    
    \section{References}
    \renewcommand{\section}[2]{}%
    \bibliographystyle{plain}
    \bibliography{wg21}
    
    \begin{thebibliography}{9}
        
        
        \bibitem[Unicode]{Unicode}
        Unicode 13\newline
        \url{http://www.unicode.org/versions/Unicode13.0.0/}
        
        
        \bibitem[N4861]{N4861}
        Richard Smith
        \emph{Working Draft, Standard for Programming Language C++}\newline
        \url{https://wg21.link/N4861}
        
    \end{thebibliography}
\end{document}
