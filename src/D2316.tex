% !TeX program = luatex
% !TEX encoding = UTF-8


\RequirePackage{luatex85}


\documentclass{wg21}


\usepackage{luatexja-fontspec}

\setmainjfont{Noto Sans CJK KR}


\newcommand{\UnicodeLetter}[1]{\textbf{\textcolor{BrickRed}{\Large\tcode{#1}}}}

\title{	Consistent character literal encoding}
\docnumber{P2316R0}
\audience{EWG, SG-16}
\author{Corentin Jabot}{corentin.jabot@gmail.com}

\begin{document}
\maketitle

\paperquote{}

\section{Abstract}

Character literals in preprocessor conditional should behave like they do in C++ expression.

This proposal first appeared as part of \paper{P2178R1}.

\section{Motivation}

Consider the following code:

\begin{colorblock}
#if 'A' == '\x41'
//...
#endif
if ('A' == 0x41){}
\end{colorblock}

Both conditions are not guaranteed to yield a similar result, as \href{http://eel.is/c++draft/cpp#cond-12}{the value of character literals in preprocessor conditional is not required to be identical to that of character literals in expressions}.

However, a survey of the 1300+ open sources projects available on vcpkg shows that the primary use case for these macros is exactly to detect the
narrow literal encoding at compile time and all compilers available on compiler explorer treat these literals as if they were in the narrow literal encoding.

Notably, a few libraries use that pattern to detect EBCDIC or ASCII narrow literal encoding.
Of the ~50 usages of the pattern, all but one were in C libraries.

\begin{quoteblock}
For example, this code is in \href{https://github.com/sqlite/sqlite/blob/master/src/sqliteInt.h#L739}{sqlite.c}

\begin{colorblock}
/*
** Check to see if this machine uses EBCDIC.  (Yes, believe it or
** not, there are still machines out there that use EBCDIC.)
*/
#if 'A' == '\301'
# define SQLITE_EBCDIC 1
#else
# define SQLITE_ASCII 1
#endif
\end{colorblock}
\end{quoteblock}


While we think there should be a better way to detect encodings in C++ \cite{P1885R2}, there is no reason to deprecate that feature.

Instead, we recommend adopting the standard practice and user expectation of converting these literals to the narrow literal encoding before evaluating them.

\section{SG16 poll}

\begin{codeblock}
August 26th, 2020
Poll: Proposal 11 (of  @\paper{P2178R0}@): We agree that the same character encoding should be used 
for character literal in translation phase 4 and 7.

Attendees: 10
No objection to unanimous consent.
\end{codeblock}

\section{Wording}


\rSec1[cpp]{Preprocessing directives}%

\rSec2[cpp.cond]{Conditional inclusion}%

\pnum
The resulting tokens comprise the controlling constant expression
which is evaluated according to the rules of [expr.const]
using arithmetic that has at least the ranges specified
in [support.limits]. For the purposes of this token conversion and evaluation
all signed and unsigned integer types
act as if they have the same representation as, respectively,
\tcode{intmax_t} or \tcode{uintmax_t}\iref{cstdint}.
\begin{note}
    Thus on an
    implementation where \tcode{std::numeric_limits<int>::max()} is \tcode{0x7FFF}
    and \tcode{std::numeric_limits<unsigned int>::max()} is \tcode{0xFFFF},
    the integer literal \tcode{0x8000} is signed and positive within a \tcode{\#if}
    expression even though it is unsigned in translation phase
    7\iref{lex.phases}.
\end{note}
\begin{removedblock}
This includes interpreting \grammarterm{character-literal}s which may involve
converting escape sequences into execution character set members.
Whether the numeric value for these \grammarterm{character-literal}s
matches the value obtained when an identical \grammarterm{character-literal}
occurs in an expression
(other than within a
\tcode{\#if}
or
\tcode{\#elif}
directive)
is \impldef{numeric values of \grammarterm{character-literal}s in \tcode{\#if}
    directives}.
\begin{note}
    Thus, the constant expression in the following
    \tcode{\#if}
    directive and
    \tcode{if} statement\iref{stmt.if}
    is not guaranteed to evaluate to the same value in these two
    contexts:
    \begin{codeblock}
        #if 'z' - 'a' == 25
        if ('z' - 'a' == 25)
    \end{codeblock}
\end{note}
Also, whether a single-character \grammarterm{character-literal} may have a negative
value is \impldef{negative value of \grammarterm{character-literal} in preprocessor}.
\end{removedblock}
\begin{addedblock}
This include interpreting \grammarterm{character-literal} according to the rules in [lex.ccon].
\begin{note}
The associated character encodings of literals are the same in \tcode{\#if} and \tcode{\#elif} directives and in any expression.
\end{note}
\end{addedblock}

Each subexpression with type
\tcode{bool}
is subjected to integral promotion before processing continues.

\section{Acknowledgments}

\section{References}
\renewcommand{\section}[2]{}%
\bibliographystyle{plain}
\bibliography{wg21}

\begin{thebibliography}{9}

\bibitem[N4878]{N4878}
Richard Smith
\emph{Working Draft, Standard for Programming Language C++}\newline
\url{https://wg21.link/N4878}

\end{thebibliography}
\end{document}
