% !TeX program = luatex
% !TEX encoding = UTF-8

\documentclass{wg21}

\title{Iterators pair constructors for stack and queue}
\docnumber{P1425R4}
\audience{LEWG, LWG}
\author{Corentin Jabot}{corentin.jabot@gmail.com}


\begin{document}
\maketitle

\section{Abstract}

This paper proposes to add iterators-pair constructors to \tcode{std::stack} and \tcode{std::queue}

\section{Tony tables}
\begin{center}
\begin{tabular}{l|l}
Before & After\\ \hline

\begin{minipage}[t]{0.5\textwidth}
\begin{colorblock}
std::vector<int> v(42);
std::queue<int> q({v.begin(), v.end()});
std::stack<int> s({v.begin(), v.end()});
\end{colorblock}
\end{minipage}
&
\begin{minipage}[t]{0.3\textwidth}
\begin{colorblock}
std::vector<int> v(42);
std::queue q(v.begin(), v.end());
std::stack s(v.begin(), v.end());
\end{colorblock}
\end{minipage}
\\\\ \hline

\end{tabular}
\end{center}

\section{Revisions}

\subsection{R4}
\begin{itemize}
\item Replace the 2 previously suggested feature macro by \tcode{__cpp_lib_adaptor_iterator_pair_constructor}
\item Reorganize the ordfer of constructructors and put the allocators costructors on the approriate section
\item use \exposid{iter-value-type} as introduced by \paper{LWG3506}
\end{itemize}

\subsection{R3}
\begin{itemize}
    \item Add missing deduction guides for the constructors with allocators.\\
    \textbf{Since LEWG review, the deduction guides have been fixed to deduce the allocator type of the default container}.
    
\end{itemize}
\textbf{}
\subsection{R2}
\begin{itemize}
    \item Remove the \tcode{Container} argument
    \item Add allocator support - in alignment with \paper{LWG3506}
    \item Add feature test macros
\end{itemize}


\section{Motivation}

\tcode{std::stack} and \tcode{std::queue} do not provide iterators based constructors which is inconsistent.
This paper is an offshoot of \cite{P1206}, for which I conducted a review of existing containers and containers adapters constructors.

The lack of these constructors forces the implementation of \tcode{ranges::to} to special case container-adapters or to not support them.
Their absence make it also impossible to deduce their type using CTAD.

While this is a a small change, we believe its impact on the standard is low and consistent designs are less surprising and therefore easier
to use: with this change, all container-like types, whether they are \emph{Containers} or container adapters, can be constructed from 
an iterators pair, making them more compatible with \tcode{ranges}.


\section{Removal of the \tcode{Container} argument in R2}

Previous iteration had a \tcode{queue(Iterator first, Iterator last, Container c)} argument, which was added for consistency
with \tcode{priority_queue}.
However, it was specified to insert the range denoted by [first, last) at the end of \tcode{c}.
LWG astutely noted that this was confusing and wanted LEWG's input.

As a result, i decided to remove this argument entierly, as I can't think of a non-confusing fix, 
nor can I really come up with a good justification for this parameter.

\begin{itemize}
\item Changing the order of parameters would be inconsistent with \tcode{priority_queue}.
\item Mandating that the range is inserted at the begining of the container has performance drawbacks.
\item It is unclear that using the order of parameter to determine the order of insertion would actually make sense to users.
\end{itemize}

\section{Implementation}

This proposal has been \href{https://github.com/cor3ntin/llvm-project/tree/stack_queue_iterators}{Implemented in libc++}

\section{Proposed Wording}

\textbf{Note for LWG \& the editors: this wording uses the exposition only \exposid{iter-value-type} type which
is introduced by the accepted resolution to issue \paper{LWG3506}}

\rSec2[queue.defn]{Definition}
	
\begin{codeblock}
	
	
namespace std {
	template<class T, class Container = deque<T>>
	class queue {
		public:
		using value_type      = typename Container::value_type;
		using reference       = typename Container::reference;
		using const_reference = typename Container::const_reference;
		using size_type       = typename Container::size_type;
		using container_type  =          Container;
		
		protected:
		Container c;
		
		public:
		queue() : queue(Container()) {}
		explicit queue(const Container&);
		explicit queue(Container&&);
		
	   	@\added{template<class InputIterator>}@
                @\added{queue(InputIterator first, InputIterator last);}@
				
		template<class Alloc> explicit queue(const Alloc&);
		template<class Alloc> queue(const Container&, const Alloc&);
		template<class Alloc> queue(Container&&, const Alloc&);
		template<class Alloc> queue(const queue&, const Alloc&);
		template<class Alloc> queue(queue&&, const Alloc&);
        
                @\added{template<class InputIterator, class Alloc>}@
                @\added{queue(InputIterator first, InputIterator last, const Alloc\&);}@
		
		//...
	};
	
	template<class Container>
	queue(Container) -> queue<typename Container::value_type, Container>;
    
\end{codeblock}
\begin{addedblock}  
\begin{codeblock}
	
	template<class InputIterator>
	queue(InputIterator, InputIterator) -> queue<@\exposid{iter-value-type}@<InputIterator>>;

\end{codeblock}
\end{addedblock}  

\begin{codeblock}
	template<class Container, class Allocator>
	queue(Container, Allocator) -> queue<typename Container::value_type, Container>;
\end{codeblock}
\begin{addedblock}  
\begin{codeblock}
        template<class InputIterator, class Allocator>
        queue(InputIterator, InputIterator, Allocator)
        -> queue<@\exposid{iter-value-type}@<InputIterator>, deque<@\exposid{iter-value-type}@<InputIterator>, Allocator>>;
\end{codeblock}
\end{addedblock}
\begin{codeblock}
	
	template<class T, class Container>
	void swap(queue<T, Container>& x, queue<T, Container>& y) noexcept(noexcept(x.swap(y)));
	
	template<class T, class Container, class Alloc>
	struct uses_allocator<queue<T, Container>, Alloc>
	: uses_allocator<Container, Alloc>::type { };
}
\end{codeblock}

\rSec2[queue.cons]{Constructors}

\begin{itemdecl}
	explicit queue(const Container& cont);
\end{itemdecl}

\begin{itemdescr}
	\pnum
	\effects\ Initializes \tcode{c} with \tcode{cont}.
\end{itemdescr}

\begin{itemdecl}
	explicit queue(Container&& cont);
\end{itemdecl}

\begin{itemdescr}
	\pnum
	\effects\ Initializes \tcode{c} with \tcode{std::move(cont)}.
\end{itemdescr}

\begin{addedblock}

\begin{itemdecl}
    template<class InputIterator>
    queue(InputIterator first, InputIterator last);
\end{itemdecl}

\begin{itemdescr}
\pnum
\effects
Initializes \tcode{c} with \tcode{first} as the first argument and \tcode{last} as the second argument.
\end{itemdescr}

\end{addedblock}

\rSec3[queue.cons.alloc]{Constructors with allocators}

[...]

\begin{itemdecl}
    template<class Alloc> queue(const queue& q, const Alloc& a);
\end{itemdecl}

\begin{itemdescr}
    \pnum
    \effects
    Initializes \tcode{c} with \tcode{q.c} as the first argument and \tcode{a} as the
    second argument.
\end{itemdescr}

\begin{itemdecl}
    template<class Alloc> queue(queue&& q, const Alloc& a);
\end{itemdecl}

\begin{itemdescr}
    \pnum
    \effects
    Initializes \tcode{c} with \tcode{std::move(q.c)} as the first argument and \tcode{a}
    as the second argument.
\end{itemdescr}


\begin{addedblock}

\begin{itemdecl}
    template<class InputIterator, typename Alloc>
    queue(InputIterator first, InputIterator last, const Alloc & alloc);
\end{itemdecl}

\begin{itemdescr}
\pnum
\effects
Initializes \tcode{c} with \tcode{first} as the first argument, \tcode{last} as the second argument and \tcode{alloc} as the third argument.
\end{itemdescr}


\end{addedblock}

\rSec3[stack.defn]{Definition}

\begin{codeblock}
namespace std {
	template<class T, class Container = deque<T>>
	class stack {
		public:
		using value_type      = typename Container::value_type;
		using reference       = typename Container::reference;
		using const_reference = typename Container::const_reference;
		using size_type       = typename Container::size_type;
		using container_type  = Container;
		
		protected:
		Container c;
		
		public:
		stack() : stack(Container()) {}
		explicit stack(const Container&);
		explicit stack(Container&&);
		
		@\added{template<class InputIterator>}@
                @\added{stack(InputIterator first, InputIterator last);}@
		
		template<class Alloc> explicit stack(const Alloc&);
		template<class Alloc> stack(const Container&, const Alloc&);
		template<class Alloc> stack(Container&&, const Alloc&);
		template<class Alloc> stack(const stack&, const Alloc&);
		template<class Alloc> stack(stack&&, const Alloc&);
        
        
                @\added{template<class InputIterator, class Alloc>}@
                @\added{stack(InputIterator first, InputIterator last, const Alloc\&);}@
		
		//...
	};
	
	template<class Container>
	stack(Container) -> stack<typename Container::value_type, Container>;
	
	@\added{template<class InputIterator>}@
	@\added{stack(InputIterator, InputIterator)}@
	@\added{-> stack<\exposid{iter-value-type}<InputIterator>>;}@
        
	template<class Container, class Allocator>
	stack(Container, Allocator) -> stack<typename Container::value_type, Container>;
    
        @\added{template<class InputIterator, class Allocator>}@
        @\added{stack(InputIterator, InputIterator, Allocator)}@
        @\added{-> stack<\exposid{iter-value-type}<InputIterator>, 
            deque<\exposid{iter-value-type}<InputIterator>, Allocator>>}@
    
	
	template<class T, class Container, class Alloc>
	struct uses_allocator<stack<T, Container>, Alloc>
	: uses_allocator<Container, Alloc>::type { };
}
\end{codeblock}

\rSec3[stack.cons]{Constructors}

\begin{itemdecl}
	explicit stack(const Container& cont);
\end{itemdecl}

\begin{itemdescr}
	\pnum
	\effects Initializes \tcode{c} with \tcode{cont}.
\end{itemdescr}

\begin{itemdecl}
	explicit stack(Container&& cont);
\end{itemdecl}

\begin{itemdescr}
	\pnum
	\effects Initializes \tcode{c} with \tcode{std::move(cont)}.
\end{itemdescr}


\begin{addedblock}
	
\begin{itemdecl}
    template<class InputIterator>
    stack(InputIterator first, InputIterator last);
\end{itemdecl}

\begin{itemdescr}
\pnum
\effects
Initializes \tcode{c} with \tcode{first} as the first argument and \tcode{last} as the second argument.
\end{itemdescr}

\end{addedblock}

\rSec3[stack.cons.alloc]{Constructors with allocators}

[...]

\indexlibraryctor{stack}%
\begin{itemdecl}
    template<class Alloc> stack(const stack& s, const Alloc& a);
\end{itemdecl}

\begin{itemdescr}
    \pnum
    \effects
    Initializes \tcode{c} with \tcode{s.c} as the first argument and \tcode{a}
    as the second argument.
\end{itemdescr}

\indexlibraryctor{stack}%
\begin{itemdecl}
    template<class Alloc> stack(stack&& s, const Alloc& a);
\end{itemdecl}

\begin{itemdescr}
    \pnum
    \effects
    Initializes \tcode{c} with \tcode{std::move(s.c)} as the first argument and \tcode{a}
    as the second argument.
\end{itemdescr}

\begin{addedblock}

\begin{itemdecl}
    template<class InputIterator, typename Alloc>
    stack(InputIterator first, InputIterator last, const Alloc & alloc);
\end{itemdecl}

\begin{itemdescr}
\pnum
\effects
Initializes \tcode{c} with \tcode{first} as the first argument, \tcode{last} as the second argument and \tcode{alloc} as the third argument.
\end{itemdescr}

\end{addedblock}


\section{Feature test macros}

Insert into [version.syn]

\begin{addedblock}
\begin{codeblock}
#define __cpp_lib_adaptor_iterator_pair_constructor <DATE OF ADOPTION>
\end{codeblock}
\end{addedblock}

\section{Acknowledgment}

Thanks to Eric Niebler who reviewed the wording

\section{References}
\renewcommand{\section}[2]{}%
\begin{thebibliography}{9}
	
	\bibitem[P1206]{P1206}
	Corentin Jabot
	\emph{A function to convert any range to a container}\newline
	\url{https://wg21.link/P1206}

\end{thebibliography}
\end{document}
