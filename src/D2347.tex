% !TeX document-id = {9322a846-f757-4574-9231-a2e85c743b21}
% !TeX program = luatex
% !TEX encoding = UTF-8


\RequirePackage{luatex85}%
\documentclass{wg21}

\usepackage{luatexja-fontspec}
\RequirePackageWithOptions{fontspec}
\usepackage{newunicodechar}

\setmainfont{Noto Sans}
\setmainjfont{Noto Sans CJK KR}

\newfontfamily{\fallbackfont}{Noto Sans}
\DeclareTextFontCommand{\textfallback}{\fallbackfont}
\newunicodechar{ẞ}{\textfallback{ẞ}}


\newcommand{\UnicodeLetter}[1]{\textbf{\textcolor{BrickRed}{\Large\tcode{#1}}}}


\title{Type deduction for non-trailing parameter packs}
\docnumber{D2347R0}
\audience{EWG}
\author{Corentin Jabot}{corentin.jabot@gmail.com}
\authortwo{Bruno Manganelli}{bruno.manga95@gmail.com}

\begin{document}
\maketitle

\section{Abstract}

\section{Motivation}

\paper{P0478R0}

\href{https://cor3ntin.github.io/posts/variadic/}{Non-terminal variadic template parameters}

\section{Design}

The general idea is to deduce a single pack, and to deduce the size of that pack such that once expanded,
the argument list matches the size of the parameter-list, excluding any defaulted parameter.

Questions:
\begin{itemize}
    \item Should partial ordering be modified ?
    \item Should we apply the same logic to type deduction ?
\end{itemize}

\section{Wording}

\rSec3[temp.deduct.call]{Deducing template arguments from a function call}

//...

\added{Let N be the number of remaining non-defaulted template function parameters parameters and K be the numbers of remaining arguments of the call.}

For a function parameter pack \removed{that occurs at the end
of the \grammarterm{parameter-declaration-list}},
deduction is performed for \changed{each remaining}{the next K-N} argument of the call,
taking the type \tcode{P}
of the \grammarterm{declarator-id} of the function parameter pack
as the corresponding function template parameter type.
Each deduction deduces template arguments for subsequent positions in
the template parameter packs expanded by the function parameter pack.
When a function parameter pack appears in a non-deduced
context\iref{temp.deduct.type}, the type of that pack is
never deduced.
\begin{example}
\begin{removedblock}
\begin{codeblock}
    template<class ... Types> void f(Types& ...);
    template<class T1, class ... Types> void g(T1, Types ...);
    template<class T1, class ... Types> void g1(Types ..., T1);
    
    void h(int x, float& y) {
        const int z = x;
        f(x, y, z);                   // \tcode{Types} deduced as \tcode{int}, \tcode{float}, \tcode{const int}
        g(x, y, z);                   // \tcode{T1} deduced as \tcode{int}; \tcode{Types} deduced as \tcode{float}, \tcode{int}
        g1(x, y, z);                  // error: \tcode{Types} is not deduced
        g1<int, int, int>(x, y, z);   // OK, no deduction occurs
    }
\end{codeblock}   
\end{removedblock}

\begin{addedblock}
\begin{codeblock}
template<class ... Types> void f(Types& ...);
template<class T1, class ... Types> void g(T1, Types ...);
template<class T1, class ... Types> void g1(Types ..., T1);
template<class ... Types> void g2(int, Types ..., int = 0);
template<class ... Types, class T2, class... OtherTypes> void g3(Types ..., T2, OtherTypes...);

void h(int x, float& y) {
    const int z = x;
    f(x, y, z);  // \tcode{Types} deduced as \tcode{int}, \tcode{float}, \tcode{const int}
    g(x, y, z);  // \tcode{T1} deduced as \tcode{int}; \tcode{Types} deduced as \tcode{float}, \tcode{int}
    g1(x, y, z); // \tcode{Types} deduced as \tcode{float}, \tcode{int}, \tcode{T1} deduced as \tcode{int};
    g2(x, x);    // \tcode{Types} deduced as \tcode{int};
    g2(x);       // \tcode{sizeof...(Types) == 0};
    g3(x, y, z); // error: \tcode{Types} is not deduced, \tcode{OtherTypes} is not deduced
    g3<int, int, int>(x, y, z);   // OK, no deduction occurs
}
\end{codeblock}
\end{addedblock}

\end{example}

\rSec3[temp.deduct.partial]{Deducing template arguments during partial ordering}

\pnum
Using the resulting types
\tcode{P}
and
\tcode{A},
the deduction is then done as described in [temp.deduct.type].

\begin{addedblock}
Let N be the number of remaining non-defaulted template function parameters and K be the numbers of remaining arguments of the call.
\end{addedblock}

If \tcode{P} is a function parameter pack, 
the type \tcode{A} of each \changed{remaining}{of the the next {K-N}}
parameter types of the argument template is compared with the type \tcode{P} of
the \grammarterm{declarator-id} of the function parameter pack. Each comparison
deduces template arguments for subsequent positions in the template parameter
packs expanded by the function parameter pack.
Similarly, if \tcode{A} was transformed from a function parameter pack,
it is compared with each  \changed{remaining}{of the the next {N-K}} parameter type of the parameter template.
\begin{addedblock}
\begin{example}
\begin{codeblock}
template<class Args.., class TLast> void f(Args.., int i = 42);
f(0); // sizeof...(Args) == 1
f(0, 0); // sizeof...(Args) == 2
\end{codeblock}
\end{example}
\end{addedblock}


If deduction succeeds for a given type,
the type from the argument template is considered to be at least as specialized
as the type from the parameter template.
\begin{example}
    \begin{codeblock}
        template<class... Args>              void f(Args... args);        // \#1
        template<class T1, class... Args>    void f(T1 a1, Args... args); // \#2
        template<class T1, class T2>         void f(T1 a1, T2 a2);        // \#3
        template<class Args.., class TLast>  void g(Args.., int i = 42); //  \#4
        
        f();                // calls \#1
        f(1, 2, 3);         // calls \#2
        f(1, 2);            // calls \#3; non-variadic template \#3 is more specialized
        // than the variadic templates \#1 and \#2
    \end{codeblock}
\end{example}


\rSec3[temp.deduct.type]{Deducing template arguments from a type}

\pnum
The non-deduced contexts are:

\indextext{context!non-deduced}%
\begin{itemize}
\item
The
\grammarterm{nested-name-specifier}
of a type that was specified using a
\grammarterm{qualified-id}.
\item
The \grammarterm{expression} of a \grammarterm{decltype-specifier}.
\item
A non-type template argument or an array bound in which a subexpression
references a template parameter.
\item
A template parameter used in the parameter type of a function parameter that
has a default argument that is being used in the call for which argument
deduction is being done.
\item
A function parameter for which the associated argument is an
overload set\iref{over.over}, and one or more of the following apply:
\begin{itemize}
    \item
    more than one function matches the function parameter type (resulting in
    an ambiguous deduction), or
    \item
    no function matches the function parameter type, or
    \item
    the overload set supplied as an argument contains one or more function templates.
\end{itemize}
\item A function parameter for which the associated argument is an initializer
list\iref{dcl.init.list} but the parameter does not have
a type for which deduction from an initializer list is specified\iref{temp.deduct.call}.
\begin{example}
    \begin{codeblock}
        template<class T> void g(T);
        g({1,2,3});                 // error: no argument deduced for \tcode{T}
    \end{codeblock}
\end{example}
\item A function parameter pack that \changed{does not occur at the end of}{is not the only parameter pack in} the \grammarterm{parameter-declaration-list}.
\end{itemize}

// ...

\pnum

%% \added{Let N be the number of remaining template parameters and K be the numbers of remaining types.}

%If \tcode{P} has a form that contains \tcode{<T>}
%or \tcode{<i>}, then each argument $\mathtt{P}_i$ of the
%respective template argument list of \tcode{P} is compared with the
%corresponding argument $\mathtt{A}_i$ of the corresponding
%template argument list of \tcode{A}. If the template argument list
%of \tcode{P} contains a pack expansion that is not the \changed{last template argument}{only pack expansion}, the entire template argument list is a non-deduced
%context. If $\texttt{P}_i$ is a pack expansion, then the pattern
%of $\texttt{P}_i$ is compared with \changed{each remaining}{the N following} argument\added{s} in the
%template argument list of \tcode{A} \added{where N is such that the deduced parameter list has the same size as the expanded list of arguments}. Each comparison deduces
%template arguments for subsequent positions in the template parameter
%packs expanded by $\texttt{P}_i$.
%During partial ordering\iref{temp.deduct.partial}, if $\texttt{A}_i$ was
%originally a pack expansion:
%\begin{itemize}
%    \item if \tcode{P} does not contain a template argument corresponding to
%    $\texttt{A}_i$ then $\texttt{A}_i$ is ignored;
%    
%    \item otherwise, if $\texttt{P}_i$ is not a pack expansion, template argument
%    deduction fails.
%\end{itemize}
%\begin{example}
%    \begin{codeblock}
%        template<class T1, class... Z> class S;                                 // \#1
%        template<class T1, class... Z> class S<T1, const Z&...> { };            // \#2
%        template<class T1, class T2>   class S<T1, const T2&> { };              // \#3
%        S<int, const int&> s;           // both \#2 and \#3 match; \#3 is more specialized
%        
%        template<class T, class... U>            struct A { };                  // \#1
%        template<class T1, class T2, class... U> struct A<T1, T2*, U...> { };   // \#2
%        template<class T1, class T2>             struct A<T1, T2> { };          // \#3
%        template struct A<int, int*>;   // selects \#2
%    \end{codeblock}
%\end{example}

\section{Acknowledgments}



\bibliographystyle{plain}
\bibliography{wg21}


\renewcommand{\section}[2]{}%
\begin{thebibliography}{9}
    
    \bibitem[N4885]{N4885}
    Thomas Köppe
    \emph{Working Draft, Standard for Programming Language C++}\newline
    \url{https://wg21.link/N4885}
     
\end{thebibliography}



\end{document}
