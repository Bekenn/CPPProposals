% !TeX program = luatex
% !TEX encoding = UTF-8


\RequirePackage{luatex85}%
\documentclass{wg21}


\title{A nice placeholder with no name}
\docnumber{P2169R3}
\audience{EWG}
\author{Corentin Jabot}{corentin.jabot@gmail.com}
\authortwo{Michael Park}{mcypark@gmail.com}

\begin{document}
\maketitle


\section{Abstract}

In [P1110]\cite{P1110R0} Jeffrey Yesskin and JF Bastien explore the design space of placeholders
with the goal of not naming entities for which a name would provide no additional information.
In this paper, we propose a concrete solution for the use of \tcode{_} (\tcode{U+005F _ LOW LINE}) as such placeholder
for variable declarations and pattern matching \textbf{in a fully backward compatible manner}.

\section{Tony tables}
\begin{center}
\begin{tabular}{l|l}
Before & After\\ \hline

\begin{minipage}[t]{0.5\textwidth}
\begin{lstlisting}[style=color]
std::lock_guard namingIsHard(mutex);
// Structured binding
[[maybe_unused]] auto [x, y, iDontCare] = f();
// Pattern matching
inspect(foo) { __  => bar; };

\end{lstlisting}
\end{minipage}
&
\begin{minipage}[t]{0.5\textwidth}
\begin{lstlisting}[style=color]
std::lock_guard _(mutex);
// Structured binding
auto [x, y, _] = f();
// Pattern matching
inspect(foo) { _  => bar; };

\end{lstlisting}
\end{minipage}
\\\\ \hline

\end{tabular}
\end{center}

\section{Revisions}

\subsection{Revision 3}

\begin{itemize}
\item Add wording
\item R2 mentioned that we could allow _ for global variables at namespace scope in module implementation units. While true, this seems too much of a weird allowance.
\end{itemize}

\subsection{Revision 2}

\begin{itemize}
    \item A few more details about interactions with P2011
\end{itemize}

\subsection{Revision 1}

\begin{itemize}
   \item Specify that placeholders can appear as class members
   \item Make use of \tcode{_} ill-formed once a placeholder has been declared in the same scope
\end{itemize}

\section{Motivation}

Both [P1110]\cite{P1110R0}, [P1469]\cite{P1469R0} and [P0577]\cite{P0577R0} go over the motivation both for a placeholder syntax in general and
the single underscore \tcode{_}:

\begin{itemize}
\item Some variable, like locks, \tcode{scope_guard} and other RAII objects are only used for their side effects.

\begin{lstlisting}[style=color]
std::lock_guard lock(mutex);
\end{lstlisting}

\item Some structured bindings may not be used, and we would like a syntax to specify that this is the case. Introducing names for variables that
are not used to convey less meaning than \tcode{_}, which convey the "I don't care meaning".
It is furthermore not possible to apply attributes to structured bindings and as such, it is not possible to carry the developer's intent to
the compiler.
The current wording advises:

\begin{quoteblock}
Recommended practice: For an entity marked \tcode{[[maybe_unused]]}, implementations should not emit a warning that the entity or its structured bindings (if any) are used or unused. For a structured binding declaration not marked \tcode{[[maybe_unused]]}, implementations should not emit such a warning unless all of its structured bindings are unused.
\end{quoteblock}

\item We need a wildcard token for pattern matching. The proposed \tcode{_} is, as explained by [P1469]\cite{P1469R0}
an industry wide-standard and we believe having a single token \tcode{_} for both wildcards and placeholders contribute
to make C++ more consistent with itself and with other languages (making easier to teach, etc)

\end{itemize}

\section{Design considerations}

The design as presented in this paper takes the following design parameters into consideration

\begin{itemize}
\item \tcode{_} is recognized as some kind of placeholder in many existing languages, including \tcode{go}, \tcode{python}, \tcode{rust}, \tcode{C\#}.
\item \tcode{_} is used in existing code and popular frameworks.
\item \tcode{_} might be defined as a macro function in some files depending on the \tcode{gettext} library.
\item \tcode{_} already carries the meaning of "I don't want to use this variable". Variables that developers care about have more meaningful names.
\item \tcode{_} unnamed variables cannot have linkage.
\item \tcode{_} and \tcode{__} are similar enough that we should not give meaning to both.
\item Tokens are rare, using a different token for placeholder may needlessly restrict the design space for more important features.
\item For consistency, we would like to use the same token as the pattern matching placeholder.
\end{itemize}

\subsection{Proposal}

We propose that when \tcode{_} is used as the identifier for the declaration of a variable, non static class member variable, function parameter name,
lambda capture or structured binding.
the introduced name is implicitly given the \tcode{[[maybe_unused]]} attribute.

\begin{example}
\begin{lstlisting}[style=color]
auto _ = foo(); // equivalent to [[maybe_unused]] auto _  = f();
\end{lstlisting}
\end{example}

When a variable with automatic storage duration, a function parameter, a non-static class member variable, structured binding or lambda capture is introduced by the identifier \tcode{_}
it can redefine an existing declaration, [basic.scope.declarative] in the same scope. After redeclaration, if the variable is used, the programm is ill-formed.

\begin{example}
\begin{lstlisting}[style=color]
namespace a {
    auto _ = f(); // Ok, declare a variable "_"
    auto _ = f(); // error: "_" is already defined in this namespace scope
}
void f() {
    auto _ = 42; // Ok, declare a variable "_"
    auto _ = 0;  // Ok, re-declare a variable "_"
    {
        auto _ = 1; // Ok, shaddowing
        assert( _ == 1 ); // Ok
    }
    assert( _ == 42 ); // ill-formed: Use of a redeclared placeholder variables
}
\end{lstlisting}
\end{example}

In contexts where the grammar expects a pattern matching pattern, \tcode{_} represents the wildcard pattern.

\begin{lstlisting}[style=color]
inspect(i) {
    0  => 0;
    _  => 1;   // wildcard pattern
};
\end{lstlisting}

\subsection{Should declaring variables called \tcode{_} be deprecated?}

This paper does not propose to deprecate variables called \tcode{_}, as a few frameworks have legitimate and widely deployed usages of it
(see the section about google mock), and a depreciation is not necessary for the well-behaving of this proposal.
We do however think that such depreciation may be useful to consider in a longer time frame.

\subsection{Alternative Design}

A previous version of this proposal suggested that \tcode{_} should refer to the first declaration in scope,
but many prefered the approach in this current revision.

\subsection{Pattern matching}

Citing [P1371R0]{\cite{P1371R0}, \tcode{_} can be used as a pattern matching wilcard without ambiguities

\begin{quoteblock}
Even though \tcode{_} is a valid identifier, it does not introduce a name as doing so would result in redeclaration
errors in the case where multiple wildcard \tcode{_} identifiers are used.
It is possible for users to have already introduced _ as a type or variable name in the same scope where an
inspect statement is used. As a highly-visible example, the authors are aware of the use of _ as a name in
the popular “Google Mock” library. Idiomatically this is accessed by introducing _ into the current namespace
or block scope with a using declaration. Using the wildcard pattern in cases like this is unambiguous since
the expression pattern requires a ˆ introducer for primary expressions. \tcode{\^_} will always match against an
existing name and \tcode{_} will always represent the wildcard pattern. An existing \tcode{_} name can be used without
ambiguity in the matched statement to which control is passed.
Naturally, the impact of defining \tcode{_} in the pre-processor cannot be predicted or controlled by this paper and
is thus liable to result in an ill-formed program.
\end{quoteblock}

\subsection{Impact on P2011 and other pipeline rewrite musing}

Placeholders in pipeline rewrite have a "Bind to this" semantic rather than a "I don't care" semantic.
As such, these proposals could use a different syntax.
However, there are no syntactic ambiguities between the two features, P2169 impacts only declarations, not expressions.

Independently of P2169, P2011 makes the following ambiguous

\begin{lstlisting}[style=color]
int _ = 0; // Not a placeholder
f() |> g(_, _); // placeholder or variable ?
\end{lstlisting}

This is simply resolved by having \tcode{_} always be a placeholder in pipeline expression,
and any \tcode{_} variable can be access either through a reference or by wrapping it in parentheses.

\begin{lstlisting}[style=color]
int _ = 0; // Not a placeholder
auto & placeholder = _;
f() |> g(_, placeholder); // OK
f() |> g(_, (_)); // OK
\end{lstlisting}


And so,

\begin{itemize}
\item There are no adverse synergies between this proposal and pipeline rewrite
\item There are no adverse synergies between this proposal and pattern matching
\item This proposal and pattern matching use the \tcode{_} token for similar semantics ("don't care"), whether the pipeline rewrite has a different semantic ("bind to this argument").
However, both are placeholders and an argument can also be made to use the same syntax in both context, it really boils down to what we think is less surprising.
\end{itemize}


\section{Impact on existing code}

\subsection{Google Mock}

Google Mock appear to constitute the most common use of the \tcode{identifier}.
With this proposal, Google Mock continues to work.
Variables named \tcode{_} may shadow the \tcode{testing::_} global variable introduced by google mock:

\begin{lstlisting}[style=color]
namespace testing {
    const internal::AnythingMatcher _ = {};
}
\end{lstlisting}

Because the proposal limits the use of placeholder to function scope (because of linker considerations),
there is little risk that increased use of \tcode{_} causes shadowing issues for users of this framework.

It is possible to use the feature proposed in this paper along with google mock as long as the \tcode{using namespace testing;}
appears before any declaration of \tcode{_} in that code.

\href{https://godbolt.org/z/EghbHF}{Compiler explorer link}

\subsection{Gettext}

Some projects using Gettext define \tcode{_} as a function-like macro.

\begin{lstlisting}[style=color]
# define _(msgid) gettext (msgid)
\end{lstlisting}

It is important to know that this is not provided in any gettext header.
It is also technically undefined as the \tcode{_} is reserved at global scope.
But, regardless, defining this macro would not prevent the use of the proposed syntax, as \tcode{_} is only
expanded by the preprocessor in this case if followed by parentheses:

\begin{lstlisting}[style=color]
constexpr const char* gettext(int) { return nullptr;}
#define _(msgid) gettext (msgid)

int main() {
    constexpr auto _ = _(42);
    auto _ = 42;
    static_assert(_ == nullptr);
}

\end{lstlisting}

\href{https://godbolt.org/z/FRFg9-}{Compiler explorer link}

\section{Implementation}

R0 of this proposal has been implemented in Clang and the implementation is available on \href{https://godbolt.org/z/5lmnfN}{Compiler Explorer}.

The implementation consists of ignoring existing \tcode{_} variables in the same scope when declaring placeholder variables.
Note that name lookup is not affected by this proposal, some variables are simply skipped over based on their name.
The variables otherwise conserve their name for diagnostics purposes.

The current version of this proposal can be emulaterd with the \tcode{-Werror} flag \href{https://godbolt.org/z/_C6X7L}{Compiler Explorer}.


\section{Wording}

\rSec2[basic.scope.scope]{General}

\begin{addedblock}
A declaration is \grammarterm{name independant} if its name is \tcode{_},
and it declares a variable with automatic storage duration, a structured binding, a function parameter, the variable introduced by an \grammarterm{init-capture}, or a non-static data member.

The \grammarterm{attribute-token} \tcode{maybe_unused} is applied to name independant declarations.

%\recommended
%An implementation should not emit a warning that
%a \grammarterm{redefinable underscore entity}
%is unused.
\end{addedblock}

Two declarations \defn{potentially conflict}
if they correspond and
cause their shared name to denote different entities\iref{basic.link}.
The program is ill-formed
if, in any scope, a name is bound to two declarations \added{A and B}
that potentially conflict and \changed{one}{A} precedes \changed{the other}{B} \iref{basic.lookup}\added{, unless B is name independant}.

\begin{addedblock}
\begin{note}
An \grammarterm{id-expression} that denotes a name independant declaration is ill-formed when another declaration with the same name inhabits the same scope.
\end{note}
\end{addedblock}


\begin{note}
	Overload resolution can consider potentially conflicting declarations
	found in multiple scopes
	(e.g. via \grammarterm{using-directive}s or for operator functions),
	in which case it is often ambiguous.
\end{note}
\begin{example}
	\begin{codeblock}
		void f() {
			int x,y;
			void x();             // error: different entity for \tcode{x}
			int y;                  // error: redefinition

                        @\added{    int _;}@
			@\added{    _ = 0;                  // OK}@
                        @\added{    int _;                  // OK}@
			@\added{    _ = 0;                  // error: id-expression '_' denoting a \grammarterm{redefinable underscore entity}}@
		}
		enum { f };             // error: different entity for \tcode{::f}
		namespace A {}
		namespace B = A;
		namespace B = A;        // OK, no effect
		namespace B = B;        // OK, no effect
		namespace A = B;        // OK, no effect
		namespace B {}          // error: different entity for \tcode{B}
	\end{codeblock}
\end{example}

\textcolor{noteclr}{[...]}

\rSec2[basic.scope.block]{Block scope}

\textcolor{noteclr}{[...]}

\pnum
If a declaration \added{that is not a name independant declaration and} whose target scope is the block scope $S$ of a
\begin{itemize}
    \item
    \grammarterm{compound-statement} of a \grammarterm{lambda-expression},
    \grammarterm{function-body}, or \grammarterm{function-try-block},
    \item
    substatement of a selection or iteration statement
    that is not itself a selection or iteration statement, or
    \item
    \grammarterm{handler} of a \grammarterm{function-try-block}
\end{itemize}
potentially conflicts with a declaration
whose target scope is the parent scope of $S$,
the program is ill-formed.
\begin{example}
    \begin{codeblock}
        if (int x = f()) {
            int x;            // error: redeclaration of \tcode{x}
        }
        else {
            int x;            // error: redeclaration of \tcode{x}
        }
    \end{codeblock}
\end{example}

\textcolor{noteclr}{[...]}

\rSec1[basic.link]{Program and linkage}%

\ednote{Modify [basic.link]/p8 as follow}

\pnum
Two declarations of entities declare the same entity
if, considering declarations of unnamed types to introduce their names
for linkage purposes, if any\iref{dcl.typedef,dcl.enum},
they correspond\iref{basic.scope.scope},
have the same target scope that is not a function or template parameter scope,
\added{neither is a name independant declaration, }
and either
\begin{itemize}
    \item
    they appear in the same translation unit, or
    \item
    they both declare names with module linkage and are attached to the same module, or
    \item
    they both declare names with external linkage.
\end{itemize}
\begin{note}
    There are other circumstances in which declarations declare the same entity%
    \iref{dcl.link,temp.type,temp.spec.partial}.
\end{note}

\textcolor{noteclr}{[...]}

\rSec1[namespace.udecl]{The \tcode{using} declaration}%

\textcolor{noteclr}{[...]}

\pnum
If a declaration \added{A} named by a \grammarterm{using-declaration}
that inhabits the target scope of another declaration \added{B}
potentially conflicts with it \added{, either B is not name independant or A doesn't precedes B}\iref{basic.scope.scope}, and
either is reachable from the other, the program is ill-formed.
If two declarations named by \grammarterm{using-declaration}{s}
that inhabit the same scope potentially conflict,
either is reachable from the other, and
they do not both declare functions or function templates,
the program is ill-formed.
\begin{note}
    Overload resolution possibly cannot distinguish
    between conflicting function declarations.
\end{note}


%\rSec2[basic.lookup.general]{General}%
%\pnum
%The name lookup rules apply uniformly to all names (including
%\grammarterm{typedef-name}{s}\iref{dcl.typedef},
%\grammarterm{namespace-name}{s}\iref{basic.namespace}, and
%\grammarterm{class-name}{s}\iref{class.name}) wherever the grammar allows
%such names in the context discussed by a particular rule. Name lookup
%associates the use of a name with a set of declarations\iref{basic.def} of
%that name.
%Unless otherwise specified,
%the program is ill-formed if no declarations are found.
%If the declarations found by name lookup
%all denote functions or function templates,
%the declarations are said to form an \defn{overload set}.
%
%\begin{addedblock}
%Otherwise, declarations found by name lookup are ambiguous unless they either:
%\begin{itemize}
%\item all denote the same entity, or
%\item all denote \grammarterm{redeclarable placeholder entities} in the same scope.
%\end{itemize}
%If name lookup finds ambiguous declarations, the program is ill-formed.
%\end{addedblock}
%
%\removed{Otherwise, if the declarations found by name lookup do not all denote the same entity,
%\indextext{lookup!ambiguous}%
%they are \defn{ambiguous} and the program is ill-formed.}
%Overload resolution\iref{over.match,over.over}
%takes place after name lookup has succeeded. The access rules\iref{class.access}
%are considered only once name lookup and
%function overload resolution (if applicable) have succeeded. Only after
%name lookup, function overload resolution (if applicable) and access
%checking have succeeded
%are the semantic properties introduced by the declarations
%used in further processing.

\section{Rebuttal some wilder ideas explored by P1110}

[P1110] explores the different places where a placeholder may be used.


\subsection{Enums}

The code on the left (P1110) can be written in standard C++ in a way that expresses the intent as clearly, if not more.

\begin{center}
\begin{tabular}{l|l}
\begin{minipage}[t]{0.5\textwidth}
\begin{lstlisting}[style=color]
enum MmapBits {
    Shared,
    Private,
    _,
    _,
    Fixed,
    Rename,
    //...
};

\end{lstlisting}
\end{minipage}
&
\begin{minipage}[t]{0.5\textwidth}
\begin{lstlisting}[style=color]
enum MmapBits {
    Shared,
    Private,
    Fixed = Private + 3,
    Rename,
    //...
};

\end{lstlisting}
\end{minipage}
\end{tabular}
\end{center}

\subsection{Concept-constrained declarations}

The following example from P1110 is a historical artifact as C++20 uniformizes \tcode{auto} for this purpose.

\begin{lstlisting}[style=color]
template<Integral __ N> class integral_constant { ... };
Numeric multiplyAdd(Numeric __ x, Numeric __ y, Numeric __ z) {
    Numeric __ multiplied = x * y;
    return multiplied + z;
}
\end{lstlisting}

Here is the C++20 equivalent

\begin{lstlisting}[style=color]
template<Integral auto N> class integral_constant { ... };
    Numeric multiplyAdd(Numeric auto x, Numeric auto y, Numeric auto z) {
    Numeric auto multiplied = x * y;
    return multiplied + z;
}
\end{lstlisting}


Other explored areas, including placeholders for function names, type declaration, using and concepts are not
useful, either because C++ does already provide the desired facilities such as anonymous type, or because the entity needs a name
(concepts, functions and using declarations without a name cannot be used)

\section{Acknowledgments}

Thanks to Miro Knejp and Tony Van Eerd for providing valuable feedback on R0 on very short notice!
Thanks to Davis for helping woth the wording!

\section{References}
\renewcommand{\section}[2]{}%
\bibliographystyle{plain}
\bibliography{wg21}

\begin{thebibliography}{9}

    \bibitem[N4861]{N4861}
    Richard Smith
    \emph{Working Draft, Standard for Programming Language C++}\newline
    \url{https://wg21.link/N4861}

\end{thebibliography}
\end{document}
