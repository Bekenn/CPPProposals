% !TeX program = luatex
% !TEX encoding = UTF-8


\documentclass{wg21}

\title{A plan for better template meta programming facilities in C++26}
\docnumber{D2632R0}
\audience{EWG}
\author{Corentin Jabot}{corentin.jabot@gmail.com}

\begin{document}
\maketitle

\section{Abstract}

Features such as concepts, \tcode{if constexpr} and template parameters packs have made Template Meta Programming more powerful and accessible than it was 15 years ago, or even 5 years ago.

However, some programms are still very difficult, if not impossible to express. Using template meta programming often require a high level of expertise.
Shortcomings of the core language have been addressed through library features with high compile time costs and poor user experience.

These problems can be solved by extending the core language, or lifting restrictions therein.
The overarching goals is to present a set of features to make generic programming and template meta programming easier,
faster to compile, with better compile time diagnostics.

Invidual features described in this document are of will be the object of separate proposals.


\section{Revisions}

\section{Scope and Goals}

\subsection{Not a paradigm shift}

The C++ committee likes big, bold complex and paradigm shifting proposals, like concepts and reflection.
This is no such proposal.
Most of the proposed features and highlighted proposals are logical extensions or generalization of existing features.
Most are patterns that many people have tried to express only to find they can't.
As such the combined goal of the presented features is not as much to make C++ more exciting as it is to make it less surprising and frustrating.

\subsection{Iterrative Improvements}

This paper touches on pack handling, template template parameters, type deduction, diagnostics, \tcode{constexpr} features, etc.
The features presented have various degrees of maturity and implementation experience.
We need to find a balance between having a good long term vision of the design space, and incremental improvements, such that well understood features,
are not held back by more controversial or less mature proposals.
It is not our intention to present a large proposal, but rather a number of small to medium sized proposals wth compounded benefits.
We should strive to standardize many of these individual features for C++26.

\subsection{Interraction with reflection}

There is some overalap between what is considered Template Meta Programming, \tcode{constexpr} facilities, and reflection.
The features presented in this paper (\paper{P1240R2}) are orthogonal to reflection and do not intend to compete with reflections
features.
As reflection can be used as a mechanism to manipulate types, there will inevitably be some problems that can be solved by both reflection
and template metaprogramming techniques, or a combination of the two. The best tool to use for a given task will be context dependant,
and this paper does not try to held any particular tool as the best.
Waiting for reflection should not stop us from improving the rest of the language.

\subsection{Circle}

Many of the features presented in this paper were designed and implemented in Circle by Sean Baxter.
There is a lot of commonality between this paper and the work presented by Sean Baxter at CppNow in May 2022.
This is very much intentional: We were really impressed by this body of work and the underlying philosophy that the language should be incrementally improved
to fit the need of library and user types, rather that relying on arcane, slow to compile, hard to diagnose template meta-programming features.


\section{Template parameters}

\subsection{Universal template parameters}

\subsection{Variable and concept template template parameters}

\section{Parameter Packs}

Packs are fundamentally sequences of expressions or types.
There exist, however no standard facilities to interate, index, search, or modify these sequences.
Most techniques to manipulate these sequences rely on recursion and rely on the instantiation of
types that are linerar - sometimes quadratic - with the size of the packs, leasding to high development costs and long compile times.
By improving language support for packs we can, all at once improve compile times, diagnostic messages, ease of development.

\subsection{Indexing}

Once of the most fundamental operation that can be done with a sequence is to index it:
Given an index \tcode{n} and a pack of type, we want to know the \tcode{n}th type, and given a pack of expression, we want the nth expression.

Here is a meta programming facility extracting the Nth type:

\begin{colorblock}
template <std::size_t I, typename T, typename ...Ts>
struct nth_element_impl {
    using type = typename nth_element_impl<I-1, Ts...>::type;
};

template <typename T, typename ...Ts>
struct nth_element_impl<0, T, Ts...> {
    using type = T;
};

template <std::size_t I, typename ...Ts>
using nth_element = typename nth_element_impl<I, Ts...>::type;
\end{colorblock}

This code was taken from a \href{https://ldionne.com/2015/11/29/efficient-parameter-pack-indexing/}{blog post} written by Louis Dione in 2015.
This is the most straigntforward approach presented.

It's a recursive solution, which is already non-trivial to write for a lot of C++ developer and it is the simple solution presented.
Most importantly, Louis's benchmark show that this very slow to compile.

The conclusion of the article, led to Clang implementing a built-in \tcode{__type_pack_element} builtin.

\begin{colorblock}
template <std::size_t N, typename... T>
using nth_element = __type_pack_element<N, T...>;
\end{colorblock}

\href{https://gcc.gnu.org/bugzilla/show_bug.cgi?id=100157}{GCC is considering} a similar facility.

An \href{https://gcc.gnu.org/bugzilla/show_bug.cgi?id=100157}{internet search reveals} a large number of people searching for answer to this exact problem.

Both \paper{P1858R2} and Cicle propose to solve this problem with a pack subscripting operator:

\begin{colorblock}
template<std::size_t N, typename... T>
auto get(T&&... args) -> std::remove_cvref_t<T...[N]> // Indexes a pack of type {
    return args...[N]; // Indexes a pack of expression
}
\end{colorblock}

This can then be used to implement \tcode{std::get} like facilities, simplify standard and library times interfaces, etc.
The following code, for example, implement \tcode{std::variant::variant()}.

\begin{colorblock}
template<class... Types>
class variant {
public:
    constexpr variant()
    noexcept(std::is_nothrow_default_constructible_v<Types...[0]>)
    requires(std::is_default_constructible_v<Types...[0]>);

};
\end{colorblock}

\subsubsection{Negative Indexing}

Both Circle and P1858 propose negative indexing

\begin{colorblock}
template <typename... T>
using Last = T...[-1];
static_assert(std::same_as<Last<int, double, long>, long>);
\end{colorblock}

\tcode{T...[-1]} is equivalent to \tcode{T...[sizeof...(T)-1]}.


This is fairly useful as getting the last element of a pack is a frequent operation.


Pack indexing is covered in more details in \paper{P1858R2}. \paper{P1858R2} Demonstrate how pack indexing can be used to simplify
\tcode{std::tuple} and tuple-adjacent facilities.

The proposed syntax is valid today however, \href{https://www.open-std.org/jtc1/sc22/wg21/docs/papers/2020/p1858r2.html#pack-indexing-ambiguity}{as explained by Barry},
The current meaning (declaring a pack of C arrays without name) is neither super useful, nor used, nor implemented by all compilers (\href{https://godbolt.org/z/6obhffPTP}{[Compiler Explorer]}).

Pack indexing can be standardized and implemented independently of other features proposed in this paper.

\subsection{Pack slicing}

Slicing, or subsetting a pack is also a fairly common operation.
In particular, many constraints or requirements apply to all elements but the first, or all elements but the last.

For examples, \tcode{std::ranges::cartesian_product_view}, and almost all supporting functions is defined as

\begin{colorblock}
template<input_range First, forward_range...Vs>
class cartesian_product_view : public view_interface<cartesian_product_view<First, Vs...>> {};
\end{colorblock}

Because of the different requirements on \tcode{First} and the necessity to be able to extract \tcode{First}
is many places.

Having the ability to take a subset of the pack can simplify that code:

\begin{colorblock}
template<input_range...Vs>
requires (forward_range<Vs...[1:]> && ...)
class cartesian_product_view : public view_interface<cartesian_product_view<Vs...>> {};
\end{colorblock}

Circle supports a third step parameter, which can be used to stride the pack, or reverse it,
and that seems less useful. It is not proposed by \paper{P1858R2}.

It is important to note that slicing yields a pack rather than its elements, so there might be dots on both sides

\begin{colorblock}
auto f(int a, int b);
int g(auto... pack) {
    return f(pack...[1:3]...); // pack...[1:3] slices the pack and ... expands it
}
\end{colorblock}

\subsection{Member Packs}

\subsection{Non-trailing packs}

\subsection{Multi-conditional operator}

\subsection{Packs in structured bindings}

\subsection{Packs as return values}

\subsection{Integer sequences and their indexing}

Cirle offers the following facilities to generate a list of integers as a pack of variables:

\begin{itemize}
\item \tcode{int...(N)} : Generate a pack of \tcode{N} consecutive integers starting at 0
\item \tcode{int...(begin:end:step)} : Generate a pack of integers in the range \range{begin}{end} with a step of \tcode{step}.
\item \tcode{int...}: yields the index of the current element in the pack expansion
\end{itemize}

These features are nice to have but far from necessary.

First, it is important to note that most use cases for integer sequences can be replaced by other feature presented in this paper,
such as pack indexing and expansion statements, and that, once we have these facilities, it is not clear that the need for a
pack of integers will be frequent enough to warrant a language feature.

Second, the syntax choosen by circle is entierly workable but also a bit suprising.
Ideally, we would want a pack size type to be \tcode{std::size_t}, and so tso token \tcode{int...} may lead to confusion.

But lastly, it is not clear that producing a pack in that way is the better solution, and we feel more exploration is needed
before commiting time to a feature that may or not be useful.

For example, expansion statements (\paper{P1306R1}) can be used in conjonction of \tcode{views::iota} to create a loop
of N elements, that can be used in the same contexts.

We could imagine that structured bindings - or even packs - could be initialized from a range.

Or other similar looking solution.

In the meantime, it is possible to extend \tcode{std::integer_sequence} to be compatible with the tuple protocole.

\begin{colorblock}
namespace std {
    template <class T, T... N>
    struct tuple_size<integer_sequence<T, N...>>
    : integral_constant<size_t, sizeof...(N)>
    { };

    template <size_t I,  class T, T... N>
    struct tuple_element<I, integer_sequence<T, N...>> {
        using type = T;
    };

    template <std::size_t I,  class T, T... N>
    constexpr auto get(std::integer_sequence<T, N...>) {
        return N...[I];
    }
}
\end{colorblock}

\section{Constexpr facilities}

\subsection{\tcode{constexpr} structured bindings}

\subsection{\tcode{constexpr} ternary operator}

\section{Priorities and status}

\begin{tabular}{|l|c|c|c|}
    \hline
    Feature & Paper & Implemented in Circle? & Implemented In Clang? \\
    \hline
    Pack Indexing & \paper{P1858R2} & \checkmark &  \\
    \hline
    Pack Slicing& \paper{P1858R2} & \checkmark &  \\
    \hline
    &  &  &  \\
    \hline
    &  &  &  \\
    \hline
    &  &  &  \\
    \hline
    &  &  &  \\
    \hline
    &  &  &  \\
    \hline
    &  &  &  \\
    \hline
    &  &  &  \\
    \hline
    &  &  &  \\
    \hline
\end{tabular}


\section{References}
\renewcommand{\section}[2]{}%
\bibliographystyle{plain}
\bibliography{wg21, extra}

\begin{thebibliography}{9}

\bibitem[N4885]{N4885}
Thomas Köppe
\emph{Working Draft, Standard for Programming Language C++}\newline
\url{https://wg21.link/N4885}


\end{thebibliography}

\end{document}