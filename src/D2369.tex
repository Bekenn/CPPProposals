% !TeX program = luatex
% !TEX encoding = UTF-8


\RequirePackage{luatex85}%
\documentclass{wg21}
\usepackage{threeparttable}
\usepackage{adjustbox}

\usepackage{array}
\newcolumntype{?}{!{\vrule width 1pt}}

\title{Relaxing preconditions on \tcode{coroutine\_handle::from\_address}}
\docnumber{D2369R0}
\audience{EWG}
\author{Lewis Baker}{lewissbaker@gmail.com}
\authortwo{Corentin Jabot}{corentin.jabot@gmail.com}

\begin{document}
\maketitle


\section{Abstract}
This paper avoids the storing of an additional pointer in \tcode{std::generator} by relaxing preconditions on \tcode{coroutine_handle::from_address}.


\section{Motivation}

\tcode{std::generator} as presented in \paper{P2168R4} supports yielding elements of a nested generator, which should be implemented via symmetric transfer for performance reason,
as well as allocator support.

We want to be able to yields elements of a different allocator type

\begin{colorblock}
    std::generator<int, int, std::allocator> g();
    std::generator<int, int, my_allocator> f() {
        co_yield elements_of(g());
    }
\end{colorblock}


Because the allocator is part of the promise type, f and g do not have the same promise type.
A good, if not the only implementation strategy for \tcode{std::generator} is for the promise types to share a base_class so that generators of different allocator types can yield each other.

\begin{colorblock}
template <typename Ref>
struct promise_base {
   // data members
};

template <typename Ref, typename Allocator>
struct promise : promise_base<Ref,Allocator> {
    // NO DATA members
};
\end{colorblock}

The implementation of symmetric transfer then needs to get access to a promise_base pointer so that the yielded value, exceptions pointer, etc
are set on the topmost/root generator.
Within the current wording rules, we have to store that promise_base pointer on the generator itself, as well as on any iterator instance (so that the iterator can access the value),
because getting a coroutine_handle to a base object is a precondition violation, even if

An implementation is currently required to store a pointer.

By relaxing requierements on \tcode{coroutine_handle::from_address} we can avoid storing that extra pointer, and instead get to the promise through
\tcode{std::coroutine_handle<promise_base>::from_address(handle.address())}.

\section{Alternatives considered}
For the purpose of \tcode{std::generator}, implementations could pretend they do not invoke UB, however, we think this change would benefit third-party code.

\section{Priority}

Given that \tcode{std::generator} is a P1 paper, and the change proposed in this paper could not be adopted without ABI break,
we recommend treating this paper as a P1 paper.

\section{Implementation}

(One of) our \tcode{std::generator} implementation successfully assumed that this undefined behavior did not exist.

\section{Impact}

This affects coroutine support, so both EWG and LEWG might want to look at it.

\section{Wording}

Modify 17.12.4.4 [coroutine.handle.export.import] as follow

\begin{quote}
\indexlibrarymember{from_address}{coroutine_handle}%
\begin{itemdecl}
static constexpr coroutine_handle<Promise> coroutine_handle<Promise>::from_address(void* addr);
\end{itemdecl}

\begin{itemdescr}
\expects
\tcode{addr} was obtained via a prior call to \tcode{address}
on an object of type \cv \changed{\tcode{coroutine\char`_handle<Promise>}}{\tcode{coroutine\char`_handle<T>}
        where \tcode{T} is any type such that \tcode{is\char`_pointer\char`_interconvertible\char`_base\char`_of\char`_v<T, Promise>} is \tcode{true}\\
        and \tcode{alignment\char`_of\char`_v<T> == alignment\char`_of\char`_v<Promise>} is \tcode{true}}.

\end{itemdescr}
\end{quote}

\section{Feature test macros}

Bump the value of \tcode{__cpp_lib_coroutine} to the date of adoption.

\section{References}
\renewcommand{\section}[2]{}%
\bibliographystyle{plain}
\bibliography{wg21}

\begin{thebibliography}{9}
    \bibitem[Implementation]{Implementation}
    Lewis Baker, Corentin Jabot
    \emph{\tcode{std::generator} implementation}\newline
    \url{https://github.com/lewissbaker/generator/}

    \bibitem{P2168R4}
    Corentin Jabot and Lewis Baker.
    \newblock {P2168R4}: generator: A synchronous coroutine generator compatible
    with ranges.
    \newblock \url{https://wg21.link/p2168r4}, 4 2021.

    \bibitem[N4885]{N4885}
    Thomas Köppe
    \emph{Working Draft, Standard for Programming Language C++}\newline
    \url{https://wg21.link/N4885}


\end{thebibliography}

\end{document}
