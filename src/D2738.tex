% !TeX program = luatex
% !TEX encoding = UTF-8


\documentclass{wg21}

\title{\tcode{constexpr} cast from \tcode{void*}: towards \tcode{constexpr} type-erasure}
\docnumber{D2738R0}
\audience{EWG}
\author{Corentin Jabot}{corentin.jabot@gmail.com}

\newcommand{\replaceucs}{\changed{UCS}{Unicode}}

\usepackage{color, colortbl}
\begin{document}
\maketitle

\section{Abstract}

We propose to allow a limited form of casting from \tcode{void*} to support type erasure in constexpr.

\section{Revisions}

\subsection{R0}

Initial revision

\section{Motivation}

Using the proposed feature, we were able to get \tcode{std::format} working at compile times.
Other standard facilities could be made constexpr such as \tcode{function_ref}, \tcode{function}, \tcode{any}.

Storing \tcode{void*} instead of a concrete type is commonly used as a compilation firewall technique to reduce template
instantiations, and more generally in any sort of context where type erasure is used.

Allowing that sort of capability in constexpr would allow more library facilities (such as \tcode{`format`}) at constexpr,
without having to support an entierly different \tcode{constexpr} implementation where type erasure would not be used.

\pagebreak

The following example, curtesy of Jason Turner, illustrates the use of this feature

\begin{colorblock}
#include <string_view>

struct Sheep {
    constexpr std::string_view speak() const noexcept { return "Baaaaaa"; }
};

struct Cow {
    constexpr std::string_view speak() const noexcept { return "Mooo"; }
};

class Animal_View {
    private:
    const void *animal;
    std::string_view (*speak_function)(const void *);

    public:
    template <typename Animal>
    constexpr Animal_View(const Animal &a)
    : animal{&a}, speak_function{[](const void *object) {
            return static_cast<const Animal *>(object)->speak();
    }} {}

    constexpr std::string_view speak() const noexcept {
        return speak_function(animal);
    }
};

// This is the key bit here. This is a single concrete function
// that can take anything that happens to have the "Animal_View"
// interface
std::string_view do_speak(Animal_View av) { return av.speak(); }

int main() {
    // A Cow is a cow. The only think that makes it special
    // is that it has a "std::string_view speak() const" member
    constexpr Cow cow;
    // cannot be constexpr because of static_cast
    [[maybe_unused]] auto result = do_speak(cow);
    return static_cast<int>(result.size());
}
\end{colorblock}

Facilities like \tcode{format} and \tcode{function_ref} use a similar pattern.

\section{Design}

We propose to allow casting from a pointer of \tcode{void*} to a pointer of type \tcode{T} in constexpr if the type of the object at that address is exactly of type \tcode{T}.
In particular, we are not proposing allowing conversion to a pointer that would be interconvertible, let alone unrelated.
Indeed, most constexpr evaluator implementations are based on value, rather than memory, and anything that would require reinterpreting the object as another type is generally not possible.

However, most implementations have a way to know the type of an object pointed to by a given pointer (for diagnostics, constexpr allocation, virtual dispatch, or other reasons), and so a cast from \tcode{void*} to a pointer of the type of the pointed to object is implementable.

\subsection{Do we want to support conversion to pointer to base?}

At first approach, it would make sense to support casts to base classes.
After all, casting to a base class is possible in constexpr contexts.
However, in a non-constexpr context, consider

\begin{colorblock}
struct A {
    virtual void f() {};
    int a;
};
struct B {
    int b;
};
struct C: B, A {};

int main() {
    C c;
    void* v = &c;
    assert(static_cast<B*>(v) == static_cast<B*>(static_cast<C*>(v))); // #1
}
\end{colorblock}

\tcode{\#1} does not hold. Both expressions return different addresses.
So casting to a derived class then its base is not isomorphic to casting to the base directly so, for consistency and simplicity we are not proposing
to cast to allow cast from \tcode{void*} to base classes either.

\section{Implementation Experience}

\subsection{Clang}

Implementing this in Clang was trivial. Indeed, clang does support constexpr conversions from \tcode{void*} but limits their use to inside of \tcode{std::allocator::allocate}, as part of the constexpr allocation machinery. Lifting that restriction doesn't present any particular challenge.

The other clang constexpr interpreter (based on bytecode) also tracks the origin of pointers and can implement this proposal.

\subsection{MSVC \& GCC}

Front-end engineers from both GCC and MSVC indicated this proposal offered no particular implementation challenge as their respective implementations already track the full type information of all pointers.

\subsection{EDG}

Someone from EDG indicated this proposal should be implementable, albeit with performance inefficiencies. Their implementation does not track enough
information for the type to be immediately available and they would have to reconstruct it by walking the AST of the enclosing whole object.

\ednote{TODO}

\subsection{Impact on future implementations}

This constrains an evaluator to know about the type of an object a pointer points to.
Adding this information to an implementation that does not have it could be challenging in the future, and, as more implementation start looking at
interpreting \tcode{constexpr} code, knowing they must preserve that information will inform their design. We should therefore do that change sooner than later to guarantee it remains implementable.

\section{Wording}

\rSec1[expr.const]{Constant expressions}%
\indextext{expression!constant}

\pnum
An expression $E$ is a \defnadj{core constant}{expression}
unless the evaluation of $E$, following the rules of the abstract
machine\iref{intro.execution}, would evaluate one of the following:

\textcolor{noteclr}{[...]}
\begin{itemize}
\item
a conversion from type \cvqual{cv\added{1}}~\tcode{\keyword{void}*} to a pointer-to-object type \added{\tcode{T} unless $E$ evaluates to the address of an object of type \cvqual{cv2} \tcode{T} where \cvqual{cv2} is the same cv-qualification as, or greater cv-qualification than, \cvqual{cv1}.}

%\cvqual{cv2} \tcode{T2} has the same cv-qualification as, or greater cv-qualification than, \cvqual{cv1} and \tcode{T} is either the same type as \tcode{T2} or is a non-virtual, non-ambiguous and accessible base of \tcode{T2}};

\item
a \keyword{reinterpret_cast}\iref{expr.reinterpret.cast};

\item
a modification of an object\iref{expr.ass,expr.post.incr,expr.pre.incr}
unless it is applied to a non-volatile lvalue of literal type
that refers to a non-volatile object
whose lifetime began within the evaluation of $E$;

\end{itemize}

\section{Feature test macro}

\ednote{In \tcode{[tab:cpp.predefined.ft]}, bump the value of\tcode{__cpp_­constexpr} to the date of adoption}.


\section{Acknowledgments}

Thans to Jason Merril, Daveed Vandevoorde and Cameron DaCamara for their input in the implementability of this feature.

\section{References}

\renewcommand{\section}[2]{}%
\bibliographystyle{plain}
\bibliography{wg21, extra}

\begin{thebibliography}{9}

\bibitem[N4892]{N4892}
Thomas Köppe
\emph{Working Draft, Standard for Programming Language C++}\newline
\url{https://wg21.link/N4892}


\end{thebibliography}

\end{document}



