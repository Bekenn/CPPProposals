% !TeX program = luatex
% !TEX encoding = UTF-8


\documentclass{wg21}

\title{\tcode{views::enumerate}}
\docnumber{P2164R2}
\audience{LEWG}
\author{Corentin Jabot}{corentin.jabot@gmail.com}

\begin{document}
\maketitle


\section{Abstract}

We propose a view \tcode{enumerate} whose value type is a struct with 2 members \tcode{index} and \tcode{value}
representing respectively the position and value of the elements in the adapted range. 

\section{Revisions}

\subsection{R2, following mailing list reviews}
\begin{itemize}
    \item Make \tcode{value_type} different from  \tcode{reference} to match other views
    \item Remove inconsistencies between the wording and the description
    \item Add relevant includes and namespaces to the examples
\end{itemize}


\subsection{R1}
\begin{itemize}
\item Fix the index type
\end{itemize}

\section{Tony tables}
\begin{center}
\begin{tabular}{l|l}
Before & After\\ \hline

\begin{minipage}[t]{0.5\textwidth}
\begin{colorblock}

std::vector days{"Mon", "Tue", 
  "Wed", "Thu", "Fri", "Sat", "Sun"};

int idx = 0;
for(const auto & d : days) {
    print("{} {} \n", idx, d);
    index++;
}

\end{colorblock}
\end{minipage}
&
\begin{minipage}[t]{0.5\textwidth}
\begin{colorblock}
#include <ranges>

std::vector days{"Mon", "Tue", 
  "Wed", "Thu", "Fri", "Sat", "Sun"};

for(const auto & [idx, d] 
      : std::views::enumerate(days)) {
    print("{} {} \n", idx, d);
}

\end{colorblock}
\end{minipage}
\\\\ \hline

\end{tabular}
\end{center}

\section{Motivation}

The impossibility to extract an index from a range-based for loop leads to the use of non-range based for loop, 
or the introduction of a variable in the outer scope. This is both more verbose and error-prone: in the example above, the type of \tcode{index} is incorrect. 

\tcode{enumerate} is a library solution solving this problem, enabling the use of range-based for loops in more cases.

It also composes nicely with other range facilities:
The following creates a map from a vector using the position of each element as key.
 
\begin{colorblock}
my_vector | views::enumerate | ranges::to<map>;
\end{colorblock}

This feature exists in some form in Python, Rust, Go (backed into the language), and in many C++ libraries: ranges-v3, folly, boost::ranges (\tcode{indexed}).


The existence of this feature or lack thereof is the subject of recurring stackoverflow questions.  


\section{Design}

\subsection{The result is a simple aggregate}

Following the trend of using meaningful names instead of returning pairs or tuples, this proposal uses a simple aggregate return type.
This design was previously discussed by LEWGI in Belfast in the context of \paper{P1894R0}.

\begin{colorblock}

struct reference {
    count_type index;
    range_reference_t<Base> value;
};
\end{colorblock}

\tcode{count_type} is defined as follow:
\begin{itemize}
    \item \tcode{ranges::range_size_t<Base>} if \tcode{Base} models \tcode{ranges::sized_range}
    \item Otherwise, \tcode{make_unsigned_t<ranges::range_difference_t<Base>>}
\end{itemize}

This is consistent with ranges-v3 and allows the view to support both sized and non sized ranges.

\subsection{constness}

The \tcode{index} is always const, \tcode{value} is conditionnally const like all other views

\subsection{Performance}

An optimizing compiler can generate the same machine code for \tcode{views::enumerate} as it would for an equivalent \tcode{for} loop.  \href{https://godbolt.org/z/2Kxo8d}{Compiler Explorer}


\subsection{Implementation}

This proposal has been implemented (\href{https://github.com/cor3ntin/rangesnext/blob/master/include/cor3ntin/rangesnext/enumerate.hpp}{Github})
There exist an implementation in ranges-v3 (where the enumerate view uses zip_with and a pair value type).

\section{Proposal}

We propose a view \tcode{enumerate} whose value type is a struct with 2 members \tcode{index} and \tcode{value}
representing respectively the position and value of the elements in the adapted range. 

\section{Wording}

\begin{addedblock}

\rSec2[range.enumerate]{Enumerate view}

\rSec3[range.enumerate.overview]{Overview}

\pnum
\tcode{enumerate_view} presents a \libconcept{view} with a value type that represents both the position and value of the adapted \libconcept{view}'s value-type.

\pnum
The name \tcode{views::enumerate} denotes a
range adaptor object\iref{range.adaptor.object}.
Given the subexpressions \tcode{E}
the expression \tcode{views::enumerate(E)} is expression-equivalent to \tcode{enumerate_view\{E\}}.

\pnum
\begin{example}
\begin{codeblock}
vector<int> vec{ 1, 2, 3 };
for (auto [index, value] : enumerate(vec) )
    cout << index << ":" << value ' '; // prints: 0:1 1:2 2:3
\end{codeblock}
\end{example}
        
\rSec3[range.enumerate.view]{Class template \tcode{enumerate_view}}


\begin{codeblock}
namespace std::ranges {
    template<@\libconcept{input_range}@ V>
    requires view<V>
    class enumerate_view : public view_interface<enumerate_view<V>> {
    
      private:
        V @\exposid{base_}@ = {};
        
        template <bool Const>
        class @\exposid{iterator}@; // \expos
        template <bool Const>
        struct @\exposid{sentinel}@; // \expos
        
       public:
        
        constexpr enumerate_view() = default;
        constexpr enumerate_view(V base);
        
        constexpr auto begin() requires (!simple_view<V>)
        { return iterator<false>(ranges::begin(base_), 0); }
        
        constexpr auto begin() const requires simple_view<V>
        { return iterator<true>(ranges::begin(base_), 0); }
        
        constexpr auto end()
        { return sentinel<false>{end(base_)}; }
        
        constexpr auto end()
        requires common_range<V> && sized_range<V>
        { return iterator<false>{ranges::end(base_), 
                 static_cast<range_difference_t<V>>(size()) }; }
        
        constexpr auto end() const
        requires range<const V>
        { return sentinel<true>{ranges::end(base_)}; }
        
        constexpr auto end() const
        requires common_range<const V> && sized_range<V>
        { return iterator<true>{ranges::end(base_), 
                 static_cast<range_difference_t<V>>(size())}; }
        
        constexpr auto size()
        requires sized_range<V>
        { return ranges::size(base_); }
        
        constexpr auto size() const
        requires sized_range<const V>
        { return ranges::size(base_); }
        
        
        constexpr V base() const & requires copy_constructible<V> { return @\exposid{base_}@; }
        constexpr V base() && { return move(@\exposid{base_}@); }
    };
    template<class R>
    enumerate_view(R&&) -> enumerate_view<views::all_t<R>>;

\end{codeblock}

\begin{itemdecl}
    constexpr enumerate_view(V base);
\end{itemdecl}

\begin{itemdescr}
    \pnum
    \effects
    Initializes \exposid{base_} with \tcode{move(base)}.
\end{itemdescr}

\rSec3[range.enumerate.iterator]{Class \tcode{enumerate_view::\exposid{iterator}}}

\begin{codeblock}
namespace std::ranges {
    template<@\libconcept{input_range}@ V>
    requires view<V>
    template<bool Const>
    class enumerate_view<V>::@\exposid{iterator}@ {
        
        using Base = conditional_t<Const, const V, V>;
        using count_type = @\seebelow@;
    
        iterator_t<Base> current_ = iterator_t<Base>();
        count_type pos_ = 0;
        
        
      public:
        using iterator_category = typename iterator_traits<iterator_t<Base>>::iterator_category;
        
        struct reference {
            const count_type index;
            range_reference_t<Base> value;
        };
        
        struct value_type {
            const count_type index;
            range_value_t<Base> value;
        };
 
        using difference_type = range_difference_t<Base>;
        
        
        iterator() = default;
        constexpr explicit iterator(iterator_t<Base> current, range_difference_t<Base> pos);
        constexpr iterator(iterator<!Const> i)
        requires Const && convertible_to<iterator_t<V>, iterator_t<Base>>;
        
        
        constexpr iterator_t<Base> base() const&
        requires copyable<iterator_t<Base>>;
        constexpr iterator_t<Base> base() &&;
        
        constexpr decltype(auto) operator*() const {
             return reference{pos_, *current_};
        }
    
        constexpr iterator& operator++();
        constexpr void operator++(int) requires (!forward_range<Base>);
        constexpr iterator operator++(int) requires forward_range<Base>;
        
        constexpr iterator& operator--() requires bidirectional_range<Base>;
        constexpr iterator operator--(int) requires bidirectional_range<Base>;
        
        constexpr iterator& operator+=(difference_type x)
        requires random_access_range<Base>;
        constexpr iterator& operator-=(difference_type x)
        requires random_access_range<Base>;
        
        constexpr decltype(auto) operator[](difference_type n) const
        requires random_access_range<Base>
        { return reference{static_cast<difference_type>(pos_ + n), *(current_ + n) }; }
        
        
        friend constexpr bool operator==(const iterator& x, const iterator& y)
        requires equality_comparable<iterator_t<Base>>;
        
        friend constexpr bool operator<(const iterator& x, const iterator& y)
        requires random_access_range<Base>;
        friend constexpr bool operator>(const iterator& x, const iterator& y)
        requires random_access_range<Base>;
        friend constexpr bool operator<=(const iterator& x, const iterator& y)
        requires random_access_range<Base>;
        friend constexpr bool operator>=(const iterator& x, const iterator& y)
        requires random_access_range<Base>;
        friend constexpr auto operator<=>(const iterator& x, const iterator& y)
        requires random_access_range<Base> && three_way_comparable<iterator_t<Base>>;
        
        friend constexpr iterator operator+(const iterator& x, difference_type y)
        requires random_access_range<Base>;
        friend constexpr iterator operator+(difference_type x, const iterator& y)
        requires random_access_range<Base>;
        friend constexpr iterator operator-(const iterator& x, difference_type y)
        requires random_access_range<Base>;
        friend constexpr difference_type operator-(const iterator& x, const iterator& y)
        requires random_access_range<Base>;
    };
}
\end{codeblock}


\tcode{iterator::count_type} is defined as follow:
\begin{itemize}
    \item \tcode{ranges::range_size_t<Base>} if \tcode{Base} models \tcode{ranges::sized_range}
    \item Otherwise, \tcode{make_unsigned_t<ranges::range_difference_t<Base>>}
\end{itemize}



\begin{itemdecl}
    constexpr explicit @\exposid{iterator}@(iterator_t<@\exposid{Base}@> current, range_difference_t<Base> pos = 0);
\end{itemdecl}

\begin{itemdescr}
    \pnum
    \effects
    Initializes \tcode{current_} with \tcode{move(current)} and \exposid{pos} with \tcode{static_cast<count_type>(pos)}.
\end{itemdescr}

\begin{itemdecl}
    constexpr @\exposid{iterator}@(@\exposid{iterator}@<!Const> i)
    requires Const && @\libconcept{convertible_to}@<iterator_t<V>, iterator_t<@\exposid{Base}@>>;
\end{itemdecl}

\begin{itemdescr}
    \pnum
    \effects
    Initializes \exposid{current_} with \tcode{move(i.\exposid{current_})} and \exposid{pos} with \tcode{i.pos_}.
\end{itemdescr}

\begin{itemdecl}
    constexpr iterator_t<@\exposid{Base}@> base() const&
    requires copyable<iterator_t<@\exposid{Base}@>>;
\end{itemdecl}

\begin{itemdescr}
    \pnum
    \effects
    Equivalent to: \tcode{return \exposid{current_};}
\end{itemdescr}

\begin{itemdecl}
    constexpr iterator_t<@\exposid{Base}@> base() &&;
\end{itemdecl}

\begin{itemdescr}
    \pnum
    \effects
    Equivalent to: \tcode{return move(\exposid{current_});}
\end{itemdescr}

\begin{itemdecl}
    constexpr @\exposid{iterator}@& operator++();
\end{itemdecl}

\begin{itemdescr}
    \pnum
    \effects
    Equivalent to:
    \begin{codeblock}
        ++@\exposid{pos}@;
        ++@\exposid{current_}@;
        return *this;
    \end{codeblock}
\end{itemdescr}

\begin{itemdecl}
    constexpr void operator++(int) requires (!@\libconcept{forward_range}@<@\exposid{Base}@>);
\end{itemdecl}

\begin{itemdescr}
    \pnum
    \effects
    Equivalent to:
    \begin{codeblock}
        ++@\exposid{pos}@;
        ++@\exposid{current_}@;
    \end{codeblock}
\end{itemdescr}

\begin{itemdecl}
    constexpr @\exposid{iterator}@ operator++(int) requires @\libconcept{forward_range}@<@\exposid{Base}@>;
\end{itemdecl}

\begin{itemdescr}
    \pnum
    \effects
    Equivalent to:
    \begin{codeblock}
        auto temp = *this;
        ++@\exposid{pos}@;
        ++@\exposid{current_}@;
        return temp;
    \end{codeblock}
\end{itemdescr}

\begin{itemdecl}
    constexpr @\exposid{iterator}@& operator--() requires @\libconcept{bidirectional_range}@<@\exposid{Base}@>;
\end{itemdecl}

\begin{itemdescr}
    \pnum
    \effects
    Equivalent to:
    \begin{codeblock}
        --@\exposid{pos_}@;
        --@\exposid{current_}@;
        return *this;
    \end{codeblock}
\end{itemdescr}

\begin{itemdecl}
    constexpr @\exposid{iterator}@ operator--(int) requires @\libconcept{bidirectional_range}@<@\exposid{Base}@>;
\end{itemdecl}

\begin{itemdescr}
    \pnum
    \effects
    Equivalent to:
    \begin{codeblock}
        auto temp = *this;
        --@\exposid{current_}@;
        --@\exposid{pos_}@;
        return temp;
    \end{codeblock}
\end{itemdescr}

\begin{itemdecl}
    constexpr @\exposid{iterator}@& operator+=(difference_type n);
    requires @\libconcept{random_access_range}@<@\exposid{Base}@>;
\end{itemdecl}

\begin{itemdescr}
    \pnum
    \effects
    Equivalent to:
    \begin{codeblock}
        @\exposid{current_}@ += n;
        @\exposid{pos_}@ += n;
        return *this;
    \end{codeblock}
\end{itemdescr}

\begin{itemdecl}
    constexpr @\exposid{iterator}@& operator-=(difference_type n)
    requires @\libconcept{random_access_range}@<@\exposid{Base}@>;
\end{itemdecl}

\begin{itemdescr}
    \pnum
    \effects
    Equivalent to:
    \begin{codeblock}
        @\exposid{current_}@ -= n;
        @\exposid{pos_}@ -= n;
        return *this;
    \end{codeblock}
\end{itemdescr}

\begin{itemdecl}
    friend constexpr bool operator==(const @\exposid{iterator}@& x, const @\exposid{iterator}@& y)
    requires equality_comparable<@\exposid{Base}@>;
\end{itemdecl}

\begin{itemdescr}
    \pnum
    \effects
    Equivalent to: \tcode{return x.\exposid{current_} == y.\exposid{current_};}
\end{itemdescr}

\begin{itemdecl}
    friend constexpr bool operator<(const @\exposid{iterator}@& x, const @\exposid{iterator}@& y)
    requires @\libconcept{random_access_range}@<@\exposid{Base}@>;
\end{itemdecl}

\begin{itemdescr}
    \pnum
    \effects
    Equivalent to: \tcode{return x.\exposid{current_} < y.\exposid{current_};}
\end{itemdescr}

\begin{itemdecl}
    friend constexpr bool operator>(const @\exposid{iterator}@& x, const @\exposid{iterator}@& y)
    requires @\libconcept{random_access_range}@<@\exposid{Base}@>;
\end{itemdecl}

\begin{itemdescr}
    \pnum
    \effects
    Equivalent to: \tcode{return y < x;}
\end{itemdescr}

\begin{itemdecl}
    friend constexpr bool operator<=(const @\exposid{iterator}@& x, const @\exposid{iterator}@& y)
    requires @\libconcept{random_access_range}@<@\exposid{Base}@>;
\end{itemdecl}

\begin{itemdescr}
    \pnum
    \effects
    Equivalent to: \tcode{return !(y < x);}
\end{itemdescr}

\begin{itemdecl}
    friend constexpr bool operator>=(const @\exposid{iterator}@& x, const @\exposid{iterator}@& y)
    requires @\libconcept{random_access_range}@<@\exposid{Base}@>;
\end{itemdecl}

\begin{itemdescr}
    \pnum
    \effects
    Equivalent to: \tcode{return !(x < y);}
\end{itemdescr}

\begin{itemdecl}
    friend constexpr auto operator<=>(const @\exposid{iterator}@& x, const @\exposid{iterator}@& y)
    requires @\libconcept{random_access_range}@<@\exposid{Base}@> && @\libconcept{three_way_comparable}@<iterator_t<@\exposid{Base}@>>;
\end{itemdecl}

\begin{itemdescr}
    \pnum
    \effects
    Equivalent to: \tcode{return x.\exposid{current_} <=> y.\exposid{current_};}
\end{itemdescr}

\begin{itemdecl}
    friend constexpr @\exposid{iterator}@ operator+(const @\exposid{iterator}@& x, difference_type y)
    requires @\libconcept{random_access_range}@<@\exposid{Base}@>;
\end{itemdecl}

\begin{itemdescr}
    \pnum
    \effects
    Equivalent to: \tcode{return iterator\{x\} += y;}
\end{itemdescr}

\begin{itemdecl}
    friend constexpr @\exposid{iterator}@ operator+(difference_type x, const @\exposid{iterator}@& y)
    requires @\libconcept{random_access_range}@<@\exposid{Base}@>;
\end{itemdecl}

\begin{itemdescr}
    \pnum
    \effects
    Equivalent to: \tcode{return y + x;}
\end{itemdescr}

\begin{itemdecl}
    constexpr @\exposid{iterator}@ operator-(const @\exposid{iterator}@& x, difference_type y)
    requires @\libconcept{random_access_range}@<@\exposid{Base}@>;
\end{itemdecl}

\begin{itemdescr}
    \pnum
    \effects
    Equivalent to: \tcode{return iterator\{x\} -= y;}
\end{itemdescr}

\begin{itemdecl}
    constexpr difference_type operator-(const @\exposid{iterator}@& x, const @\exposid{iterator}@& y)
    requires @\libconcept{random_access_range}@<@\exposid{Base}@>;
\end{itemdecl}

\begin{itemdescr}
    \pnum
    \effects
    Equivalent to: \tcode{return x.\exposid{current_} - y.\exposid{current_};}
\end{itemdescr}


\rSec3[range.enumerate.sentinel]{Class template \tcode{enumerate_view::\exposid{sentinel}}}

\begin{codeblock}
namespace std::ranges {
    template<input_range V, size_t N>
    requires view<V>
    template<bool Const>
    class enumerate_view<V, N>::@\exposid{sentinel}@ {                 // \expos
        private:
        using @\exposid{Base}@ = conditional_t<Const, const V, V>;      // \expos
        sentinel_t<@\exposid{Base}@> @\exposid{end_}@ = sentinel_t<@\exposid{Base}@>();         // \expos
        public:
        @\exposid{sentinel}@() = default;
        constexpr explicit @\exposid{sentinel}@(sentinel_t<@\exposid{Base}@> end);
        constexpr @\exposid{sentinel}@(@\exposid{sentinel}@<!Const> other)
        requires Const && convertible_to<sentinel_t<V>, sentinel_t<@\exposid{Base}@>>;
        
        constexpr sentinel_t<@\exposid{Base}@> base() const;
        
        friend constexpr bool operator==(const @\exposid{iterator}@<Const>& x, const @\exposid{sentinel}@& y);
        
        friend constexpr range_difference_t<@\exposid{Base}@>
        operator-(const @\exposid{iterator}@<Const>& x, const @\exposid{sentinel}@& y)
        requires sized_sentinel_for<sentinel_t<@\exposid{Base}@>, iterator_t<@\exposid{Base}@>>;
        
        friend constexpr range_difference_t<@\exposid{Base}@>
        operator-(const @\exposid{sentinel}@& x, const @\exposid{iterator}@<Const>& y)
        requires sized_sentinel_for<sentinel_t<@\exposid{Base}@>, iterator_t<@\exposid{Base}@>>;
    };
}
\end{codeblock}

\begin{itemdecl}
    constexpr explicit @\exposid{sentinel}@(sentinel_t<@\exposid{Base}@> end);
\end{itemdecl}

\begin{itemdescr}
    \pnum
    \effects
    Initializes \exposid{end_} with \tcode{end}.
\end{itemdescr}

\begin{itemdecl}
    constexpr @\exposid{sentinel}@(@\exposid{sentinel}@<!Const> other)
    requires Const && @\libconcept{convertible_to}@<sentinel_t<V>, sentinel_t<@\exposid{Base}@>>;
\end{itemdecl}

\begin{itemdescr}
    \pnum
    \effects
    Initializes \exposid{end_} with \tcode{move(other.\exposid{end_})}.
\end{itemdescr}

\begin{itemdecl}
    constexpr sentinel_t<@\exposid{Base}@> base() const;
\end{itemdecl}

\begin{itemdescr}
    \pnum
    \effects
    Equivalent to: \tcode{return \exposid{end_};}
\end{itemdescr}

\begin{itemdecl}
    friend constexpr bool operator==(const @\exposid{iterator}@<Const>& x, const @\exposid{sentinel}@& y);
\end{itemdecl}

\begin{itemdescr}
    \pnum
    \effects
    Equivalent to: \tcode{return x.\exposid{current_} == y.\exposid{end_};}
\end{itemdescr}

\begin{itemdecl}
    friend constexpr range_difference_t<@\exposid{Base}@>
    operator-(const @\exposid{iterator}@<Const>& x, const @\exposid{sentinel}@& y)
    requires sized_sentinel_for<sentinel_t<@\exposid{Base}@>, iterator_t<@\exposid{Base}@>>;
\end{itemdecl}

\begin{itemdescr}
    \pnum
    \effects
    Equivalent to: \tcode{return x.\exposid{current_} - y.\exposid{end_};}
\end{itemdescr}

\begin{itemdecl}
    friend constexpr range_difference_t<@\exposid{Base}@>
    operator-(const @\exposid{sentinel}@& x, const @\exposid{iterator}@<Const>& y)
    requires sized_sentinel_for<sentinel_t<@\exposid{Base}@>, iterator_t<@\exposid{Base}@>>;
\end{itemdecl}

\begin{itemdescr}
    \pnum
    \effects
    Equivalent to: \tcode{return x.\exposid{end_} - y.\exposid{current_};}
\end{itemdescr}

\end{addedblock}

\section{References}
\renewcommand{\section}[2]{}%
\bibliographystyle{plain}
\bibliography{../wg21}

\begin{thebibliography}{9}

    \bibitem[N4861]{N4861}
    Richard Smith
    \emph{Working Draft, Standard for Programming Language C++}\newline
    \url{https://wg21.link/N4861}

\end{thebibliography}
\end{document}
