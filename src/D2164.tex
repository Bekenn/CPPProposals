% !TeX program = luatex
% !TEX encoding = UTF-8


\documentclass{wg21}

\title{\tcode{views::enumerate}}
\docnumber{P2164R8}
\audience{LEWG, LWG}
\author{Corentin Jabot}{corentin.jabot@gmail.com}

\definecolor{addtuplecolor}{rgb}{0.72, 0.53, 0.04}
\newcommand{\addedtuple}[1]{\textcolor{addtuplecolor}{\uline{#1}}}

\begin{document}
\maketitle
\paperquote{A struct with 2 members, how hard can it be?!}

\section{Abstract}

We propose a view \tcode{enumerate} whose value type is a struct with 2 members \tcode{index} and \tcode{value}
representing respectively the position and value of the elements in the adapted range.

\section{Revisions}

\subsection{R8}
\begin{itemize}
	\item remove the wording for enumerate result following SG9 consensus to use std::tuple.
	\item Qualify std::move
	\item Italicize base_ and simple-view
	\item Unconditionally use range_difference_type as the index_type
	\item Model the sentinel wording on elements_view in order to support mixed comparison
	\item  Add missing explicit on the one parameter constructor
	\item Improve the specification of the begin/end methods
	\item Add missing \tcode{enable_borrowed_range} specialization
	\item Greatly simplify the comparison operators
	\item Propose an \tcode{index} methods to avoid deferencing the underlying iterator when not needed. The \tcode{*} operator
	of the underlying range might be expensive, if the user only cares about the index, the \tcode{index} method could be used
	to query the index cheaply in all cases. It is trivial to specify, implement, and could be provided later at no cost. But I'm happy
	to do that change if it increases consensus.
	\item Other minor wording fixes.
\end{itemize}

\subsection{R7}
\begin{itemize}
    \item This version asks LEWG to choose between \tcode{tuple} or \tcode{enumerate_result} as the reference and value types of \tcode{enumerate_view}.
    The approach presented in previous revisions of having a value type and a reference type of different types proved not workable.
    We need to pick one of the two options. Wording is provided for both.

    \item Add missing \tcode{iter_move} and \tcode{iter_swap} functions.
    \item Add the markup for freestanding
    \item Add feature test macro
    \item Formatting fixes.
\end{itemize}

\subsection{R6}
Wording changes:
\begin{itemize}
    \item Add \tcode{enumerate_result} to the list of \tcode{tuple-like} types
    \item Because \tcode{enumerate_view::iterator::operator*} returns values,  \tcode{enumerate_view::iterator}
    cannot be a \tcode{Cpp17ForwardIterator}. Change \tcode{iterator_category} and add \tcode{iterator_concept} accordingly.
\end{itemize}

\subsection{R5}

Instead of adding complexity to \tcode{enumerate_result}, we assume changes made by \paper{P2165R2}.
\paper{P2165R2} makes \tcode{pair} constructible from \placeholder{pair-like} objects, and associative containers deduction guides work
with ranges of \placeholder{pair-like} objects.
With these changes, \tcode{enumerate_result} can remain a simple aggregate. We just need to implement the tuple protocol for it (\tcode{get}, \tcode{tuple_element}, \tcode{tuple_size}).

\paper{P2165R2} ensures a common reference exists between \tcode{enumerate_result} and \tcode{std::tuple} as long as one exists between each element.

\tcode{count_type} is renamed to \tcode{index_type}. I am not sure why I ever chosed \tcode{count_type} as the initial name.

\subsection{R4}

This revision is intended to illustrate the effort necessary to support named fields for \tcode{index} and \tcode{value}.
In previous revisions, the value and reference types were identical, a regrettable blunder that made the wording and implementation efforts smaller than they are.
\tcode{reference} and \tcode{value_type} types however needs to be different, if only to make the \tcode{ranges::to} presented in this very paper.

If that direction is acceptable, better wording will be provided to account for these new \tcode{reference} and \tcode{value_type} types.

This revision also gets rid of the \tcode{const} \tcode{index} value as LEWG strongly agreed that it was a terrible idea to begin with, one that would make composition with other views cumbersome.

\subsection{R3}

\begin{itemize}
    \item Typos and minor wording improvements
\end{itemize}

\subsection{R2, following mailing list reviews}
\begin{itemize}
    \item Make \tcode{value_type} different from  \tcode{reference} to match other views
    \item Remove inconsistencies between the wording and the description
    \item Add relevant includes and namespaces to the examples
\end{itemize}


\subsection{R1}
\begin{itemize}
\item Fix the index type
\end{itemize}

\section{Tony tables}
\begin{center}
\begin{tabular}{l|l}
Before & After\\ \hline

\begin{minipage}[t]{0.5\textwidth}
\begin{colorblock}

std::vector days{"Mon", "Tue",
  "Wed", "Thu", "Fri", "Sat", "Sun"};

int idx = 0;
for(const auto & d : days) {
    print("{} {} \n", idx, d);
    idx++;
}

\end{colorblock}
\end{minipage}
&
\begin{minipage}[t]{0.5\textwidth}
\begin{colorblock}
#include <ranges>

std::vector days{"Mon", "Tue",
  "Wed", "Thu", "Fri", "Sat", "Sun"};

for(const auto & [index, value]
	   : std::views::enumerate(days)) {
    print("{} {} \n", index, value);
}

\end{colorblock}
\end{minipage}
\\\\ \hline

\end{tabular}
\end{center}

\section{Motivation}

The impossibility to extract an index from a range-based for loop leads to the use of non-range-based \tcode{for} loops, or the introduction of a variable in the outer scope. This is both more verbose and error-prone: in the example above, the type of \tcode{idx} is incorrect.

\tcode{enumerate} is a library solution solving this problem, enabling the use of range-based for loops in more cases.

It also composes nicely with other range facilities:
The following creates a map from a vector using the position of each element as key.

\begin{colorblock}
my_vector | views::enumerate | ranges::to<map>;
\end{colorblock}

This feature exists in some form in Python, Rust, Go (backed into the language), and in many C++ libraries: ranges-v3, folly, boost::ranges (\tcode{indexed}).


The existence of this feature or lack thereof is the subject of recurring StackOverflow questions.


\section{Design}

\subsection{\tcode{std::tuple} vs aggregate with name members}

Following the trend of using meaningful names instead of returning pairs or tuples, one option is to use a struct with named public members.
\begin{colorblock}
struct enumerate_result {
    count index;
    T value;
};
\end{colorblock}

This design was previously discussed by LEWGI in Belfast in the context of \paper{P1894R0}, and many people have expressed a desire for such struct with names.

\begin{colorblock}
std::vector<double> v;
enumerate(view) | to<std::vector>(); // std::vector<std::tuple<std::size_t, double>>.
enumerate(view) | to<std::map>();    // std::map<std::size_t, double>.
\end{colorblock}

\subsubsection{Why not just always return a \tcode{tuple} and rely on structure binding?}

If a range reference type is convertible to the index type,
it is error-prone whether one should write

\begin{colorblock}
for(auto && [value, index] : view | std::views::enumerate)
for(auto && [index, value] : view | std::views::enumerate)
\end{colorblock}

Having named members avoids this issue.
The feedback I keep getting is "we should use a struct if we can". Which is consistent with previous LEWG guidelines to avoid using pair
when a more meaningful type is possible.


\subsubsection{Why use a tuple?}

The drawback of using a distinct type is that
\begin{colorblock}
auto vec = enumerate(view) | ranges::to<std::vector>();
\end{colorblock}

would produce a \tcode{vector<enumerate_result<std::size_t, range_value_t<decltype(view)>>}

where ideally, I think it should produce a tuple.

\subsection{Why not \tcode{enumerate_result} as reference type and \tcode{tuple} as value type?}

This was the approached proposed by the previous revision of the paper and my preferred solution. Best of both world.
It only has a smal drawback: it doesn't work.

Many algorithms end up requiring \tcode{invocable<F\&, iter_value_t<I>\&> \&\& invocable<F\&, iter_reference_t<I>>} (where F is a function),
which would require \tcode{std::tuple<std::size_t, Foo>\&} to be convertible to \tcode{enumerate_result<std::size_t, Foo>}.

In practice, this means that many algorithms are not utilisable if reference and values are not the same type

\begin{colorblock}
std::ranges:::find(enumerate(/*...*/), [](auto const& p) { // constraints not satisfied.
    return /*...*/;
})
\end{colorblock}

This is simply not acceptable.

Tomasz also observed that it would interract pourly with \tcode{as_const}.

\begin{colorblock}
for (auto const& p : enumerate(/*...*/)) {
    something(p.value); // OK
}

for (auto const& p : enumerate(/*...*/) | views|as_const) {
    something(p.value); // KO decltype(p) is tuple<std::size_t, const Foo&>
}
\end{colorblock}

Which is... not great.
The unfortunate \tcode{invocable<F\&, iter_value_t<I>\&>} constraints exists as some algorithms (not find) may constructs value types out of the elements of the range.

\subsection{Where do we go from here?}

We need to choose between using \tcode{tuple} or \tcode{enumerate_result}, and that type would be used for both the value type and the reference type.

\textbf{POLL: Does LEWG prefer using \tcode{enumerate_result} (Option 1) rather than a \tcode{tuple} (Option 2) as the value and reference type of \tcode{enumerate_view::iterator}?}

The wording provides both options.

\subsection{Why not \tcode{zip(iota, view)}?}
Zipping the view with iota \href{https://github.com/ericniebler/range-v3/issues/1141}{does not actually work}
(see also \href{http://www.open-std.org/jtc1/sc22/wg21/docs/papers/2020/p2214r0.html#enumerates-first-range}{\paper{P2214R0}}),
and a custom \tcode{index_view} would need to be used as the first range composed with \tcode{zip}, so a custom \tcode{enumerate} view with
appropriately named members is not adding a lot of work.

\tcode{enumerate} as presented here is slightly less work for the compiler, but both solutions generate similar assembly.


\subsection{Performance}

An optimizing compiler can generate the same machine code for \tcode{views::enumerate} as it would for an equivalent \tcode{for} loop.  \href{https://godbolt.org/z/2Kxo8d}{Compiler Explorer} \ednote{This implementation is a prototype not fully reflective of the proposed design}.


\subsection{Implementation}

This proposal has been implemented (\href{https://github.com/cor3ntin/rangesnext/blob/master/include/cor3ntin/rangesnext/enumerate.hpp}{Github})
There exist an implementation in ranges-v3 (where the enumerate view uses zip_with and a pair value type).

\section{Proposal}

We propose a view \tcode{enumerate} whose value type is a struct with 2 members \tcode{index} and \tcode{value}
representing respectively the position and value of the elements in the adapted range.

\section{Wording}

\rSec1[ranges.syn]{Header \tcode{<ranges>} synopsis}

\begin{codeblock}

template<class R>
using keys_view = elements_view<R, 0>;                                          // freestanding
template<class R>
using values_view = elements_view<R, 1>;                                        // freestanding

namespace views {
    template<size_t N>
    inline constexpr @\unspecnc@ elements = @\unspecnc@;                          // freestanding
    inline constexpr auto keys = elements<0>;                                       // freestanding
    inline constexpr auto values = elements<1>;                                     // freestanding
}
\end{codeblock}
\begin{addedblock}

\begin{codeblock}
template <input_range View>
requires view<View>
class enumerate_view; // freestanding

template<class View>
inline constexpr bool enable_borrowed_range<enumerate_view<View>> =                   // freestanding
enable_borrowed_range<View>;

namespace views { inline constexpr @\unspecnc@ enumerate = @\unspecnc@; }      // freestanding
\end{codeblock}
\end{addedblock}
\begin{codeblock}


// \ref{range.zip}, zip view
template<input_range... Views>
requires (view<Views> && ...) && (sizeof...(Views) > 0)
class zip_view;                                                                   // freestanding

template<class... Views>
inline constexpr bool enable_borrowed_range<zip_view<Views...>> =               // freestanding
(enable_borrowed_range<Views> && ...);

namespace views { inline constexpr @\unspecnc@ zip = @\unspecnc@; }               // freestanding

@\textcolor{noteclr}{[...]}@

}

\end{codeblock}



\begin{addedblock}


\ednote{Add the following new section before [range.zip]}

\rSec2[range.enumerate]{Enumerate view}

\rSec3[range.enumerate.overview]{Overview}

\pnum
\tcode{enumerate_view} presents a \libconcept{view} whose elements represents both the position and value of the adapted \libconcept{view}'s elements.

\pnum
The name \tcode{views::enumerate} denotes a
range adaptor object\iref{range.adaptor.object}.
Given the subexpression \tcode{E}
the expression \tcode{views::enumerate(E)} is expression-equivalent to \tcode{enumerate_view<views​::​all_­t<decltype((E))>>(E)}.

\pnum
\begin{example}
\begin{codeblock}
vector<int> vec{ 1, 2, 3 };
for (auto [index, value] : enumerate(vec) )
    cout << index << ":" << value << ' '; // prints: 0:1 1:2 2:3
\end{codeblock}
\end{example}


\rSec3[range.enumerate.view]{Class template \tcode{enumerate_view}}


\begin{codeblock}
    template<@\libconcept{input_range}@ V>
    requires view<V>
    class enumerate_view : public view_interface<enumerate_view<V>> {

      private:
        V @\exposid{base_}@ = V(); // \expos

        template <bool Const>
        class @\exposid{iterator}@; // \expos
        template <bool Const>
        class @\exposid{sentinel}@; // \expos

       public:

        constexpr enumerate_view() requires default_­initializable<V> = default;
        constexpr explicit enumerate_view(V base);

        constexpr auto begin() requires (!@\exposid{simple-view}@<V>)
        { return @\exposid{iterator}@<false>(ranges::begin(@\exposid{base_}@), 0); }

        constexpr auto begin() const requires range<const V>
        { return @\exposid{iterator}@<true>(ranges::begin(@\exposid{base_}@), 0); }

        constexpr auto end() requires (!@\exposid{simple-view}@<V>) {
        	if constexpr (common_­range<V>  && sized_range<V> ) {
        		return @\exposid{iterator}@<false>{ranges::end(@\exposid{base_}@), ranges::distance(@\exposid{base_}@)};
        	} else {
        		return @\exposid{sentinel}@<false>{ranges::end(@\exposid{base_}@)};
        	}
        }

    	constexpr auto end() const {
    	  if constexpr (common_­range<const V>  && sized_range<const V> ) {
    		return @\exposid{iterator}@<true>{ranges::end(@\exposid{base_}@), ranges::distance(@\exposid{base_}@)};
    	  } else {
    	    return @\exposid{sentinel}@<true>{ranges::end(@\exposid{base_}@)};
    	  }
    	}

        constexpr auto size()
        requires sized_range<V>
        { return ranges::size(@\exposid{base_}@); }

        constexpr auto size() const
        requires sized_range<const V>
        { return ranges::size(@\exposid{base_}@); }


        constexpr V base() const & requires copy_constructible<V> { return @\exposid{base_}@; }
        constexpr V base() && { return std::move(@\exposid{base_}@); }
    };
    template<class R>
    enumerate_view(R&&) -> enumerate_view<views::all_t<R>>;
}
\end{codeblock}



\begin{itemdecl}
    constexpr enumerate_view(V base);
\end{itemdecl}

\begin{itemdescr}
    \pnum
    \effects
    Initializes \exposid{base_} with \tcode{std::move(base)}.
\end{itemdescr}

\rSec3[range.enumerate.iterator]{Class \tcode{enumerate_view::\exposid{iterator}}}

\begin{codeblock}
namespace std::ranges {
    template<@\libconcept{input_range}@ V>
    requires view<V>
    template<bool Const>
    class enumerate_view<V>::@\exposid{iterator}@ {
      public:
		using iterator_category = input_iterator_tag;
		using iterator_concept  = @\seebelow@;
		using value_type = tuple<index_type, range_value_t<@\exposid{Base}@>>;
		using difference_type = range_difference_t<@\exposid{Base}@>;
		using index_type = difference_type;

   private:
        using @\exposid{Base}@ = @\exposid{maybe-const}@<Const, V>;
        iterator_t<@\exposid{Base}@> @\exposid{current_}@ = iterator_t<@\exposid{Base}@>(); // \expos
        index_type  @\exposid{pos_}@ = 0; // \expos

        constexpr explicit @\exposid{iterator}@(iterator_t<@\exposid{Base}@> current,
                                                                             index_type pos);  // \expos

    public:

        @\exposid{iterator}@() = default;

        constexpr iterator(@\exposid{iterator}@<!Const> i)
        requires Const && convertible_to<iterator_t<V>, iterator_t<@\exposid{Base}@>>;

        constexpr iterator_t<@\exposid{Base}@> base() const&
        requires copyable<iterator_t<@\exposid{Base}@>>;
        constexpr iterator_t<@\exposid{Base}@> base() &&;

        @\addedTwo{constexpr index_type index() const; }@

        constexpr auto operator*() const {
             return tuple<index_type, range_reference_t<Base>>{ @\exposid{pos_}@, *@\exposid{current_}@};
        }

        constexpr @\exposid{iterator}@& operator++();
        constexpr void operator++(int);
        constexpr @\exposid{iterator}@ operator++(int) requires forward_range<@\exposid{Base}@>;

        constexpr @\exposid{iterator}@& operator--() requires bidirectional_range<@\exposid{Base}@>;
        constexpr @\exposid{iterator}@ operator--(int) requires bidirectional_range<@\exposid{Base}@>;

        constexpr @\exposid{iterator}@& operator+=(difference_type x)
        requires random_access_range<@\exposid{Base}@>;
        constexpr @\exposid{iterator}@& operator-=(difference_type x)
        requires random_access_range<@\exposid{Base}@>;

        constexpr auto operator[](difference_type n) const
        requires random_access_range<@\exposid{Base}@>
        { return reference{@\exposid{pos_}@ + n, *(@\exposid{current_}@ + n) }; }

        friend constexpr bool operator==(const @\exposid{iterator}@& x, const @\exposid{iterator}@& y);
        friend constexpr auto operator<=>(const @\exposid{iterator}@& x, const @\exposid{iterator}@& y);

        friend constexpr iterator operator+(const @\exposid{iterator}@& x, difference_type y)
        requires random_access_range<@\exposid{Base}@>;
        friend constexpr @\exposid{iterator}@ operator+(difference_type x, const @\exposid{iterator}@& y)
        requires random_access_range<@\exposid{Base}@>;
        friend constexpr @\exposid{iterator}@ operator-(const @\exposid{iterator}@& x, difference_type y)
        requires random_access_range<@\exposid{Base}@>;
        friend constexpr difference_type operator-(const @\exposid{iterator}@& x, const @\exposid{iterator}@& y);

        friend constexpr auto iter_move(const @\exposid{iterator}@& i)
        noexcept(noexcept(ranges::iter_move(i.@\exposid{current_}@))) {
            return tuple<index_type,
                       range_rvalue_reference_t<Base>>{@\exposid{pos_}@, ranges::iter_move(i.@\exposid{current_}@)};
        }
    };
}
\end{codeblock}

\tcode{iterator::iterator_concept} is defined as follows:
\begin{itemize}
\item If \exposid{Base} models \libconcept{random_access_range},
    then \tcode{iterator_concept} denotes \tcode{random_access_iterator_tag}.
    \item
    Otherwise, if \exposid{Base} models \libconcept{bidirectional_range},
    then \tcode{iterator_concept} denotes \tcode{bidirectional_iterator_tag}.
    \item
    Otherwise, if \exposid{Base} models \libconcept{forward_range},
    then \tcode{iterator_concept} denotes \tcode{forward_iterator_tag}.
     \item
    Otherwise, \tcode{iterator_concept} denotes \tcode{input_iterator_tag}.
\end{itemize}

\begin{itemdecl}
    constexpr explicit @\exposid{iterator}@(iterator_t<@\exposid{Base}@> current, index_type pos = 0);
\end{itemdecl}

\begin{itemdescr}
    \pnum
    \effects
    Initializes \tcode{current_} with \tcode{std::move(current)} and \exposid{pos_} with \tcode{pos}.
\end{itemdescr}

\begin{itemdecl}
    constexpr @\exposid{iterator}@(@\exposid{iterator}@<!Const> i)
    requires Const && @\libconcept{convertible_to}@<iterator_t<V>, iterator_t<@\exposid{Base}@>>;
\end{itemdecl}

\begin{itemdescr}
    \pnum
    \effects
    Initializes \exposid{current_} with \tcode{std::move(i.\exposid{current_})} and \exposid{pos_} with \tcode{i.\exposid{pos_}}.
\end{itemdescr}

\begin{itemdecl}
    constexpr iterator_t<@\exposid{Base}@> base() const&
    requires copyable<iterator_t<@\exposid{Base}@>>;
\end{itemdecl}

\begin{itemdescr}
    \pnum
    \effects
    Equivalent to: \tcode{return \exposid{current_};}
\end{itemdescr}

\begin{itemdecl}
    constexpr iterator_t<@\exposid{Base}@> base() &&;
\end{itemdecl}

\begin{itemdescr}
    \pnum
    \effects
    Equivalent to: \tcode{return std::move(\exposid{current_});}
\end{itemdescr}

\begin{addedblockTwo}
\begin{itemdecl}
constexpr index_type index() const;
\end{itemdecl}

\begin{itemdescr}
	\pnum
	\effects
	Equivalent to: \tcode{return \exposid{pos_};}
\end{itemdescr}
\end{addedblockTwo}

\begin{itemdecl}
    constexpr @\exposid{iterator}@& operator++();
\end{itemdecl}

\begin{itemdescr}
    \pnum
    \effects
    Equivalent to:
    \begin{codeblock}
    	++@\exposid{current_}@;
    	++@\exposid{pos_}@;
        return *this;
    \end{codeblock}
\end{itemdescr}

\begin{itemdecl}
    constexpr void operator++(int);
\end{itemdecl}

\begin{itemdescr}
    \pnum
    \effects
    Equivalent to:
    \begin{codeblock}
    ++@\exposid{current_}@;
    ++@\exposid{pos_}@;
    \end{codeblock}
\end{itemdescr}

\begin{itemdecl}
    constexpr @\exposid{iterator}@ operator++(int) requires @\libconcept{forward_range}@<@\exposid{Base}@>;
\end{itemdecl}

\begin{itemdescr}
    \pnum
    \effects
    Equivalent to:
    \begin{codeblock}
        auto temp = *this;
        ++*this;
        return temp;
    \end{codeblock}
\end{itemdescr}

\begin{itemdecl}
    constexpr @\exposid{iterator}@& operator--() requires @\libconcept{bidirectional_range}@<@\exposid{Base}@>;
\end{itemdecl}

\begin{itemdescr}
    \pnum
    \effects
    Equivalent to:
    \begin{codeblock}
    	--@\exposid{current_}@;
    	--@\exposid{pos_}@;
        return *this;
    \end{codeblock}
\end{itemdescr}

\begin{itemdecl}
    constexpr @\exposid{iterator}@ operator--(int) requires @\libconcept{bidirectional_range}@<@\exposid{Base}@>;
\end{itemdecl}

\begin{itemdescr}
    \pnum
    \effects
    Equivalent to:
    \begin{codeblock}
        auto temp = *this;
        --*this;
        return temp;
    \end{codeblock}
\end{itemdescr}

\begin{itemdecl}
    constexpr @\exposid{iterator}@& operator+=(difference_type n);
    requires @\libconcept{random_access_range}@<@\exposid{Base}@>;
\end{itemdecl}

\begin{itemdescr}
    \pnum
    \effects
    Equivalent to:
    \begin{codeblock}
        @\exposid{current_}@ += n;
        @\exposid{pos_}@ += n;
        return *this;
    \end{codeblock}
\end{itemdescr}

\begin{itemdecl}
    constexpr @\exposid{iterator}@& operator-=(difference_type n)
    requires @\libconcept{random_access_range}@<@\exposid{Base}@>;
\end{itemdecl}

\begin{itemdescr}
    \pnum
    \effects
    Equivalent to:
    \begin{codeblock}
        @\exposid{current_}@ -= n;
        @\exposid{pos_}@ -= n;
        return *this;
    \end{codeblock}
\end{itemdescr}

\begin{itemdecl}
    friend constexpr bool operator==(const @\exposid{iterator}@& x, const @\exposid{iterator}@& y);
\end{itemdecl}

\begin{itemdescr}
    \pnum
    \effects
    Equivalent to: \tcode{return x.\exposid{pos_} == y.\exposid{pos_};}
\end{itemdescr}

\begin{itemdecl}
    friend constexpr auto operator<=>(const @\exposid{iterator}@& x, const @\exposid{iterator}@& y);
\end{itemdecl}

\begin{itemdescr}
    \pnum
    \effects
    Equivalent to: \tcode{return x.\exposid{pos_} <=> y.\exposid{pos_};}
\end{itemdescr}

\begin{itemdecl}
    friend constexpr @\exposid{iterator}@ operator+(const @\exposid{iterator}@& x, difference_type y)
    requires @\libconcept{random_access_range}@<@\exposid{Base}@>;
\end{itemdecl}

\begin{itemdescr}
    \pnum
    \effects
    Equivalent to: \tcode{return \exposid{iterator}\{x\} += y;}
\end{itemdescr}

\begin{itemdecl}
    friend constexpr @\exposid{iterator}@ operator+(difference_type x, const @\exposid{iterator}@& y)
    requires @\libconcept{random_access_range}@<@\exposid{Base}@>;
\end{itemdecl}

\begin{itemdescr}
    \pnum
    \effects
    Equivalent to: \tcode{return y + x;}
\end{itemdescr}

\begin{itemdecl}
    constexpr @\exposid{iterator}@ operator-(const @\exposid{iterator}@& x, difference_type y)
    requires @\libconcept{random_access_range}@<@\exposid{Base}@>;
\end{itemdecl}

\begin{itemdescr}
    \pnum
    \effects
    Equivalent to: \tcode{return iterator\{x\} -= y;}
\end{itemdescr}

\begin{itemdecl}
    constexpr difference_type operator-(const @\exposid{iterator}@& x, const @\exposid{iterator}@& y);
\end{itemdecl}

\begin{itemdescr}
    \pnum
    \effects
    Equivalent to: \tcode{return x.\exposid{pos_} - y.\exposid{pos_};}
\end{itemdescr}


\rSec3[range.enumerate.sentinel]{Class template \tcode{enumerate_view::\exposid{sentinel}}}

\begin{codeblock}
namespace std::ranges {
    template<input_range V, size_t N>
    requires view<V>
    template<bool Const>
    class enumerate_view<V, N>::@\exposid{sentinel}@ {                 // \expos
     private:
        using @\exposid{Base}@ = @\exposid{maybe-const}@<Const, V>;                 // \expos
        sentinel_t<@\exposid{Base}@> @\exposid{end_}@ = sentinel_t<@\exposid{Base}@>();         // \expos
     public:
        @\exposid{sentinel}@() = default;
        constexpr explicit @\exposid{sentinel}@(sentinel_t<@\exposid{Base}@> end);
        constexpr @\exposid{sentinel}@(@\exposid{sentinel}@<!Const> other)
        requires Const && @\libconcept{convertible_to}@<sentinel_t<V>, sentinel_t<@\exposid{Base}@>>;

        constexpr sentinel_t<@\exposid{Base}@> base() const;

        template<bool OtherConst>
        requires @\libconcept{sentinel_for}@<sentinel_t<@\exposid{Base}@>, iterator_t<@\exposid{maybe-const}@<OtherConst, V>>>
        friend constexpr bool operator==(const @\exposid{iterator}@<OtherConst>& x, const @\exposid{sentinel}@& y);

        template<bool OtherConst>
        requires @\libconcept{sized_sentinel_for}@<sentinel_t<@\exposid{Base}@>, iterator_t<@\exposid{maybe-const}@<OtherConst, V>>>
        friend constexpr range_difference_t<@\exposid{maybe-const}@<OtherConst, V>>
        operator-(const @\exposid{iterator}@<OtherConst>& x, const @\exposid{sentinel}@& y);

        template<bool OtherConst>
        requires @\libconcept{sized_sentinel_for}@<sentinel_t<@\exposid{Base}@>, iterator_t<@\exposid{maybe-const}@<OtherConst, V>>>
        friend constexpr range_difference_t<@\exposid{maybe-const}@<OtherConst, V>>
        operator-(const @\exposid{sentinel}@& x, const @\exposid{iterator}@<OtherConst>& y);
    };
}
\end{codeblock}

\begin{itemdecl}
	constexpr explicit @\exposid{sentinel}@(sentinel_t<@\exposid{Base}@> end);
\end{itemdecl}

\begin{itemdescr}
	\pnum
	\effects
	Initializes \exposid{end_} with \tcode{end}.
\end{itemdescr}

\indexlibraryctor{elements_view::\exposid{sentinel}}%
\begin{itemdecl}
	constexpr @\exposid{sentinel}@(@\exposid{sentinel}@<!Const> other)
	requires Const && @\libconcept{convertible_to}@<sentinel_t<V>, sentinel_t<@\exposid{Base}@>>;
\end{itemdecl}

\begin{itemdescr}
	\pnum
	\effects
	Initializes \exposid{end_} with \tcode{std::move(other.\exposid{end_})}.
\end{itemdescr}

\indexlibrarymember{base}{elements_view::\exposid{sentinel}}%
\begin{itemdecl}
	constexpr sentinel_t<@\exposid{Base}@> base() const;
\end{itemdecl}

\begin{itemdescr}
	\pnum
	\effects
	Equivalent to: \tcode{return \exposid{end_};}
\end{itemdescr}

\indexlibrarymember{operator==}{elements_view::\exposid{sentinel}}%
\begin{itemdecl}
	template<bool OtherConst>
	requires @\libconcept{sentinel_for}@<sentinel_t<@\exposid{Base}@>, iterator_t<@\exposid{maybe-const}@<OtherConst, V>>>
	friend constexpr bool operator==(const @\exposid{iterator}@<OtherConst>& x, const @\exposid{sentinel}@& y);
\end{itemdecl}

\begin{itemdescr}
	\pnum
	\effects
	Equivalent to: \tcode{return x.\exposid{current_} == y.\exposid{end_};}
\end{itemdescr}

\indexlibrarymember{operator-}{elements_view::\exposid{sentinel}}%
\begin{itemdecl}
	template<bool OtherConst>
	requires @\libconcept{sized_sentinel_for}@<sentinel_t<@\exposid{Base}@>, iterator_t<@\exposid{maybe-const}@<OtherConst, V>>>
	friend constexpr range_difference_t<@\exposid{maybe-const}@<OtherConst, V>>
	operator-(const @\exposid{iterator}@<OtherConst>& x, const @\exposid{sentinel}@& y);
\end{itemdecl}

\begin{itemdescr}
	\pnum
	\effects
	Equivalent to: \tcode{return x.\exposid{current_} - y.\exposid{end_};}
\end{itemdescr}

\indexlibrarymember{operator-}{elements_view::\exposid{sentinel}}%
\begin{itemdecl}
	template<bool OtherConst>
	requires @\libconcept{sized_sentinel_for}@<sentinel_t<@\exposid{Base}@>, iterator_t<@\exposid{maybe-const}@<OtherConst, V>>>
	friend constexpr range_difference_t<@\exposid{maybe-const}@<OtherConst, V>>
	operator-(const @\exposid{sentinel}@& x, const @\exposid{iterator}@<OtherConst>& y);
\end{itemdecl}

\begin{itemdescr}
	\pnum
	\effects
	Equivalent to: \tcode{return x.\exposid{end_} - y.\exposid{current_};}
\end{itemdescr}

\subsection{Feature test macro}

\ednote{define \tcode{__cpp_lib_ranges_enumerate} set to the date of adoption in \tcode{<version>} and \tcode{<ranges>}}.

\end{addedblock}


\section{Acknowledgments}

Thanks a lot to \tcode{Tomasz Kamiński} for finding defects in the design proposed in earlier revisions, as well as his invaluable wording feedbacks.
Thanks to Barry Revzin and Christopher Di Bella for their inputs on the design.

\section{References}
\renewcommand{\section}[2]{}%
\bibliographystyle{plain}
\bibliography{wg21, extra}

\begin{thebibliography}{9}

\bibitem[N4885]{N4885}
Thomas Köppe
\emph{Working Draft, Standard for Programming Language C++}\newline
\url{https://wg21.link/N4885}


\end{thebibliography}
\end{document}