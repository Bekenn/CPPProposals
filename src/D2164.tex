% !TeX program = luatex
% !TEX encoding = UTF-8


\documentclass{wg21}

\title{\tcode{views::enumerate}}
\docnumber{P2164R5}
\audience{SG-9}
\author{Corentin Jabot}{corentin.jabot@gmail.com}

\begin{document}
\maketitle


\section{Abstract}

We propose a view \tcode{enumerate} whose value type is a struct with 2 members \tcode{index} and \tcode{value}
representing respectively the position and value of the elements in the adapted range.

\section{Revisions}

\subsection{R5}

Instead of adding complexity to \tcode{enumerate_result}, we assume changes made by \paper{P2165R2}.
\paper{P2165R2} makes \tcode{pair} constructible from \placeholder{pair-like} objects, and associative containers deduction guides work
with ranges of \placeholder{pair-like} objects.
With these changes, \tcode{enumerate_result} can remain a simple aggregate. We just need to implement the tuple protocol for it (\tcode{get}, \tcode{tuple_element}, \tcode{tuple_size}).

For simplicity, consistency with \tcode{zip} and \tcode{carthesian_product} and to avoid \tcode{enumerate_result} propagating,
the reference type of \tcode{enumerate_view} is \tcode{enumerate_result}  and its value type is \tcode{tuple}.

\paper{P2165R2} ensures a common reference exists as long as one exists between each element.

\subsection{R4}

This revision is intended to illustrate the effort necessary to support named fields for \tcode{index} and \tcode{value}.
In previous revisions, the value and reference types were identical, a regrettable blunder that made the wording and implementation efforts smaller than they are.
\tcode{reference} and \tcode{value_type} types however needs to be different, if only to make the \tcode{ranges::to} presented in this very paper.

If that direction is acceptable, better wording will be provided to account for these new \tcode{reference} and \tcode{value_type} types.

This revision also gets rid of the \tcode{const} \tcode{index} value as LEWG strongly agreed that it was a terrible idea to begin with, one that would make composition with other views cumbersome.

\subsection{R3}

\begin{itemize}
    \item Typos and minor wording improvements
\end{itemize}

\subsection{R2, following mailing list reviews}
\begin{itemize}
    \item Make \tcode{value_type} different from  \tcode{reference} to match other views
    \item Remove inconsistencies between the wording and the description
    \item Add relevant includes and namespaces to the examples
\end{itemize}


\subsection{R1}
\begin{itemize}
\item Fix the index type
\end{itemize}

\section{Tony tables}
\begin{center}
\begin{tabular}{l|l}
Before & After\\ \hline

\begin{minipage}[t]{0.5\textwidth}
\begin{colorblock}

std::vector days{"Mon", "Tue",
  "Wed", "Thu", "Fri", "Sat", "Sun"};

int idx = 0;
for(const auto & d : days) {
    print("{} {} \n", idx, d);
    idx++;
}

\end{colorblock}
\end{minipage}
&
\begin{minipage}[t]{0.5\textwidth}
\begin{colorblock}
#include <ranges>

std::vector days{"Mon", "Tue",
  "Wed", "Thu", "Fri", "Sat", "Sun"};

for(const auto & e : std::views::enumerate(days)) {
    print("{} {} \n", e.index, e.value);
}

\end{colorblock}
\end{minipage}
\\\\ \hline

\end{tabular}
\end{center}

\section{Motivation}

The impossibility to extract an index from a range-based for loop leads to the use of non-range-based \tcode{for} loops, or the introduction of a variable in the outer scope. This is both more verbose and error-prone: in the example above, the type of \tcode{idx} is incorrect.

\tcode{enumerate} is a library solution solving this problem, enabling the use of range-based for loops in more cases.

It also composes nicely with other range facilities:
The following creates a map from a vector using the position of each element as key.

\begin{colorblock}
my_vector | views::enumerate | ranges::to<map>;
\end{colorblock}

This feature exists in some form in Python, Rust, Go (backed into the language), and in many C++ libraries: ranges-v3, folly, boost::ranges (\tcode{indexed}).


The existence of this feature or lack thereof is the subject of recurring StackOverflow questions.


\section{Design}

\subsection{The reference type is a simple aggregate with name members}

Following the trend of using meaningful names instead of returning pairs or tuples, this proposal uses a struct with named public members.
\begin{colorblock}
struct enumerate_result {
    count index;
    T value;
};
\end{colorblock}

This design was previously discussed by LEWGI in Belfast in the context of \paper{P1894R0}, and many people have expressed a desire for such struct with names.
Using this struct for both the reference type and the value type would add significant complexity, as the value and reference type need to share a \tcode{common_reference} (see \paper{P2164R4}).

Instead, we propose that the reference type is \tcode{enumerate_result<index, range_reference_t<Base>>} and the value type is
\tcode{tuple<index, range_value_t<Base>>}.

With is design, only \tcode{get}, \tcode{tuple_element}, \tcode{tuple_size} need to be implemented for \tcode{enumerate_result},
and \tcode{enumerate_result} remains a simple aggregate.

This design works nicely with \tcode{ranges::to} as it will create a container based on the \tcode{value} type:

\begin{colorblock}
std::vector<double> v;
enumerate(view) | to<std::vector>(); // std::vector<std::tuple<std::size_t, double>>.
enumerate(view) | to<std::map>();    // std::map<std::size_t, double>.
\end{colorblock}

This gives us some consistency: \tcode{enumerate}'s value type is a \tcode{tuple}, similar to that of \tcode{zip}, \tcode{carthesian_product},
while retaining the ease of use and added benefits of a struct with named members while iterating over an \tcode{enumerate_view}.

\subsection{Why not just always return a tuple/pair and rely on structure binding?}

If a range reference type is convertible to the index type,
it is error-prone whether one should write

\begin{colorblock}
for(auto && [value, index] : view | std::views::enumerate)
for(auto && [index, value] : view | std::views::enumerate)
\end{colorblock}

Having named members avoids this issue.
The feedback I keep getting is "we should use a struct if we can". Which is consistent with previous LEWG guidelines to avoid using pair
when a more meaningful type is possible.

And we can. The proposed design in R5 is not involved.
Keep in mind that zipping the view with iota \href{https://github.com/ericniebler/range-v3/issues/1141}{does not actually work}
(see also \href{http://www.open-std.org/jtc1/sc22/wg21/docs/papers/2020/p2214r0.html#enumerates-first-range}{\paper{P2214R0}}),
and a custom \tcode{index_view} would need to be used as the first range composed with \tcode{zip}, so a custom \tcode{enumerate} view with
appropriately named members is not adding a lot of work if we pursue \paper{P2165R2}.

Granted, \paper{P2165R2} and this paper justify each other, and \paper{P2165R2} is not a trivial amount of work.
However, P2165 offers further benefits besides enabling a slightly nicer enumerate, so if we think P2165 is generally useful,
we can pursue this paper. If we don't, we can quickly respecify \tcode{enumerate} in terms of \tcode{zip} and some \tcode{index_view}, for which we
have usage experience.

\tcode{enumerate} as presented here is slightly less work for the compiler, but both solutions generate similar assembly.

\subsection{index_type}

\tcode{index_type} is defined as follow:
\begin{itemize}
    \item \tcode{ranges::range_size_t<Base>} if \tcode{Base} models \tcode{ranges::sized_range}
    \item Otherwise, \tcode{make_unsigned_t<ranges::range_difference_t<Base>>}
\end{itemize}

This is consistent with ranges-v3 and allows the view to support both sized and non-sized ranges.


\subsection{Performance}

An optimizing compiler can generate the same machine code for \tcode{views::enumerate} as it would for an equivalent \tcode{for} loop.  \href{https://godbolt.org/z/2Kxo8d}{Compiler Explorer} \ednote{This implementation is a prototype not fully reflective of the proposed design}.


\subsection{Implementation}

This proposal has been implemented (\href{https://github.com/cor3ntin/rangesnext/blob/master/include/cor3ntin/rangesnext/enumerate.hpp}{Github})
There exist an implementation in ranges-v3 (where the enumerate view uses zip_with and a pair value type).

\section{Proposal}

We propose a view \tcode{enumerate} whose value type is a struct with 2 members \tcode{index} and \tcode{value}
representing respectively the position and value of the elements in the adapted range.

\section{Wording}

\begin{addedblock}

\ednote{TODO: ranges synopsis}

\rSec2[range.enumerate]{Enumerate view}

\rSec3[range.enumerate.overview]{Overview}

\pnum
\tcode{enumerate_view} presents a \libconcept{view} with a value type that represents both the position and value of the adapted \libconcept{view}'s value-type.

\pnum
The name \tcode{views::enumerate} denotes a
range adaptor object\iref{range.adaptor.object}.
Given the subexpressions \tcode{E}
the expression \tcode{views::enumerate(E)} is expression-equivalent to \tcode{enumerate_view\{E\}}.

\pnum
\begin{example}
\begin{codeblock}
vector<int> vec{ 1, 2, 3 };
for (auto [index, value] : enumerate(vec) )
    cout << index << ":" << value ' '; // prints: 0:1 1:2 2:3
\end{codeblock}
\end{example}

\rSec3[range.enumerate.view]{Class template \tcode{enumerate_view}}


\begin{codeblock}
namespace std::ranges {
    template <class Index, class Value>
    struct enumerate_result {
        Index index;
        Value value;
    };

    template<size_t I, class Index, class Value>
    constexpr tuple_element_t<I, enumerate_result<Index, Value>>&
    get(enumerate_result<Index, Value>&) noexcept;

    template<size_t I, class Index, class Value>
    constexpr tuple_element_t<I, enumerate_result<Index, Value>>&&
    get(enumerate_result<Index, Value>&&) noexcept;

    template<size_t I, class Index, class Value>
    constexpr const tuple_element_t<I, enumerate_result<Index, Value>>&
    get(const enumerate_result<Index, Value>&) noexcept;

    template<size_t I, class Index, class Value>
    constexpr const tuple_element_t<I, enumerate_result<Index, Value>>&&
    get(const enumerate_result<Index, Value>&&) noexcept;

    template<@\libconcept{input_range}@ V>
    requires view<V>
    class enumerate_view : public view_interface<enumerate_view<V>> {

      private:
        V @\exposid{base_}@ = {};

        template <bool Const>
        class @\exposid{iterator}@; // \expos
        template <bool Const>
        struct @\exposid{sentinel}@; // \expos

       public:

        constexpr enumerate_view() = default;
        constexpr enumerate_view(V base);

        constexpr auto begin() requires (!simple_view<V>)
        { return iterator<false>(ranges::begin(base_), 0); }

        constexpr auto begin() const requires simple_view<V>
        { return iterator<true>(ranges::begin(base_), 0); }

        constexpr auto end()
        { return sentinel<false>{end(base_)}; }

        constexpr auto end()
        requires common_range<V> && sized_range<V>
        { return iterator<false>{ranges::end(base_),
                 static_cast<range_difference_t<V>>(size()) }; }

        constexpr auto end() const
        requires range<const V>
        { return sentinel<true>{ranges::end(base_)}; }

        constexpr auto end() const
        requires common_range<const V> && sized_range<V>
        { return iterator<true>{ranges::end(base_),
                 static_cast<range_difference_t<V>>(size())}; }

        constexpr auto size()
        requires sized_range<V>
        { return ranges::size(base_); }

        constexpr auto size() const
        requires sized_range<const V>
        { return ranges::size(base_); }


        constexpr V base() const & requires copy_constructible<V> { return @\exposid{base_}@; }
        constexpr V base() && { return move(@\exposid{base_}@); }
    };
    template<class R>
    enumerate_view(R&&) -> enumerate_view<views::all_t<R>>;
}

namespace std {

   template<class Index, class Value>
   struct tuple_size<ranges::enumerate_result<Index, Value>> : integral_constant<size_t, 2> { };

   template<size_t I, class Index, class Value>
   struct tuple_element<I, ranges::enumerate_result<Index, Value>> {
       using type = see below ;
   };

}

\end{codeblock}

\begin{itemdecl}
template<size_t I, class Index, class Value>
struct tuple_element<I, ranges::enumerate_result<Index, Value>> {
    using type = @\seebelow@;
};
\end{itemdecl}

\begin{itemdescr}
\mandates \tcode{I < 2}.

\ctype The type \tcode{Index} if \tcode{I} is \tcode{0}, otherwise the type \tcode{Value}.
\end{itemdescr}


\begin{itemdecl}
template<size_t I, class Index, class Value>
constexpr tuple_element_t<I, enumerate_result<Index, Value>>&
get(enumerate_result<Index, Value>& r) noexcept;

template<size_t I, class Index, class Value>
constexpr tuple_element_t<I, enumerate_result<Index, Value>>&&
get(enumerate_result<Index, Value>&& r) noexcept;

template<size_t I, class Index, class Value>
constexpr const tuple_element_t<I, enumerate_result<Index, Value>>&
get(const enumerate_result<Index, Value>& r) noexcept;

template<size_t I, class Index, class Value>
constexpr const tuple_element_t<I, enumerate_result<Index, Value>>&&
get(const enumerate_result<Index, Value>&& r) noexcept;
\end{itemdecl}

\begin{itemdescr}
\mandates \tcode{I < 2}.
\returns
\begin{itemize}
\item if {I} is 0, returns a reference to \tcode{r.index}.
\item if {I} is 1, returns a reference to \tcode{r.value}.
\end{itemize}
\end{itemdescr}


\begin{itemdecl}
    constexpr enumerate_view(V base);
\end{itemdecl}

\begin{itemdescr}
    \pnum
    \effects
    Initializes \exposid{base_} with \tcode{move(base)}.
\end{itemdescr}

\rSec3[range.enumerate.iterator]{Class \tcode{enumerate_view::\exposid{iterator}}}

\begin{codeblock}
namespace std::ranges {
    template<@\libconcept{input_range}@ V>
    requires view<V>
    template<bool Const>
    class enumerate_view<V>::@\exposid{iterator}@ {

        using Base = conditional_t<Const, const V, V>;
        using index_type = @\seebelow@;

        iterator_t<Base> current_ = iterator_t<Base>();
        index_type pos_ = 0;


      public:
        using iterator_category = typename iterator_traits<iterator_t<Base>>::iterator_category;

        using reference  = enumerate_result<index_type, range_reference_t<Base>>;
        using value_type = tuple<index_type, range_value_t<Base>>;

        using difference_type = range_difference_t<Base>;

        iterator() = default;
        constexpr explicit iterator(iterator_t<Base> current, range_difference_t<Base> pos);
        constexpr iterator(iterator<!Const> i)
        requires Const && convertible_to<iterator_t<V>, iterator_t<Base>>;


        constexpr iterator_t<Base> base() const&
        requires copyable<iterator_t<Base>>;
        constexpr iterator_t<Base> base() &&;

        constexpr decltype(auto) operator*() const {
             return reference{pos_, *current_};
        }

        constexpr iterator& operator++();
        constexpr void operator++(int) requires (!forward_range<Base>);
        constexpr iterator operator++(int) requires forward_range<Base>;

        constexpr iterator& operator--() requires bidirectional_range<Base>;
        constexpr iterator operator--(int) requires bidirectional_range<Base>;

        constexpr iterator& operator+=(difference_type x)
        requires random_access_range<Base>;
        constexpr iterator& operator-=(difference_type x)
        requires random_access_range<Base>;

        constexpr decltype(auto) operator[](difference_type n) const
        requires random_access_range<Base>
        { return reference{static_cast<difference_type>(pos_ + n), *(current_ + n) }; }


        friend constexpr bool operator==(const iterator& x, const iterator& y)
        requires equality_comparable<iterator_t<Base>>;

        friend constexpr bool operator<(const iterator& x, const iterator& y)
        requires random_access_range<Base>;
        friend constexpr bool operator>(const iterator& x, const iterator& y)
        requires random_access_range<Base>;
        friend constexpr bool operator<=(const iterator& x, const iterator& y)
        requires random_access_range<Base>;
        friend constexpr bool operator>=(const iterator& x, const iterator& y)
        requires random_access_range<Base>;
        friend constexpr auto operator<=>(const iterator& x, const iterator& y)
        requires random_access_range<Base> && three_way_comparable<iterator_t<Base>>;

        friend constexpr iterator operator+(const iterator& x, difference_type y)
        requires random_access_range<Base>;
        friend constexpr iterator operator+(difference_type x, const iterator& y)
        requires random_access_range<Base>;
        friend constexpr iterator operator-(const iterator& x, difference_type y)
        requires random_access_range<Base>;
        friend constexpr difference_type operator-(const iterator& x, const iterator& y)
        requires random_access_range<Base>;
    };
}
\end{codeblock}


\tcode{iterator::index_type} is defined as follow:
\begin{itemize}
    \item \tcode{ranges::range_size_t<Base>} if \tcode{Base} models \tcode{ranges::sized_range}
    \item Otherwise, \tcode{make_unsigned_t<ranges::range_difference_t<Base>>}
\end{itemize}



\begin{itemdecl}
    constexpr explicit @\exposid{iterator}@(iterator_t<@\exposid{Base}@> current, range_difference_t<Base> pos = 0);
\end{itemdecl}

\begin{itemdescr}
    \pnum
    \effects
    Initializes \tcode{current_} with \tcode{move(current)} and \exposid{pos} with \tcode{static_cast<index_type>(pos)}.
\end{itemdescr}

\begin{itemdecl}
    constexpr @\exposid{iterator}@(@\exposid{iterator}@<!Const> i)
    requires Const && @\libconcept{convertible_to}@<iterator_t<V>, iterator_t<@\exposid{Base}@>>;
\end{itemdecl}

\begin{itemdescr}
    \pnum
    \effects
    Initializes \exposid{current_} with \tcode{move(i.\exposid{current_})} and \exposid{pos} with \tcode{i.pos_}.
\end{itemdescr}

\begin{itemdecl}
    constexpr iterator_t<@\exposid{Base}@> base() const&
    requires copyable<iterator_t<@\exposid{Base}@>>;
\end{itemdecl}

\begin{itemdescr}
    \pnum
    \effects
    Equivalent to: \tcode{return \exposid{current_};}
\end{itemdescr}

\begin{itemdecl}
    constexpr iterator_t<@\exposid{Base}@> base() &&;
\end{itemdecl}

\begin{itemdescr}
    \pnum
    \effects
    Equivalent to: \tcode{return move(\exposid{current_});}
\end{itemdescr}

\begin{itemdecl}
    constexpr @\exposid{iterator}@& operator++();
\end{itemdecl}

\begin{itemdescr}
    \pnum
    \effects
    Equivalent to:
    \begin{codeblock}
        ++@\exposid{pos}@;
        ++@\exposid{current_}@;
        return *this;
    \end{codeblock}
\end{itemdescr}

\begin{itemdecl}
    constexpr void operator++(int) requires (!@\libconcept{forward_range}@<@\exposid{Base}@>);
\end{itemdecl}

\begin{itemdescr}
    \pnum
    \effects
    Equivalent to:
    \begin{codeblock}
        ++@\exposid{pos}@;
        ++@\exposid{current_}@;
    \end{codeblock}
\end{itemdescr}

\begin{itemdecl}
    constexpr @\exposid{iterator}@ operator++(int) requires @\libconcept{forward_range}@<@\exposid{Base}@>;
\end{itemdecl}

\begin{itemdescr}
    \pnum
    \effects
    Equivalent to:
    \begin{codeblock}
        auto temp = *this;
        ++@\exposid{pos}@;
        ++@\exposid{current_}@;
        return temp;
    \end{codeblock}
\end{itemdescr}

\begin{itemdecl}
    constexpr @\exposid{iterator}@& operator--() requires @\libconcept{bidirectional_range}@<@\exposid{Base}@>;
\end{itemdecl}

\begin{itemdescr}
    \pnum
    \effects
    Equivalent to:
    \begin{codeblock}
        --@\exposid{pos_}@;
        --@\exposid{current_}@;
        return *this;
    \end{codeblock}
\end{itemdescr}

\begin{itemdecl}
    constexpr @\exposid{iterator}@ operator--(int) requires @\libconcept{bidirectional_range}@<@\exposid{Base}@>;
\end{itemdecl}

\begin{itemdescr}
    \pnum
    \effects
    Equivalent to:
    \begin{codeblock}
        auto temp = *this;
        --@\exposid{current_}@;
        --@\exposid{pos_}@;
        return temp;
    \end{codeblock}
\end{itemdescr}

\begin{itemdecl}
    constexpr @\exposid{iterator}@& operator+=(difference_type n);
    requires @\libconcept{random_access_range}@<@\exposid{Base}@>;
\end{itemdecl}

\begin{itemdescr}
    \pnum
    \effects
    Equivalent to:
    \begin{codeblock}
        @\exposid{current_}@ += n;
        @\exposid{pos_}@ += n;
        return *this;
    \end{codeblock}
\end{itemdescr}

\begin{itemdecl}
    constexpr @\exposid{iterator}@& operator-=(difference_type n)
    requires @\libconcept{random_access_range}@<@\exposid{Base}@>;
\end{itemdecl}

\begin{itemdescr}
    \pnum
    \effects
    Equivalent to:
    \begin{codeblock}
        @\exposid{current_}@ -= n;
        @\exposid{pos_}@ -= n;
        return *this;
    \end{codeblock}
\end{itemdescr}

\begin{itemdecl}
    friend constexpr bool operator==(const @\exposid{iterator}@& x, const @\exposid{iterator}@& y)
    requires equality_comparable<@\exposid{Base}@>;
\end{itemdecl}

\begin{itemdescr}
    \pnum
    \effects
    Equivalent to: \tcode{return x.\exposid{current_} == y.\exposid{current_};}
\end{itemdescr}

\begin{itemdecl}
    friend constexpr bool operator<(const @\exposid{iterator}@& x, const @\exposid{iterator}@& y)
    requires @\libconcept{random_access_range}@<@\exposid{Base}@>;
\end{itemdecl}

\begin{itemdescr}
    \pnum
    \effects
    Equivalent to: \tcode{return x.\exposid{current_} < y.\exposid{current_};}
\end{itemdescr}

\begin{itemdecl}
    friend constexpr bool operator>(const @\exposid{iterator}@& x, const @\exposid{iterator}@& y)
    requires @\libconcept{random_access_range}@<@\exposid{Base}@>;
\end{itemdecl}

\begin{itemdescr}
    \pnum
    \effects
    Equivalent to: \tcode{return y < x;}
\end{itemdescr}

\begin{itemdecl}
    friend constexpr bool operator<=(const @\exposid{iterator}@& x, const @\exposid{iterator}@& y)
    requires @\libconcept{random_access_range}@<@\exposid{Base}@>;
\end{itemdecl}

\begin{itemdescr}
    \pnum
    \effects
    Equivalent to: \tcode{return !(y < x);}
\end{itemdescr}

\begin{itemdecl}
    friend constexpr bool operator>=(const @\exposid{iterator}@& x, const @\exposid{iterator}@& y)
    requires @\libconcept{random_access_range}@<@\exposid{Base}@>;
\end{itemdecl}

\begin{itemdescr}
    \pnum
    \effects
    Equivalent to: \tcode{return !(x < y);}
\end{itemdescr}

\begin{itemdecl}
    friend constexpr auto operator<=>(const @\exposid{iterator}@& x, const @\exposid{iterator}@& y)
    requires @\libconcept{random_access_range}@<@\exposid{Base}@> && @\libconcept{three_way_comparable}@<iterator_t<@\exposid{Base}@>>;
\end{itemdecl}

\begin{itemdescr}
    \pnum
    \effects
    Equivalent to: \tcode{return x.\exposid{current_} <=> y.\exposid{current_};}
\end{itemdescr}

\begin{itemdecl}
    friend constexpr @\exposid{iterator}@ operator+(const @\exposid{iterator}@& x, difference_type y)
    requires @\libconcept{random_access_range}@<@\exposid{Base}@>;
\end{itemdecl}

\begin{itemdescr}
    \pnum
    \effects
    Equivalent to: \tcode{return iterator\{x\} += y;}
\end{itemdescr}

\begin{itemdecl}
    friend constexpr @\exposid{iterator}@ operator+(difference_type x, const @\exposid{iterator}@& y)
    requires @\libconcept{random_access_range}@<@\exposid{Base}@>;
\end{itemdecl}

\begin{itemdescr}
    \pnum
    \effects
    Equivalent to: \tcode{return y + x;}
\end{itemdescr}

\begin{itemdecl}
    constexpr @\exposid{iterator}@ operator-(const @\exposid{iterator}@& x, difference_type y)
    requires @\libconcept{random_access_range}@<@\exposid{Base}@>;
\end{itemdecl}

\begin{itemdescr}
    \pnum
    \effects
    Equivalent to: \tcode{return iterator\{x\} -= y;}
\end{itemdescr}

\begin{itemdecl}
    constexpr difference_type operator-(const @\exposid{iterator}@& x, const @\exposid{iterator}@& y)
    requires @\libconcept{random_access_range}@<@\exposid{Base}@>;
\end{itemdecl}

\begin{itemdescr}
    \pnum
    \effects
    Equivalent to: \tcode{return x.\exposid{current_} - y.\exposid{current_};}
\end{itemdescr}


\rSec3[range.enumerate.sentinel]{Class template \tcode{enumerate_view::\exposid{sentinel}}}

\begin{codeblock}
namespace std::ranges {
    template<input_range V, size_t N>
    requires view<V>
    template<bool Const>
    class enumerate_view<V, N>::@\exposid{sentinel}@ {                 // \expos
        private:
        using @\exposid{Base}@ = conditional_t<Const, const V, V>;      // \expos
        sentinel_t<@\exposid{Base}@> @\exposid{end_}@ = sentinel_t<@\exposid{Base}@>();         // \expos
        public:
        @\exposid{sentinel}@() = default;
        constexpr explicit @\exposid{sentinel}@(sentinel_t<@\exposid{Base}@> end);
        constexpr @\exposid{sentinel}@(@\exposid{sentinel}@<!Const> other)
        requires Const && convertible_to<sentinel_t<V>, sentinel_t<@\exposid{Base}@>>;

        constexpr sentinel_t<@\exposid{Base}@> base() const;

        friend constexpr bool operator==(const @\exposid{iterator}@<Const>& x, const @\exposid{sentinel}@& y);

        friend constexpr range_difference_t<@\exposid{Base}@>
        operator-(const @\exposid{iterator}@<Const>& x, const @\exposid{sentinel}@& y)
        requires sized_sentinel_for<sentinel_t<@\exposid{Base}@>, iterator_t<@\exposid{Base}@>>;

        friend constexpr range_difference_t<@\exposid{Base}@>
        operator-(const @\exposid{sentinel}@& x, const @\exposid{iterator}@<Const>& y)
        requires sized_sentinel_for<sentinel_t<@\exposid{Base}@>, iterator_t<@\exposid{Base}@>>;
    };
}
\end{codeblock}

\begin{itemdecl}
    constexpr explicit @\exposid{sentinel}@(sentinel_t<@\exposid{Base}@> end);
\end{itemdecl}

\begin{itemdescr}
    \pnum
    \effects
    Initializes \exposid{end_} with \tcode{end}.
\end{itemdescr}

\begin{itemdecl}
    constexpr @\exposid{sentinel}@(@\exposid{sentinel}@<!Const> other)
    requires Const && @\libconcept{convertible_to}@<sentinel_t<V>, sentinel_t<@\exposid{Base}@>>;
\end{itemdecl}

\begin{itemdescr}
    \pnum
    \effects
    Initializes \exposid{end_} with \tcode{move(other.\exposid{end_})}.
\end{itemdescr}

\begin{itemdecl}
    constexpr sentinel_t<@\exposid{Base}@> base() const;
\end{itemdecl}

\begin{itemdescr}
    \pnum
    \effects
    Equivalent to: \tcode{return \exposid{end_};}
\end{itemdescr}

\begin{itemdecl}
    friend constexpr bool operator==(const @\exposid{iterator}@<Const>& x, const @\exposid{sentinel}@& y);
\end{itemdecl}

\begin{itemdescr}
    \pnum
    \effects
    Equivalent to: \tcode{return x.\exposid{current_} == y.\exposid{end_};}
\end{itemdescr}

\begin{itemdecl}
    friend constexpr range_difference_t<@\exposid{Base}@>
    operator-(const @\exposid{iterator}@<Const>& x, const @\exposid{sentinel}@& y)
    requires sized_sentinel_for<sentinel_t<@\exposid{Base}@>, iterator_t<@\exposid{Base}@>>;
\end{itemdecl}

\begin{itemdescr}
    \pnum
    \effects
    Equivalent to: \tcode{return x.\exposid{current_} - y.\exposid{end_};}
\end{itemdescr}

\begin{itemdecl}
    friend constexpr range_difference_t<@\exposid{Base}@>
    operator-(const @\exposid{sentinel}@& x, const @\exposid{iterator}@<Const>& y)
    requires sized_sentinel_for<sentinel_t<@\exposid{Base}@>, iterator_t<@\exposid{Base}@>>;
\end{itemdecl}

\begin{itemdescr}
    \pnum
    \effects
    Equivalent to: \tcode{return x.\exposid{end_} - y.\exposid{current_};}
\end{itemdescr}

\end{addedblock}


\section{References}
\renewcommand{\section}[2]{}%
\bibliographystyle{plain}
\bibliography{wg21, extra}

\begin{thebibliography}{9}
    \bibitem[N4878]{N4878}
    Richard Smith
    \emph{Working Draft, Standard for Programming Language C++}\newline
    \url{https://wg21.link/N4878}

\end{thebibliography}
\end{document}