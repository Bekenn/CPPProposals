% !TeX program = luatex
% !TEX encoding = UTF-8


\RequirePackage{luatex85}%
\documentclass{wg21}

\usepackage{xcolor}
\usepackage{soul}
\usepackage{ulem}
\usepackage{fullpage}
\usepackage{parskip}
\usepackage{listings}
\usepackage{enumitem}
\usepackage{makecell}
\usepackage{longtable}
\usepackage{graphicx}
\usepackage{amssymb}
\RequirePackage{fontspec}
\usepackage{newunicodechar}
\usepackage{url}

\setmonofont{Inconsolatazi4}
\setmainfont{Noto Serif}
\usepackage{csquotes}


\lstdefinestyle{CPP}{language=C++,
    keywordstyle=\color{blue}\ttfamily,
    stringstyle=\color{red}\ttfamily,
    commentstyle=\color{OliveGreen}\ttfamily,
    morecomment=[l][\color{magenta}]{\#},
    morekeywords={reflexpr,valueof,exprid,typeof, concept,constexpr,consteval,unqualid}
}

\lstdefinestyle{ASM}{
    basicstyle=\tiny\ttfamily,
    numbers=left, 
    numberstyle=\tiny\ttfamily, 
    keywordstyle=\color{blue}\ttfamily,
    stringstyle=\color{red}\ttfamily,
    commentstyle=\color{OliveGreen}\ttfamily,
    breaklines=true,
    columns=fullflexible,
    literate={\_}{{\_}}1,
    postbreak=\mbox{\textcolor{red}{$\hookrightarrow$}\space},
    frame = single, 
    language=[x86masm]{Assembler}, 
    % with these extra keywords:
    morekeywords={CDQE,CQO,CMPSQ,CMPXCHG16B,JRCXZ,LODSQ,MOVSXD, %
        POPFQ,PUSHFQ,SCASQ,STOSQ,IRETQ,RDTSCP,SWAPGS, %
        rax,rdx,rcx,rbx,rsi,rdi,rsp,rbp, %
        r8,r8d,r8w,r8b,r9,r9d,r9w,r9b, %
        r10,r10d,r10w,r10b,r11,r11d,r11w,r11b, %
        r12,r12d,r12w,r12b,r13,r13d,r13w,r13b, %
        r14,r14d,r14w,r14b,r15,r15d,r15w,r15b}
} % etc.


\title{Add a step parameter to \tcode{iota\_view}}
\docnumber{P2016R0}
\audience{LEWG, LEWGI}
\author{Corentin Jabot}{corentin.jabot@gmail.com}

\begin{document}
    
\maketitle


\section{Abstract}

We propose adding a step parameter to \tcode{iota\_view}.

\section{Proposal}

We propose adding a step parameter to \tcode{iota_view} such that

\begin{itemize}
\item \tcode{views::iota} can generate increasing values with an increment different than 1
\item \tcode{views::iota} can generate decreasing values, by the mean of a negative increment (-1 or arbitrary integral value) 
\end{itemize}

\begin{lstlisting}[style=CPP]
view::iota(0, 10, 2); // [0, 4, 6, 8]
view::iota(10, 0, -2); // [10, 8, 6, 6, 2]
view::iota(0, unreacheable_t{}, -1) | views::take(5); // [0, -1, -2, -3, -4]
\end{lstlisting}

\section{Motivation}

I suggested including a step parameter to \tcode{iota_view} in Belfast while discussing P1894 \cite{P1894R0}.
This would allow \tcode{views::iota} to offer a feature set equivalent to that of Python's \tcode{range} function.

It is notably useful to iterate over interlaced data such as matrices, images or audio signals.
It can also be used to implements counters, paging and any other use case requiring linear sequences.
That feature has been requested in Range-v3 \cite{RangeV3}.


However, \tcode{iota_view} can be expressed in terms of \tcode{view::stride} \cite{P1899R0} and \tcode{views::reverse}.
While \tcode{iota_view} can be more efficient (and less verbose) than combining multiple views, a combination of \tcode{views::stride} and \tcode{views::enumerate} should be preferred over using \tcode{iota_view} when iterating over a container.

As such, despite the relative simplicity of the proposal and the general usefulness of the feature 
demonstrated by other languages and libraries, it is unclear that the cost/benefit ratio of the present proposal plays
in its favor.
In any case, \tcode{view::stride} \cite{P1899R0} or something similar \cite{Conor} should be prioritized over the present proposal.

\section{Implementation}
An implementation is available  \url{https://github.com/ericniebler/range-v3/pull/1392}
\section{Wording}

No wording is provided at this time, the paper intends to gauge interest.

\bibliographystyle{plain}
\bibliography{../wg21}

\let\oldsection=\section
\renewcommand{\section}[2]{}%
\begin{thebibliography}{9}    
    
    \bibitem[RangeV3]{RangeV3}
    \emph{step size for monotonically increasing values}\newline
    \url{https://github.com/ericniebler/range-v3/issues/723} 
    
    \bibitem[Conor]{Conor}
    \emph{CppCon 2019: Conor Hoekstra “23 Ranges: slide \& stride”}\newline
    \url{https://www.youtube.com/watch?v=-_lqZJK2vjI} 
\end{thebibliography}



\let\section=\oldsection


\end{document}
