\documentclass{wg21}

\usepackage{xcolor}
\usepackage{soul}
\usepackage{ulem}
\usepackage{fullpage}
\usepackage{parskip}
\usepackage{csquotes}
\usepackage{listings}
\usepackage{minted}
\usepackage{enumitem}
\usepackage{minted}


\lstdefinestyle{base}{
language=c++,
breaklines=false,
basicstyle=\ttfamily\color{black},
moredelim=**[is][\color{green!50!black}]{@}{@},
escapeinside={(*@}{@*)}
}

\newcommand{\cc}[1]{\mintinline{c++}{#1}}
\newminted[cpp]{c++}{}


\title{Range constructor for \tcode{std::span}}
\docnumber{P1394R3}
\audience{LWG}
\author{Corentin Jabot}{corentin.jabot@gmail.com}
\authortwo{Casey Carter}{casey@carter.net}


\begin{document}
\maketitle

\section{Abstract}

This paper proposes that \tcode{span} be constructible from any contiguous forwarding-range with a compatible element type.
The idea was extracted from P1206.

\section{Tony tables}
\begin{center}
\begin{tabular}{l|l}
Before & After\\ \hline

\begin{minipage}[t]{0.5\textwidth}
\begin{minted}[fontsize=\footnotesize]{cpp}
std::vector<int> v(42);
std::span<int> foo =
    v | view::take(3); //ill-formed
\end{minted}
\end{minipage}
&
\begin{minipage}[t]{0.5\textwidth}
\begin{minted}[fontsize=\footnotesize]{cpp}
std::vector<int> v(42);
std::span foo = v | view::take(3); //valid
\end{minted}
\end{minipage}
\\\\ \hline


\begin{minipage}[t]{0.5\textwidth}
\begin{minted}[fontsize=\footnotesize]{cpp}
std::vector<int> v(42);
std::span bar(v.begin(), 3); // ill-formed
\end{minted}
\end{minipage}
&
\begin{minipage}[t]{0.5\textwidth}
\begin{minted}[fontsize=\footnotesize]{cpp}
std::vector<int> v(42);
std::span bar(v.begin(), 3); // valid

\end{minted}
\end{minipage}
\\\\ \hline

\begin{minipage}[t]{0.5\textwidth}
\begin{minted}[fontsize=\footnotesize]{cpp}
std::vector<int> get_vector();
void foo(std:::span<int>);
void bar(std::span<const int>);
bar(get_vector()); //valid
foo(get_vector()); //ill-formed
\end{minted}
\end{minipage}
&
\begin{minipage}[t]{0.5\textwidth}
\begin{minted}[fontsize=\footnotesize]{cpp}
std::vector<int> get_vector();
void foo(std::span<int>);
void bar(std::span<const int>);
bar(get_vector()); //valid
foo(get_vector()); //ill-formed
\end{minted}
\end{minipage}
\\\\ \hline
\end{tabular}
\end{center}


\section{Motivation}

\tcode{std::span} is specified to be constructible from \tcode{Container} types.
However, while defined, \tcode{Container} is not a concept and as such \tcode{ContiguousRange} is more expressive.
Furthermore, there exist some non-container ranges that would otherwise be valid ranges to construct span from.
As such span as currently specified fits poorly with the iterators / ranges model of the rest of the standard library.

The intent of span was always to be constructible from a wide number of compatible types,
whether standard contiguous containers, non-standard equivalent types, or views.
This proposal ensure that span, especially when used as parameter of a function will be constructible
from all compatible types while offering stronger and more consistent (in regard to Range) lifetime guarantees.


\section{Design considerations}

We propose to specify all constructors currently accepting a container or pointers in terms of \tcode{ContiguousRange} and \tcode{ContiguousIterator}
respectively as well as to add or modify the relevant deduction guides for these constructors.

\section{Future work}

\begin{itemize}
    \item We suggest that both the wording and the implementation of span would greatly benefit from a trait to detect whether a type has a static extent.
Because \tcode{std::extent} equals to 0 for types without static extent, and because 0 is a valid extent for containers, \tcode{std::extent} proved too limited. However we do not propose a solution in the present paper.
\end{itemize}

\section{Proposed wording}


Change in \textbf{[views.span] 21.7.3}:
\begin{quote}
\begin{codeblock}

// [span.cons], constructors, copy, and assignment
constexpr span() noexcept;
@\added{template <class It>}@
constexpr span( @\removed{pointer ptr}@ @\added{It begin}@, index_type count);
@\removed{constexpr span(pointer first, pointer last);}@
@\added{template <class It, class End>}@
@\added{constexpr span(It first, End last);}@

template<size_t N>
constexpr span(element_type (&arr)[N]) noexcept;
template<size_t N>
constexpr span(array<value_type, N>& arr) noexcept;
template<size_t N>
constexpr span(const array<value_type, N>& arr) noexcept;
@\removed{template<class Container>}@
@\removed{constexpr span(Container\& cont);}@
@\removed{template<class Container>}@
@\removed{constexpr span(const Container\& cont);}@
@\added{template <class R>}@
@\added{constexpr span(R\&\& r);}@

constexpr span(const span& other) noexcept = default;
template<class OtherElementType, ptrdiff_t OtherExtent>
constexpr span(const span<OtherElementType, OtherExtent>& s) noexcept;

...

}


template<class T, size_t N>
span(T (&)[N]) -> span<T, N>;
template<class T, size_t N>
span(array<T, N>&) -> span<T, N>;
template<class T, size_t N>
span(const array<T, N>&) -> span<const T, N>;
@\added{template <class It, class End>}@
@\added{span(It, End) -> span<remove_reference_t<iter_reference_t<It>>>;}@

template<class T, size_t N>
span(const array<T, N>&) -> span<const T, N>;
@\removed{template<class Container>}@
@\removed{span(Container\&) -> span<typename Container::value_type>;}@
@\removed{template<class Container>}@
@\removed{span(const Container\&) -> span<const typename Container::value_type>;}@
@\added{template<class R>}@
@\added{span(R\&\&) -> span<ranges::range_value_t<R>>;}@

\end{codeblock}
\end{quote}

In 21.7.3.2 [span.cons]

\begin{quote}


\begin{itemdecl}
    constexpr span() noexcept;
\end{itemdecl}
\begin{itemdescr}
    \pnum
    \ensures
    \tcode{size() == 0 \&\& data() == nullptr}.

    \pnum
    \remarks
    This constructor shall not participate in overload resolution
    unless \tcode{Extent <= 0} is \tcode{true}.
\end{itemdescr}
\begin{removedblock}
\begin{itemdecl}
constexpr span(pointer ptr, index_type count);
\end{itemdecl}
\end{removedblock}
\begin{addedblock}
\begin{itemdecl}
template <class It>
constexpr span(It first, index_type count);
\end{itemdecl}
\end{addedblock}
\begin{itemdescr}
    \begin{addedblock}
    \pnum
    \constraints
    \begin{itemize}
    	\item \tcode{It} satisfies \tcode{ContiguousIterator},
    	\item \tcode{is_convertible_v<remove_reference_t<iter_reference_t<It>>(*)[], element_type(*)[]>} is true.
    	 \begin{note}The intent is to allow only qualification conversions of the iterator reference type to \tcode{element_type} \end{note}
    \end{itemize}
    \end{addedblock}

    \expects
\begin{itemize}
    \item \range{\changed{ptr}{first}}{ \changed{ptr}{first} + count} shall be a valid range.
\begin{addedblock}
    \item \tcode{It} models \tcode{ContiguousIterator},
\end{addedblock}
    \item If \tcode{extent} is not equal to \tcode{dynamic_extent},
    then \tcode{count} shall be equal to \tcode{extent}.
\end{itemize}
    

    \pnum
    \effects
    Constructs a \tcode{span} that is a view over the range \range{\changed{ptr}{first} }{\changed{ptr}{first}  + count}.

    \pnum
    \ensures
    \begin{removedblock}
    	\tcode{size() == count \&\& data() == ptr}.
    \end{removedblock}
    \begin{addedblock}
    \begin{itemize}
    	\item \tcode{size() == count} is true
    	\item \tcode{to_address(first) == data()} is true
    %%	\item if \tcode{count} is not equal to \tcode{0}, \tcode{data() == addressof(*first)} is true, otherwise, \tcode{data() == nullptr} is true.
    \end{itemize}
	\end{addedblock}	
    		

    \pnum
    \throws
    Nothing.

\end{itemdescr}


\begin{removedblock}
\begin{itemdecl}
constexpr span(pointer first, pointer last);
\end{itemdecl}
\end{removedblock}
\begin{removedblock}

\begin{itemdescr}
    \pnum
    \requires
    \range{first}{last} shall be a valid range.
    If \tcode{extent} is not equal to \tcode{dynamic_extent},
    then \tcode{last - first} shall be equal to \tcode{extent}.

    \pnum
    \effects
    Constructs a span that is a view over the range \range{first}{last}.

    \pnum
    \ensures
    \tcode{size() == last - first \&\& data() == first}.

    \pnum
    \throws
    Nothing.
\end{itemdescr}
\end{removedblock}

\begin{addedblock}
\begin{itemdecl}
template <class It, class End>
constexpr span(It first, End last);
\end{itemdecl}
\end{addedblock}

\begin{addedblock}

\begin{itemdescr}
	 \pnum
	\constraints
	\begin{itemize}
		\item \tcode{is_convertible_v<remove_reference_t<iter_reference_t<It>>(*)[], element_type(*)[]>} is true,
		\begin{note}The intent is to allow only qualification conversions of the iterator reference type to \tcode{element_type} \end{note},
		\item \tcode{It} satisfies \tcode{ContiguousIterator},
		\item \tcode{End} satisfies \tcode{SizedSentinel<It>},
		\item \tcode{is_convertible_v<End, size_t>} is false.
	\end{itemize}

    \expects
    \begin{itemize}
    \item
    If \tcode{extent} is not equal to \tcode{dynamic_extent},
    then \tcode{last - first} is equal to \tcode{extent},
    \item \range{first}{last} is a valid range,
    \item \tcode{It} models \tcode{ContiguousIterator},
    \item \tcode{End} models \tcode{SizedSentinel<It>}.
    \end{itemize}

    \pnum
    \effects
    Constructs a span that is a view over the range \range{first}{last}.

    \pnum
    \ensures
    
    \begin{itemize}
    	\item \tcode{size() == last - first} is true,
    	\item \tcode{to_address(first) == data()} is true.
    	%% \item if \tcode{first != last} \tcode{data() == addressof(*first)} is true, otherwise, \tcode{data() == nullptr} is true.
    \end{itemize}
   
    \pnum
    \throws
    Nothing.


\end{itemdescr}
\end{addedblock}

\begin{itemdecl}
template<size_t N> constexpr span(element_type (&arr)[N]) noexcept;
template<size_t N> constexpr span(array<value_type, N>& arr) noexcept;
template<size_t N> constexpr span(const array<value_type, N>& arr) noexcept;
\end{itemdecl}
\begin{itemdescr}
    \pnum
    \effects
    Constructs a \tcode{span} that is a view over the supplied array.

    \pnum
    \ensures
    \tcode{size() == N \&\& data() == data(arr)}.

    \pnum
    \remarks
    These constructors shall not participate in overload resolution unless:
    \begin{itemize}
        \item \tcode{extent == dynamic_extent || N == extent} is \tcode{true}, and
        \item \tcode{remove_pointer_t<decltype(data(arr))>(*)[]} is convertible to \tcode{element_type(*)[]}.
    \end{itemize}
\end{itemdescr}

\begin{removedblock}
\begin{itemdecl}
template<class Container> constexpr span(Container& cont);
template<class Container> constexpr span(const Container& cont);
\end{itemdecl}

\begin{itemdescr}
\pnum
\constraints
\begin{itemize}
	\item \tcode{extent == dynamic_extent} is \tcode{true},
	\item \tcode{Container} is not a specialization of \tcode{span},
	\item \tcode{Container} is not a specialization of \tcode{array},
	\item \tcode{is_array_v<Container>} is \tcode{false},
	\item \tcode{data(cont)} and \tcode{size(cont)} are both well-formed, and
	\item \tcode{remove_pointer_t<decltype(data(cont))>(*)[]} is convertible to \tcode{ElementType(*)[]}.
\end{itemize}

\pnum
\expects
\range{data(cont)}{data(cont) + size(cont)} is a valid range.

\pnum
\effects
Constructs a \tcode{span} that is a view over the range \range{data(cont)}{data(cont) + size(cont)}.

\pnum
\ensures
\tcode{size() == size(cont) \&\& data() == data(cont)}.

\pnum
\throws
What and when \tcode{data(cont)} and \tcode{size(cont)} throw.
\end{itemdescr}
\end{removedblock}


\begin{addedblock}
\begin{itemdecl}
template <class R>
constexpr span(R&& r);
\end{itemdecl}
\end{addedblock}

\begin{addedblock}
\begin{itemdescr}
    \pnum
    \constraints
    \begin{itemize}
       \item \tcode{extent == dynamic_extent} is true,
       \item \tcode{R} satisfies \tcode{ranges::ContiguousRange} and \tcode{ranges::SizedRange},
       \item either \tcode{R} satisfies \placeholder{forwarding-range} or \tcode{is_const_v<element_type>} is true,
       \item \tcode{remove_reference_t<R>} is not a specialization of \tcode{span},
       \item \tcode{remove_reference_t<R>} is not a specialization of \tcode{array},
       \item \tcode{is_array_v<remove_reference_t<R>>} is \tcode{false},
       \item \tcode{is_convertible_v<remove_reference_t<iter_reference_t<ranges::iterator_t<R>>>(*)[], element_type(*)[]>} is true
       \begin{note}The intent is to allow only qualification conversions of the iterator reference type to \tcode{element_type} \end{note}.
    \end{itemize}

	\expects \tcode{R} models \tcode{ranges::ContiguousRange} and \tcode{ranges::SizedRange}.

    \pnum
    \ensures
    \tcode{size() == ranges::size(r) \&\& data() == ranges::data(r)} is true.

    \pnum
    \throws
    What and when \tcode{ranges::data(r)} and \tcode{ranges::size(r)} throw.

\end{itemdescr}
\end{addedblock}

\begin{itemdecl}
constexpr span(const span& other) noexcept = default;
\end{itemdecl}
\begin{itemdescr}
    \pnum
    \ensures
    \tcode{other.size() == size() \&\& other.data() == data()}.
\end{itemdescr}

\end{quote}


Add a new section [span.deduction] to describe the following deduction guides:


\begin{quote}
\begin{addedblock}
\begin{itemdecl}
template <class It, class End>
span(It, End) -> span(It, End) -> span<remove_reference_t<iter_reference_t<It>>>;
\end{itemdecl}
\begin{itemdescr}
    \constraints
\begin{itemize}
    \item \tcode{It} satisfies \tcode{ContiguousIterator},
    \item \tcode{End} satisfies \tcode{SizedSentinel<It>}.
\end{itemize}

\end{itemdescr}

\begin{itemdecl}
template<class R>
span(R&&) -> span<ranges::range_value_t<R>>;
\end{itemdecl}
\begin{itemdescr}
    \constraints \tcode{R} satisfies \tcode{ranges::ContiguousRange}.
\end{itemdescr}


\end{addedblock}

\end{quote}

\section{References}
\renewcommand{\section}[2]{}%
\begin{thebibliography}{9}

    \bibitem[P1419]{P1419}
    Casey Carter, Corentin Jabot
    \emph{A SFINAE-friendly trait to determine the extent of statically sized containers}\newline
    \url{https://wg21.link/P1419}

    \bibitem[P1391]{P1391}
    Corentin Jabot
    \emph{Range constructor for std::string\_view}\newline
    \url{https://wg21.link/P1391}


    \bibitem[P1474]{P1474}
    Casey Carter
    \emph{Helpful pointers for ContiguousIterator}\newline
    \url{https://wg21.link/P1474}


\end{thebibliography}
\end{document}

\end{document}